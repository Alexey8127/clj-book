\chapter{Тесты в Clojure}

\begin{teaser}
В последней главе мы поговорим о тестировании приложений. Читатель узнает, что
такое тесты и каких видов они бывают. Постараемся избежать лишней сложности: не
будем злоупотреблять терминами вроде TDD и BDT. Покажем, что в мире Clojure
легко писать и поддерживать тесты.
\end{teaser}

\section{Основные понятия}

На тему тестов написано много книг и статей, придуманы разные методологии. Их
сторонники отстаивают позиции в бесконечных спорах. Начинающих сбивает с толку
множество терминов и мнений. Поможем читателю погрузиться в тему: расскажем
основные моменты тестирования простыми словами.

Тест~--- это код, который проверяет другой код. Например, мы написали функцию
перевода температуры из Цельсия в Фаренгейты:

\begin{verbatim}
(defn ->fahr [cel]
  (+ (* cel 1.8) 32))
\end{verbatim}

Вызвали ее несколько раз убедились, что результаты такие же, как в
справочнике. Зафиксируем расчеты в функции проверки. Функция сравнивает вызов
\spverb|->fahr| с каноничными значениями. Их посчитали заранее и расценивают как
эталон.

\begin{verbatim}
(defn test-fahr []
  (assert (= (int (->fahr 20)) 68))
  (assert (= (int (->fahr 100)) 212)))
\end{verbatim}

Макрос \spverb|assert| бросит исключение, если тело вычисляется в ложь. В данный
момент \spverb|(test-fahr)| вернет \spverb|nil| без ошибок. Если внести в
расчеты изменения, получим ошибку:

\begin{verbatim}
(defn ->fahr [cel]
  (+ (* cel 1.9) 32))

(test-fahr)
Assert failed: (= (int (->fahr 20)) 68)
\end{verbatim}

Функция \spverb|test-fahr| и есть тест. Она проверяет, что никто не изменил код
\spverb|->fahr| так, что получится другой результат. В наших интересах вызвать
\spverb|(test-fahr)| перед сборкой программы. Так мы не допустим, чтобы код с
ошибкой попал в производство.

Техническая деталь: в тесте мы сравниваем результат \spverb|->fahr| с целым
числом. Без обертки в \spverb|(int ...)| функция вернет число с плавающей
запятой, которое трудно сравнить с другим таким же числом. Поэтому значение
приводят к целому.

Тест не знает устройство функции, которую проверяет. Мы вправе менять алгоритм
до тех пор, пока тест выполняется без ошибок. Предположим, мы тестируем функцию
факториала. В первой версии мы линейно умножаем числа, что неэффективно. Но уже
на этой стадии добавим тест, что $5! = 120$. Если заменить линейный алгоритм на
дерево или таблицу значений, тест пройдет без ошибок. Иначе в новом алгоритме
ошибка.

\subsection{Кейсы и покрытие}

Близкие по семантике тесты объединяются в \emph{кейсы}. Вспомним школьную задачу
с квадратным уравнением. Требуется найти корни по заданным коэффициентам $a$,
$b$, $c$. Особенность в том, что ее логика задачи ветвится. В зависимости от
параметров может быть два корня, один или ни одного.

\begin{verbatim}
(defn square-roots [a b c]
  (let [D (- (* b b) (* 4 a c))]
    (cond
      (pos? D) [(/ (+ (- b) (Math/sqrt D)) (* 2 a))
                (/ (- (- b) (Math/sqrt D)) (* 2 a))]
      (zero? D) (/ (- b) (* 2 a))
      (neg? D) nil)))
\end{verbatim}

Чтобы проверить алгоритм, тест вызывает функцию \spverb|square-roots| минимум
три раза. Подберем параметры таким образом, чтобы сработала каждая ветка. Еще
лучше, если на каждую ветку приходится отдельный тест с разными
значениями. Каждый тест можно расширить в будущем. Тесты ниже образуют кейс,
который проверяет алгоритм целиком.

%% TODO check ints

\begin{verbatim}
(defn test-square-roots-two-roots []
  (let [[x1 x2] (square-roots 1 -5 6)]
    (assert (= [(int x1) (int x2)] [3 2]))))

(defn test-square-roots-one-root []
  (assert (= (square-roots 1 6 9) -3)))

(defn test-square-roots-no-roots []
  (assert (nil? (square-roots 2 4 7))))
\end{verbatim}

В объектно-ориентированных языках кейсы это классы, а тесты~--- их методы. В
Clojure тест это функция, а пространство похоже на кейс.

В тестах часто упоминают \emph{покрытие}. Под этим словом имеют в виду долю
кода, которая сработала в момент запуска теста. Предположим, мы написали тесты
только для двух корней. В этом случае ветки \spverb|(zero? D)| и \spverb|(neg? D)|
не выполнятся. Тестовый фреймворк считает общее число строк, число выполненных строк
и коэффициент покрытия. Некоторые фреймворки строят отчет, где выполненные строки
отмечены зеленым (плюсом), а пропущенные красным (минусом). Обычно покрытие выносят
в отдельный модуль фреймворка.

\begin{verbatim}
+(defn square-roots [a b c]
+  (let [D (- (* b b) (* 4 a c))]
+    (cond
+      (pos? D) [(/ (+ (- b) (Math/sqrt D)) (* 2 a))
+                (/ (- (- b) (Math/sqrt D)) (* 2 a))]
-      (zero? D) (/ (- b) (* 2 a))
-      (neg? D) nil)))
\end{verbatim}

В функции \spverb|square-roots| всего 7 строк. При запуске теста сработали 5 из
них. Покрытие составит 5/7, что приблизительно 71\%.

Принято считать, что покрытия 80\% достаточно для того, чтобы код работал
надежно. Значения ниже говорят о малом покрытии. Это значит, в проекте
встречается код, который не защищен от спонтанных изменений. При попытке
приблизиться к 100\% тесты становятся слишком затратными. В каждой команде
придерживаются той величины, которая удобна бизнесу.

Процентная величина не должна затмевать здравый смысл. В примере выше цифра 71\%
кажется большой только на первый взгляд. Из покрытия видно, что мы проверяем
одну ветку алгоритма из трех. Если в двух других ошибки, мы не узнаем о них. При
оценке покрытия смотрят не на процентную величину, а на ветвление
алгоритма. Надежный тест гарантирует, что сработала каждая ветка.

\subsection{Не только числа}

У начинающих складывается мнение, что тесты пишут для математических
расчетов. Это не так. Тесты подходят для всех функций, которые хотят
зафиксировать. Это может быть что угодно: обход коллекции, криптография, поиск в
тексте.

Рассмотрим пример с подписью HTTP-запроса. Функция \spverb|sign-params|
принимает словарь параметров и секретный ключ. Алгоритм подписи следующий:

\begin{itemize}

\item
  отсортировать параметры по ключам;

\item
  составить строку \spverb|param1=value1&param2=value2...|;

\item
  экранировать некоторые символы: пробел, плюс, процент и другие;

\item
  получить сигнатуру строки по алгоритму \spverb|HMAC SHA256| и секретному
  ключу;

\item
  вернуть словарь параметров с полем \spverb|:signature|.

\end{itemize}

Так работает авторизация в API популярных сервисов. Напишем тест для функции
подписи. Проверим, что входные параметры дополнены сигнатурой, которую мы
посчитали заранее.

\begin{verbatim}
(defn test-sign-params []
  (let [api-key "2Ag48&@%776^634Tsdf23"
        params {:action :postComment
                :user_id 42
                :post_id 1999
                :comment "This is a great article!"}
        signature "e36b331823b..."]
    (assert (= (assoc params :signature signature)
               (sign-params params api-key)))))
\end{verbatim}

Заметим, что \spverb|sign-params| работает не с числами, а коллекциями и
криптографией. Это не помешало написать для нее тест. Чтобы улучшить нашу
работу, добавьте проверку на пустой ключ, экранирование символов, кириллицу,
эмодзи и так далее.

\subsection{Что тестировать}

Иногда разработчики не понимают, какие функции покрыть тестами, и тестов не
хватает. Бывает и другая крайность: тесты пишут строго для каждой функции. Это
приводит к их избыточности и замедляет разработку. Верное решение, как всегда,
посередине.

Тестируйте \emph{действия}, которые атомарны с точки зрения логики. Например,
подпись запроса состоит из мелких функций: сортировки, экранирования и
других. Если проверять каждую из них, получится слишком много кода. Позже
алгоритм подписи поменяют, а тесты выкинут. И наоборот, когда тестируют огромный
пласт кода, трудно понять, почему возникла ошибка. Тест становится хрупким:
малейшее изменение рушит его. Крупные тесты разбивают на несколько поменьше.

Подробнее на вопрос отвечает пирамида
тестов\footurl{https://martinfowler.com/bliki/TestPyramid.html}. Это вымышленная
модель, где тесты разделены на слои. Чем выше слой, тем б\'{о}льшую часть логики
он покрывает. Говорить о пирамиде еще преждевременно; мы вернемся к ней ближе к
концу главы.

\subsection{О чистоте функций}

Функцию удобно тестировать, если у нее нет побочных эффектов. Это значит,
функция не обращается к диску, сети и глобальным переменным. Результат зависит
\emph{только} от входных параметров.

Представим, что функция читает файл. Перед запуском теста мы должны записать
файл в нужное место, а потом удалить его. Технически это возможно, но тесты
усложняются: теперь они зависят от окружения. Чтобы облегчить тестирование,
функцию разделяют на две. Первая читает содержимое файла и передает строку во
вторую. Функция до исправления:

\begin{verbatim}
(defn process-csv [path]
  (let [content (slurp path)]
    (for [line (clojure.string/split content #"\n")]
      (remap-line line))))
\end{verbatim}

\noindent
и после:

\begin{verbatim}
(defn process-csv-content [content]
  (for [line (clojure.string/split content #"\n")]
    (remap-line line)))

(defn process-csv [path]
  (process-csv-content (slurp path)))
\end{verbatim}

Напишем тест для \spverb|process-csv-content|. Не придется писать файл на диск,
потому что функция принимает строку, которую объявим тут же.

\begin{verbatim}
(def CONTENT
  (str "Ivan;ivan@test.ru;http://example.ru"
       \newline
       "John;john@test.com;http://example.com"))

(assert (= [{:name "Ivan" :email ...}
            {:name "John" :email ...}]
           (process-csv-content CONTENT)))
\end{verbatim}

Другой способ в том, чтобы заменить функцию на мультиметод. Его реализация
зависит от первого аргумента. Если это \spverb|java.io.File|, мы читаем файл и
вызываем метод с содержимым. Для \spverb|String| запускается разбор данных.

Чистая функция это идеальный случай тестирования. На практике вы столкнетесь с
тем, что промышленный код трудно изолировать от побочных эффектов. В некоторых
случаях это даже невозможно. Чтобы тестировать код с эффектами, прибегают к
расширенным практикам: используют заглушки (стабы), подменяют объекты в рантайме
(моки). Эти и другие приемы мы рассмотрим ниже.

Заметим, что подмена среды это вынужденная мера, и к ней прибегают в последнюю
очередь. Перед тестированием проверьте, можно ли выделить из кода его чистую
часть. Если да, задайте отдельную функцию и тестируйте ее.

%% ------------

\subsection{Производство}

Тесты не только проверяют логику приложения. Они помогают проекту на разных
стадиях производства. Например, ветку с изменениями нельзя слить в мастер до тех
пор, пока не пройдут все тесты. Этим вы обезопасите код от случайных изменений.

Если разработчик добавил код, не не покрыл его тестами, это снизит долю
покрытия. Можно задать правило, что при значении ниже заданного слияние такой
ветки запрещено. Сервисы GitHub, CircleCI и другие гибко реагируют на ошибки в
тестах. Они шлют письма с отчетом, пишут сообщения в Slack, всячески
сигнализируют о проблеме.

В некоторых случаях тесты заменяют документацию. Это утверждение нельзя
воспринимать буквально~--- настоящую документацию не заменит ничто. Но в open
source-проектах ее недостаток иногда компенсируют тестами. Это нормально, потому
что на качественную документацию уходит столько же времени, сколько на
техническую часть. Если вы не уверены, как работает та или иная функция,
возможно, ее поведение прояснится из тестов.

\subsection{Устройство тестов}

Функции с макросом assert это сильно упрощенные тесты. Они пригодятся на раннем
этапе разработки, когда автор только обозначил контуры программы. Assert-ы это
временные меры чтобы сдержать код от изменений. Для полноценной работы с тестами
понадобится фремворк. Это библиотека, которая определяет, где и как описаны
тесты.

Когда выполнить команду lein test в папке проекта, тем самым мы запустим
тестовый фреймворк. Это сложный цикл из нескольких шагов. Мы рассмотрим основные
из них, чтобы читатель понимал так называемый стек тестов.

На первом этапе фреймворк ищет тесты в кодовой базе проекта. По-другому фаза
называется test discovery. Это не простая задача: фреймворк загружает все модули
и ищет среди них тесты. Каждый найденный тест попадает в глобальный регистр.

Тест должен носить какой-то признак, по которому его можно отличить от обычной
функции. В примере выше test-sign-params это именно функция, а не тест (префикс
test- не делает функцию тестом). Далее мы рассмотрим, как превратить ее в тест.

Когда тесты найдены, система определяет, какие из них будут запущены. Если
фильтры не заданы, сработают все тесты. Иногда нужно выполнить тесты только из
определенного модуля или по признаку. Ести такие параметры заданы, сработает
фильтрация тестов.

На этапе прогона фреймворк выполняет тесты один за другим как функции. Система
фиксирует их количество и имена. Вызов теста обернут в try/catch, чтобы
исключение не прервало процесс. Каждая проверка попадает в глобальную
переменную. Эти сведения необходимы, чтобы построить отчет.

На этапе отчета система выводит накопленные данные в консоль. Качественный
фреймворк отличает удобство вывода~--- насколько он понятен человеку. Проблемные
тесты должны быть обозначены красным цветом. Для неудачных сравнений мы ожидаем
полные формы, а не финальные значения. Например, \spverb|(= (int (->fahr 20)) 68)|
вместо \spverb|(= 69 68)|.

Промышленные фреймворки показывают отчет в разных форматах. Кроме текста в
консоли это может быть HTML-файл. Систему сборки настраивают так, что отчет о
тестах публикуется на внутреннем веб-сервере по адресу
\spverb|/<project>-<artifact>-test.html|. Часть \spverb|<artifact>| это имя ветки или хэш
коммита.

Стандарт XUnit определяет, как записать отчет в виде XML-файла. Промышленные
системы вроде TeamCity и Atlassian понимают, как вывести этот файл в удобном
виде. XUnit удобен тем, что это стандарт. У вас могут быть тесты на Python,
JavaScript или Clojure, но если прогон запущен с флагом~---xunit (или подобным),
отчеты будут выглядеть одинаково.

Это был поверхностный обзор того, как устроен обычный тестовый фреймворк. В
последующих разделах мы рассмотрим, как управлять им на каждом этапе.

\subsection{Типы тестов}

Прежде чем писать тесты, разберемся каких видов они бывают. Возможно, читатель
слышал термины "юнит", "интеграционные" и другие. На первый взгляд кажется, что
это излишняя классификация. На практике тесты из разных категорий отличаются
стилем, окружением и порядком запуска.

Понятие "юнит-тест" происходит от английского unit, модуль. Термин пришел из
мира Java и фреймворка JUnit. Согласно его идеологии, тесты группируют по
модулям. На каждый модуль из кодовой базы приходится одноименный тестовый,
который проверяет функциональность первого.

Юнит-тесты это простой способ зафиксировать логику приложения. Чаще всего проект
включает только эту категорию тестов. Юнит-тестирование поощряет чистые
функции. Чем меньше требований к окружению тестов, тем они удобней в
поддержке. Рассматривайте юнит-тесты как проверку отдельных частей приложения
(но не всего разом). Примеры выше подходят под категорию юнит-тестов.

Интеграционные тесты решают проблему, с которой не справляются юнит-тесты. Даже
когда отдельные части приложения проверены, в их совместной работе может что-то
пойти не так. Это известный феномен; про него говорят, что целое больше суммы
его частей.

%% TODO

В качестве примера вспомним регистрацию на сайте. Это сложный процесс из
нескольких шагов:

- проверить, что такого пользователя нет в базе;
- создать временную учетную запись;
- сгенерировать ссылку для подтверждения записи;
- отправить письмо с этой ссылкой;
- проверить, что ссылка не подделана и не вышел срок ее действия;
- активировать учетную запись.

Даже если каждый шаг обильно покрыт тестами, это не гарантирует работу всей
цепи. Например, письмо не уйдет адресату или попадет в спам. В одном из
браузеров не сработает AJAX-запрос или поедет верстка. Причины такого уровня
слишком абстрактны, чтобы решать их на уровне юнит-тестов.

Интеграционные тесты проверяют, что отдельные части системы согласованы друг с
другом. Такой тест выполняется в среде, максимально приближенный к
реальности. Это главное отличие юнит-тестов от интеграционных: в первом случае
окружение сужают, во втором расширяют.

Чтобы проиграть сценарий регистрации, понадобится запущенная система и смежные
ресурсы: веб-сервер, почтовый сервер, база данных. Интеграционный тест
подключается к браузеру по протоколу WebDriver и командует им. Открывает
страницу регистрации, вводит данные и нажимает кнопку. Проверяет, что появилось
сообщение с просьбой проверить почту. По протоколу SMTP скачивает последнее
письмо. Ищет ссылку активации регулярным выражением. Открывает ссылку в новой
вкладке. Переходит на страницу авторизации и вводит почту и пароль. Убеждается,
что попал в личный кабинет.

Интеграционные тесты сложнее юнит-аналогов. Их труднее организовать, потому что
требуется настроить компоненты друг на друга и привести к нужному
состоянию. Например, если выполнить тест повторно, пользователь уже будет в базе
данных, и сценарий оборвется.

Интеграционные тесты выполняются дольше. Открыть страницу, дождаться ее загрузки
и заполнить поля занимает больше времени, чем отправить запрос машинным
способом. Иногда управление браузером намеренно замедляют, чтобы согласовать
команды с содержимым. Например, не кликнуть на кнопку, которую браузер еще не
успел отрисовать.

Ошибки в интеграционных тестах требуют расследования. Юнит-тесты выводят отчет
об ошибках, где четко указано, какое выражение не равно чему. В интеграционных
тестах причина ошибки чаще всего не ясна. Например, после отправки формы мы
ожидаем сообщение с просьбой проверить почту. Но оно не появилось. Была ли это
ошибка на сервере? И пользователь уже существует? Или в форме новое обязательное
поле, которое мы не заполнили? Или сообщение изменилось так, что не ищется
селектором? Ответ требует вмешательства человека.

Чтобы проверить, как проект ведет себя под нагрузкой, прибегают к нагрузочному
тестированию. Приложению посылают запросы и фиксируют множество показателей:
время ответа, общее число запросов, сколько из них параллельны, потребление
процессора и памяти. По этим данным строят отчеты и графики.

Нагрузочное тестирование находит медленные части в проекте. С помощью него можно
узнать, насколько готова новая функциональность к промышленному
запуску. Например, если боевой сервер обслуживает в среднем сто запросов в
секунду, а новая страница выдает только двадцать, ее отправят на доработку.

К другим категориям тестов относятся Smoke-, Sanity-, регрессионные и
другие. Полный их список состоит из десяти и более пунктов. Мы не ставим цель
рассмотреть их все. В этой главе мы глубоко изучим юнит- и интеграционные
тесты. На этих двух категориях основаны остальные. Читателю хватит знаний, чтобы
разобраться с ними самостоятельно.

\section{Тесты в Clojure}

Переходим к практике. В этом разделе мы познакомимся со стандартным тестовым
фреймворком Clojure. Мы напишем настоящие тесты~--- такие, которые отвечают
требованиям фреймворка. Будем тестировать функции Фаренгейта и квадратного
уравнения из примеров выше.

Предположим, что функция ->fahr находится в модуле src/book/util.clj. Создайте
файл \spverb|test/book/util_test.clj| с содержимым:

\begin{verbatim}
(ns book.util-test
  (:require [book.util :refer [->fahr]]
            [clojure.test :refer [deftest testing is]]))

(deftest test-fahr
  (is (= (int (->fahr 20)) 68))
  (is (= (int (->fahr 100)) 212)))
\end{verbatim}

Получился тестовый модуль. Он импортирует ->fahr и объявляет для нее один тест
test-fahr. Макросы is и deftest заимствованы из пакета clojure.test. Это
тестовый фреймворк из поставки Clojure.

Шесть строк выше несут много новой информации. У читателя появятся резонные
вопросы. Почему тесты лежат в папке test, а не src? Разве не логично хранить
тесты рядом с тем, что они проверяют? Как система найдет их? Почему тест
объявляют особым макросом deftest? Автор утверждал, что тест это функция, или
это не так? Зачем было менять assert на is? Ответим на эти вопросы по порядку.

В Clojure разделяют код приложения и тестов. Код приложения находится в папке
src/, а тесты в test/. Такой подход несет преимущества. Во-первых, делит код на
смысловые части. Чтобы найти неисправный тест, мы сужаем область поиска до папки
test/, а не ищем по всему проекту. Во-вторых, когда тесты отделены, это снижает
риск того, что они станут частью скомпилированного результата. В этом нет
смысла, потому что тесты запускают только на этапе разработки. Мы только
увеличим время сборки и объем конечного файла.

Во время тестов проект запускается с особым параметром. Он определяет, где еще
кроме src/ искать исходные файлы. Технически это список, в котором по умолчанию
один путь src/. В режиме тестов система добавляет к нему путь test/. Когда
компилятор ищет файл, он по очереди просматривает пути из списка. С точки зрения
программиста пути как будто сливаются друг с другом. Далее в главе мы
рассмотрим, как настроить пути под нужды проекта.

Мы упоминали, что фреймворку необходим какой-то признак, чтобы отличить тест от
обычной функции. Макрос deftest объявляет функцию и сообщает ее метаданными
особое поле :test. Чтобы выбрать все тесты, фреймворк сканирует пространства
имен и читает метаданные каждой переменной. Если среди полей есть :test,
переменная считается тестом.

Заметим, что deftest не оставляет возможности задать аргументы теста. Это
сделано нарочно: в Clojure тест не зависит от внешних данных. Если тесту нужны
особые условия, их задают фикстурами. Мы рассмотрим в фикстуры в отдельном
разделе.

Макрос is отличается от assert побочными эффектами. Assert это чистое сравнение:
мы получим либо nil, либо исключение. Макрос is изменяет глобальные
переменные. Он записывает данные о том, сколько раз и с какими параметрами его
вызвали. Из этих данных фреймворк строит отчет. Например, покажет все выражения
как они записаны в коде, которые не прошли проверку.

Пока что мы только написали тест, но не запустили его. Clojure предлагает
несколько способов сделать это.

\subsection{Командная строка}

Самый простой способ~--- поручить прогон утилите для сопровождения проекта. Если
проект основан на Leiningen, достаточно вызвать в терминале lein
test. Потребуется некоторое время, чтобы загрузить код из папок src/ и test/,
найти тесты и выполнить их. Вы увидите следующий отчет:

\begin{verbatim}
lein test book.util-test

Ran 1 tests containing 2 assertions.
0 failures, 0 errors.
\end{verbatim}

Clojure группирует тесты по пространствам имен. Под каждым пространством сводная
информация о том, сколько выполнено тестов (объявлений deftest) и проверок
(вызовов is). В нашем случае это один тест test-fahr с двумя проверками is.

Выражени lein test book.util-test напечатано для того, чтобы его можно было
скопировать и запустить в терминале. Тогда сработают тесты только из
пространства book.util-test.

Фреймворк различает т.н. failures и errors. Failure (анг. неудача)~--- это ошибка
в проверке. Например, мы утверждаем, что (= (int (->fahr 20)) 68). Если формула
изменится и значения станут не верны, счетчик failures увеличится на
единицу. Каждое неудачное утверждение помнит о том, с какими параметрами его
запустили. Если была хотя бы одна неудача, мы увидим отчет об ошибке.

Откройте определение ->fahr и измените коэффициент 1.8 на 1.9. Сохраните файл и
запустите в терминале lein test:

\begin{verbatim}
$ lein test

lein test :only book.util-test/test-fahr

FAIL in (test-fahr) (util_test.clj:6)
expected: (= (int (->fahr 20)) 68)
  actual: (not (= 70 68))

lein test :only book.util-test/test-fahr

FAIL in (test-fahr) (util_test.clj:7)
expected: (= (int (->fahr 100)) 212)
  actual: (not (= 222 212))

Ran 1 tests containing 2 assertions.
2 failures, 0 errors.
Tests failed.
\end{verbatim}

Отчет показывает 2 failures, что верно: в теле test-fahr макрос is встречается
два раза. Обе проверки неверны, потому что опираются на ->fahr. Для каждой из
них отчет показывает исходное выражение (= (int (->fahr 100)) 212) и конечные
значения (= 222 212).

Выражение

\begin{verbatim}
lein test :only book.util-test/test-fahr
\end{verbatim}

, которое отчет выводит над каждой проверкой, запустит только указанный
тест. Это полезно, когда вы работаете над определенным участком кода запускаете
один и тот же тест много раз.

Errors или ошибки это непойманные исключения, которые случились во время работы
теста. Чаще всего ошибки возникают, когда в функции передают некорректные
данные. Например, мы решили проверить поведение ->fahr с nil. Добавим второй
тест:

\begin{verbatim}
(deftest test-fahr-nil
  (is (nil? (->fahr nil))))
\end{verbatim}

Отчет будет таким:

\begin{verbatim}
$ lein test

lein test :only book.util-test/test-fahr-nil

ERROR in (test-fahr-nil) (Numbers.java:3849)
expected: (nil? (->fahr nil))
  actual: java.lang.NullPointerException: null
 at clojure.lang.Numbers.multiply (Numbers.java:3849)
    book.util$__GT_fahr.invokeStatic (util.clj:5)
    book.util$__GT_fahr.invoke (util.clj:4)
    book.util_test$fn__370.invokeStatic (util_test.clj:10)
    book.util_test/fn (util_test.clj:9)
    ... <truncated>

Ran 2 tests containing 3 assertions.
0 failures, 1 errors.
Tests failed.
\end{verbatim}

Вызов (->fahr nil) приводит к выбросу NullPointerException. Фреймворк
оборачивает тест в try/catch и запоминает исключение. Для ошибок отчет выводит
их стек-трейс. В примере выше мы сократили его для экономии места.

Непойманные исключения говорят о том, что тесты спроектированы неудачно. Скорее
всего, тест вышел из-под контроля и делает что-то не то. Например, вы не
проверили, что результат функции положительный перед тем, как передать его в
другую функцию.

Иногда мы ожидаем, что функция действительно бросит исключение при определенных
параметрах. Фреймворк готов к такому сценарию. Макрос is, записанный в особой
форме, выражает это как обычное утверждение. Мы рассмотрим пример с исключениями
в следующем разделе.

\subsection{REPL}

У тестов в консоли один недостаток: это долго. Код на Clojure, когда загружен в
память, выполняется быстро, но его загрузка занимает от трех до десяти
секунд. Время зависит от подключенных библиотек, размера кодовой базы и мощности
компьютера. Даже если вы запускаете один тест, lein стартует новую копию
проекта, что неэффективно.

Когда вы работаете с проектом, скорее всего, вы подключены к его сеансу
REPL. Будет удобно выполнить тест прямо в REPL по команде, не запуская новый
проект. Тогда издержки на загрузку кода станут равны нулю.

Наивный способ выполнить тест~--- запустить его как функцию. Выражение (test-fahr)
вернет nil, что означает, что все утверждения сработали верно. Теперь исправим
тест: пусть одно из утверждений заведомо ошибочно:

\begin{verbatim}
(is (= (int (->fahr 20)) 999))
\end{verbatim}

Даже если сохранить файл и вызвать (test-fahr), по-прежнему получим результат
без ошибок. Это происходит потому, что мы только сохранили файл, но не загрузили
новый теста в память JVM. То, как изменения в файле перетекают в память
Lisp-машины, зависит от вашего окружения.

Если это Emacs/CIDER, выполните одно из действий. Переместите курсор за
последнюю закрывающую скобку формы (deftest test-fahr ...) и нажмите C-c
C-e. Сработает команда cider-eval-last-sexp, которая выполняет последнее
S-выражение. Другой способ в том, чтобы выполнить весь буфер. Независимо от
того, где сейчас курсор, нажмите C-c C-k (или M-x cider-eval-buffer <RET>). Эта
команда равносильна тому, чтобы скопировать буфер и вставить в сеанс REPL.

Повторный вызов (test-fahr) вернет nil, но в консоли появятся строки:

\begin{verbatim}
FAIL in (test-fahr) (util_test.clj:14)
expected: 68
  actual: (999)
\end{verbatim}

Если вызвать тест (test-fahr-nil), который мы все еще не исправили, увидим
данные об исключении:

\begin{verbatim}
ERROR in (test-fahr-nil) (Numbers.java:3849)
expected: (nil? (->fahr nil))
  actual: java.lang.NullPointerException: null
 at clojure.lang.Numbers.multiply (Numbers.java:3849)
    book.util$__GT_fahr.invokeStatic (form-init3606582116051051694.clj:5)
    book.util$__GT_fahr.invoke (form-init3606582116051051694.clj:4)
    ...
\end{verbatim}

Заметим, это только вывод в консоль, а не брошенное исключение. Тест устроен
так, что его тело обернуто в try/catch. Поэтому дополнительно оборачивать вызов
теста не нужно.

Когда тест запускают вручную как функцию, к нему не применяются фикстуры. Это
пре- и пост-обработчики, которые подготавливают среду для теста. О фикстурах мы
поговорим в дальнейших разделах главы. Пока что отметим, что функция test-vars
выполняет тесты с учетом фикстур. Она принимает вектор объектов Var:

\begin{verbatim}
(require '[clojure.test :refer [test-vars]])
(test-vars [#'test-fahr #'test-fahr-nil])
\end{verbatim}

Функция run-test запускает все тесты пространства (текущего или
указанного). Технически она сводится к вызову test-vars со всеми переменными из
этого пространства:

\begin{verbatim}
(require '[clojure.test :refer [run-tests]])
(run-tests)
\end{verbatim}

Еще одна функция run-all-tests из выполняет тесты для всех загруженных
пространств:

\begin{verbatim}
(require '[clojure.test :refer [run-all-tests]])
(run-all-tests)
\end{verbatim}

Запустив это выражение, вы увидете, как тестовый фреймворк перебирает все
загруженные модули:

\begin{verbatim}
Testing nrepl.middleware.interruptible-eval
Testing cider.nrepl.middleware.util.nrepl
Testing clojure.test
Testing clojure.core.server
Testing clojure.core.specs.alpha
Testing book.util
...
\end{verbatim}

Чтобы отбросить лишние пространства, в run-all-tests передают регулярное
выражение. С этим выражением сверяют имена пространств. Например, чтобы
выполнить только наши тесты, укажем, что имя модуля начинается с book:

\begin{verbatim}
(run-all-tests #"$book")
\end{verbatim}

Emacs/CIDER

Модуль CIDER для редактора Emacs упрощает работу с тестами. CIDER предлагает
команды и комбинации клавиш, чтобы выполнить один или несколько тестов. Он же
показывает отчет об ошибках в отдельном буфере. Вызвать тест клавишей гораздо
быстрее, чем печатать команду в REPL.

Чтобы запустить один тест, переместите на него курсор и нажмите C-c C-t t (или
M-x cider-test-run-test). Курсор может быть в любом месте (deftest ...). Если
тест прошел без ошибок, вы увидите в области сообщений зеленый текст. Если
случились ошибки, откроется специальный буфер *cider-test-report*.

Этот буфер выводит отчет в более удобной форме. Ошибки в утверждениях подсвечены
красным цветом, а непойманные исключения желтым. Буфер интерактивен: если
подвести курсор к блоку с ошибкой и нажать Enter, соседний буфер покажет
исходный код теста. То же самое работает для исключений: чтобы не засорять
отчет, CIDER показывает только класс и сообщение. По Enter открывается отдельный
буфер с полным стек-трейсом.

Перечислим другие полезные команды:

- \spverb|cider-test-rerun-failed-tests (C-c C-t f)| повторно выполняет неудачный тест
  из прошлого запуска. Это удобно, если в первый раз вы запустили блок тестов, и
  ошибка случилась где-то на середине.

- \spverb|cider-test-run-ns-tests (C-c C-t n)| выполняет тесты для определенного
  пространства. С помощью этой команды можно запустить тесты без перехода в их
  модуль. Чтобы сопоставить пространство и тесты, CIDER добавляет к имени
  "-test". Например, для book.util целевой модуль получится \spverb|book.util-test|.

Запуск тестов это рутинная операция; избегайте ее ручного вызова. Чтобы не
терять время, изучите, как это сделать напрямую из редактора или IDE.

\section{Полезные практики}

В этом разделе мы рассмотрим несколько практик по написанию тестов. Поскольку
тесты это тоже код, логично, что они нуждается в поддержке. Небрежные тесты
причиняют столько же проблем, сколько и основное приложение. Следующие приемы
сделают тесты удобней в чтении, навигации и поддержке.

\subsection{Testing}

Макрос \spverb|testing| оборачивает произвольное тело строкой. Это сообщение, которое
говорит о том, что мы собираемся делать:

\begin{verbatim}
(deftest test-square-roots
  (testing "Two roots"
    (let [result (square-roots 1 -5 6)]
      (is (= (mapv int result) [3 2]))))
  (testing "One root"
    (is (= (square-roots 1 6 9) -3)))
  (testing "No roots"
    (is (nil? (square-roots 2 4 7)))))
\end{verbatim}

Сообщение важно по двум причинам: визуальной и технической. Оно разбивает тест
на логические блоки, заменяет комментарии, делает тест понятнее. Если
утверждение не сработало, в отчете будет указано сообщение, в которое оно
обернуто.

\begin{verbatim}
(is (= (square-roots 1 6 10) -3))

Fail in test-square-roots
One root
expected: nil
  actual: -3
\end{verbatim}

Макросы testing могут быть вложенными. Тогда каждое следующее сообщение
добавляется к предыдущему через пробел. Тест ниже проверяет, что API createUser
действительно заносит пользователя его в базу данных. Дополнительно мы
убеждаемся, что пароль в базе зашифрован. Если отбросить основной код, получим
скелет теста из форм testing:

\begin{verbatim}
(deftest test-some-api
  (testing "API call"
    ...
    (testing "HTTP response"
      ...))
  (testing "DB checks"
    ...
    (testing "common user fields"
      ...)
    (testing "password is hashed"
      ...)))
\end{verbatim}

\subsection{Исключения}

Мы все еще не решили вопрос о том, как проверять код с исключениями. Хотелось бы
убедиться, что вызов (->fahr nil) порождает ошибку, а не возвращает nil. Если
теста на исключение нет, код не защищен от таких изменений:

\begin{verbatim}
(defn ->fahr [cel]
  (when cel
    (+ (* cel 1.8) 32)))
\end{verbatim}

В этом случае nil провалится ниже по стеку вызовов, что усложнит поиск причины.

Пример ниже с try/catch решает проблему лишь отчасти. В таком виде (is true)
действительно сработает. Если же заменить (->fahr nil) на (->fahr 1), то
исключения не произойдет, и ветка (catch...) будет отброшена вместе с
содержимым. Это значит, что исключение, которое не выбросили там, где его
ожидают,~--- тоже ошибка.

\begin{verbatim}
(deftest test-fahr-nil
  (try
    (->fahr nil)
    (catch NullPointerException e
      (is true))))
\end{verbatim}

Макрос \spverb|(is (thrown?...))| решает эту проблему. Форма thrown? принимает класс
исключения и произвольное тело. Если в момент исполнения было брошено исключение
этого класса (с учетом наследования), внешний (is...) сработает
положительно. Если исключения не было, получим ошибку в утверждении:

\begin{verbatim}
(deftest test-fahr-nil
  (is (thrown? NullPointerException
               (->fahr nil))))
\end{verbatim}

Исправьте nil на любое число и убедитесь, что тест не проходит.

Иногда одной проверки на класс недостаточно. Когда тестируют большой участок
кода, NullPointerException возникает на разных уровнях программы. Например, если
кто-то изменит ->fahr так, что она возвращает nil, ошибка придет из другой
функции, которая принимает результат ->fahr. Это приводит к ложному
тестированию: тест проходит, но на самом деле не фиксирует поведение программы.

Проблему решают двумя шагами. Первый~--- исправить функцию так, чтобы она бросала
что-то более осмысленное, чем NPE. Например, специальное исключение
IllegalArgumentException. Его дополняют сообщением о том, что пошло не так:

\begin{verbatim}
(defn ->fahr [cel]
  (if cel
    (+ (* cel 1.8) 32)
    (throw (new IllegalArgumentException
            "Fahrenheit temperature should be a real number"))))
\end{verbatim}

Второй шаг~--- убедиться, что исключение пришло именно из ->fahr. Форма (is
(thrown-with-msg?...)) проверяет, что текст исключения совпадает с регулярным
выражением. Тест ниже покрывает все перечисленные требования:

\begin{verbatim}
(deftest test-fahr-nil
  (is (thrown-with-msg?
       IllegalArgumentException #"Fahrenheit temperature"
       (->fahr nil))))
\end{verbatim}

\subsection{Пакетная проверка}

Вспомним, как выглядит test-fahr:

\begin{verbatim}
(deftest test-fahr
  (is (= (int (->fahr 20)) 68))
  (is (= (int (->fahr 100)) 212)))
\end{verbatim}

Второе утверждение это копия первого и отличается только числами. Чтобы добавить
новую проверку, придется снова копировать одну из форм (is...). Подход с
копированием зашумляет код и в целом выходит боком. Очевидно, что из похожих
(is...) можно выделить постоянную и переменную части и переписать тест в
короткой форме.

Макрос are (анг. множественная форма глагола is) выполняет несколько is по
шаблону. Он принимает форму связывания, шаблон выражения и произвольные
аргументы. Число аргументов должно быть кратно переменным из формы
связывания. На каждом шаге макрос связывает часть аргументов с переменными и
выполняет шаблон в рамках is:

\begin{verbatim}
(deftest test-fahr
  (are [f c] (= (int (->fahr f)) c)
    20 68
    100 212))
\end{verbatim}

Аргументы записывают столбиком по принципу одна строка~--- один is. В примере выше
f и c означают цифры по Фаренгейту и Цельсию, то есть исходное и ожидаемое
значения. Выражение (int (->fahr f)) переехало в шаблон, поэтому нет смысла
повторять его каждый раз.

Макрос are удобен для небольших выражений. С ростом логики он становится трудным
в поддержке. В этом случае тест переписывают на doseq, который пробегает по
данным.

Предположим, мы тестируем HTTP API для создания пользователя. Один из тестов
проверяет входные параметры. Мы ожидаем негативный ответ для случаев, когда нет
обязательного поля или оно в неверном формате. В переменную params-ok запишем
удачные параметры:

\begin{verbatim}
(def params-ok {:name "John Smith" :email "john@test.com"})
\end{verbatim}

Чтобы проверить негативные случаи, объявим вектор params-variations. Каждый его
элемент это пара: словарь и текст. Словарь это измененные параметры, которые мы
добавим к params-ok функцией merge. Текст это сообщение с описанием проблемы,
например, неверный адрес почты, слишком длинное имя и тд.

\begin{verbatim}
(def params-variations
  [[{:name nil} "Empty name"]
   [{:name (apply str (repeat 999 "A"))} "Name is too long"]
   [{:email "dunno"} "Wrong email"]
   [{:email nil} "No email"]
   [{:extra 42} "Extra field"]])
\end{verbatim}

Тест ниже пробегает по params-variations. На каждом шаге он вызывает
api-create-user с испорченными параметрами. Обратите внимание, что вызов обернут
в форму testing с сообщением. Если на одном из шагов утверждение не сработает,
мы узнаем причину по сообщению из отчета.

\begin{verbatim}
(deftest test-api-create-user-bad-params
  (testing "Sending bad parameters"
    (doseq [[params* description] params-variations]
      (testing description
        (let [params (merge params-ok params*)
              response (api-create-user params)
              {:keys [status]} response]
          (is (= 400 status)))))))
\end{verbatim}

Когда тест разделяет данные и проверку, его легко поддерживать. Если в API
появятся новые поля, достаточно расширить params-variations, не изменяя тест.

\subsection{Именование}

Приучите себя к правилу: имя теста всегда начинается с префикса test-. Мы
упоминали, что с точки зрения Clojure это необязательно. Тестовый фремворк
находит тесты по метаданным, а не имени. Поэтому выражения (defn
test-if-suspended [user]) и (deftest if-suspended) останутся функцией и тестом.

Префикс необходим для навигации по коду. В модуле может быть гораздо больше
тестов, чем влезает на экран. Вперемешку с тестами идут вспомогательные
переменные и функции. С ростом модуля станет трудно понять, где функция, а где
тест. С префиксом такой ошибки быть не может.

Emacs предлагает навигацию по коду командой imenu и ее улучшенной версией
helm-imenu. При вызове команды открывается буфер со списком определений. С
вводом текста в буфере останутся только те элементы, имена которых содержат
текст. Называйте определения так, чтобы по команде M-x helm-imenu <RET> test в
списке остались только тесты.

В общем случае имя теста устроено по правилу \spverb|test-<what>-<case>|, где \spverb|<what>|
означает что мы тестируем, а \spverb|<case>|~--- сценарий. Например,
\spverb|test-create-user-ok| означает, что это положительный сценарий для создания
пользователя. Имя \spverb|test-create-user-bad-params| говорит о попытке создать
пользователя с неправильными параметрами. От теста
\spverb|test-user-login-signature-expired| мы ожидаем, что пользователь не смог
авторизоваться, потому что подпись устарела.

\section{Фикстуры}

До сих пор мы рассматривали простые тесты. Они запускаются в любой момент,
потому что не зависят от сторонних ресурсов. Выдержать это условие удается не
всегда. Чем сложнее код, тем больше у него требований к базе данных, файлам и
сети. Все вместе это называется окружением теста.

В этом разделе мы рассмотрим, как сообщить тесту его окружение. Для этого служат
фикстуры. Термин пришел из промышленности и означает стенд, на котором собирают
и испытывают изделие. В программировании фикстура это объект или функция,
которая работает до или после теста. Фикстура готовит окружение и очищает его.

Перечислим несколько случаев, когда фикстуры полезны.

\emph{Файлы на диске.} Код, который мы тестируем, ожидает, что по заданному пути
находится файл с данными. Фикстура копирует файл в нужную папку и по завершению
теста удаляет его. Такой подход гарантирует, что не оставит следов на диске
после работы.

\emph{Данные в базе}~--- частный случай фикстуры по подготовке данных. Например,
чтобы проверить авторизацию, необходимо заранее записать в базу пользователя с
известным паролем. Возможно, понадобятся и другие данные о пользователе: его
профиль или история заказов. Фикстура, которая готовит данные в базе, очищает
таблицы перед записью, чтобы следующий тест не зависел от действий предыдущего.

\emph{Запуск и остановка системы.} Иногда фикстуры управляют глобальным состоянием
программы. Например, для интеграционного теста необходимы все компоненты системы
(веб-сервер, база, почта и тд). Логику запуска и остановки системы логично
вынести в фикстуру.

По ходу главы мы рассмотрим эти и другие сценарии, когда прибегают к
фикстурам. Прямо сейчас разберемся, как они устроены технически.

С точки зрения фреймворка фикстура это функция, которая принимает тест. Функция
выполняет любые действия, в том числе вызывает тест. Вопрос в какой момент это
сделать остается на усмотрение разработчика. Например, если фикстура готовит
данные в базе, логично сперва очистить таблицы и выполнить несколько запросов:

\begin{verbatim}
(def db {:dbtype "postgres" :dbname "book"
         ;; other JDBC fields go here...
         })

(defn fix-db-data [t]
  ;; purge
  (jdbc/execute! db "truncate users cascade;")
  (jdbc/execute! db "truncate orders cascade;")
  ;; write
  (jdbc/insert! db :users user-data)
  (jdbc/insert! db :profile profile-data)
  ;; execute
  (t))
\end{verbatim}

В этом примере мы сначала готовим окружение, а затем вызываем тест. Но бывает и
наоборот: тест нуждается в финальных штрихах. Предположим, код из теста пишет
промежуточные файлы на диск. Чтобы очистить папку при выходе из теста, напишем
фикстуру:

\begin{verbatim}
(import 'java.io.File
        'org.apache.commons.io.FileUtils)

(defn fix-clear-files [t]
  (t)
  (FileUtils/cleanDirectory (new File "/tmp/tests/reports")))
\end{verbatim}

Иногда фикстура сообщает тесту и предварительные, и финальные шаги. В этом
случае вызов (t) располагается где-то посередине.

Надеемся, читатель заметил, что имена фикстур начинаются с префикса fix-. В
отношении фикстур действует то же правило, что и для тестов. Это особые
сущности, поэтому их отделяют префиксом от обычных функций.

\subsection{Регистрация}

Если запустить тест прямо сейчас, фикстура не сработает. Мы только объявили
фикстуру, но не указали фреймворку взять ее в работу. Это делают отдельным шагом
-- регистрацией фикстуры. На первый взгляд это выглядит странно. Если фреймворк
сам находит тесты, почему бы ему не искать фикстуры и применять их?

Причина в том, что фикстура может быть запущена на разных этапах
теста. Фреймворк Clojure выделяет стадии :each и :once. Зарегистрировать
фикстуру означает связать ее с одним из этих ключей. Значение :each означает,
что фикстура сработает для каждого теста из текущего пространства. С ключом
:once фикстура выполняется только однажды в рамках пространства.

Одна и та же фикстура может быть :each или :once в зависимости от семантики
теста. Вспомним фикстуру fix-clear-files, которая очищает директорию. Если
каждый тест порождает случайные имена файлов, логично очистить их единоразово в
конце. Это значит, что фикстуру регистрируют с ключом :once. В случае, если
имена файлов одинаковы, возрастает риск их коллизии (чтение файла из другого
теста и тд). Тогда фикстуру связывают с ключом :each.

Другой пример~--- фикстура базы данных. На время теста она добавляет в таблицы
данные и очищает их. Если точно известно, что тесты только читают данные,
фикстуру регистрируют с :once. Это дает выигрыш в производительности: для
тридцати тестов фикстура сработает единожды. Для тестов на запись важно, чтобы
ни один из них не влиял на другой. Поэтому фикстуру регистрируют с :each. Чтобы
не было коллизий, тесты на чтение и запись разносят по разным модулям.

Функция use-fixtures из пакета clojure.test принимает ключ и переменное число
фикстур:

\begin{verbatim}
(use-fixtures :once fix-db-server fix-clear-files)
(use-fixtures :each fix-db-data)
\end{verbatim}

Эти выражения обычно ставят в верхней части файла до тестов. Технически они
могут располагаться и в конце. Но чем раньше программист узнает о том, какие
тесты участвуют в тестах, тем лучше.

\subsection{Пример}

Чтобы лучше понять, в какой момент срабатывает фикстура, добавим в нее побочный
эффект. Функция fix-factory принимает тип фикстуры, ее номер и возвращает
функцию-фикстуру. Такая фикстура обрамляет тест выводом в консоль.

\begin{verbatim}
(defn fix-factory [type number]
  (fn [t]
    (println (format "%s %s starts" type number))
    (t)
    (println (format "%s %s ends" type number))))
\end{verbatim}

Добавим по две фикстуры каждого типа:

\begin{verbatim}
(use-fixtures :once
  (fix-factory :once 1)
  (fix-factory :once 2))

(use-fixtures :each
  (fix-factory :each 3)
  (fix-factory :each 4))
\end{verbatim}

Если запустить тесты, увидим следующий текст. Фикстуры 1 и 2 типа :once
сработали однажды на границах. Обратите внимание, что в момент завершения из
порядок меняется на противоположный: сначала завершается фикстура 2 и только
потом 1. В нашем модуле 4 теста, поэтому фикстуры 3 и 4 встречаются столько же
раз. Чтобы отобразить процесс наглядней, мы добавили отступы в каждое
сообщение. С ними становится ясна иерархия фикстур.

\begin{verbatim}
:once 1 starts
 :once 2 starts
  :each 3 starts   +
   :each 4 starts  | test1
   :each 4 ends    |
  :each 3 ends     +
  :each 3 starts   +
   :each 4 starts  | test2
   :each 4 ends    |
  :each 3 ends     +
  :each 3 starts   +
   :each 4 starts  | test3
   :each 4 ends    |
  :each 3 ends     +
  :each 3 starts   +
   :each 4 starts  | test4
   :each 4 ends    |
  :each 3 ends     +
 :once 2 ends
:once 1 ends
\end{verbatim}

\subsection{На нижнем уровне}

Пытливому читателю будет интересно узнать, где лежит информация о
фикстурах. Очевидно, вызов use-fixtures меняет глобальное состояние. Это не атом
и не переменная, а нечто другое, с чем мы еще не работали~--- метаданные текущего
пространства. Проверим его ключи:

\begin{verbatim}
(keys (meta *ns*))
(:clojure.test/each-fixtures :clojure.test/once-fixtures)
\end{verbatim}

Под каждым ключом находится список функций, который мы передали в use-fixtures с
:each или :once. Вызов use-fixtures заменяет в метаданных все фикстуры этого
типа. Важно, что это полная замена, а не запись в конец списка. Например, чтобы
отказаться от всех фикстур типа :each без перезагрузки REPL, запустите
выражение:

\begin{verbatim}
(use-fixtures :each nil)
\end{verbatim}

Другой способ~--- удалите эту строку и выполните весь буфер целиком. Тогда
определение \spverb|(ns...)| сработает еще раз с новыми метаданными.

Другие языки и фреймворки выделяют еще одну стадию :session. Это когда фикстура
запускается в разрезе всего прогона тестов. Можно сказать, что это
фикстура-синглтон: система гарантирует, что даже для нескольких модулей она
сработает один раз. В Clojure нет такой стадии. Если бы она была, это бы
значило, что несколько пространств связаны друг с другом, что идеологически
неверно. Начинающим на Clojure часто не хватает session-фикстур, но мы научимся
обходиться без них.

\subsection{Связанные переменные}

Особенно выразительны фикстуры в связке с динамическими переменными. Мы
рассматривали их в главе про изменяемость. Напомним, динамические переменные это
формы def с тегом \spverb|^:dynamic| и \spverb|*ушками*| по краям имени. Макрос
binding выполняет тело в рамках нового значения этой переменной.

Форма deftest порождает функцию без аргументов; мы не можем сообщить тесту его
параметры как обычной функции. А ведь тесту нужны имена файлов, сессия БД и
многое другое. Чтобы пробросить эти данные, применяют binding-фикстуры. Это
фикстура, которая выполняет тест в теле binding. При этом глобальная переменная
связана с актуальным значением. Сослаться на переменную можно только из теста.

Предположим, мы тестируем функцию, которая принимает путь к файлу и пишет в него
PNG-картинку (карту, график). Объявим фикстуру, которая на время теста свяжет
переменную *file* с временным файлом. С точки зрения теста это *file* будет
экземпляром java.util.File. При выходе из теста он будет удален, а переменная
восстановится в nil.

\begin{verbatim}
(defonce ^:dynamic *file* nil)

(defn with-fix-tmp-file [t]
  (let [^File tmp-file (TmpFile/createFile "....")]
    (binding [*file* tmp-file]
      (t))
    (.delete tmp-file)))

(use-fixture :each with-fix-tmp-file)
\end{verbatim}

Тест мог бы выглядеть следующим образом. Мы вызываем plot-chart с текущим
временным файлом. Остается проверить, что функция действительно записала в него
PNG-изображение. Класс PngImage предлагает простейший парсер PNG. Мы читаем
разрешение файла, его размер и проверяем их.

\begin{verbatim}
(import 'some.path.PngImage)

(deftest test-plot-chart-png
  (let [dataset [[...] [...] [...]]
        filepath (.getAbsPath *file*)]
    (plot-chart dataset filepath)
    (let [png (new PngImage *file*)
          width (.getWidth png)
          height (.getHeight png)])
    (is (= [640 480] [width height]))))
\end{verbatim}

Для полноты картины рассмотрим случай с базой данных. Пусть это будет Cassandra,
а не привычный JDBC-драйвер. Фикстура with-fix-db связывает *db* с текущей
сессией. Поскольку установка соединения это дорогая операция, вынесем ее в
разовую фикстуру:

\begin{verbatim}
(defonce ^:dynamic ^Session *db* nil)

(defn with-fix-db [t]
  (let [cluster ...
        session ...]
    (binding [*db* session]
      (t))
    (.close session)
    (.close cluster)))

(use-fixture :once with-fix-db)
\end{verbatim}

Другая фикстура восстанавливает данные для каждого теста. Фикстуры :once
запускаются раньше, чем :each. Поэтому при подготовке данных мы уже будем внутри
\spverb|(binding [*db* ...])| из тела with-fix-db. Это значит, фикстура
fix-db-prepare-data свободно обращается к \spverb|*db*| как к сессии:

\begin{verbatim}
(defn fix-db-prepare-data [t]
  (alia/execute! *db* "truncate project.users;")
  (alia/execute! *db* "truncate project.orders;")
  (alia/execute! *db* "insert into project.users (name, email) values (%s, %s)" user-data)
  (t))

(use-fixture :each fix-db-prepare-data)
\end{verbatim}

\section{Мульти-фикстуры}

Любопытный вопрос: что произойдет, фикстура вызывает тест несколько раз?
Например, объявим такую фикстуру:

\begin{verbatim}
(defn fix-multi [t]
  (t) (t) (t))

(use-fixtures :each fix-multi)
\end{verbatim}

и запустим один тест. С точки зрения фреймворка прошло три теста, о чем написано
в отчете:

\begin{verbatim}
book.util-test: Ran 6 assertions, in 3 test functions. 0 failures, 0 errors.
\end{verbatim}

На первый взгляд в этом нет никакого смысла. Многократный прогон теста не несет
пользы, а только потребляет ресурсы и время. (Если же число запусков теста на
что-то влияет, это плохой тест.) Но в паре с динамическими переменными такая
фикстура дает полезный эффект. Один и тот же тест можно запустить в разных
окружениях.

Предположим, наша программа работает с базой данных. В системных требованиях
указано, что это может быть PostgreSQL или MySQL. Мы должны убедиться, что
программа действительно поддерживает обе СУБД. Чтобы не дублировать тесты под
каждый тип базы, эту логику выносят в фикстуру.

Объявим JDBC-подключения к базам данных:

\begin{verbatim}
(def db-pg {:dbtype "postgresql" :host "..."})
(def db-mysql {:dbtype "mysql" :host "..."})
\end{verbatim}

Добавим динамическую переменную \spverb|*db*|. Новая фикстура перебирает список
подключений и связывает их с \spverb|*db*| на каждом шаге. Когда переменная связана,
запускается тест. Он зависит от текущего подключения и таким образом обращается
либо в PostgreSQL, либо в MySQL.

\begin{verbatim}
(defonce ^:dynamic *db* nil)

(defn fix-multi-db-backend [t]
  (doseq [db [db-pg db-mysql]]
    (binding [*db* db]
      (testing (format "Testing with DB: %s" (:dbtype *db*))
        (t)))))
\end{verbatim}

Обратите внимание, что вызов (t) дополнительно обернут в форму testing. В ней мы
указываем тип базы. Если произошла ошибка, нам важно знать, с какой базой мы
работали в тот момент. Пример теста на чтение пользователя:

\begin{verbatim}
(defn test-get-user-by-id
  (let [user (com.system.orm/get-user *db* 1)]
    (is (= user {:name "Ivan"}))))
\end{verbatim}

Перечислим другие сценарии для мульти-фикстур. Это может быть запуск
интеграционных тестов в нескольких браузерах (Chrome, Firefox). Или мы хотим
убедиться, что логика не зависит от формата передачи данных (JSON, YAML,
XML). Если программа работает с изображениями, тесты прогоняют на разных типах
файлов (PNG, jpeg) и разрешениях.

\subsection{Минутка неадекватности}

(Просим не воспринимать этот раздел слишком серьезно.) Перейдем в другую
крайность: что случится, если фикстура не вызовет тест ни разу? Например:

\begin{verbatim}
(defn fix-mute [t])
(use-fixtures :each fix-mute)
\end{verbatim}

Это приведет к тому, что ни один тест не будет выполнен. Объясним это на
техническом уровне. Такая фикстура не отменяет тесты, и они по-прежнему видны
тестовому фреймворку. В отчете вы увидите их список, но для каждого будет
указано 0 assertions. Формально мы запустили тест, но его тело так и не
сработало из-за фикстуры fix-mute. Она заглушает любой тест в текущем
пространстве.

Если включить fix-mute и запустить тест из Emacs, мы увидим предупреждение:

\begin{verbatim}
No assertions (or no tests) were run.Did you forget to use ‘is’ in your tests?
\end{verbatim}

Модуль Cider считает подозрительным, что в тесте не было ни одного is. Это
справедливо: без is тест может подать сигнал об ошибке только исключением, что
ведет к ряду проблем. Исключение и неверный результат это разные
сигналы. Например, функция может давать верный результат, а деление на ноль
происходит только в особых случаях. Если заменить деление на умножение, мы
больше никогда не получим ArithmeticException, но результат будет далек от
истины.

Чтобы обмануть Cider, улучшим фикстуру. Вместо запуска теста добавим в нее is,
который всегда истинен. Даже если тест с ошибками, его тело не сработает, но в
отчете будет как минимум 1 assertion. Назовем такую фикстуру fix-this-is-fine:

\begin{verbatim}
(defn fix-this-is-fine [t]
  (is true))

(use-fixtures :each fix-this-is-fine)
\end{verbatim}

С ее помощью мы пройдем тест Оруэлла:

\begin{verbatim}
(deftest test-1984
  (is (= (* 2 2) 5)))
;; OK, ran 1 assertions
\end{verbatim}

Задание читателю: напишите фикстуру, которая портит любой тест. Добейтесь, чтобы
100% тестов завершились с ошибкой. Если ваша система версий поддерживает
отложенные коммиты, запланируйте такую фикстуру на 1 апреля.

Теперь вернемся к серьезному повествованию.

\subsection{Фикстуры с условиями}

Из примеров выше ясно, что фикстура не только вызывает тест, но и несет
дополнительную логику. Обращение к тесту может быть обернуто в цикл, условие или
другую форму. Разберемся с условием. В каких случаях тест ставят под if или when
и какой в этом смысл?

К условной фикстуре прибегают, когда мы не знаем заранее, сможем ли выполнить
этот тест. Например, часть программы не работает на платформе Windows или
Mac. Тогда фикстура проверяет тип операционной системы и не вызывает тест, если
он заведомо не работает.

\begin{verbatim}
(defn fix-mac-only [t]
  (when (= (System/getProperty "os.name") "Mac OS X")
    (t)))
\end{verbatim}

Эта фикстура выполнит тест только на платформе Mac семейства X. Недостаток в
том, что мы сравниваем конкретные значения в лоб. Это наивный подход. В системах
Windows или Linux имена различаются даже в рамках одного семейства. Чтобы
определить платформу правильно, свойство "os.name" проверяют на вхождение строки
или регулярным выражением.

Доработаем фикстуру: во-первых, сделаем ее подробнее. Обернем тест в форму
testing, чтобы видеть в отчете, в рамках какой ОС был выполнен тест. Если мы не
поддерживаем систему, добавим заведомо ложное утверждение:

\begin{verbatim}
(defn fix-mac-only [t]
  (let [os (System/getProperty "os.name")]
    (if (= os "Mac OS X")
      (testing (format "OS: %s" os)
        (t))
      (testing (format "Unsupported OS: %s" os)
        (is false)))))
\end{verbatim}

Каким способом лучше зарегистрировать фикстуру~--- :once или :each? Это зависит от
характера условия. В нашем примере тип системы не меняется от теста к тесту,
поэтому условие может быть выполнено лишь однажды (:once). Если же фикстура
проверяет файл на диске, это условие выполняют для каждого теста.

Рассмотрим другие сценарии, когда фикстуры несут условия. Например, это версия
базы данных: если она ниже определенного значения, мы не можем выполнить
тест. Некоторые приложения зависят от системных утилит, которые вызывают в
отдельных процессах. Фикстура определяет, установлена ли такая утилита или
нет. Иногда тест работает с большими файлами. Фикстура проверяет, что на диске
достаточно места до того, как скачать файл с сетевого ресурса.

\section{Метки и селекторы}

Фикстуры с условиями похожи на систему фильтров. С их помощью запускают не все
тесты, а подмножество, которое подходит текущему окружению. Однако, избегайте
ситуаций, когда фикстуры перегружены условиями сверх меры. Clojure предлагает
особые средства для выборочного прогона. Это метки и селекторы тестов.

Метки устроены проще, чем фикстуры с условием. Это метаданные, которые вы
сообщаете тесту при его описании. Например, тесту ниже

\begin{verbatim}
(deftest ^:special test-special-case
  (is true))
\end{verbatim}

мы сообщили флаг \spverb|:special| со значением \spverb|true|. Напомним, что
выражение \spverb|^:field| это сокращенная форма \spverb|^{:field true}|.
Сокращенная форма полезна, чтобы задать несколько флагов за один раз:

\begin{verbatim}
(deftest ^:special ^:backend ^:no-db
  test-special-case
  (is true))
\end{verbatim}

Когда тегов два и более, имя теста переносят на следующую строку, чтобы
визуально отделить их.

По такому набору тегов легко догадаться, что это тест на серверную логику
(backend); он не зависит от базы данных (no-db); это какой-то особый тест
(special). Семантика тегов зависит от соглашений в команде. Проверим флаги:
прочитаем метаданные переменной 'test-special-case. Среди прочих полей вы
найдете special и остальные:

\begin{verbatim}
(meta #'test-special-case)
{:special true :backend true :no-db true ...}
\end{verbatim}

Логично, что если тестам назначены теги, можно выбрать подмножество
тестов. Например, только особые (special) или на серверную логику
(backend). Такая выборка называется селектором тестов. Селекторы полезны по
нескольким причинам.

Подмножество тестов отработает быстрее, чем полный набор. Если мы работаем над
задачей и точно знаем, какие тесты она затрагивает, то нет смысла запускать все
доступные тесты. Назначим тег и будем запускать только подмножество. Иногда
окружение разработчика не настроено должным образом, и выполнить тесты целиком
невозможно.

Даже если окружение в порядке, отдельные тесты могут занять много
времени. Например, интеграционный тест длится на порядок дольше модульного и
вдобавок требует инфраструктуру: очередь задач, сторонние API. Поэтому
интеграционные тесты выносят в отдельный шаг Continuous Integration-процесса. В
общих чертах он выглядит так:

- запускается минимальное окружение (база данных);
- выполняются модульные тесты (селектор :default);
- если не было ошибок, то запускается дополнительное окружение (Kafka, заглушки
  API);
- выполняются интеграционные тесты (селектор :integration).

Чем сложнее проект, тем больше в нем тестов и дополнительных шагов по их
запуску.

В тестовом фреймворке Clojure нет селекторов. Он запускает либо один тест, либо
пространство, либо все тесты целиком. Селекторы доступны в сторонних библиотеках
и утилитах. Рассмотрим, как задать их в проекте на базе Leiningen.

Откройте файл project.clj. Внутри defproject добавьте ключ :test-selectors. Это
словарь, где ключ это метка селектора, а значение~--- функция одного
аргумента. Если при запуске тестов сообщить метку, каждый тест проходит проверку
этой функцией. В нее передают метаданные теста. Если функция вернула ложь или
nil, тест не попадает в набор.

Поскольку кейворд в Clojure это функция, селектором может быть сам тег. Пример
ниже можно прочитать как "набор тестов :special, у которых тег :special":

\begin{verbatim}
:test-selectors {:special :special
                 :backend :backend}
\end{verbatim}

Чтобы запустить тесты по селектору, передайте его метку в команду test:

\begin{verbatim}
lein test :special
\end{verbatim}

Если метка не задана, фреймворк назначит ей значение :default. Этому селектору
удовлетворяет любой тест (функция всегда вернет true). Ниже пример из реального
проекта. Для :default мы задали селектор, который вернет все не интеграционные
тесты. Этим мы добьемся того, что команда lein test запустит только модульные
тесты. На случай, когда нужно прогнать весь проект, мы завели метку
:all. Функция identity вернет исходный словарь метаданных, который вырождается в
true.

\begin{verbatim}
:test-selectors {:default (complement :integration)
                 :integration :integration
                 :all identity}
\end{verbatim}

Селектор может быть и полноценной функцией, а не только ключом. Предположим,
программа должна работать с разными версиями БД, чтобы обеспечить
совместимость. Для каждого теста укажем минимальную версию базы. Это поле
:pg/version и числовое значение. Тогда можно задать селектор так, чтобы выбрать
тесты по версиям. В примере ниже тест, который проверяет экспериментальные
разработки на новой версии базы.

\begin{verbatim}
(deftest ^{:pg/version 11}
  test-db-experimental-feature
  (is true))
\end{verbatim}

С точки зрения селектора тест экспериментальный, если поле :pg/version больше
или равно 11:

\begin{verbatim}
:test-selectors
{:db-experimental
 (fn [test-meta]
   (some-> test-meta :pg/version (>= 11)))}
\end{verbatim}

Запустите их командой \spverb|lein test :db-experimental|.

Когда в модуле много тестов, расставлять теги утомительно. Вдобавок повышается
риск забыть тег, и тест выпадет из набора. Если все тесты модуля связаны по
смыслу, тег назначают не тесту, а пространству. С точки зрения lein тест
наследует теги пространства, в котором он объявлен:

\begin{verbatim}
(ns ^:integration
  book.integration-test
  (:require [clojure.test :refer :all]))

(deftest test-user-login-ok
  ...)
\end{verbatim}

В примере выше нет смысла указывать \spverb|^:integration| каждому тесту,
поскольку это сделано на уровне модуля. Селектор интеграционных тестов не
изменится: его метка и функция равны ключу \spverb|:integration|. Тесты
запускаются как обычно:

\begin{verbatim}
lein test :integration
\end{verbatim}

\subsection{Тесты в deps.edn}

Не все Clojure-проекты используют lein; некоторые перешли на deps.edn. Это
утилита, разработанная в фирме Cognitect. Основная задача утилиты~--- управлять
зависимостями проекта. С версии 1.9 deps.edn идет в поставке с Clojure.

Если проект работает на deps, мы теряем часть функциональности lein (в отдельных
случаях их удается скрестить, но об этом в другой раз). В том числе это прогон
тестов, команда lein test. В deps замещают возможности lein сторонними
библиотеками. Проект test-runner это библиотека, которая делает то же самое:
находит тесты, запускает их и выводит отчет.

Чтобы включить библиотеку в проект, добавьте в файл deps.edn новую сущность:

\begin{verbatim}
:aliases
{:test
 {:extra-paths ["test"]
  :extra-deps {com.cognitect/test-runner
               {:git/url "https://github.com/cognitect-labs/test-runner.git"
                :sha "209b64504cb3bd3b99ecfec7937b358a879f55c1"}}
  :main-opts ["-m" "cognitect.test-runner"]}}
\end{verbatim}

Объясним этот словарь. Ключ :aliases это примерно то же самое, что профили в
lein: дополнительные опции, которые включаются, если указан этот профиль. Ключ
:extra-paths это список дополнительных путей для поиска файлов с кодом. Поле
:extra-deps указывает зависимости, с которыми запустится проект с этим
профилем. Поскольку test-runner нужен только для тестов, нет смысла вносить его
в глобальные зависимости. В примере выше библиотека скачивается прямиком с
GitHub; поле :sha означает номер коммита. Это коренное отличие deps от lein:
последний скачивает зависимости только из репозиториев (maven, clojars). Ключ
:main-opts задает входную точку программы. Это функция -main из пространства
cognitect.test-runner.

Библиотека предлагает несколько ключей для выборки тестов с логикой "только с
этим тегом" (-i, include, включить ) или "без этого тега" (-e, exclude,
исключить). Команда ниже выполнит все тесты кроме интеграционных:

\begin{verbatim}
clj -Atest -e :integration
\end{verbatim}

В плане селекторов test-runner устроен проще lein. Он учитывает только
логические теги и не работает со сложным функциями, как мы делали это в lein на
примере версий БД. С другой стороны, test-runner умеет искать пространства по
регулярному выражению (ключ \spverb|-r|, namespace-regex). Например, выражение
\spverb|#"project\.api\.\w+-test$"| найдет тесты из модулей \spverb|project.api.users-test|,
\spverb|project.api.tasks-test| и так далее.

\section{Проблема окружения}

До сих пор мы писали примитивные тесты, которые проверяют вычисления. На
практике вы столкнетесь с проблемой: в обычном коде преобладают не вычисления, а
ввод-вывод данных. Это обращение к сторонним ресурсам: базе данных, очереди
задач, HTTP API. Такой код трудно тестировать по двум причинам.

Во-первых, не всегда возможно настроить локальное окружение. У вас может не быть
ключей доступа или сертификатов. Во-вторых, тесты плохо работают с данными,
которые не контролируют. Например, если тест посылает запросы к стороннему
сервису, то в какой-то момент данные изменятся, и мы получим сломанный билд на
ровном месте.

Чтобы тест зависел от окружения, применяют две техники: моки и стабы. В
следующих разделах мы рассмотрим их устройство, преимущества и недостатки.

\subsection{Моки}

Мок (анг. mock~--- ложный, фиктивный)~--- это объект, который временно заменяет
другой объект. Цель такой подмены в том, чтобы при обращении к объекту сработал
не исходный код, а тот, который нужен именно сейчас. Чаще всего моки накладывают
на методы и функции с доступом в сеть. С их помощью проверяют, как поведет себя
функция при различных сценариях: данные получены, пришел код ошибки, соединение
не удалось.

В ООП-языках моки заменяют не только объект, но и отдельные его поля, методы или
даже класс. В Clojure основной строительный элемент это функция, поэтому ниже мы
рассмотрим, как заменять функции на время тестирования.

Представим, что пишем веб-сервер для мобильного приложения. Его тематика~---
развлечение и отдых. На главном экране пользователь видит карту, на которую
нанесены кафе, рестораны и ближайшие события: фильмы, выставки,
фестивали. Сервер работает по протоколу HTTP/JSON API. Данные для главного
экрана возвращает функция view-main-page. Она извлекает из запроса координаты
пользователя, собирает сведения о местах общепита и событиях и отсылает
мобильному приложению:

\begin{verbatim}
(ns book.views
  (:require [clj-http.client :as client]))

(defn view-main-page [request]
  (let [location (-> request :params (select-keys [:lat :lon]))
        sites (get-sites-by-location location)
        events (get-events-by-location location)]
    {:status 200
     :body {:sites sites :events events}}))
\end{verbatim}

Проблема в том, что у сервера нет этих данных. Мы извлекаем из сторонних
сервисов, например, условных Яндекс.Карт и Яндекс.Афиши. Функции get-sites... и
get-events... общаются с ними по протоколу HTTP. Вот так может выглядеть функция
get-sites-by-location для поиска кафе и ресторанов в радиусе 300 метров:

\begin{verbatim}
(defn get-sites-by-location
  [{:keys [lat lon]}]
  (-> {:method :get
       :url "https://maps.yandex.ru/search/v1/"
       :as :json
       :query-params {:apikey "....."
                      :lat lat :lon lon :distance 300
                      :type "cafe,restaurant"}}
      client/request
      :body))
\end{verbatim}

Функция get-events-by-location для поиска мероприятий аналогична первой. Разница
лишь в URL-адресе (не maps.yandex.ru, a events.yandex.ru) и query-параметрах.

Очевидно, view-main-page совершает два сетевых вызова, что затрудняет ее
тестирование. Это значит, нам нужно два API-ключа; если запускать тесты часто,
мы исчерпаем квоты на число запросов. Поскольку мы получим настоящие, живые
данные, мы не можем быть уверены в их надежности (завтра в этом месте откроется
новое кафе или закроется старое). Решим эту проблему моками.

В главе про изменяемость мы работали с макросом with-redefs. Он заменяет
переменную по ее пути на что-то другое. Простейший мок~--- это подмена функции с
сетевым вызовом на анонимную функцию, которая возвращает заведомо известный
результат. Напомним, что получить функцию из результата можно с помощью формы
constantly.

Напишем тест главного экрана с двумя моками. Данные моков удобно вынести в
отдельные переменные в let. Важно, что view-main-page должна быть выполнена
внутри with-redefs, иначе эффект моков теряется.

\begin{verbatim}
(deftest test-main-page
  (let [sites [{:name "Cafe1"} {:name "Cafe2"}]
        events [{:name "Event1"} {:name "Event2"}]]
    (with-redefs
      [book.views/get-sites-by-location (constantly sites)
       book.views/get-events-by-location (constantly events)]
      (let [request {:params {:lat 55.751244
                              :lon 37.618423}}
            result (view-main-page request)]
        (is (= (:body result)
               {:sites sites :events events}))))))
\end{verbatim}

С этим подходом мы избежали похода в сеть. Тест не зависит от API-ключей, прав
доступа и новых заведений. Можно запустить его в любое время дня и ночи, и
результат будет одинаковым. Теперь рассмотрим, как улучшить этот тест.

\subsection{Вынести данные в файл}

Переменные sites и events, которые мы якобы получили от сторонних сервисов,
крайне скудны. Будет правильным сообщить им данные, скопированные из реального
источника, например, десять ресторанов и семь мероприятий. Чтобы не засорять код
огромными массивами данных, поместите их в .json-файлы в папку с
ресурсами. Считайте их один раз на уровне модуля в переменную:

\begin{verbatim}
(def data-events
  (-> "data/events.json"
      clojure.java.io/resource
      slurp
      (cheshire.core/parse-string true)))
\end{verbatim}

В промышленном запуске функция get-sites-by-location устроена сложнее, чем в
нашем примере. Скорее всего, данные от сервиса карт придется обработать согласно
бизнес-логике нашего приложения. Например, отбросить заведения с низкой оценкой
и те, что сейчас закрыты. Это логику можно проверить тестом: добавить в
json-файл заведение с низким рейтингом и убедиться, что в его нет в ответе.

\subsection{Перенести мок в фикстуру}

Если требуется больше одного теста на главный экран, оборачивать каждый из них в
with-redefs будет утомительно. Логично вынести мок в фикстуру. Тем самым мы
снизим число повторов в коде, а другие тесты смогут ее переиспользовать:

\begin{verbatim}
(defn fix-mock-main-view [t]
  (with-redefs [...]
    (t)))

(use-fixtures :each fix-mock-main-view)
\end{verbatim}

Недостаток фикстуры в том, что для всех тестов она устанавливает позитивный
сценарий (обе функции сработали без ошибок). Для полноты картины мы должны
проверить, что случится, если сервисы недоступны. Об этом~--- в следующем пункте.

\subsection{Негативные сценарии}

Недоступность одного сервиса не должна влиять на работу приложения в целом. Если
возникла накладка с кафе, пользователь все равно должен увидеть мероприятия и
наоборот. Возможно, он даже не заметит проблемы: если ответ не был получен,
приложение покажет старые данные.

Чтобы проверить главный экран на устойчивость, необходимы негативные сценарии:

- получили кафе, но не мероприятия;
- наоборот: с кафе все в порядке, ошибка с мероприятиями;
- не работают оба сервиса.

Кроме того, под общими словами "не работают" имеют в виду варианты:

- проблемы доступа или квот: статусы 403 и 429;
- недоступность сервиса: статусы 500 и 503;
- проблемы связи: исключение по таймауту и поиску хоста.

Чтобы сократить код с with-redefs, напишем его краткую версию. Макрос with-mock
принимает путь к функции, результат функции-мока и произвольное тело.

\begin{verbatim}
(defmacro with-mock
  [path result & body]
  `(with-redefs
     [~path (fn [& ~'_] ~result)]
     ~@body))
\end{verbatim}

Важный момент: для замены path мы создаем анонимную функцию с помощью fn, а не
constantly. Иначе невозможно смоделировать выброс исключения. Поскольку
constantly это функция, ее аргументы вычисляются до того, как она
сработает. Поэтому выражение

\begin{verbatim}
(constantly (throw (new Exception "error")))
\end{verbatim}

бросит исключение еще на этапе связывания.

С помощью with-mock мы собираем комбинации успехов и неудач. Тест ниже проверяет
сценарий, когда сервис карт сработал без ошибок, а с мероприятиями что-то пошло
не так. Если запустить тест, окажется, что test-main-page не отлавливает
исключения. Это значит, пользователь не увидит ничего.

\begin{verbatim}
(deftest test-sites-ok-events-err
  (with-mock book.views/get-sites-by-location [...]
    (with-mock book.views/get-events-by-location
      (throw (new java.net.UnknownHostException "DNS error"))
      (let [response (view-main-page {})]
        (is (:body response) {...})))))
\end{verbatim}

Доработайте view-main-page так, чтобы тест увидел ответ со статусом 200 и полем
:sites. Добавьте тесты с другими комбинациями: ошибка соединения в get-sites,
статусы 403 и 500, недоступность сразу двух сервисов.

\subsection{Сбор данных}

Не заглядывая в конец главы, подумайте, как организовать мок промышленного
уровня. Такой мок не только возвращает результат, но и:

- считает, сколько раз его вызвали;
- запоминает аргументы каждого вызова;
- принимает список результатов и возвращает их поочередно, например, в первый и
  второй раз словари, а на третий раз исключение;
- предлагает удобный доступ к этим данным.

Моки с такими возможностями следят за особой логикой программы. Например, поиск
ресторанов вернет только базовые сведения о них. Чтобы получить детальную
информацию о заведении, придется слать отдельный запрос к сервису карт. Чтобы
сетевой трафик не рос линейно, мы извлекаем данные только для трех лучших
ресторанов. Продвинутый мок для условной функции get-site-details проверяет, что
ее вызвали не более трех раз. В конце главы мы рассмотрим продвинутые решения
для мокинга в Clojure.

\subsection{Недостатки}

На первый взгляд кажется, что моки решили проблему окружения. Теперь тесты
автономны и не зависят от сторонних сервисов. Но есть у моков и недостатки;
перечислим основные из них.

Мок нарушает принцип закрытости (черного ящика). При написании теста мы
подглядываем в исходный код, чтобы узнать, какие функции подменить. Это не
критично, но чем меньше тест знает о коде, который тестирует, тем он надежней.

Мок исключает сетевое взаимодействие, что иногда выходит боком. Передача данных
по HTTP это многоступенчатый процесс, где на каждом участке может возникнуть
ошибка. Подмена HTTP-запроса на чистую функцию это грубое приближение. С его
помощью нельзя предугадать все возможные ошибки.

Наконец, мок повышает связанность тестов с логикой. Если переименовать функцию
get-sites-by-location, форма (with-redefs...) перестанет работать. Косметические
исправления в коде повлекут правки в тестах, что неудобно и неверно
идеологически.

Эти проблемы решают стабы, о которых речь в следующем разделе.

\subsection{Стабы}

Стаб (анг. stub, заглушка)~--- это сущность, которая заменяет объект или часть
системы на время тестов. От моков она отличается тем, что тест не может ей
управлять. Если в тестах с моками мы сами указываем что и чем заменить, то стаб
это черный ящик с точки зрения теста.

Удачный пример стаба это локальный HTTP-сервер, имитирующий сторонний сервис,
например, поиск мест на карте. Такой сервер запущен на порту 8808 и по адресу
\spverb|/search/v1/| возвращает JSON-данные, которые мы взяли из настоящего
источника. Теперь мы не будем мокать функцию get-sites-by-location. Изменим ее
так, чтобы поле :url запроса указывало не на \spverb|https://maps.yandex.ru/|, а на
\spverb|http://127.0.0.1:8808/|. Если запустить тест, функция выполнит запрос к
локальному серверу и прочитает его ответ.

Подход со стабами несет колоссальное преимущество. Легко увидеть, что стаб
сводит к нулю зависимость теста от логики. Теперь тест ничего не знает о
функциях приложения; ему и не положено это знать. От нас требуется только
перенацелить запросы на другой адрес. Это вопрос конфигурации, а в логике
программы ничего не меняется.

Другое достоинство в том, что во время теста происходит настоящий обмен данными
по сети. Да, это локальный хост без зашифрованного соединения и проверки
сертификатов. Но основная часть HTTP-взаимодействия протекает так же, как и в
промышленном запуске. Это разбор заголовков, чтение и парсинг JSON-данных,
логика на базе статуса ответа.

Локальные сервера-заглушки еще называют фейки (анг. fake, подделка). С помощью
фейка программист моделирует нештатное поведение сервера. Например, ситуацию,
когда в ответе не JSON-данные, а просто текст. Или когда сервер принимает
соединение, но не отвечает. Или полную недоступность сервера, когда мы выключаем
его перед тестом.

Достичь таких эффектов моком невозможно. С его помощью мы можем только кидать
исключения, которые, согласно документации, бросает настоящий HTTP-клиент. Но
тесты на исключениях это хрупки и далеки от реального положения дел.

С технической стороны стаб это фикстура. До запуска теста она включает сервер, а
после выключает его. Сервер это Jetty или другой адаптер для Ring-обработчика
запроса. Мы подробно рассмотрели Ring и Jetty в главе по веб-разработку. Напишем
фикстуру для фейкового сервера карт.

В блок require добавим зависимости:

\begin{verbatim}
[ring.middleware.json :refer [wrap-json-response]]
[ring.middleware.params :refer [wrap-params]]
[ring.middleware.keyword-params :refer [wrap-keyword-params]]
[ring.adapter.jetty :refer [run-jetty]]
\end{verbatim}

Напишем обработчик запроса. Это функция, которая принимает запрос. Для пути
search/v1 она возвращает JSON-данные, для всего остального ответ 404:

\begin{verbatim}
(defn sites-handler* [{:keys [uri]}]
  (case uri
    "/search/v1/"
    {:status 200
     :body [{:name "Cafe1" :address "..."}
            {:name "Cafe2" :address "..."}]}
    {:status 404
     :body "page not found"}))
\end{verbatim}

Обернем функцию в набор middleware. Во-первых, мы хотим, чтобы тело-коллекция
приводилось к JSON автоматически. Во-вторых, позже нам понадобятся параметры
строки:

\begin{verbatim}
(def sites-handler
  (-> sites-handler*
      wrap-keyword-params
      wrap-params
      wrap-json-response))
\end{verbatim}

Напишем и подключим фикстуру. Сервер не меняет состояние от теста к тесту (мы
только читаем данные), поэтому фикстура срабатывает один раз (ключ :once):

\begin{verbatim}
(defn fix-fake-sites-server [t]
  (let [opt {:port 8808 :join? false}
        server (run-jetty sites-handler opt)]
    (t)
    (.stop server)))

(use-fixtures :once fix-fake-sites-server)
\end{verbatim}

Исправим функцию get-sites-by-location так, чтобы клиент обращался к локальному
хосту. Очевидно, базовый URL сервиса приходит из конфигурации. Вы уже знаете,
как устроена конфигурация из предыдущих глав, поэтому весь код приводить мы не
будем. Для краткости представим, что глобальная переменная config это словарь
параметров. Тогда полный URL для сервиса карт формируется так:

\begin{verbatim}
:url (str (:maps-base-url config ) "/search/v1/")
\end{verbatim}

Напишем тест главного экрана. На этот раз нам не нужны моки, поэтому тест
сводится к вызову view-main-page. Прямо сейчас тест не пройдет, потому что мы
решили проблему только с поиском заведений. Поиск событий по-прежнему обращается
к чужому серверу. Чтобы убедиться, стаб работает нормально, на время
закомментируем вызов get-events-by-location, а поле :events сделаем nil. После
этого тест сработает без ошибок: в поле :sites ответа окажутся данные, что
возвращает стаб.

\begin{verbatim}
(deftest test-main-page
  (let [request {:params {:lat 55.751244
                          :lon 37.618423}}
        result (view-main-page request)]
    (is (= (:body result) {:sites [...] :events nil}))))
\end{verbatim}

Теперь верните все как было и напишите похожую заглушку для поиска
событий. Оформите ее в фикстуру и убедитесь, что тест проходит. Это задание для
самостоятельной работы. Далее мы рассмотрим, что еще интересного предлагают
стабы.

Как и в случае с моками, удобно, когда данные лежат в файлах. Чтобы фейковый
сервер читал файл, измените ответ следующим образом:

\begin{verbatim}
{:status 200
 :body (-> "data/events.json"
           clojure.java.io/resource
           clojure.java.io/file)}
\end{verbatim}

В файл resources/data/events.json запишите ответ реального сервера.

На текущий момент стаб всегда возвращает успешный ответ. Расширим его логику:
сделаем так, чтобы при особых условиях мы получали негативные ответы. Проще
всего это сделать в зависимости от входных параметров. Договоримся, что
некоторые координаты, которые передает пользователь, особые. Например, для пары
0,0 сервер вернет пустой результат; при 66,66 получим ошибку доступа. С
координатами 42,42 сервер отвечает с задержкой в 35 секунд, что больше
стандартного ожидания в 30 секунд. Усложним \spverb|sites-handler*|:

\begin{verbatim}
(defn sites-handler* [request]
  (let [{:keys [uri params]} request
        {:keys [lat lon]} params]
    (case uri
      "/search/v1/"
      (case [lat lon]
        ["0" "0"]   {:status 200 :body []}
        ["66" "66"] {:status 403 :body {:error "ACCESS_ERROR"}}
        ["42" "42"] (do (Thread/sleep (* 1000 35))
                        {:status 200 :body []})
        {:status 200
         :body [{:name "Cafe1" :address "..."}
                {:name "Cafe2" :address "..."}]})
      {:status 404 :body "page not found"})))
\end{verbatim}

Теперь напишите тесты для этих координат. Особенно интересен случай с долгим
ответом (пара 42). Мы должны убедиться, что если сервис карт не отвечает, мы
ждем какое-то разумное время, а не стандартные 30 секунд. Передайте в параметры
\spverb|(client/get)| поля \spverb|:socket-timeout| и \spverb|:connection-timeout| значения 5000, что
равно пяти секундам. Этого ожидания достаточно для промышленного
запуска. Замерьте время выполнения view-main-page. Добавьте в тест утверждение
is, что ожидание не превышает 5 секунд с небольшой погрешностью.

Можно добавить тест для экстремального случая: что произойдет, если сервер
физически недоступен. Чтобы смоделировать эту ситуацию, фикстура должна
предоставить доступ к переменной server. Проще всего это сделать через
динамическую переменную:

\begin{verbatim}
(defonce ^:dynamic *server* nil)

(defn fix-fake-sites-server [t]
  (let [opt {:port 8808 :join? false}]
    (binding [*server* (run-jetty sites-handler opt)]
      (t)
      (.stop *server*))))
\end{verbatim}

Напишем тест, в котором мы временно отключим сервер. Обратите внимание, что в
конце его нужно включить, иначе мы испортим все следующие тесты.

\begin{verbatim}
(deftest test-the-website-is-down
  (.stop *server*)
  (let [request {:params {:lat 1 :lon 2}}
        result (view-main-page request)]
    (is (= (:body result) {...})))
  (.start *server*))
\end{verbatim}

Сейчас тест не сработает из-за ошибки "Connection refused". Доработайте логику
так, чтобы пользователю ушел пустой результат.

\subsection{Ресурсы и пути}

Выше мы упомянули, что данные для моков или стабов хранят в файлах. Так мы не
загромождаем код объемными списками и словарями. Файлы в папке resources
называются ресурсами приложения и играют особую роль: при сборке jar-файла они
становятся его частью. Код, запущенный из jar получает доступ к той его части,
где располагаются ресурсы. С точки зрения системы это один файл, а изнутри JVM
нам доступны все те файлы, что были в resources на этапе сборки.

В одном из примеров мы поместили файл events.json в папку resources/data и
читали его как ресурс. Недостаток в том, что эти данные нужны только для
тестов. Если собрать проект, events.json станет частью выходного jar-файла. Это
напрасно увеличит его объем.

Чтобы отбросить лишние ресурсы при сборке, в проекте настраивают
:resource-paths. Это поле-список, в котором перечислены возможные пути поиска
ресурсов. По умолчанию оно равно вектору ["resources"]. При объединении профилей
вектора соединяются в один. Если другой профиль задал ресурсы иначе:

\begin{verbatim}
:resource-paths ["resources_test"]
\end{verbatim}

, то в итоге программа будет искать их в папках resources и \spverb|resources_test|. Во
время финальной сборки оставим только resources, чтобы не вбирать в jar ресурсы
для тестов.

Заметим, что путь \spverb|resources_test| выбран неудачно. Существует другой, более
удобный способ связать файловую систему и окружение. Это так называемая
env-директория, в которой мы говорили в главе про изменяемость. На ее первом
уровне находятся папки с именами профилей, а внутри src и resources с кодом и
ресурсами. Эти пути задают профилям в описании проекта.

Создайте нужные пути и переместите данные для теста:

\begin{verbatim}
mkdir -p env/test/resources/data
mv resources/data/events.json env/test/resources/data/
rm -rf resources/data
\end{verbatim}

В настройках проекта задайте профили и их пути к ресурсам:

\begin{verbatim}
:profiles {:test {:resource-paths ["env/test/resources"]}
           :dev  {:resource-paths ["env/test/resources"]}}
\end{verbatim}

На первый взгляд странно, что для :dev и :test мы указали один и тот же
путь. Это потому, что по умолчанию сеанс REPL запускается с профилем dev, но не
test. В режиме разработки у нас не будет доступа к ресурсам из env/test. Это
влечет неудобства: например, мы исправили код и тут же вызвали тест из REPL, но
он не проходит из-за неверных путей. Проблему решают двумя способами.

Первый~--- запускают REPL с ключом \spverb|with-profile +test|, что значить добавить к
профилю по умолчанию test. Знак плюса означает выполнить слияние; без него
стандартный профиль будет отброшен. Второй способ, который мы уже рассмотрели~---
продублировать пути из test в dev.

В обоих случаях выражение

\begin{verbatim}
(def data-events
  (-> "data/events.json"
      clojure.java.io/resource
      slurp
      (cheshire.core/parse-string true)))
\end{verbatim}

будет работать одинаково. Физически вы переместили файл с данными, но с точки
зрения JVM это один и тот же ресурс.

Особо любознательным предлагаем эксперимент. Скомпилируйте uberjar с разными
ресурсами: в первый раз как обычно, а во второй опцией :resource-paths:

\begin{verbatim}
:profiles {:uberjar {:resource-paths ["env/test/resources"]}
\end{verbatim}

Перед второй сборкой скопируйте файл target/uberjar/<project>.jar в другую
папку, чтобы не затереть его. Распакуйте оба архива командой jar:

\begin{verbatim}
jar xf <project>.jar
\end{verbatim}

Среди файлов из второго архива вы найдете data/events.json. Убедитесь, что его
нет в первом случае. Очевидно, если вы распространяете программу в виде
jar-архива, любой желающий может распаковать его. Поэтому в ресурсах не должно
быть приватных данных, например, конфигурации с паролями и ключами. Это касается
и данных для тестирования~--- включать их в jar будет ошибкой.

Пользуясь случаем, исследуйте другие файлы, которые вы получили из jar, их
структуру и содержимое.

\subsection{База данных}

При написании тестов вы столкнетесь с проблемой. Почти каждое действие требует,
чтобы в базе были определенные данные. Например, для создания пользователя нужна
учетка администратора. Чтобы добавить пользователя в группу, нужна эта
группа. Чтобы оформить заказ, нужны товары, история покупок для расчета скидки и
многое другое. Чем шире логика приложения, тем больше понадобится тестовых
данных.

Если каждый тест создает данные под себя, у вас возникнут проблемы с поддержкой
таких тестов. Очевидно, требуется единая точка входа, такая, чтобы каждый тест
имел доступ к одинаковым данным. В идеале они заданы декларативно в файле. Чтобы
добавить новые записи, мы должны исправить проект только в одном месте. Когда
тесты закончились, все, что мы записали в базу, бесследно исчезает. Недопустимо,
чтобы один тест зависел от изменений, выполненных в другом тесте.

В этом разделе мы рассмотрим, как работать с базой во время тестов. Начнем с
первого шага~--- подготовки новой базы. Под этим мы понимаем создание всех таблиц,
индексов и других сущностей без вставки данных.

Физически базу создают еще до того, как запустились тесты. Способ зависит от
того, как работает база: локально или в Докере. Если локально, достаточно
вызвать несколько утилит. В примере ниже createuser и createdb входят в поставку
PostgreSQL.

\begin{verbatim}
createuser book_test -S -W
createdb -O book_test book_test
\end{verbatim}

Первая команда создает нового пользователя БД с именем \spverb|book_test|. Ключ -S
означает, что это суперпользователь, т.е. обладатель высших
привилегий. Супердоступ необходим, чтобы подключать расширения: триграммный
поиск, прогрев индексов и другие. В промышленном запуске, наоборот, приложению
выделяют минимальные права, но для тестов это неважно. Вторая команда создает
одноименную пустую базу. Ключ -O (owner) задает владельца базы. Владелец имеет
полный доступ ко всем ее сущностям.

Если база работает в Докере, обратитесь к главе про системы. В одноименном
разделе мы рассмотрели, как задать параметры базы в .yaml-файле и переменными
среды.

Пустую базу наполняют таблицами. Если приложение поддерживает миграции,
например, с помощью библиотек migratus или ragtime, их запускают до тестов из
командной строки:

\begin{verbatim}
lein migratus migrate
lein test
\end{verbatim}

Шага с миграциями можно избежать, если скопировать их в папку Докер-образа
/docker-entrypoint-initdb.d. Специальный скрипт выполнит все sql-файлы из этой
папки на старте образа. Убедитесь, что скопировали только up-миграции, иначе их
down-версии сведут пользу на нет.

К этому моменту в базе уже созданы таблицы и индексы, но нет данных. Мы
переходим к основной задаче: перечислим способы их вставки и очистки во время
тестов.

\emph{Ручная вставка}

В простейшем случае данные для базы хранят в виде коллекции. Специальная функция
пробегает по списку словарей; каждый словарь это запись в базе. Функцию
оборачивают в фикстуру, которая запускает тест после вставки всех записей.

\begin{verbatim}
(def db-data
  [[:users {:name "Ivan" :email "ivan@test.com"}]
   [:users {:name "Juan" :email "Juan@test.com"}]
   [:groups {:name "Dog fans" :topics 6}]
   [:groups {:name "Cat fans" :topics 7}]])
\end{verbatim}

Здесь и ниже мы полагаем, что глобальная переменная \spverb|*db*| указывает на
JDBC-спеку. Мы не будем задаваться вопросом, откуда пришла эта
переменная. Читатель уже знаком с системами и конфигурацией и вправе задать
\spverb|*db*| так, как это удобно в проекте.

\begin{verbatim}
(defn load-data []
  (doseq [[table row] db-data]
    (jdbc/insert! *db* table row)))

(defn with-db-data [t]
  (load-data) (t))
\end{verbatim}

Заметим, что для каждой записи нужно знать ее таблицу. Это может быть элемент
пары или поле метаданных. Когда записей одного типа много, имя таблицы
становится избыточным. Логично сгруппировать записи по таблицам и вставлять их
не по одной, а разом, что быстрее.

\begin{verbatim}
(def db-data
  [[:users [{:name "Ivan" :email "ivan@test.com"}
            {:name "Juan" :email "Juan@test.com"}]]
   [:groups [{:name "Dog fans" :topics 6}
             {:name "Cat fans" :topics 7}]]])

(defn load-data []
  (doseq [[table rows] db-data]
    (jdbc/insert-multi! *db* table rows)))
\end{verbatim}

\emph{Вставка из CSV}

Иногда записей бывает так много, что хранить их в виде словарей
затруднительно. В этом случае поможет формат CSV. Это файл с текстом, где каждая
строка означает запись, а поля разделены запятой. В первой строке обычно
размещают заголовок~--- имена полей с тем же разделителем.

У CSV несколько преимуществ. Записи в нем это массивы, а не словари, поэтому они
не дублируют имена полей. В целом CSV компактней JSON и аналогов. С ним работают
табличные редакторы, например Excel и OpenOffice. Это значит, всегда можно
открыть CSV как лист ячеек, добавить или удалить столбец, пересчитать ячейки по
формуле и сохранить результат.

Наконец, отдельные базы данных читают и пишут CSV напрямую. Чтобы получить
данные из определенной таблицы с прода, выборку из нее пишут в CSV, а из него
восстанавливают в локальную базу. На больших объемах вставка из CSV работает
быстрее, чем обычный INSERT.

Предположим, нам скинули данные о пользователях с прода в CSV-файле. Вот
несколько его первых строк:

\begin{verbatim}
name,email
ivan,ivan@test.me
juan,juan@example.com
ioan,ioan@dunno.org
\end{verbatim}

Данные для тестов это ресурс, сохраним его адресу
env/test/resources/data/users.csv, как делали это раньше. Есть два способа
поместить данные из него в таблицу: запросом и кодом. Рассмотрим оба из них.

Вариант с запросом: составим SQL-выражение с командой COPY. Нужно указать
таблицу, путь к файлу с данными и его свойства. Под свойствами понимают тип
разделителя (это может быть не только запятая, но и точка с запятой или
табуляция), формат файла и признак заголовка. Последний флаг означает, что
первую строку следует пропустить.

\begin{verbatim}
(jdbc/execute! *db*
  "
  COPY users(name, email)
  FROM '/Users/ivan/work/book/env/test/resources/data/users.csv'
  DELIMITER ',' CSV HEADER
  ")
\end{verbatim}

Заметим, что запрос ожидает полный путь к файлу, что мешает командной
работе. Вероятность того, что коллеги держат проект в папке
\spverb|/Users/ivan/work/book| равна нулю. Доработайте запрос так, чтобы вместо
локального пути подставлялась текущая рабочая директория.

Выполним то же самое средствами Clojure. Усложним задачу тем, что в файле
users.csv очень много записей, и потому он сжат архиватором gzip. Это хорошая
практика: все большие файлы в репозитории должны быть сжаты. Чтобы читать такой
файл на лету (т.е. без выгрузки во временную папку), понадобятся импорты:

\begin{verbatim}
(:import java.io.FileInputStream
         java.util.zip.GZIPInputStream
         org.postgresql.copy.CopyManager)
\end{verbatim}

Класс CopyManager делает то же, что и команда COPY. Чтобы создать его экземпляр,
требуется соединение с базой (именно TCP-соединение, а не словарь
спеки). Получим его функцией jdbc/get-connection. Метод copyIn принимает
SQL-запрос и стрим с данными CSV. Этот стрим мы получим из ресурса
\spverb|data/users.csv.gz|, пропустив его через серию функций и классов. Последний
GZIPInputStream занимается тем, что по мере чтения приводит сжатые данные к их
нормальному виду. В конце работы мы закрываем соединение с базой.

\begin{verbatim}
(defn load-data-gz []
  (let [conn (jdbc/get-connection *db*)
        copy (CopyManager. conn)
        stream (-> "data/users.csv.gz"
                   clojure.java.io/resource
                   clojure.java.io/file
                   FileInputStream.
                   GZIPInputStream.)]
    (.copyIn copy "COPY users(name, email)
                   FROM STDIN (FORMAT CSV, HEADER true)"
             stream)
    (.close conn)))
\end{verbatim}

Нельзя загрузить CSV-файл сразу в несколько таблиц. Соблюдайте правило "один
файл~--- одна таблица". Если файлов несколько, перепишите функцию так, чтобы она
принимала путь к файлу и имя таблицы. Помните, что импорт в разные таблицы может
протекать параллельно с помощью футур и pmap.

\emph{Проблема ключей}

Редко бывает так, что таблицы существуют сами по себе. Чаще всего они логически
связаны: пользователь ссылается на профиль, заказ на пользователя и так далее. В
классических базах для этого служат первичный и внешний ключи. Первичный ключ
(PK, Primary key) это поле, которое однозначно определяет запись в
таблице. Внешний ключ (FK, Foreign key) это другое поле, которое ссылается на
один из PK из другой таблицы. Все вместе это называют отношением таблиц.

Как правило, первичные ключи это целые числа с автоматической нумерацией. Чтобы
добавить запись в таблицу, не нужно подбирать новый PK. Если ключ с
автонумерацией, база данных ведет для него уникальный счетчик. Для очередной
записи мы получим новый PK, увеличив счетчик на единицу. Имя счетчика строится
по шаблону \spverb|<table>_<pk>_seq|, например, \spverb|users_id_seq|.

При вставке связанных данных мы должны убедиться, что их PK и FK равны. Проще
всего задать им одинаковые значения, например, 3. Тогда в примере ниже
пользователь действительно ссылается на группу:

\begin{verbatim}
(jdbc/insert! *db* :groups {:id 3 :name "Clojure fans"})
(jdbc/insert! *db* :users {:group_id 3 :name "Ivan"})
\end{verbatim}

Проблема в том, что случайно взятый PK может вступить в конфликт со
счетчиком. Предположим, что в момент вставки groups ее счетчик был равен
нулю. Поскольку мы задали id явно, счетчик не увеличился. Но в одном из тестов
мы создали еще несколько групп, чтобы проверить какой-то сценарий (например,
максимальное число групп, в которых можно состоять без платной подписки). С
первыми двумя группами проблем не возникнет: для них база вычислит ключи 1 и 2
из счетчика. Для третьей группы получим ошибку:

\begin{verbatim}
ERROR:  duplicate key value violates unique constraint 'groups_pkey'
DETAIL:  Key (id)=(3) already exists
\end{verbatim}

Новый ключ 3 вступил в конфликт с записью, которую мы добавили раньше. Избежать
этого можно двумя способами: коррекцией счетчиков и особым загрузчиком
данных. Разница в том, кто контролирует счетчики: вы или база данных.

Под ручной коррекцией имеем в виду следующее. В тестовых данных вы указываете
первичные ключи для всех записей. Для удобства некоторые из них выносят в
переменную. Например, если на пользователя-админа ссылаются несколько сущностей,
логично записать это так:

\begin{verbatim}
(def id-user-admin 1)

(def db-data
  [[:users [{:id id-user-admin :name "Ivan"}]]
   [:profiles [{:id 1 :user_id id-user-admin :avatar "..."}]
    :posts [{:id 1 :user_id id-user-admin :title "New book"}
            {:id 2 :user_id id-user-admin :title "Some post"}]]])
\end{verbatim}

После вставки данных поправьте счетчики так, чтобы они перескочили значения,
которые мы использовали. Предположим, если в тестовый набор входит семь
пользователей, и мы назначили им ключи от 1 до 7. Тогда установка счетчика в
значение 100 решит проблему. Все дальнейшие записи в таблице users получат ключи
100, 101 и так далее.

Чтобы сбросить счетчик, выполните запрос ниже:

\begin{verbatim}
(jdbc/execute! *db* "ALTER SEQUENCE users_id_seq RESTART WITH 100")
\end{verbatim}

Если все первичные ключи называются id (что рекомендуется), достаточно знать
только имя таблицы. Их даже не обязательно перечислять: таблицы указаны в
db-data. Добавьте в конец функции load-data код:

\begin{verbatim}
(let [value 100
      tables (set (map first db-data))
      query "ALTER SEQUENCE %s_id_seq RESTART WITH %s"]
  (doseq [table tables]
    (jdbc/execute! *db* (format query (name table) value))))
\end{verbatim}

Он подхватит все таблицы, указанные в db-data и выставит счетчик в точное
значение. На больших данных, например, из CSV, цифру 100 придется увеличить на
порядок или два. Технически мы можем отличить исходные записи от добавленных в
процессе, если это потребуется (id < 100 и наоборот).

Теперь рассмотрим свой загрузчик. В этом случае мы полагаемся на ключи, которые
вернула база данных. Конкретные их значения не важны. Главное, чтобы у связанных
сущностей первичный и внешний ключи были одинаковы. Для этого пойдем на
хитрость.

Для начала удалим первичные ключи из тестовых данных. Ни в одно словаре не
должно быть поля :id с конкретным значением. Сущностям, на которые нужно
ссылаться, присвоим псевдонимы. Проще всего это сделать через метаданные (поле
:entity, :alias и т.д). В связных сущностях в поле-ссылке укажем псевдоним.

\begin{verbatim}
;; user
^{:entity :users/admin} {:name "Ivan" :email "ivan@test.com"}
;; profile
{:user_id :users/admin :avatar "/images/ivan.png"}
;; posts
{:user_id :users/admin :title "New book"}
{:user_id :users/admin :title "Some post"}
\end{verbatim}

Загрузчик перебирает записи и делает следующие проверки. Если у записи
псевдоним, мы связываем его с ключом, который вернула база. На старте загрузчик
объявляет атом с пустым словарем и наполняет его в процессе, например,
{:users/admin 42}. Ключ мы получим из ответа функции jdbc/insert!. В зависимости
от типа базы ответ будет разным:

\begin{verbatim}
(jdbc/insert! *db* :users {:name "Ivan"})
({:id 42 :name "Ivan}) ;; for PostgreSQL
({:generated_key 42})  ;; for MariaDB
\end{verbatim}

Перед вставкой загрузчик проверяет поля записи. Если это ключ, например,
\spverb|{:user_id :users/admin}|, мы считаем, что это ссылка. Заменяем ссылку на значение
из атома по ключу. Если в атоме нет такого ключа, кидаем исключение. Предлагаем
читателю написать такой загрузчик самостоятельно.

\emph{Удаление данных}

Мы достаточно много сказали про вставку данных и переходим к их
удалению. Очистка базы важна не меньше, чем ее наполнение. Она должна вызывать
минимум усилий и не оставлять после себя артефактов.

Наивный способ очистить базу~--- послать запрос \spverb|DELETE FROM <table>| для всех
таблиц, с которыми мы работаем. Проблема DELETE в том, что он учитывает
зависимость ключей. Например, если запись из другой таблицы ссылается на users,
удалить пользователей не получится. Придется вызвать DELETE в правильном
порядке: сначала для связной таблицы, затем users. Чем больше у вас таблиц, тем
сложнее будет заполнить порядок.

Такое поведение не относится к недостаткам: наоборот, без контроля за ссылками
база станет неконсистентной. Параметры контроля можно задать в объявлении
внешнего ключа (см. справку по foreign key). Но вообще, нас интересует как
очистить данные независимо от того, как объявлены ссылки.

Команда TRUNCATE (анг. "усечь, подрезать") нужна для быстрой очистки таблицы. В
отличии от DELETE она не вызывает триггеры удаления и не сканирует всю
таблицу. TRUNCATE принимает несколько несколько таблиц за раз. Еще одно ее
преимущество в каскадном режиме. Если передан флаг CASCADE, все таблицы, которые
так или иначе входят в граф связей, тоже очищаются. На практике выходит, что
каскадная очистка пары таблиц вызывает цепную реакцию по всей базе.

Напишем функцию очистки. Она посылает запрос, в котором через запятую
перечислены все таблицы из данных для тестов. Добавьте ее в фикстуру
with-db-data сразу после вызова (t).

\begin{verbatim}
(defn delete-data []
  (let [tables (set (map first db-data))
        tables-comma (clojure.string/join "," (map name tables))
        query (format "TRUNCATE %s CASCADE" tables-comma)]
    (jdbc/execute! *db* query)))
\end{verbatim}

\subsection{Транзакция с откатом}

Другой способ избавиться от изменений в базе~--- обернуть все действия с ней в
особую транзакцию. Такая транзакция завершается оператором не \spverb|COMMIT|, а
\spverb|ROLLBACK|, что значит откатить все команды. С точки зрения базы это выглядит
так:

\begin{verbatim}
BEGIN;
INSERT INTO users ...
INSERT INTO profiles ...
UPDATE users SET name=...
ROLLBACK;
\end{verbatim}

При выходе из транзакции мы не увидим последствий \spverb|INSERT|, \spverb|UPDATE| и других
команд, выполненных в ее рамках.

В пакет JDBC входит функция \spverb|db-set-rollback-only!|. Она принимает
транзакционное соединение и выставляет ему флаг rollback. Если флаг установлен,
JDBC завершает блок транзакции оператором ROLLBACK.

Вы уже знакомы с макросом \spverb|with-db-transaction|: внутри его тела действует
транзакционное соединение, которые получают из JDBC-спеки. Напишем свой макрос
with-db-rollback, который делает то же самое, но дополнительно устанавливает
флаг отката:

\begin{verbatim}
(defmacro with-db-rollback
  [[t-conn & bindings] & body]
  `(jdbc/with-db-transaction [~t-conn ~@bindings]
     (jdbc/db-set-rollback-only! ~t-conn)
     ~@body))
\end{verbatim}

Пример его работы:

\begin{verbatim}
(with-db-rollback [tx *db*]
  (println "Inserting the data...")
  (jdbc/insert! tx :users {:name "Ivan"})
  (let [...]
    (do-something)))
\end{verbatim}

Разработчик следит за тем, чтобы все действия~--- вставка данных, логика теста~---
протекали через \spverb|tx|, а не \spverb|*db*|. Иначе изменения в рамках обычного соединения
останутся в базе. Прямо сейчас загрузчик load-data данных ссылается на
глобальную переменную \spverb|*db*|. Чтобы сообщить ему транзакционное соединение,
придется либо передать параметр, либо связать \spverb|*db*| с tx формой binding.

Пример с параметром: все действия с базой принимают tx, который мы установили на
вершине теста.

\begin{verbatim}
(deftest test-user-with-rollback
  (with-db-rollback [tx *db*]
    (load-data tx)
    (let [user (get-user-by-name tx "Ivan")]
      (is (= "Ivan" (:name user))))))
\end{verbatim}

Вариант со связанной переменной. В этом случае мы считаем, что все функции
ссылаются на глобальную \spverb|*db*|. Внутри макроса она станет транзакционным
соединением с откатом.

\begin{verbatim}
(deftest test-user-with-rollback
  (with-db-rollback [tx *db*]
    (binding [*db* tx]
      (load-data)
      (let [user (get-user-by-name "Ivan")]
        (is (= "Ivan" (:name user)))))))
\end{verbatim}

Реализация зависит от того, как в проекте устроена работа с базой. Решение с
откатом хорошо подходит для mount и похожей архитектуры, когда база это
глобальная переменная. В случае с системой компонентов могут возникнуть
трудности с передачей tx от теста к логике и наоборот.

В свободное время подумайте, как оформить фикстуру с макросом
with-db-rollback. Будет ли она работать с системой компонентов? Что необходимо в
этом случае?

\section{Тестирование веб-приложений}

До сих пор мы тестировали отдельные функции, в основном связанные с
расчетами. Такие тесты, как говорят математики, необходимы, но недостаточны. Они
защищают проект от спонтанных изменений, но не обещают, что система
устойчива. Поднимемся на уровень выше и рассмотрим, как тестировать приложение
целиком.

В главе про веб-разработку мы пришли к важному выводу. Веб-приложение на каждом
уровне остается функций одного аргумента. Где-то внизу это обработчик конкретной
страницы, условный \spverb|(view-profile-page [request])|. Комбинация маршрутов и
middleware не меняет это соглашение. Конечный объект app, много раз обернутый в
middleware, по-прежнему принимает словарь запроса и возвращает словарь ответа.

Это определяет, как писать тесты для приложения. Типичный тест составляет запрос
и вызывает приложение как функцию. В ответе ищут HTTP-статус и проверяют на
успех (200, 201) или неудачу (404, 403). Если это JSON-ответ, его тело считывают
в коллекцию и сравнивают с образцом.

Вспомним приложение из первой главы. Отдельные страницы мы соединили в маршруты
с помощью Compojure. Получилось "голое" приложение. Так мы назвали его потому,
что оно многого не умеет. Например, читать параметры запроса, работать с JSON,
куками, сессиями и так далее. Приложение узнает все это из middleware, в которые
мы его оборачиваем.

\begin{verbatim}
(defroutes app-naked
  (GET "/"      request (page-index request))
  (GET "/hello" request (page-hello request))
  page-404)

(def app
  (-> app-naked
      wrap-session
      wrap-keyword-params
      wrap-params
      wrap-json-body
      wrap-json-response))
\end{verbatim}

Напишем несколько тестов для этого приложения. Пусть это будет главная страница
и любая другая, которой нет в дереве маршрутов. Для экономии места проверим
только статус ответа.

\begin{verbatim}
(deftest test-app-index
  (let [request {:request-method :get :uri "/"}
        response (app request)
        {:keys [status body]} response]
    (is (= 200 status))))

(deftest test-app-page-not-found
  (let [request {:request-method :get :uri "/missing"}
        response (app request)
        {:keys [status body]} response]
    (is (= 404 status))))
\end{verbatim}

Как видно из примеров, писать тесты для веб-приложения нетрудно. Однако, мы не
бросим читателя на этом месте. Перечислим несколько приемов, которые облегчат
работу с тестами.

**Приложение целиком.** Избегайте ситуации, когда тест вызывает не приложение, а
один из обработчиков. Плохой пример: (page-index request) вместо app. На текущем
уровне вызов конкретной функции ничего не дает. Даже если страница работает по
отдельности, нет гарантии, что запрос пройдет сквозь маршруты и все
middleware. В боевых проектах middleware несут весомую логику. Это права
доступа, разбор токенов и JWT, данные из прошлых запросов в сессии. Убрав все
это из теста, вы тем самым обманываете себя. Объект app, который вы тестируете,
должен быть максимально "заряжен", т.е. близок к настоящему веб-серверу.

**Библиотека запросов.** Выше мы объявили запрос в виде словаря. Это работает
только для простых случаев, когда нет ни параметров строки, ни тела. Если не
позаботиться об инструментах, вы дойдете до уровня

\begin{verbatim}
/users/?page=2&order=name&name_contains=smith&search_type=relevance
\end{verbatim}

, что совершенно нечитаемо и тяжело в поддержке. В словаре запроса легко
перепутать ключи :request-method и :method, что автор и сделал при написании
книги.

Чтобы избежать подобных ошибок, воспользуйтесь \spverb|ring-mock|~--- библиотекой для
запросов к Ring-приложениям. Ее функции покрывают основные сценарии в
тестах. Так, функция request принимает метод и путь. Если добавить словарь
параметров, то для GET они станут строкой запроса, а для POST~--- его
телом. URL-кодировку библиотека берет на себя. Другая функция json-body пишет в
запрос тело из коллекции. Рассмотрим несколько примеров.

Простая страница, запрос \spverb|GET /help|:

\begin{verbatim}
(mock/request :get "/help")
\end{verbatim}

Ввод данных в форму поиска, \spverb|GET /movies?search=batman&page=1|:

\begin{verbatim}
(mock/request :get "/movies" {:search "batman" :page 1})
\end{verbatim}

Новый пользователь из формы, \spverb|POST /users|. В этом запросе тело (ключ :body)
станет классом \spverb|ByteArrayInputStream|. Заголовок Content-Type примет значение
\spverb|"application/x-www-form-urlencoded"|.

\begin{verbatim}
(mock/request :post "/users" {:name "Ivan" :email "test@test.com"})
\end{verbatim}

Случай для JSON-API. Адрес /users ожидает не поля формы, а JSON-тело.Такой
запрос составляют из двух функций:

\begin{verbatim}
(-> (mock/request :post "/users")
    (mock/json-body {:name "Ivan" :email "test@test.com"}))
\end{verbatim}

\spverb|Проверка тела.| В тестах выше мы проверяли только статус ответа. На практике
одного статуса недостаточно: число 200 еще не говорит, что в ответе именно то,
что нужно. Проверка тела зависит от типа содержимого. Если это текст или HTML,
иногда хватит и регулярного выражения. Например, по фразе "Login" мы определим,
что на этой странице пользователь не авторизован.

Гораздо интересней вариант с JSON-API. В этом случае нужно восстановить данные
из ответа и сравнить с образцом. Для простоты вызовем приложение
sites-handler. Это заглушка, которой мы пользовались для тестировании
карт. Проверка ниже гарантирует, что при изменении ответа мы получим
ошибку. По-другому говорят, что ответ зафиксирован:

\begin{verbatim}
(let [request (mock/request :get "/search/v1/" {:lat 11 :lon 22})
      response (sites-handler request)
      body (-> response :body (cheshire.core/parse-string true))]
  (is (= {...} body)))
\end{verbatim}

Недостаток в том, что мы сравниваем данные как есть. Прием не сработает, если в
ответе даты, уникальные идентификаторы (UUID) или машинные номера. Перед
сравнением их исключают с помощью dissoc и map. Представим, что поиск кафе
вернул результат в таком виде:

\begin{verbatim}
{:sites [{:name "Site1" :date-updated "2019-11-12" :id 42}
         {:name "Site2" :date-updated "2019-11-10" :id 99}]}
\end{verbatim}

Код ниже вернет только те данные, которые не меняются от запроса к запросу. Их и
сравнивают с образцом.

\begin{verbatim}
(update body :sites
        (fn [sites]
          (map (fn [site]
                 (dissoc site :id :date-updated))
               sites)))
;; {:sites ({:name "Site1"} {:name "Site2"})}
\end{verbatim}

Иногда проверяют не конкретные значения, а структуру ответа. Это удобно на
больших данных, например, когда в каждом элементе двадцать и более полей. В
таком случае ответ проверяют спекой или JSON-схемой.

\begin{verbatim}
(let [;; obtain the response
      body (-> response :body (cheshire.core/parse-string true))]
  (is (s/valid? :api.search/handler body)))
\end{verbatim}

Затраты на спеку или схему окупаются в будущем. Ими проверяют входные параметры,
генерируют данные для тестов, описывают REST API (Swagger, RAML).

\section{Тестирование систем}

Коротко о том, как пишут тесты в проектах с системами, о которых мы говорили в
отдельной главе. Напомним, система это набор компонентов со связями между
ними. Покрыть каждый компонент тестами нетрудно; проблемы возникают на стыке при
взаимодействии. В проекте обязательно должен быть тест, где система работает как
единое целое.

Очевидно, чтобы тест прошел, кто-то должен запустить систему и остановить ее
после. На эту роль подходит фикстура. Будем полагать, что переменная системы и
функции ее запуска и останова находятся в модуле system.clj. Напишем фикстуру
\spverb|fix-system|:

\begin{verbatim}
(defn fix-system [t]
  (system/start!)
  (t)
  (system/stop!))
\end{verbatim}

На время ее работы в переменной \spverb|system/system| будет рабочая система. Другие
фикстуры, например, для работы с базой, могут обратиться к ее компонентам
напрямую. Важно только, чтобы в вызове use-fixtures они шли в правильном порядке
(левее~--- раньше), иначе вы получите Null-ошибки и другие неприятности.

\begin{verbatim}
(defn fix-db-data [t]
  (let [{:keys [db]} system/system]
    (prepare-test-data db)
    (t)
    (clear-test-data db)))

(use-fixtures :once fix-system fix-db-data)
\end{verbatim}

Читатель заметит, что это противоречит правилу, о котором мы говорили в главе
про системы. Что к системе нельзя обращаться напрямую и копаться в ее
компонентах. Что ж, признаем, для тестов в этом плане действуют
послабления. Обращаться к системе нельзя в промышленном коде. Но тесты это не
промышленный код, поэтому на небольшие нарушения в них порой закрывают
глаза. Главное, чтобы программист понимал, что он намерен выиграть. В нашем
случае выигрыш в том, что мы не пробрасываем компонент базы в фикстуру. Это было
бы честнее, но заняло больше строк кода.

Фикстура \spverb|fix-system| неслучайно стоит под ключом \spverb|:once|. Запуск и остановка
системы занимают много времени. В наших интересах прогнать как можно больше
тестов, пока система работает. Если же делать это поштучно, процесс затянется
надолго. Именно с этой проблемой вы столкнетесь при запуске тестов из CIDER. При
попытке выполнить один тест вы будете ждать, пока сработают все фикстуры, в том
числе fix-system.

Кажется, что пять-десять секунд это немного. Но представьте, что работаете над
задачей и запускаете тест раз за разом~--- подобные паузы раздражают и сбивают с
ритма. Поделимся с читателем техникой, которая решит эту проблему.

Потребуется два шага. Первый~--- научить систему, чтобы она знала о своем
состоянии. Например, включена ли она сейчас или выключена. Проще всего это
сделать через поле с метаданными. Вынесем имя поля в отдельную
переменную. Перепишем start! так, чтобы в метаданных системы появился флаг со
значением true. Аналогично работает stop!, только флаг принимает ложь. Функция
started? возвращает флаг значение из текущей системы.

\begin{verbatim}
(def state-field ::started?)

(defn start! []
  (let [sys (-> system
                component/start-system
                (with-meta {state-field true}))]
    (alter-var-root #'system (constantly sys))))

(defn started? []
  (some-> system meta (get state-field)))
\end{verbatim}

Второй шаг~--- перед тем, как включить систему в фикстуре, определить, была ли она
запущена вручную. Если нет, фикстура работает как обычно: запуск, тест,
остановка. Если да, это значит, что системой управляют в ручном режиме. В таком
случае фикстура только выполняет тест, что гораздо быстрее, чем полный цикл.

\begin{verbatim}
(defn fixture-system [t]
  (let [started-manually? (system/started?)]
    (when-not started-manually?
      (start!))
    (t)
    (when-not started-manually?
      (stop!))))
\end{verbatim}

Выполните в REPL (system/start!). Теперь можно вызвать тест сколько угодно раз~---
не придется ждать, пока запустится система.

\section{Интеграционные тесты}

Не протяжении главы мы постепенно усложняли тесты. С каждым шагом они все меньше
зависят от технических деталей и делают упор на бизнес-логику. По-другому этот
принцип называют пирамидой тестов. В ее основании лежат юнит-тесты~--- множество
отдельных проверок. Поднимаясь к вершине, мы абстрагируемся от технического
слоя: в какой-то момент тестируем не отдельные функции, а часть приложения.

Каждый уровень в пирамиде требует специальных знаний. Полагаем, читатель готов к
тому, чтобы подняться на последний этаж~--- освоить интеграционное
тестирование. Это особая фаза тестов, когда окружение максимально похоже на
промышленное. По-другому их еще называют UI- или Selenium-тестами в честь
одноименного фреймворка. Для большей реалистичности мы не шлем запросы
программно, а имитируем действия человека. Например, управляем браузером: вводим
данные в форму, нажимаем кнопку и проверяем, что появился нужный текст.

Интеграционные тесты выполняются медленно, потому что включают полный цикл
приложения. Это загрузка страницы, выполнение скриптов, ожидание запроса, во
время которого сервер ходит в базу или очередь задач. Если возникнет ошибка, ее
трудно расследовать из-за длины цепи. Представьте, что вы нажали на кнопку, но
текст не появился. Возможны десятки причин, почему этого не произошло.

В этом разделе мы рассмотрим, как писать UI-тесты на Clojure. С подготовительной
частью вы уже знакомы: запускаем систему и наполняем базу тестовыми данными. А
вот тест ведет себя по-новому. Он захватывает контроль над браузером и командует
им. Например, открывает страницу \spverb|http://127.0.0.1:8080/| и щелкает по
ссылкам. В любой момент мы получим адрес страницы, ее заголовок и HTML-код. В
тест добавляют формы \spverb|(is (= ...))|, чтобы проверить, на какой странице мы
оказались или видит ли пользователь этот виджет.

Чтобы управлять браузером, понадобится драйвер и библиотека к нему. Под
драйвером понимают утилиту командной строки. Когда драйвер запущен, он принимает
запросы от библиотеки по протоколу HTTP. Одновременно драйвер запускает браузер
в особом режиме, и между ними образуется связь. Фактически, драйвер это
посредник между двумя акторами. Его работа сводится к переводу HTTP-команд в
бинарный протокол браузера и наоборот.

Каждый браузер работает со своим драйвером. Для Chrome он называется
\spverb|chromedriver|, для FireFox~--- \spverb|geckodriver|. Одноименные утилиты ставятся из
пакетных менеджеров \spverb|apt|, \spverb|yum| или \spverb|brew| в зависимости от операционной
системы. Пользователи Windows скачают бинарные файлы с сайта проекта. Драйвер к
Safari называется \spverb|safaridriver|. С версии 13 он идет в комплекте с Mac OS.

Для работы с драйвером подойдет библиотека Etaoin\footurl{https://github.com/igrishaev/etaoin}.
Добавьте ее в зависимости профиля \spverb|:dev| (библиотека нужна только для тестов):

\begin{verbatim}
:dev {:dependencies [[etaoin "0.3.6"]]}
\end{verbatim}

Убедитесь, что драйвер находится в одной из папок, перечисленных в PATH,
например, \spverb|/usr/local/bin|. Другими словами, его можно вызвать в терминале,
просто выполнив \spverb|chromedriver| или \spverb|geckodriver|. Путь до драйвера можно задать
в опциях библиотеки, но пока что мы сами положим его в нужное место.

Теперь напишем первый тест. Представим, что локальный сервер работает на
порту 8080. Тест открывает форму входа, заполняет поля и нажимает кнопку
"Login". Страница обновляется, наверху видно приветствие. Пользователь видит
элементы, которые прежде были скрыты (ссылки "My profile", "Logout").

\begin{verbatim}
(ns project.integration-tests
  (:require [etaoin.api :as e]))

(deftest test-ui-login-ok
  (e/with-chrome {} driver
    (e/go driver "http://127.0.0.1:8080/login")
    (e/wait-visible driver {:fn/has-text "Login"})
    (e/fill driver {:tag :input :name :email} "test@test.com")
    (e/fill driver {:tag :input :name :password} "J3QQ4-H7H2V")
    (e/click driver {:tag :button :fn/text "Login"})
    (e/wait-visible driver {:fn/has-text "Welcome"})
    (is (e/visible? driver {:tag :a :fn/text "My profile"}))
    (is (e/visible? driver {:tag :button :fn/text "Logout"}))))
\end{verbatim}

Разберем отдельные выражения из этого теста. Форма \spverb|with-chrome| это макрос,
который запускает Хром на время исполнения тела. Макрос необходим, чтобы
гарантированно закрыть драйвер при выходе или в случае ошибки. Без него пришлось
бы писать код с ручным \spverb|try/catch|, что порождает вложенность и в целом
неудобно:

\begin{verbatim}
(let [driver (e/chrome)]
  (try
    (e/go driver "http://...")
    (e/click driver {:tag :button})
    (finally
      (e/quit driver))))
\end{verbatim}

Функция \spverb|wait-visible| ждет до тех пор, пока элемент не появится на экране. К
ней прибегают довольно часто, чтобы дождаться, пока браузер отрисует
верстку. Если не разделить две инструкции ожиданием, между ними будет разница в
несколько миллисекунд. Второе действие либо не успеет выполниться, либо будет
отброшено.

Ожидание в UI-тестах это нормально. Основное время уходит на то, чтобы дождаться
загрузки и выполнить действия с задержкой, как это свойственно
человеку. \spverb|Wait-visible| это лишь одна из семейства wait-функций. В их число
входит \spverb|wait-has-text| (дождаться текст на экране), \spverb|wait-has-class| (ждать,
пока у элемента не появится класс) и многие другие.

Драйвер ищет элементы на странице с помощью селекторов. Это строки с особыми
выражениями; различают CSS- и XPath-селекторы. Сейчас мы не будем разбирать их
синтаксис: это долго и заслуживает отдельной главы. Для простоты рассмотрим
альтернативный прием. На элемент можно сослаться, задав его свойства
словарем. Ключи tag и id означают имя тега и его идентификатор. Другие ключи
становятся атрибутами тега. В примере выше селектор \spverb|{:tag :input :name :email}|
становится выражением \spverb|input[@name='email']| на языке XPath.

Подумаем, как улучшить наш тест? Для начала рассмотрим порт 8080. Мы уже знаем,
что подобные значения приходят из конфигурации. Исправьте тест так, чтобы и
сервер, и драйвер работали с одним и тем же портом.

Вспомним, как работает \spverb|with-chrome|: он создает новый драйвер, выполняет тело и
выключает его. Если каждый тест обернут в \spverb|with-chrome|, мы теряем время,
многократно повторяя эти шаги. Сделаем так, чтобы драйвер работал на протяжении
всего прогона тестов. Объявим динамическую переменную и фикстуру, которая
связывает драйвер на время тестов. Зарегистрируем ее с ключом \spverb|:once|.

\begin{verbatim}
(defonce ^:dynamic *driver* nil)

(defn fix-chrome [t]
  (e/with-chrome {} driver
    (binding [*driver* driver]
      (t))))
\end{verbatim}

Мы тестируем приложение в Хроме, самом популярном браузере. Но вот приходит
задача~--- убедиться, что мы также поддерживаем Firefox. Технически это значит,
что все тесты, которые мы написали для Хрома, должны сработать еще раз в другом
браузере. Скопировать их и заменить with-chrome на with-firefox~--- сомнительное
решение. Представьте, что через месяц нас попросят добавить Safari. Должен быть
способ, который не влечет за собой разрастание кода.

Поможет мульти-фикстура, с которой мы знакомились в середине главы. Она
пробегает по списку типов браузеров; в примере ниже это \spverb|:firefox| и
\spverb|:chrome|. Макрос \spverb|with-driver| это общий случай \spverb|with-chrome| и
аналогов. Отличие в том, что \spverb|with-driver| ожидает первым аргументом тип
браузера. На каждом шаге фикстура связывает драйвер и выполняет тест.

\begin{verbatim}
(defn fix-multi-driver [t]
  (doseq [driver-type [:firefox :chrome]]
    (e/with-driver driver-type {} driver
      (binding [*driver* driver]
        (testing (format "Browser %s" (name driver-type))
          (t))))))
\end{verbatim}

Теперь тесты по очереди сработают в каждом из браузеров. Для ясности мы
оборачиваем тест в сообщение о том, в рамках какого браузера его
вызывают. Поддержка нового браузера сводится к тому, чтобы добавить в список
ключ \spverb|:safari|, \spverb|:edge| и другие.

Еще один способ улучшить тесты~--- вынести одинаковые действия в фикстуру или
функцию. Например, каждый тест начинается с авторизации и заканчивается выходом
из системы. Чтобы не копировать эти действия каждый раз, напишем фикстуру
\spverb|fix-login-logout|. В отличии от предыдущих фикстур, ее регистрируют с ключом
\spverb|:each|.

\begin{verbatim}
(defn fix-login-logout [t]
  (doto *driver*
    (e/go "http://127.0.0.1:8080/login")
    (e/fill {:tag :input :name :email} "test@test.com")
    (e/click {:tag :button :fn/text "Login"}))
  (t)
  (doto *driver*
    (e/click {:tag :button :fn/text "Logout"})
    (e/wait-has-text "Login")))
\end{verbatim}

Попутно мы внедрили еще одну хорошую практику. Когда несколько функций принимают
одинаковый первый аргумент, их объединяют в макрос doto. Он подставит \spverb|*driver*|
на второе место в каждый список тела. С \spverb|doto| код становится немного короче и
чище.

\section{Другие решения}

В последнем разделе мы перечислим другие библиотеки, полезные для тестов. Мы не
будем рассматривать их досконально: ограничимся кратким описанием и примером
кода. Все ответы ищите в документации к проектам.

\subsection{Продвинутые моки}

На минуту вернемся к мокам~--- подмене функции через \spverb|with-redefs|. Этот макрос
слишком многословен, чтобы работать с ним напрямую. Появились библиотеки,
которые описывают мокинг короче и выразительнее. Одна из них называется
\spverb|mockery|. Библиотека предлагает макрос \spverb|with-mock| следующего вида:

\begin{verbatim}
(with-mock mock
  {:target :project.path/get-geo-point
   :return {:lat 14.23 :lng 52.52}}
  (get-geo-point "cafe" "200m"))
\end{verbatim}

В примере выше мы <<замокали>> \spverb|get-geo-point|, которая, судя по названию,
обращается к стороннему сервису карт. Внутри макроса объект \spverb|mock| это атом,
внутри которого словарь. Он наполняется данными по мере того, как мы вызываем
цель. Например, сколько раз ее вызвали и с какими аргументами. В выражении ниже
мы добавили проверки на то, что функцию вызвали один раз с аргументами "cafe" и
"200m".

\begin{verbatim}
(let [{:keys [called? call-count call-args]} @mock]
  (is called?)
  (is (= 1 call-count))
  (is (= '("cafe" "200m") call-args)))
\end{verbatim}

Библиотека \spverb|spy| устроена похожим образом. На функцию навешивается <<шпион>>,
который копит данные о вызове.

\begin{verbatim}
(defn adder [x y] (+ x y))
(def spy-adder (spy/spy adder))

(testing "calling the function"
  (is (= 3 (spy-adder 1 2))))

(testing "calls to the spy can be accessed via spy/calls"
  (is (= [[1 2]] (spy/calls spy-adder))))
\end{verbatim}

\subsection{Альтернативный синтаксис}

Проект \spverb|midje| предлагает другой способ писать тесты. В этой библиотеке мы имеем
дело с фактами. Факт это набор проверок, сгруппированных по смыслу. В примере
ниже факты о функции \spverb|split|:

\begin{verbatim}
(facts "about split"
 (str/split "a/b/c" #"/") => ["a" "b" "c"]
 (str/split "" #"irrelvant") => [""])
\end{verbatim}

Стрелка между выражениями это особый оператор, который называется extended
equality, продвинутое равенство. С ее помощью сравнение величин записывается
короче. Например, форма \spverb|1 => even?| приходит к виду \spverb|(even? 1)|.
В \spverb|midje| встречаются и другие, более сложные стрелки для проверки коллекций и макросов.

\subsection{Вывод XUnit}

Плагин \spverb|test2junit|\footurl{https://github.com/ruedigergad/test2junit}
делает так, что отчет о тестах пишется в XML-файл формата
XUnit. Системы непрерывной интеграции, например, CircleCI или TeamCity понимают,
как отобразить его графически. Такой отчет легче просматривать, чем вывод
консоли. Проблемные места выделены красным, стектрейсы спрятаны под выпадающие
элементы. Плагину нужно задать путь к папке, куда писать файл.

\begin{verbatim}
:plugins [[test2junit "1.1.2"]]
:test2junit-output-dir "target/test2junit"
\end{verbatim}

\subsection{Генерация данных}

Возможно, вы столкнетесь с тем, что для тестов нужен большой объем данных,
например, сто или двести тысяч записей. При этом данные должны быть разнообразны
— нас не устроит один и тот же набор, скопированный тысячу раз. Поможет
библиотека \spverb|test.check|. Ее модуль gen генерирует случайные данные по заданным
правилам. Особенно полезна генерация записей. В примере ниже мы получаем кортеж
строки, числа и булева типа. Затем применяем его к конструктору \spverb|->User|.

\begin{verbatim}
(defrecord User [user-name user-id active?])

(def user-gen
  (gen/fmap (partial apply ->User)
            (gen/tuple (gen/not-empty gen/string-alphanumeric)
                       gen/nat
                       gen/boolean)))

(last (gen/sample user-gen))
;; => #user.User{:user-name "dfgJKSHF3"
;;               :user-id 5
;;               :active? false}
\end{verbatim}

Библиотека \spverb|clojure.spec|, которой мы посвятили главу, идет еще дальше. С
помощью \spverb|test.check| она генерирует данные по спеке. Так проявляется еще одно
свойство спек: кроме проверки, они подходят для тестовых данных.

\begin{verbatim}
(s/def :user/id int?)
(s/def :user/name string?)
(s/def :user/active? boolean?)
(s/def ::user (s/keys :req-un [:user/id :user/name :user/active?]))

(gen/generate (s/gen ::user))
{:id 88546920, :name "Z4MO7GH80k3mRD", :active? true}
\end{verbatim}

Возможности \spverb|spec.gen| обширны. С ее помощью порождают связанные данные,
например, пользователей, которые ссылаются на профили и наоборот. Вместо
случайных величин можно опираться на допустимые значения (списки имен,
городов). Спеки бывают быть любой вложенности, что открывает поле для
экспериментов.

%% TODO: order

\section{Порядок аргументов}

Необычный вопрос: как писать правильно, \spverb|(is (= 200 status))| или
\spverb|(is (= status 200))|? На первый взгляд это абсурд. Разве может порядок
аргументов влиять на равенство? Значения либо равны, либо нет. Однако, макрос
\spverb|is| устроен сложнее, чем мы думаем. Он разбирает выражение \spverb|(= 200 status)|
и выделяет ожидаемую и фактическую части. По-английски они
называются \emph{expected} и \emph{actual}.

Ожидаемое это значение, на которое рассчитывает тест. Как правило, это готовое
число или коллекция, которую посчитали заранее. Фактическое значение~--- то, к
которому мы пришли самостоятельно, например, вызвав функцию. Так, число 68 это
ожидаемое, а \spverb|(int (->fahr 20))|~--- действительное. Статус 200 это ожидаемое, а
\spverb|(:status response)|~--- действительное.

Такое разделение необходимо для отчетов. Когда значения не равны, нам бы
хотелось увидеть, где мы ошиблись. Предположим, что в отчете написано:
\spverb|failed (= 200 403)|. Не совсем ясно, как это трактовать. Мы ожидали
успешный ответ, но не хватило прав доступа? Или это брешь в безопасности~---
ожидали, что прав на эту страницу нет, но пользовать все-таки ее увидел? Если же
написано expected 200, got 403, то все ясно~--- это первый случай (не хватило
прав).

Теперь запомните правило: ожидаемое стоит на первом месте, а действительное на
втором. Поэтому пишите \spverb|(is (= 200 status))| вместо \spverb|(is (= status 200))|. Автор
согласен, что это непривычно и противоречит здравому смыслу. Как правило,
фактическое это число, а действительное~--- длинное выражение, поэтому хочется
записать их как слева. Увы, придется побороть себя и писать как справа:

\begin{verbatim}
;; wrong                     ;; correct
(= (int (->fahr 20)) 68)     (= 68 (int (->fahr 20)))
\end{verbatim}

Это не ошибка дизайна; правило уходит корнями в прошлое. Первый тестовый
фреймворк \spverb|JUnit| утвердил именно такой порядок в методах сравнения. Хорошо это
или плохо, судить уже поздно~--- принцип <<expected слева>> стал промышленным
стандартом. Аналогичное правило работает в языках Python, Ruby и
других. Отдельные фреймворки предлагают модули, чтобы <<перестать говорить как
Йодо>>, то есть поменять семантику аргументов. Технически это возможно и в
Clojure, но сейчас мы не будем это рассматривать.

Особенность \emph{expected} и \emph{actual} видна при запуске тестов в
CIDER. Один и тот же тест проверяет статус ответа на 200. Пока все идет хорошо,
нет разницы, в каком порядке мы записали аргументы. Но в случе ошибки вариант
слева вносит путаницу. Согласно ему, нормальным считается статус 404, а не
200. Вариант справа выводит статусы правильно.

\begin{verbatim}
;; wrong                ;; correct
(is (= status 200))     (is (= 200 status))

Fail in test-...        Fail in test-...
expected: 404           expected: 200
  actual: 200             actual: 404
    diff: - 404             diff: - 200
          + 200                   + 404
\end{verbatim}

Тем не менее, не стоит соблюдать это правило слишком рьяно. Иногда равенство с
перепутанными аргументами смотрится лучше, и потому код легче
поддерживать. Следить за порядком или нет остается на усмотрение команды. Автор
признаётся, что на этапе черновика перепутал аргументы везде, где это возможно.

%% TODO: заключение
