\chapter{Тесты}

\label{chapter-tests}

\begin{teaser}
В последней главе мы поговорим о тестировании кода. Читатель узнает, что такое
тесты и каких видов они бывают. Обойд\"{е}мся без лишней сложности: опустим
термины вроде <<TDD>> и <<BDT>>. Покажем, что в~Clojure легко писать и
поддерживать тесты.
\end{teaser}

\section{Основные понятия}

На тему тестов написано много книг и статей, придуманы методологии. Их
сторонники отстаивают позиции в долгих спорах. Начинающих сбивает с толку
фрагментация терминов и мнений. Чтобы погрузиться в тему, расскажем о тестах
простыми словами.

Тест~--- это код, который проверяет другой код. Напишем функцию для перевода
температуры из Цельсия в Фаренгейты:

\begin{english}
  \begin{clojure}
(defn ->fahr [cel]
  (+ (* cel 1.8) 32))
  \end{clojure}
\end{english}

Мы вызвали ее несколько раз убедились, что результаты такие же, как в
справочнике. Зафиксируем расчеты в функции проверки. Она сравнивает вызов
\spverb|->fahr| с каноничными значениями. Их посчитали заранее и расценивают как
эталон.

\begin{english}
  \begin{clojure}
(defn test-fahr []
  (assert (= (int (->fahr 20)) 68))
  (assert (= (int (->fahr 100)) 212)))
  \end{clojure}
\end{english}

Примечание: в тесте мы сравниваем результат \spverb|->fahr| с целым числом. Без
обертки в \spverb|(int ...)| функция вернет число с плавающей запятой, которое
трудно сравнить с другим таким же числом. Для удобства значение приводят к
целому.

Макрос \spverb|assert| бросит исключение, если тело вычисляется в ложь. Пока что
\spverb|(test-fahr)| вернет \spverb|nil| без ошибок. Если изменить расчеты,
получим ошибку:

\begin{english}
  \begin{clojure}
(defn ->fahr [cel]
  (+ (* cel 1.9) 32))

(test-fahr)
;; Assert failed: (= (int (->fahr 20)) 68)
  \end{clojure}
\end{english}

Функция \spverb|test-fahr| и есть тест. Она проверяет, что код \spverb|->fahr|
не изменили так, что получится другой результат. В наших интересах вызвать
\spverb|(test-fahr)| перед сборкой программы. Так мы не допустим, чтобы код с
ошибкой попал в производство.

Тест не знает устройство функции, которую проверяет. Мы вправе менять алгоритм
до тех пор, пока тест работает без ошибок. Предположим, мы тестируем функцию
факториала. В первой версии мы линейно умножаем числа, что неэффективно. Уже на
этой стадии добавим тест, что $5! = 120$. Если заменить линейный алгоритм на
дерево или таблицу значений, тест пройдет без ошибок. В противном случае в новом
алгоритме ошибка.

\subsection{Кейсы и покрытие}

Близкие по семантике тесты объединяются в \emph{кейсы}. Вспомним школьную задачу
с квадратным уравнением: найти корни по заданным коэффициентам $a$, $b$,
$c$. Особенность задачи в том, что е\"{е} логика ветвится. В зависимости от
параметров может быть два корня, один или ни одного.

\begin{english}
  \begin{clojure}
(defn square-roots [a b c]
  (let [D (- (* b b) (* 4 a c))]
    (cond
      (pos? D) [(/ (+ (- b) (Math/sqrt D)) (* 2 a))
                (/ (- (- b) (Math/sqrt D)) (* 2 a))]
      (zero? D) [(/ (- b) (* 2 a))]
      (neg? D) nil)))
  \end{clojure}
\end{english}

Чтобы проверить алгоритм, тест вызывает функцию \spverb|square-roots| минимум
три раза. Подберем параметры так, чтобы сработала каждая ветка. Еще лучше, если
на каждую ветку приходится отдельный тест, чтобы расширить их в будущем. Тесты
ниже образуют кейс, который проверяет алгоритм целиком.

\begin{english}
  \begin{clojure}
(defn test-square-roots-two-roots []
  (let [[x1 x2] (square-roots 1 -5 6)]
    (assert (= [3 2] [(int x1) (int x2)]))))

(defn test-square-roots-one-root []
  (let [[x1 x2] (square-roots 1 6 9)]
    (assert (= [-3] [(int x1)]))
    (assert (nil? x2))))

(defn test-square-roots-no-roots []
  (assert (nil? (square-roots 2 4 7))))
  \end{clojure}
\end{english}

В объектно-ориентированных языках кейсы это классы, а тесты~--- их методы. В
Clojure тест это функция, а роль кейса играет пространство.

В тестах часто упоминают \emph{покрытие}. Так называют долю кода, которая
сработала в тесте. Предположим, мы написали тест только для двух корней. В этом
случае ветки \spverb|(zero? D)| и \spverb|(neg? D)| не сработают. Фреймворк
считает общее число строк и сколько из них выполнились. Покрытие это отношение
второй величины к первой, число от нуля до единицы. Некоторые библиотеки строят
отчет, где выполненные строки отмечены зеленым (плюсом), а пропущенные красным
(минусом). Подсчет покрытия это трудная задача, поэтому ее выносят в расширение
фреймворка.

\begin{english}
  \begin{diff}
+(defn square-roots [a b c]
+  (let [D (- (* b b) (* 4 a c))]
+    (cond
+      (pos? D) [(/ (+ (- b) (Math/sqrt D)) (* 2 a))
+                (/ (- (- b) (Math/sqrt D)) (* 2 a))]
-      (zero? D) [(/ (- b) (* 2 a))]
-      (neg? D) nil)))
  \end{diff}
\end{english}

В функции \spverb|square-roots| семь строк. При запуске теста сработали пять из
них. Покрытие составит 5/7, что приблизительно 71\%. Считается, что покрытия
80\% достаточно, чтобы код работал надежно. Значения ниже говорят о малом
покрытии. Это значит, в проекте встречается код без защиты от спонтанных
изменений. При попытке приблизиться к 100\% тесты становятся затратными в
поддержке. В каждой команде придерживаются той величины, которая удобна
процессу.

Величина в процентах не должна затмевать здравый смысл. Цифра 71\% кажется
большой только на первый взгляд. Из покрытия видно, что мы проверяем только одну
ветку алгоритма из трех. Если в двух других ошибки, мы не узнаем о них. При
оценке покрытия смотрят не на проценты, а на развилки алгоритма. Надежный тест
гарантирует, что сработала каждая ветка.

\subsection{Не только числа}

У начинающих складывается мнение, что тесты пишут только для математических
расчетов. На самом деле тесты подходят для всех действий, которые хотят
зафиксировать. Это может быть обход коллекции, криптография, поиск в тексте.

Рассмотрим подпись HTTP-запроса. Функция \spverb|sign-params| принимает словарь
параметров и секретный ключ. Алгоритм подписи следующий:

\begin{itemize}

\item
  отсортировать параметры по ключам;

\item
  составить строку \spverb|param1=value1&param2=value2...|;

\item
  экранировать пробел, плюс, процент и другие служебные символы;

\item
  получить сигнатуру строки по алгоритму HMAC-SHA256 и секретному ключу;

\item
  вернуть словарь параметров с полем \spverb|:signature|.

\end{itemize}

Так работает авторизация в популярных сервисах. Напишем тест для функции
подписи. Проверим, что входные параметры дополнены сигнатурой, которую посчитали
заранее.

\begin{english}
  \begin{clojure}
(defn test-sign-params []
  (let [public-key "2Ag48776s634Tsdf"
        secret-key "fnsd809SDf2dfs3t"
        params {:action :postComment
                :post_id 217
                :comment "This is a great article!"
                :api_key public-key}
        signature "01b03dbf1fa5a4c5"]
    (assert (= (sign-params params secret-key)
               (assoc params :signature signature)))))
  \end{clojure}
\end{english}

Заметим, что \spverb|sign-params| работает не с числами, а коллекциями и
криптографией. Это не помешало написать для нее тест. Чтобы улучшить нашу
работу, добавьте проверку на пустой ключ, экранирование символов, кириллицу,
эмодзи и так далее.

\subsection{Что тестировать}

Иногда разработчики не понимают, какие функции нужно проверить, и тестов не
хватает. Другая крайность~--- писать тесты строго для каждой функции. Это
приводит к их избыточности и замедляет разработку. Верное решение, как всегда,
посередине.

Тестируйте действия, которые атомарны с точки зрения логики. Подпись запроса
состоит из мелких функций: сортировки, экранирования и других. Если проверять
каждый шаг, получится слишком много кода. В будущем алгоритм подписи изменится,
и тесты прид\"{е}тся удалить. И наоборот, когда тестируют огромный пласт кода,
трудно понять, почему возникла ошибка. Тест становится хрупким: малейшее
изменение рушит его. Крупные тесты разбивают на несколько поменьше.

Подробнее на вопрос отвечает пирамида тестов\footurl{https://martinfowler.com/bliki/TestPyramid.html}.
Это модель, где тесты разделены на слои. Чем выше слой, тем б\'{о}льше логики он
покрывает. Говорить о пирамиде преждевременно; мы вернемся к ней ближе к концу
главы.

\subsection{О чистоте функций}

Функцию удобно тестировать, если у нее нет побочных эффектов. Это значит,
функция не обращается к диску, сети и глобальным переменным. Результат зависит
только от входных параметров. Случайное число и текущая дата тоже считаются
побочными эффектами.

Пусть функция читает файл. Перед запуском теста мы пишем файл в нужное место, а
потом удаляем его. Тест становится сложнее, потому что теперь зависит от
окружения. Чтобы облегчить тестирование, функцию разделяют на две: первая читает
содержимое файла и передает во вторую. До исправления:

\begin{english}
  \begin{clojure}
(defn process-csv [path]
  (let [content (slurp path)]
    (for [line (clojure.string/split content #"\n")]
      (remap-line line))))
  \end{clojure}
\end{english}

\noindent
и после:

\begin{english}
  \begin{clojure}
(defn process-csv-content [content]
  (for [line (clojure.string/split content #"\n")]
    (remap-line line)))

(defn process-csv [path]
  (process-csv-content (slurp path)))
  \end{clojure}
\end{english}

Напишем тест для \spverb|process-csv-content|. Не придется создавать файл на
диске: функция принимает строку, которую объявим на месте.

\begin{english}
  \begin{clojure}
(def CONTENT
  (str "Ivan;ivan@test.ru;http://example.ru"
       \newline
       "John;john@test.com;http://example.com"))

(assert (= [{:name "Ivan" :email ...}
            {:name "John" :email ...}]
           (process-csv-content CONTENT)))
  \end{clojure}
\end{english}

Другой способ в том, чтобы заменить функцию на мультиметод. Реализация
зависит от первого аргумента. Если это файл, читаем его и вызываем мультиметод
со строкой. Для строки запускаем ее разбор.

Чистая функция это идеальный случай тестирования. Промышленный код трудно
изолировать от побочных эффектов, а иногда это невозможно. Чтобы тестировать код
с эффектами, прибегают к расширенным практикам: используют заглушки (стабы),
меняют объекты в рантайме (моки). Мы рассмотрим эти и другие приемы ниже.

Подмена среды это вынужденная мера, и к ней прибегают в последнюю
очередь. Проверьте, можно ли выделить в коде чистую часть и насколько это
затратно. Возможно, небольшое изменение в коде упростит тесты.

\subsection{Производство}

Тесты не только проверяют логику, но и помогают на разных стадиях
производства. Ветку с изменениями нельзя слить в мастер до тех пор, пока не
пройдут тесты. Так вы обезопасите код от случайных изменений.

Если разработчик добавил код без тестов, это снизит долю покрытия. Можно задать
правило, что при покрытии ниже порога слияние запрещено. GitHub, CircleCI и
другие сервисы гибко реагируют на ошибки в тестах. Они шлют письма, пишут
сообщения в Slack, всячески сигналят о проблеме.

Иногда тесты заменяют документацию. В открытых проектах е\"{е} нехватку
компенсируют тестами, потому что на качественный текст уходит столько же
времени, сколько на разработку. Если вы не уверены, как работает функция,
возможно, ее поведение прояснится в тесте.

\subsection{Устройство тестов}

Функция с макросом \spverb|assert| это не тест, а приближение. Она пригодится на
раннем этапе, когда автор только обозначил контуры программы. Для полноценных
тестов понадобится фремворк~--- библиотека, которая определяет, где и как писать
тесты.

Если выполнить команду \spverb|lein test| в папке проекта, запустится тестовый
фреймворк. Это сложный цикл из нескольких шагов. Рассмотрим основные из них,
чтобы читатель понял его устройство.

На первом этапе фреймворк ищет тесты в коде проекта. По-другому фаза называется
\emph{test discovery}. Каждый найденный тест попадает в глобальный регистр.

Тест должен носить признак, которым он отличается от обычной функции. В примере
выше \spverb|test-sign-params| это именно функция, а не тест (префикс
\spverb|test-| ни на что не влияет). Далее мы рассмотрим, как превратить ее в
тест.

Когда тесты найдены, система определяет, какие из них вызвать. Если не заданы
фильтры, сработают все тесты. Иногда нужно выполнить их подмножество по имени
или тегу. В этом случае происходит отбор тестов.

На этапе прогона фреймворк выполняет тесты один за другим как функции. Система
фиксирует их количество и имена. Тест обернут в \spverb|try/catch|, чтобы
исключение не оборвало процесс. Каждая проверка оседает в накопителе. Эти
сведения необходимы, чтобы построить отчет.

На этапе отчета система выводит данные в консоль. Отчет должен быть понятен
человеку. Проблемные тесты обозначены красным цветом. Для неудачных сравнений
ожидают полные формы, а не финальные значения. Например,
\spverb|(= (int (->fahr 20)) 68)| вместо \spverb|(= 69 68)|.

Фреймворки выводят отчет в разных форматах. Кроме консоли это может быть файл
HTML. Систему настраивают так, что отчет публикуется на сервере по адресу
\spverb|/<project>/<artifact>/tests.html|. Часть \spverb|<artifact>| это имя
ветки или хэш коммита.

Стандарт XUnit определяет, как записать отчет в XML-файл. Системы вроде TeamCity
и CircleCI понимают, как показать его в удобном виде. У вас могут быть тесты на
Python, JavaScript или Clojure, но если прогон запущен с флагом \spverb|--xunit|
(или подобным), отчеты будут похожи.

Это был поверхностный обзор того, как работает тестовый фреймворк. Далее мы
рассмотрим, как управлять им на каждом этапе.

\subsection{Типы тестов}

Прежде чем писать тесты, разберемся, каких видов они бывают. Возможно, читатель
слышал термины <<юнит>>, <<интеграционные>> и другие. На первый взгляд это
излишняя классификация. На практике тесты из разных категорий отличаются стилем,
окружением и запуском.

Понятие <<юнит-тест>> происходит от английского <<unit>>, модуль. Термин пришел
из мира Java и фреймворка JUnit. Согласно ему, тесты группируют по модулям. На
каждый модуль приложения приходится одноименный тестовый, который его проверяет.

\textbf{Юнит-тесты} фиксируют логику приложения. Это базовые, самые необходимые
тесты. Они поощряют чистые функции: чем меньше требований к окружению, тем
удобней поддержка. Юнит-тесты проверяют код по частям. Примеры с температурой и
уравнением были юнит-тестами.

\textbf{Интеграционные тесты} решают проблему, с которой не справляются
юнит-тесты. Даже если отдельные части проверены, это не значит, что они
согласованы друг с другом. Рассмотрим регистрацию на сайте. Это сложный процесс
из нескольких шагов:

\begin{itemize}

\item
  проверить, что почта еще не занята;

\item
  создать временную учетную запись;

\item
  сгенерировать ссылку для подтверждения записи;

\item
  отправить письмо со ссылкой;

\item
  получить с почтового сервера последнее письмо;

\item
  найти в нем ссылку и перейти по ней;

\item
  проверить, что ссылка не подделана и не устарела;

\item
  активировать учетную запись.

\end{itemize}

Интеграционные тесты выполняют в среде, максимально близкой к реальности. Это
главное отличие от юнит-тестов: в первом случае окружение сужают, во втором
расширяют. Их труднее организовать из-за обширного состояния. Если выполнить
тест ещ\"{е} раз, пользователь уже будет в системе, и сценарий не пройдет.

Ошибки в интеграционных тестах требуют расследования. Юнит-тесты выводят отчет,
где четко указано, какое выражение не равно чему. В интеграционных тестах
причина чаще всего не ясна. После нажатия на кнопку мы ожидали текст с просьбой
проверить почту, но он не появился. Была ли это ошибка на сервере? Пользователь
существует? Не заполнили нужное поле? Текст изменился, и селектор его не
находит? Ответ требует внимания человека.

Чтобы проверить проект под нагрузкой, прибегают к \textbf{нагрузочному тестированию}.
Приложению посылают запросы и считают время ответа, число запросов, сколько
из них параллельны, потребление процессора и памяти. По этим данным строят
отчеты и графики.

Нагрузочное тестирование находит медленные части в проекте. С помощью него
проверяют, насколько <<фича>> готова к запуску. Если боевой сервер держит сто
запросов в секунду, а новый раздел только двадцать, его отправят на доработку.

К другим типам относятся Smoke-, Sanity-, регрессионные тесты и другие. Мы не
ставим цель рассмотреть их все. В этой главе изучим юнит- и интеграционные
тесты. К концу книги читателю хватит знаний, чтобы разобрать другие методы
самостоятельно.

\section{Тесты в Clojure}

Познакомимся с тестовым фреймворком Clojure. Напишем тесты, которые отвечают его
требованиям. За основу возьмем функции Фаренгейта и квадратного уравнения,
которые написали выше.

Пусть функция \spverb|->fahr| находится в модуле \spverb|src/book/util.clj|.
Создайте файл \spverb|test/book/util_test.clj| с кодом:

\begin{english}
  \begin{clojure}
(ns book.util-test
  (:require [book.util :refer [->fahr]]
            [clojure.test :refer [deftest testing is]]))

(deftest test-fahr
  (is (= 68 (int (->fahr 20))))
  (is (= 212 (int (->fahr 100)))))
  \end{clojure}
\end{english}

Получился тестовый модуль. Он импортирует \spverb|->fahr| и объявляет тест
\spverb|test-fahr|. Макросы \spverb|is| и \spverb|deftest| приходят из пакета
\spverb|clojure.test|. Это тестовый фреймворк из поставки Clojure.

Шесть строк несут много новой информации, и у читателя появятся вопросы. Почему
тесты лежат в папке \spverb|test|, а не \spverb|src|? Разве не логично хранить
тесты рядом с тем, что они проверяют? Как система найдет их?  Почему тест
объявляют макросом \spverb|deftest|? Автор утверждал, что тест это функция, или
это не так? Зачем менять \spverb|assert| на \spverb|is|? Это правильные вопросы,
ответим на них по порядку.

В Clojure разделяют код приложения и тестов. Приложение находится в папке
\spverb|src|, а тесты в \spverb|test|. У подхода несколько
преимуществ. Во-первых, код делится на смысловые части. Чтобы найти неисправный
тест, мы сужаем область поиска до папки \spverb|test|, а не ищем по всему
проекту. Во-вторых, когда тесты отделены, снижается риск, что они попадут в
скомпилированный файл. В этом нет смысла, потому что тесты запускают только в
разработке. Мы напрасно увеличим время сборки и объем файла.

Во время тестов проект запускается с особым параметром. Он определяет, где еще
кроме \spverb|src| искать код. Это список, в котором по умолчанию
\spverb|"src"|. В режиме тестов система добавляет к нему путь
\spverb|"test"|. Когда компилятор ищет файлы, он по очереди обходит пути из
списка. Позже мы узнаем, как настроить пути под нужды проекта.

Фреймворку нужен признак, чтобы отличить тест от обычной функции. Макрос
\spverb|deftest| объявляет функцию и сообщает ей метаданные с полем
\spverb|:test|. Чтобы найти тесты, фреймворк сканирует модули и читает
метаданные каждой переменной. Если среди полей есть \spverb|:test|, переменная
считается тестом.

Заметим, что \spverb|deftest| не позволит задать аргументы теста. Это сделано
нарочно: в Clojure тест не зависит от внешних данных. Если тесту нужно особое
окружение, его задают фикстурами. Мы рассмотрим фикстуры в отдельном разделе.

Макрос \spverb|is| отличается от \spverb|assert| побочными
эффектами. \spverb|Assert| это чистое сравнение: получим либо \spverb|nil|, либо
исключение. Макрос \spverb|is| считает, сколько раз и с какими аргументами его
вызвали. Позже из этих данных фреймворк строит отчет. Например, покажет все
выражения, которые не прошли проверку.

Мы только написали тест, но не запустили его. Clojure предлагает несколько
способов это сделать.

\subsection{Командная строка}

Самый простой вариант~--- поручить прогон тестов утилите. Если проект основан на
Leiningen\footurl{https://leiningen.org}, запустите команду \spverb|lein test|.
Потребуется время, чтобы считать код из папок \spverb|src| и \spverb|test|,
найти тесты и выполнить их. Вы увидите следующий отчет:

\begin{english}
  \begin{text}
lein test book.util-test
Ran 1 tests containing 2 assertions.
0 failures, 0 errors.
  \end{text}
\end{english}

Clojure группирует тесты по пространствам имен. Под каждым пространством
информация о том, сколько выполнено тестов (объявлений \spverb|deftest|) и
проверок (вызовов \spverb|is|). У нас один тест \spverb|test-fahr| с двумя
\spverb|is|.

Выражение \spverb|lein test book.util-test| напечатано для того, чтобы его можно
было скопировать и запустить в терминале. Тогда сработают тесты только из модуля
\spverb|book.util-test|.

Фреймворк вводит понятия failures и errors. \textbf{Failure} \eng{неудача} это
ложное утверждение. Если сравнение в \spverb|is| вернуло ложь, счетчик
\spverb|failures| увеличится на единицу. Каждая неудача помнит, с какими
параметрами ее запустили. Если была хотя бы одна неудача, получим отчет об
ошибке.

Откройте функцию \spverb|->fahr| и измените коэффициент 1.8 на 1.9. Сохраните
файл и выполните \spverb|lein test|:

\begin{english}
  \begin{text}
$ lein test
lein test :only book.util-test/test-fahr

FAIL in (test-fahr) (util_test.clj:6)
expected: (= 68 (int (->fahr 20)))
  actual: (not (= 68 70))

lein test :only book.util-test/test-fahr

FAIL in (test-fahr) (util_test.clj:7)
expected: (= 212 (int (->fahr 100)))
  actual: (not (= 212 222))

Ran 1 tests containing 2 assertions.
2 failures, 0 errors.
Tests failed.
  \end{text}
\end{english}

Отчет показывает два \spverb|failures|, что верно: макрос \spverb|is|
встречается дважды. Обе проверки неверны, потому что опираются на
\spverb|->fahr| с ошибкой. Для каждого случая показаны исходное выражение и
результат.

Выражение с \spverb|:only| выполнит только указанный тест. Это полезно, когда вы
работаете над участком кода и запускаете тест много раз.

\begin{english}
  \begin{bash}
$ lein test :only book.util-test/test-fahr
  \end{bash}
\end{english}

\textbf{Errors} это не пойманные исключения, которые случились во время работы
теста. В основном ошибки возникают, когда в функцию подают неверные
данные. Представим, мы решили проверить \spverb|->fahr| с \spverb|nil|. Добавим
второй тест:

\begin{english}
  \begin{clojure}
(deftest test-fahr-nil
  (is (nil? (->fahr nil))))
  \end{clojure}
\end{english}

Отчет будет другим:

\begin{english}
  \begin{clojure}
$ lein test
lein test :only book.util-test/test-fahr-nil

ERROR in (test-fahr-nil) (Numbers.java:3849)
expected: (nil? (->fahr nil))
  actual: java.lang.NullPointerException: null
 at clojure.lang.Numbers.multiply (Numbers.java:3849)
    ... ;; truncated

Ran 2 tests containing 3 assertions.
0 failures, 1 errors.
Tests failed.
  \end{clojure}
\end{english}

Вызов \spverb|(->fahr nil)| приводит к \spverb|NullPointerException|. Фреймворк
оборачивает тест в \spverb|try/catch| и запоминает исключение. Для ошибок отчет
выводит их стек-трейс, который мы сократили для экономии места.

Иногда мы \emph{ожидаем}, что функция бросит исключение при особых
параметрах. Чтобы записать это в виде утверждения, используют макрос
\spverb|(is (thrown? ...))|. Мы рассмотрим пример с исключениями позже.

\subsection{REPL}

Код на Clojure работает быстро, но его загрузка занимает от трех до десяти
секунд. Точное время зависит от объема кода и оборудования. Даже если вы
запускаете один тест, проект включается с нуля, что неэффективно.

Когда вы работаете с проектом, вы подключены к REPL. Удобно выполнить тест прямо
в REPL, не запуская новый проект. В этом случае ожидание равно нулю.

Наивный способ выполнить тест~--- запустить его как функцию. Выражение
\spverb|(test-fahr)| вернет \spverb|nil|, что означает, что утверждения
сработали верно. Теперь добавим в тест ошибку:

\begin{english}
  \begin{clojure}
(is (= 999 (int (->fahr 20))))
  \end{clojure}
\end{english}

Если сохранить файл и вызвать \spverb|(test-fahr)|, \emph{по-прежнему} получим
результат без ошибок. Так происходит, потому что мы только сохранили файл, но не
загрузили изменения в память. То, как изменения в файле перетекают в REPL,
зависит от окружения.

Если это Emacs и CIDER, выполните одно из действий. Переместите курсор за
последнюю скобку формы \spverb|(deftest test-fahr...)| и нажмите \spverb|C-c C-e|.
Сработает команда \spverb|cider-eval-last-sexp|, которая выполнит последнее S-выражение.

Другой способ в том, чтобы выполнить на сервере весь буфер. Независимо от того,
где сейчас курсор, нажмите \spverb|C-c C-k| (или \spverb|M-x cider-eval-buffer <RET>|).
Команда равносильна тому, чтобы скопировать буфер, вставить в REPL и нажать ввод.

После обновления вызов \spverb|(test-fahr)| вернет \spverb|nil|, но в консоли
появятся строки:

\begin{english}
  \begin{clojure}
FAIL in (test-fahr) (util_test.clj:14)
expected: 999
  actual: (68)
  \end{clojure}
\end{english}

Если вызвать тест \spverb|(test-fahr-nil)|, который вс\"{е} еще не исправен,
увидим исключение:

\begin{english}
  \begin{clojure}
ERROR in (test-fahr-nil) (Numbers.java:3849)
expected: (nil? (->fahr nil))
  actual: java.lang.NullPointerException: null
 at clojure.lang.Numbers.multiply (Numbers.java:3849)
    book.util$__GT_fahr.invokeStatic (util.clj:5)
    ... ;; truncated
  \end{clojure}
\end{english}

Это вывод в консоль, а не брошенное исключение. Макрос \spverb|deftest|
оборачивает тело в \spverb|try/catch|.

Когда тест запускают как функцию, не работают фикстуры. Это обработчики, которые
готовят среду для теста. О фикстурах мы поговорим позже; пока что отметим, что
функция \spverb|test-vars| выполняет тесты с учетом фикстур. Она принимает
вектор объектов \spverb|Var|:

\begin{english}
  \begin{clojure}
(require '[clojure.test :refer [test-vars]])
(test-vars [#'test-fahr #'test-fahr-nil])
  \end{clojure}
\end{english}

Функция \spverb|run-test| запускает все тесты пространства (текущего или
указанного). Технически она сводится к \spverb|test-vars| со всеми переменными
пространства:

\begin{english}
  \begin{clojure}
(require '[clojure.test :refer [run-tests]])
(run-tests)
  \end{clojure}
\end{english}

Еще одна функция \spverb|run-all-tests| из выполняет тесты для всех загруженных
пространств:

\begin{english}
  \begin{clojure}
(require '[clojure.test :refer [run-all-tests]])
(run-all-tests)
  \end{clojure}
\end{english}

Запустив это выражение, вы увидете, как фреймворк перебирает загруженные модули:

\begin{english}
  \begin{text}
Testing clojure.test
Testing clojure.core.server
Testing clojure.core.specs.alpha
Testing book.util
...
  \end{text}
\end{english}

Чтобы отбросить лишнее, в \spverb|run-all-tests| передают регулярное выражение.
Пример ниже выполнит только тесты из модулей, чьи имена начинаются с book.

\begin{english}
  \begin{clojure}
(run-all-tests #"$book")
  \end{clojure}
\end{english}

\subsection{Emacs и CIDER}

Модуль CIDER\footurl{https://github.com/clojure-emacs/cider} для Emacs упрощает
работу с тестами. CIDER предлагает команды и комбинации клавиш, чтобы выполнить
один или несколько тестов. Он показывает отчет об ошибках в отдельном
буфере. Вызвать тест клавишей удобнее, чем печатать команду в REPL.

Чтобы запустить один тест, поместите на него курсор и нажмите \spverb|C-c C-t t|
(или \spverb|M-x cider-test-run-test|). Курсор может быть в любом месте
\spverb|(deftest ...)|. Если тест прошел без ошибок, внизу появится зеленый
текст, иначе откроется буфер \spverb|*cider-test-report*|.

Буфер выводит отчет в удобной форме. Ошибки в утверждениях подсвечены красным
цветом, а непойманные исключения желтым. Буфер интерактивен: если подвести
курсор к блоку с ошибкой и нажать Enter, соседнее окно покажет исходный код. То
же самое для исключений: чтобы не засорять отчет, CIDER выводит только класс и
сообщение. По Enter откроется отдельный буфер со стек-трейсом.

Перечислим другие полезные команды:

\begin{itemize}

\item
  \spverb|cider-test-rerun-failed-tests| (\spverb|C-c C-t f|) выполнит только
  неудачные тесты из прошлого запуска. Полезно, чтобы не ждать весь прогон
  снова.

\item
  \spverb|cider-test-run-ns-tests| (\spverb|C-c C-t n|) выполнит тесты текущего
  пространства. Чтобы сопоставить пространство и тесты, CIDER добавляет к имени
  \spverb|-test|. Например, для \spverb|book.util| тестовый модуль получится
  \spverb|book.util-test|.

\end{itemize}

Запуск тестов это рутина. Чтобы не терять время, изучите, как это сделать из
редактора или IDE.

\section{Полезные практики}

Тесты это тоже код, и за ним нужен уход. Небрежность в тестах несет столько же
проблем, сколько и в приложении. Приемы ниже сделают тесты удобней в чтении и
поддержке.

\subsection{Testing}

Макрос \spverb|testing| оборачивает тело строкой. Это сообщение о том, что мы
собираемся делать:

\begin{english}
  \begin{clojure}
(deftest test-square-roots
  (testing "Two roots"
    (let [result (square-roots 1 -5 6)]
      (is (= [3 2] (mapv int result)))))
  (testing "One root"
    (let [result (square-roots 1 6 9)]
      (is (= [-3] (mapv int result)))))
  (testing "No roots"
    (is (nil? (square-roots 2 4 7)))))
  \end{clojure}
\end{english}

Сообщение важно по двум причинам: визуальной и технической. Оно разбивает код на
логические блоки, заменяет комментарии, делает тест понятнее. Если \spverb|is|
не сработал, в отчете будет сообщение, в которое он об\"{е}рнут.

Макросы \spverb|testing| могут быть вложенными. Каждое сообщение наращивает
последнее через пробел. Тест ниже проверяет, что API \spverb|createUser| вносит
пользователя в базу. Дополнительно мы убеждаемся, что пароль в базе
зашифрован. Если отбросить основной код, получим скелет из форм
\spverb|testing|:

\begin{english}
  \begin{clojure}
(deftest test-some-api
  (testing "API call" ...
    (testing "HTTP response" ...))
  (testing "DB checks:" ...
    (testing "common user fields" ...)
    (testing "password is hashed" ...)))
  \end{clojure}
\end{english}

\subsection{Исключения}

Мы вс\"{е} еще не разобрались, как тестировать код с исключениями. Хотелось бы
убедиться, что вызов \spverb|(->fahr nil)| бросит ошибку, а не вернет
\spverb|nil|. Если проверки нет, код не защищен от таких изменений:

\begin{english}
  \begin{clojure}
(defn ->fahr [cel]
  (when cel
    (+ (* cel 1.8) 32)))
  \end{clojure}
\end{english}

В этом случае \spverb|nil| провалится ниже по стеку вызовов, что усложнит поиск
причины.

Об\"{е}ртка \spverb|try/catch| внутри теста решает проблему лишь отчасти. В
таком виде \spverb|(is true)| действительно сработает. Если заменить
\spverb|(->fahr nil)| на \spverb|(->fahr 1)|, исключения не будет, и тело
\spverb|(catch...)| не получит управление. Если исключение ожидают, но его нет,
это \emph{тоже ошибка}.

\begin{english}
  \begin{clojure}
(deftest test-fahr-nil
  (try
    (->fahr nil)
    (catch NullPointerException e
      (is true))))
  \end{clojure}
\end{english}

Макрос \spverb|(is (thrown?...))| решает эту проблему. Форма \spverb|thrown?|
принимает класс исключения и произвольное тело. Если брошено исключение этого
класса (с учетом наследования), внешний \spverb|is| сработает
положительно.

\begin{english}
  \begin{clojure}
(deftest test-fahr-nil
  (is (thrown? NullPointerException
               (->fahr nil))))
  \end{clojure}
\end{english}

Когда исключения не было, получим ошибку в утверждении. Исправьте \spverb|nil|
на число и убедитесь, что тест не проходит.

Иногда проверки на класс не хватает. В больших участках кода \spverb|NPE|
возникает на разных уровнях. Предположим, кто-то изменил \spverb|->fahr| так,
что для \spverb|nil| получим \spverb|nil|. Этим мы расширим область ошибки: ниже
\spverb|NPE| возникает не в \spverb|->fahr|, а в функции сложения. Это приводит
к ложному тестированию: на самом деле происходит не то, что ожидают.

\begin{english}
  \begin{clojure}
(let [temp-cel nil]
  (+ 1 (->fahr temp-cel)))
  \end{clojure}
\end{english}

Проблему решают двумя шагами. Первый~--- исправить функцию так, чтобы она
бросила что-то более осмысленное, чем \spverb|NPE|. Подойдет специальное
исключение \spverb|IllegalArgumentException|. Его дополняют сообщением о том,
что именно не так:

\begin{english}
  \begin{clojure}
(defn ->fahr [cel]
  (if (number? cel)
    (+ (* cel 1.8) 32)
    (throw (new IllegalArgumentException
            "Fahrenheit temperature should be a real number"))))
  \end{clojure}
\end{english}

Второй шаг~--- убедиться, что исключение пришло именно из \spverb|->fahr|. Форма
\spverb|(is (thrown-with-msg?...))| проверяет, что текст исключения совпадает с
регулярным выражением. Тест ниже покрывает эти требования:

\begin{english}
  \begin{clojure}
(deftest test-fahr-nil
  (is (thrown-with-msg?
       IllegalArgumentException #"Fahrenheit temperature"
       (->fahr nil))))
  \end{clojure}
\end{english}

\subsection{Пакетная проверка}

Вспомним, как выглядит \spverb|test-fahr|:

\begin{english}
  \begin{clojure}
(deftest test-fahr
  (is (= 68 (int (->fahr 20))))
  (is (= 212 (int (->fahr 100)))))
  \end{clojure}
\end{english}

Оба \spverb|is| отличается только числами. Чтобы добавить проверку, придется
копировать одну из форм. Копирование раздувает код и в целом выходит
боком. Очевидно, из похожих \spverb|(is...)| можно выделить постоянную часть и
переписать тест.

Макрос \spverb|are| (анг. множественная форма <<is>>) выполняет несколько
\spverb|is| по шаблону. Он принимает форму связывания, шаблон и аргументы. Число
аргументов должно быть кратно переменным из формы связывания. На каждом шаге
макрос связывает часть аргументов с переменными и выполняет шаблон:

\begin{english}
  \begin{clojure}
(deftest test-fahr
  (are [c f] (= c (int (->fahr f)))
    68 20
    212 100))
  \end{clojure}
\end{english}

Аргументы записывают столбиком по принципу одна строка~--- один
\spverb|is|. Выше \spverb|f| и \spverb|c| означают цифры по Фаренгейту и
Цельсию, то есть исходное и ожидаемое значения. Выражение \spverb|(int (->fahr f))|
переехало в шаблон, поэтому нет смысла копировать его.

Макрос \spverb|are| удобен для небольших выражений. Со временем он становится
трудным в поддержке. В этом случае тест переписывают на \spverb|doseq|, который
пробегает по данным.

Предположим, мы тестируем API для создания пользователя. Один из тестов
проверяет входные параметры. Мы ожидаем негативный ответ для случаев, когда нет
обязательного поля или оно в неверном формате. В переменную \spverb|params-ok|
запишем правильные параметры:

\begin{english}
  \begin{clojure}
(def params-ok {:name "John Smith" :email "john@test.com"})
  \end{clojure}
\end{english}

Для негативных случаев объявим вектор \spverb|params-variations|. Он состоит из
пар \tuple{словарь, текст}. Словарь это изменения, которые добавим к
\spverb|params-ok| функцией \spverb|merge|. Текст описывает проблему: неверная
почта, слишком длинное имя и так далее.

\begin{english}
  \begin{clojure}
(def params-variations
  [[{:name nil} "Empty name"]
   [{:name (apply str (repeat 999 "A"))} "Name is too long"]
   [{:email "dunno"} "Wrong email"]
   [{:email nil} "No email"]
   [{:extra 42} "Extra field"]])
  \end{clojure}
\end{english}

Тест пробегает по \spverb|params-variations|. На каждом шаге он вызывает
\spverb|api-create-user| с испорченными параметрами. Обратите внимание, что
вызов обернут в \spverb|testing| с сообщением. Если на одном из шагов не
сработает \spverb|is|, мы узнаем причину из отчета.

\begin{english}
  \begin{clojure}
(deftest test-api-create-user-bad-params
  (testing "Sending bad parameters"
    (doseq [[params* description] params-variations]
      (testing description
        (let [params (merge params-ok params*)
              response (api-create-user params)
              {:keys [status]} response]
          (is (= 400 status)))))))
  \end{clojure}
\end{english}

Тест разделяет данные и проверку, поэтому его легко поддерживать. Если в API
появятся новые поля, расширим \spverb|params-variations|, не меняя тест.

\subsection{Именование}

Начинайте имя теста с префикса \spverb|test-|. С точки зрения Clojure это
необязательно: фреймворк ищет тесты по метаданным, а не имени. Выражения ниже
останутся функцией и тестом даже если их назвали наоборот.

\begin{english}
  \begin{clojure}
(defn test-if-suspended [user]) ;; function
(deftest is-suspended)          ;; test
  \end{clojure}
\end{english}

Префикс необходим для навигации по коду. В модуле бывает больше тестов, чем
умещается на экран. Вперемешку с тестами идут переменные и служебные функции. С
ростом модуля станет трудно понять, где функция, а где тест. С префиксом ошибки
быть не может.

Emacs предлагает навигацию по коду командой \spverb|imenu| и ее улучшенной
версией \spverb|helm-imenu|. При их вызове открывается буфер со списком
определений. С вводом текста останутся только те элементы, которые его
содержат. Называйте определения так, чтобы по команде \spverb|M-x helm-imenu <RET> test|
в списке остались только тесты.

В общем случае тест называют по правилу \spverb|test-<what>-<case>|, где
\spverb|<what>| означает что мы тестируем, а \spverb|<case>|~---
сценарий. Например, \spverb|test-create-user-ok| означает, что создают
пользователя и ожидают успех. Имя \spverb|test-create-user-bad-params| говорит о
попытке создать пользователя с неверными параметрами. От
\spverb|test-user-login-signature-expired| ожидают, что пользователь не смог
войти в систему, потому что подпись устарела.

\subsection{Порядок аргументов}

Необычный вопрос: как писать правильно, \spverb|(is (= 200 status))| или
\spverb|(is (= status 200))|? На первый взгляд это абсурд: неужели равенство
зависит от порядка? Значения либо равны, либо нет. Но макрос \spverb|is|
сложнее, чем мы думаем: он разбивает форму \spverb|(= 200 status)| на ожидаемую
и фактическую части. По-английски они называются <<expected>> и <<actual>>.

Ожидаемое значение, которое посчитали заранее. Фактическое~--- то, на которое
тест вышел самостоятельно, вызвав функцию. Число 68 это ожидаемое, а
\spverb|(int (->fahr 20))|~--- действительное. Статус 200 это ожидаемое, а
\spverb|(:status response)|~--- действительное.

Разделение нужно для отчетов. Когда значения не равны, хотелось бы видеть, где
мы ошиблись. Предположим, в отчете написано: \spverb|failed (= 200 403)|. Как
это понимать? Ожидали успешный ответ, но не хватило прав? Или это брешь в
безопасности~--- ожидали, что доступа к странице нет, но пользователь ее увидел?
Если же написано <<expected 200, got 403>>, то ясно~--- это первый случай, когда
не хватило прав.

Правило: ожидаемое стоит на первом месте, а действительное на втором. Пишите
\spverb|(is (= 200 status))| вместо \spverb|(is (= status 200))|. Это непривычно
и противоречит здравому смыслу. Как правило, фактическое это скаляр, а
действительное~--- выражение, поэтому хочется записать их как слева. Увы,
прид\"{е}тся побороть себя и писать по образцу справа:

\noindent
\begin{tabular}{ @{}p{5cm} @{}p{5cm} }

\begin{english}
  \begin{clojure}
;; wrong
(= (int (->fahr 20)) 68)
  \end{clojure}
\end{english}

&

\begin{english}
  \begin{clojure}
;; correct
(= 68 (int (->fahr 20)))
  \end{clojure}
\end{english}

\end{tabular}

Правило уходит корнями в прошлое. Фреймворк JUnit утвердил именно такой порядок
в методах сравнения\footurl{https://junit.org/junit5/docs/current/user-guide/}.
Хорошо это или плохо, судить поздно~--- принцип <<expected слева>> стал
стандартом. Аналогично работают тесты в Python, Ruby и других языках. Отдельные
фреймворки предлагают модули, чтобы <<перестать говорить как Йодо>>, то есть
поменять семантику аргументов. Это возможно и в Clojure, но сейчас мы не будем
углубляться в детали.

Особенность expected и actual видна при запуске тестов в CIDER. Один и тот же
тест проверяет статус ответа на 200. Пока вс\"{е} хорошо, нет разницы, в каком
порядке идут аргументы \spverb|is|. В случае ошибки вариант слева вносит
путаницу. Согласно ему, нормальным считается статус 404 (expected), а не 200
(actual). Вариант справа выводит статусы правильно.

\noindent
\begin{tabular}{ @{}p{5cm} @{}p{5cm} }

\begin{english}
  \begin{clojure}
;; wrong
(is (= status 200))

Fail in test-...
expected: 404
  actual: 200
    diff: - 404
          + 200
  \end{clojure}
\end{english}

&

\begin{english}
  \begin{clojure}
;; correct
(is (= 200 status))

Fail in test-...
expected: 200
  actual: 404
    diff: - 200
          + 404
  \end{clojure}
\end{english}

\end{tabular}

\section{Фикстуры}

До сих пор мы работали с простыми тестами. Они запускаются в любой момент,
потому что не зависят от ресурсов. Так бывает не всегда: чем сложнее код, тем
больше требований к базе данных, файлам и сети. Все вместе это называется
окружением теста.

Рассмотрим, как сообщить тесту его окружение. Для этого служат фикстуры. Термин
означает стенд, на котором испытывают изделие. В программировании фикстура это
объект или функция, которая запускается до или после теста. Фикстура готовит
окружение и очищает его. Перечислим случаи, когда полезны фикстуры.

\textbf{Файлы на диске.} Код, который мы тестируем, ищет в заданной папке файл с
данными. Фикстура пишет в нее файл и удаляет после теста. Подход гарантирует,
что на диске не останется следов после работы.

\textbf{Данные в базе.} Чтобы проверить авторизацию, в базу вносят пользователя
с известным паролем. Пригодятся и другие данные: профиль, друзья, история
заказов. Фикстура пишет данные в базу и после очищает таблицы, чтобы не повлиять
на следующий тест.

\textbf{Запуск и остановка системы.} Иногда фикстура управляет глобальным
состоянием программы. Например, для интеграционного теста нужны все компоненты
(сервер, база, почта). Запуск и остановку системы логично поместить в фикстуру.

По ходу главы мы рассмотрим и другие сценарии для фикстур. Пока что разберемся,
как они устроены технически.

Фикстура это функция, которая принимает тест. Функция выполняет любые действия,
в том числе вызывает тест. Вопрос в какой момент это делать остается на ваше
усмотрение.

\begin{english}
  \begin{clojure}
(def db {:dbtype "postgres" :dbname "book"
         ;; other JDBC fields
         })

(defn fix-db-data [t]
  ;; purge tables
  (jdbc/execute! db "truncate users cascade;")
  (jdbc/execute! db "truncate orders cascade;")
  ;; add data
  (jdbc/insert! db :users user-data)
  (jdbc/insert! db :profile profile-data)
  ;; execute the test
  (t))
  \end{clojure}
\end{english}

В примере мы сначала готовим окружение, а затем вызываем тест. Бывает и
наоборот: тест нуждается в финальных действиях. Предположим, мы пишем
промежуточные файлы на диск. Чтобы очистить папку при выходе из теста, напишем
фикстуру:

\begin{english}
  \begin{clojure}
(import 'java.io.File
        'org.apache.commons.io.FileUtils)

(defn clean-dir [path]
  (FileUtils/cleanDirectory (new File path)))

(defn fix-clear-files [t]
  (t)
  (clean-dir "/tmp/tests/data"))
  \end{clojure}
\end{english}

Иногда фикстура сообщает и предварительные, и финальные шаги. В этом случае
\spverb|(t)| находится где-то посередине.

Надеемся, читатель заметил, что имена фикстур начинаются с префикса
\spverb|fix-|. Для фикстур действует то же правило именования, что и для
тестов. Это особые сущности, поэтому их выделяют префиксом.

%% ---------------

\subsection{Регистрация}

Если запустить тест сейчас, фикстура не сработает. Мы только объявили фикстуру,
но не сказали фреймворку взять ее в работу. Это делают регистрацией фикстуры.
Резонный вопрос: если фреймворк сам находит тесты, почему бы ему не искать
фикстуры и применять их?

Причина в том, что фикстуры работают на разных этапах теста. Clojure выделяет
стадии \spverb|:each| и \spverb|:once|. Зарегистрировать фикстуру значит связать
ее с одним из этих ключей. Значение \spverb|:each| означает, что фикстура
работает для каждого теста из модуля. С ключом \spverb|:once| фикстура
выполняется однажды в рамках модуля.

Одна и та же фикстура может быть \spverb|:each| или \spverb|:once|. Вспомним
фикстуру \spverb|fix-clear-files|, которая очищает папку. Если у файлов
случайные имена, логично очистить папку один раз в конце, и фикстуре указывают
\spverb|:once|. Если имена одинаковы, возрастает риск коллизии (чтение и запись
из другого теста). Тогда фикстуру связывают с ключом \spverb|:each|.

Другой пример~--- фикстура базы данных. На время теста она пишет данные в
таблицы и очищает их. Если тесты только читают данные, укажем
\spverb|:once|. Получим выигрыш в скорости: для тридцати тестов фикстура
сработает один раз. Для тестов на запись важно, чтобы ни один из них не влиял на
другой, поэтому фикстуру добавляют с \spverb|:each|.

Функция \spverb|use-fixtures| из \spverb|clojure.test| принимает ключ и любое
число фикстур:

\begin{english}
  \begin{clojure}
(use-fixtures :once fix-db-server fix-clear-files)
(use-fixtures :each fix-db-data)
  \end{clojure}
\end{english}

Выражения ставят в верхней части файла до тестов. Они могут быть и в конце, но
чем раньше мы увидим фикстуры, тем лучше.

\subsection{Пример}

Чтобы исследовать порядок фикстур, добавим побочный эффект. Функция
\spverb|fix-factory| принимает тип фикстуры, ее номер и возвращает
фикстуру. Внутри она обрамляет тест выводом в консоль.

\begin{english}
  \begin{clojure}
(defn fix-factory [type number]
  (fn [t]
    (println (format "%s %s starts" type number))
    (t)
    (println (format "%s %s ends" type number))))
  \end{clojure}
\end{english}

\noindent
Добавим по две фикстуры каждого типа:

\begin{english}
  \begin{clojure}
(use-fixtures :once
  (fix-factory :once 1)
  (fix-factory :once 2))

(use-fixtures :each
  (fix-factory :each 3)
  (fix-factory :each 4))
  \end{clojure}
\end{english}

Если запустить тесты, увидим следующее. Фикстуры 1 и 2 с типом \spverb|:once|
сработали однажды на границах. Обратите внимание, что в конце их порядок
противоположный: сначала завершается фикстура 2 и только потом 1. Фикстуры 3 и 4
оборачивают каждый тест. Чтобы выразить процесс нагляднее, добавим отступы. С
ними видна иерархия фикстур.

\begin{english}
  \begin{clojure}
:once 1 starts
  :once 2 starts
    :each 3 starts
      :each 4 starts
      :each 4 ends
    :each 3 ends
    :each 3 starts
      :each 4 starts
      :each 4 ends
    :each 3 ends
  :once 2 ends
:once 1 ends
  \end{clojure}
\end{english}

\subsection{Хранение фикстур}

Пытливому читателю интересно узнать, где лежит информация о фикстурах. Очевидно,
вызов \spverb|use-fixtures| меняет глобальное состояние. Это не атом и не
переменная, а нечто другое, с чем мы еще не работали~--- метаданные
пространства. После регистрации фикстур проверьте его ключи:

\begin{english}
  \begin{clojure}
(keys (meta *ns*))
(:clojure.test/each-fixtures :clojure.test/once-fixtures)
  \end{clojure}
\end{english}

В каждом из них список функций, которые передали в \spverb|use-fixtures| с
\spverb|:each| или \spverb|:once|. Вызов \spverb|use-fixtures| заменяет в
метаданных все фикстуры этого типа. Важно, что это полная замена, а не запись в
конец. Чтобы отказаться от всех фикстур \spverb|:each|, выполните:

\begin{english}
  \begin{clojure}
(use-fixtures :each nil)
  \end{clojure}
\end{english}

Другой способ очистить фикстуры~--- удалить форму \spverb|use-fixtures| в коде и
выполнить буфер в REPL. Тогда объявление \spverb|(ns...)| сработает еще раз с
новыми метаданными.

В других языках и фреймворках выделяют еще одну стадию фикстур~---
\spverb|:session|. Такая фикстура запускается в разрезе \emph{всего прогона}
тестов. Можно сказать, это фикстура-синглтон: даже для нескольких модулей она
сработает один раз. В \spverb|clojure.test| нет такой стадии. С ней пространства
были бы связаны друг с другом, что нарушает идею фреймворка. Начинающим не
хватает session-фикстур, но мы научимся обходиться без них.

\subsection{Связанные переменные}

Фикстуры особенно удобны в паре с динамическими переменными. Напомним,
динамические переменные это формы \spverb|def| с тегом \spverb|^:dynamic| и
\spverb|*|ушками\spverb|*|. Макрос \spverb|binding| выполняет код с новым
значением переменной. Мы рассмотрели их в главе про
изменяемость \page{dynamic-vars}.

Форма \spverb|deftest| порождает функцию без аргументов. Нельзя сообщить тесту
его параметры как обычной функции. А ведь тесту нужны файлы, база данных и
многое другое. Окружение передают в два шага. Первый~--- тест ссылается на
глобальную переменную. Второй~--- пишут фикстуру, которая оборачивает тест в
\spverb|binding| с этой переменной и нужным значением.

Предположим, мы тестируем функцию, которая принимает путь к файлу и пишет в него
картинку. Объявим фикстуру, которая на время теста свяжет переменную
\spverb|*tmp-file*| со временным файлом. На время теста \spverb|*tmp-file*|
станет экземпляром \spverb|java.util.File|. При выходе из теста удалим файл, а
переменная верн\"{е}тся в \spverb|nil|.

\begin{english}
  \begin{clojure}
(import 'java.io.File)

(defonce ^:dynamic *tmp-file* nil)

(defn fix-with-tmp-file [t]
  (let [^File tmp-file (File/createTempFile "test" "chart")]
    (binding [*tmp-file* tmp-file]
      (t))
    (.delete tmp-file)))

(use-fixture :each fix-with-tmp-file)
  \end{clojure}
\end{english}

Частичка \spverb|with| в имени фикстуры намекает на контекст. Добавляйте е\"{е},
когда фикстура обращается к \spverb|binding|, \spverb|with-redefs| или аналогам.

Внутри теста мы вызываем \spverb|plot-chart| с текущим временным
файлом. Убедимся, что функция действительно записала в него картинку. Класс
\spverb|PngImage| предлагает простейший парсер PNG. Прочитаем разрешение файла и
сверим с образцом.

\begin{english}
  \begin{clojure}
(import 'javax.imageio.ImageIO)

(deftest test-plot-chart-png
  (let [dataset [[...] [...] [...]]
        filepath (.getAbsolutePath *tmp-file*)]
    (plot-chart dataset filepath)
    (let [png (ImageIO/read *tmp-file*)
          width (.getWidth png)
          height (.getHeight png)])
    (is (= [640 480] [width height]))))
  \end{clojure}
\end{english}

Для полноты картины рассмотрим базу данных. Пусть это будет Cassandra, а не
JDBC-драйвер. Фикстура \spverb|fix-with-db| связывает \spverb|*db*| с текущей
сессией. Установка сессии это дорогая операция, поэтому вынесем ее в разовую
фикстуру.

\begin{english}
  \begin{clojure}
(require '[qbits.alia :as alia])

(defonce ^:dynamic *db* nil)

(defn fix-with-db [t]
  (let [cluster (alia/cluster {:contact-points ["localhost"]})
        session (alia/connect cluster)]
    (binding [*db* session]
      (t))
    (.close session)
    (.close cluster)))

(use-fixture :once fix-with-db)
  \end{clojure}
\end{english}

Вторая фикстура готовит данные для каждого теста. Фикстуры \spverb|:once|
запускаются раньше, чем \spverb|:each|. Подготовка данных сработает внутри
\spverb|(binding [*db* ...])| из первой фикстуры. Это значит, внутри второй мы
свободно обращаемся к \spverb|*db*| как к сессии:

\begin{english}
  \begin{clojure}
(defn fix-db-prepare-data [t]
  (alia/execute! *db* "truncate project.users;")
  (alia/execute! *db* "truncate project.orders;")
  (alia/execute! *db* "insert into project.users...")
  (t))

(use-fixture :each fix-db-prepare-data)
  \end{clojure}
\end{english}

\section{Мульти-фикстуры}

Любопытный вопрос: что произойдет, если фикстура вызовет тест несколько раз?
Напишем такую фикстуру:

\begin{english}
  \begin{clojure}
(defn fix-multi [t]
  (t) (t) (t))

(use-fixtures :each fix-multi)
  \end{clojure}
\end{english}

\noindent
и запустим один тест. С точки зрения фреймворка прошло три теста, о чем скажет
отчет:

\begin{english}
  \begin{text}
book.util-test: Ran 6 assertions, in 3 test functions.
0 failures, 0 errors.
  \end{text}
\end{english}

На первый взгляд в этом нет смысла. Множественный прогон не несет пользы, а
только потребляет ресурсы и время. Но с динамическими переменными такая фикстура
неожиданно полезна. Один и тот же тест можно выполнить в разных окружениях.

Предположим, программа работает с базой данных. В требованиях сказано, что это
может быть PostgreSQL или MySQL. Убедимся, что программа поддерживает обе
СУБД. Чтобы не дублировать тесты под каждый тип базы, логику повтора выносят в
фикстуру. Объявим подключения к базам данных:

\begin{english}
  \begin{clojure}
(def db-pg {:dbtype "postgresql" :host "..."})
(def db-mysql {:dbtype "mysql" :host "..."})
  \end{clojure}
\end{english}

Добавим динамическую переменную \spverb|*db*|. Новая фикстура перебирает
подключения и связывает их с \spverb|*db*| на каждом шаге. Когда переменная
связана, запускается тест. Он зависит от текущего подключения и поэтому
обращается либо в PostgreSQL, либо в MySQL.

\begin{english}
  \begin{clojure}
(defonce ^:dynamic *db* nil)

(defn fix-with-multi-db-backend [t]
  (doseq [db [db-pg db-mysql]]
    (binding [*db* db]
      (testing (format "Testing with DB %s" (:dbtype *db*))
        (t)))))
  \end{clojure}
\end{english}

Вызов \spverb|(t)| обернут в форму \spverb|testing|, в которой сообщаем тип
базы. Если произошла ошибка, мы узнаем, с какой базой работали в тот
момент. Тест на чтение пользователя:

\begin{english}
  \begin{clojure}
(defn test-get-user-by-id
  (let [user (project.orm/get-user *db* 1)]
    (is (= {:name "Ivan"} user))))
  \end{clojure}
\end{english}

Перечислим другие сценарии для мульти-фикстур. Это можут быть интеграционные
тесты в нескольких браузерах (Chrome, Firefox). Иногда нужно проверить, что
логика не зависит от формата передачи данных (JSON, Msgpack). Если программа
работает с изображениями, тесты прогоняют на разных типах файлов (PNG, JPEG) и
разрешениях.

\subsection{Подавление ошибок}

У этого раздела нет практической пользы, и мы приводим его ради шутки и полноты
картины. Раз уж мы рассмотрели множественный вызов теста в фикстуре, учт\"{е}м и
обратный случай. Что случится, если фикстура не вызывает тест?

\begin{english}
  \begin{clojure}
(defn fix-mute [t])
(use-fixtures :each fix-mute)
  \end{clojure}
\end{english}

Это приведет к тому, что ни один тест не сработает. Фикстура не отменяет тесты:
фреймворк по-прежнему находит и выполняет их. В отч\"{е}те увидим список тестов,
но для каждого будет указано <<0~assertions>>. Если запустить тест из Emacs,
получим предупреждение:

\begin{english}
  \begin{clojure}
No assertions (or no tests) were run.
Did you forget to use ‘is’ in your tests?
  \end{clojure}
\end{english}

CIDER считает подозрительным, что не было ни одного \spverb|is|. Чтобы обмануть
CIDER, улучшим фикстуру. Добавим в нее \spverb|is|, который всегда истинен. Даже
если тест с ошибками, его тело не выполнится, а в отч\"{е}те будет один
assertion. Назовем фикстуру \spverb|fix-this-is-fine|%
\footurl{https://knowyourmeme.com/memes/this-is-fine}:

\begin{english}
  \begin{clojure}
(defn fix-this-is-fine [t]
  (is true))

(use-fixtures :each fix-this-is-fine)
  \end{clojure}
\end{english}

\noindent
С ее помощью мы пройдем тест Оруэлла:

\begin{english}
  \begin{clojure}
(deftest test-1984
  (testing "Sometimes they are five."
    (is (= 5 (* 2 2)) )))
;; OK, ran 1 assertions
  \end{clojure}
\end{english}

По аналогии напишите фикстуру, которая портит любой тест. Добейтесь, чтобы все
тесты завершились с ошибкой.

\subsection{Фикстуры с условиями}

Фикстура не только вызывает тест, но и несет дополнительную логику. Тест может
быть обернут в цикл, условие или другую форму. Разбер\"{е}мся с условием. В
каких случаях тест ставят под \spverb|if| и какой в этом смысл?

К условной фикстуре прибегают, когда мы не знаем заранее, сможем ли выполнить
тест. Например, часть программы работает только с базой данных
PostgreSQL. Фикстура проверяет тип подключения и не вызывает тест, если он
заведомо не работает.

\begin{english}
  \begin{clojure}
(defn fix-pg-only [t]
  (when (= (:dbtype *db*) "postgresql")
    (t)))
  \end{clojure}
\end{english}

Фикстура молча пропускает тесты, которые не поддерживает. Если нужен сигнал об
ошибке, добавьте ложное утверждение. Обратите внимание на второй параметр
\spverb|is|. Это сообщение, которое появится в отч\"{е}те для этого утверждения.

\begin{english}
  \begin{clojure}
(defn fix-pg-only [t]
  (let [{:keys [dbtype]} *db*]
    (if (= dbtype "postgresql")
      (t)
      (is false (str "Unsupported DB " dbtype)))))
  \end{clojure}
\end{english}

Каким способом зарегистрировать фикстуру~--- \spverb|:once| или \spverb|:each|?
Это зависит от семантики условия. В нашем примере база не меняется от теста к
тесту, поэтому условие может быть выполнено однажды. Если фикстура проверяет
файл на диске, проверку делают для каждого теста.

Рассмотрим другие сценарии, когда фикстурам нужны условия. Это версия
операционной системы: если она ниже порога, мы не можем выполнить тест. Иногда
приложение зависит от системной утилиты, которую вызывает в отдельном
процессе. Фикстура проверяет, что утилита установлена. Когда тест работает с
большими файлами, может не хватить места на сервере CI. Перед запуском фикстура
убеждается, что места хватит.

\section{Метки и селекторы}

Фикстуры с условиями похожи на фильтры. С их помощью запускают не все тесты, а
часть, которая подходит окружению. Избегайте ситуаций, когда в фикстурах слишком
много условий. Clojure предлагает более удобный способ выборочного
прогона. Это метки и селекторы тестов.

Метки работают проще, чем фикстуры с условием. Это метаданные, которые сообщают
тесту при его создании. Тесту ниже назначили флаг \spverb|:special| со значением
\spverb|true|.

\begin{english}
  \begin{clojure}
(deftest ^:special test-special-case
  (is true))
  \end{clojure}
\end{english}

Напомним, выражение \spverb|^:flag| это сокращенная форма \spverb|^{:flag true}|.
<<Крышка>> полезна, чтобы задать несколько флагов за один раз:

\begin{english}
  \begin{clojure}
(deftest ^:special ^:backend ^:no-db
  test-special-case
  (let [...]
    (is ...)))
  \end{clojure}
\end{english}

Когда тегов два и больше, имя теста переносят на следующую строку, чтобы
визуально отделить их.

По тегам легко догадаться, что это тест на серверную логику (backend); ему не
нужна база данных (no-db); это особый тест (special). Смысл тегов зависит от
соглашений в команде. Чтобы проверить флаги, прочитаем метаданные переменной
\spverb|test-special-case|. Среди прочих полей вы найдете \spverb|:special| и
остальные:

\begin{english}
  \begin{clojure}
(meta #'test-special-case)
{:special true :backend true :no-db true ...}
  \end{clojure}
\end{english}

По тегам можно выбрать часть тестов: только особые (special) или на серверную
логику (backend). Выборка называется селектором тестов. Селекторы полезны по
нескольким причинам.

Часть тестов отработает быстрее, чем полный набор. Если мы правим участок кода
и точно знаем, какие тесты он задевает, нет смысла запускать вс\"{е}
подряд. Назначим тег и будем работать с подмножеством.

Иногда окружение не настроено должным образом, и выполнить все тесты
невозможно. Например, интеграционный тест требует браузер и драйвер к нему,
очередь задач и почтовый сервер. Тест выносят в отдельный шаг сборочной линии. В
общих чертах схема выглядит так:

\begin{itemize}

\item
  запустить минимальное окружение (база данных);

\item
  выполнить модульные тесты;

\item
  если не было ошибок, поднять дополнительное окружение (очередь задач,
  заглушки);

\item
  выполнить интеграционные тесты.

\end{itemize}

\noindent
Чем сложнее проект, тем больше в нем тестов и шагов по их запуску.

В тестовом фреймворке Clojure нет селекторов. Он запускает один тест,
пространство или вс\"{е} целиком. Селекторы идут в сторонних библиотеках и
утилитах. Рассмотрим, как задать их в Leiningen.

Откройте файл \spverb|project.clj|. Внутри \spverb|defproject| добавьте ключ
\spverb|:test-selectors|. Это словарь, где ключ это метка селектора, а
значение~--- функция одного аргумента. В функцию приходят метаданные теста. Если
функция вернула ложь или \spverb|nil|, тест пропускают.

Поскольку кейворд в Clojure это функция, селектором может быть сам тег. Пример
ниже читается как <<набор тестов \spverb|:special|, у которых тег \spverb|:special|>>:

\begin{english}
  \begin{clojure}
:test-selectors {:special :special
                 :backend :backend}
  \end{clojure}
\end{english}

\noindent
Чтобы запустить тесты по селектору, передайте метку в команду \spverb|test|:

\begin{english}
  \begin{bash}
$ lein test :special
  \end{bash}
\end{english}

Если метка не задана, фреймворк назначит ей \spverb|:default|. Под этот селектор
подходит любой тест. Хитрость в том, что его поведение можно изменить. Ниже для
\spverb|:default| задали селектор, который вернет все \textbf{не}~интеграционные
тесты. В результате \spverb|lein test| запустит только модульные тесты. Чтобы
прогнать весь проект, мы завели метку \spverb|:all|, которая всегда верн\"{е}т
истину.

\begin{english}
  \begin{clojure}
:test-selectors {:default (complement :integration)
                 :integration :integration
                 :all (constantly true)}
  \end{clojure}
\end{english}


Селектор может быть полноценной функцией. Предположим, часть программы
полагается на новые функции базы данных. Для каждого теста укажем минимальную
версию базы, поле \spverb|:db/version| с числом. Тест считают экспериментальным,
если версия базы выше 11:

\begin{english}
  \begin{clojure}
(deftest ^{:db/version 11}
  test-db-experimental-feature
  (is true))
  \end{clojure}
\end{english}

\noindent
Ниже селектор, который вернет подобные тесты. Запустите их командой
\spverb|lein test :db-experimental|:

\begin{english}
  \begin{clojure}
:test-selectors
{:db-experimental
 (fn [test-meta]
   (some-> test-meta :db/version (>= 11)))}
  \end{clojure}
\end{english}

Когда тестов много, расставлять теги утомительно. Повышается риск забыть тег, и
тест выпадет из набора. Если все тесты модуля связаны по смыслу, тег назначают
не тесту, а пространству. С точки зрения \spverb|lein| тест наследует теги
пространства, в котором объявлен:

\begin{english}
  \begin{clojure}
(ns ^:integration
  book.integration-test
  (:require [clojure.test :refer :all]))

(deftest test-user-login-ok
  ...)
  \end{clojure}
\end{english}

Не нужно указывать \spverb|^:integration| каждому тесту, поскольку мы сделали
это на уровне модуля. Селектор интеграционных тестов не изменится: его метка и
функция равны \spverb|:integration|. Тесты запускаются как обычно:

\begin{english}
  \begin{bash}
$ lein test :integration
  \end{bash}
\end{english}

\subsection{Тесты в deps.edn}

Не все проекты на Clojure используют Lein; некоторые перешли на
Deps\footurl{https://clojure.org/guides/deps\_and\_cli}, утилиту для управления
зависимостями. С версии 1.9 Deps идет в поставке Clojure.

Если проект работает на Deps, мы теряем возможности Lein, в том числе прогон
тестов. Чтобы компенсировать потери, в Deps подключают сторонние
библиотеки. Проект Test-runner\footurl{https://github.com/cognitect-labs/test-runner}
делает то же, что и \spverb|lein test|: находит тесты, запускает их и выводит
отчет.

Добавьте в файл \spverb|deps.edn| запись:

\begin{english}
  \begin{clojure}
:aliases
{:test
 {:extra-paths ["test"]
  :extra-deps
  {com.cognitect/test-runner
   {:git/url "https://github.com/cognitect-labs/test-runner.git"
    :sha "209b64504cb3bd3b99ecfec7937b358a879f55c1"}}
  :main-opts ["-m" "cognitect.test-runner"]}}
  \end{clojure}
\end{english}

Ключ \spverb|:aliases| это то же самое, что профили в \spverb|lein|: опции,
которые включаются, если указан профиль. В ключе \spverb|:extra-paths| список
путей для поиска файлов с кодом. Поле \spverb|:extra-deps| указывает зависимости
в дополнение к основным. Test-runner нужен только для тестов, поэтому не нужно
вносить его в глобальные зависимости.

Библиотека загружается с GitHub; поле \spverb|:sha| означает коммит или
тег. Коренное отличие Deps от Lein том, что репозиторием может быть GitHub. Ключ
\spverb|:main-opts| задает входную точку программы. Это функция \spverb|-main|
из модуля \spverb|cognitect.test-runner|.

Библиотека предлагает ключи для выборки тестов с логикой <<только с этим тегом>>
(\spverb|-i|, анг. <<include>>, включить) или <<без этого тега>> (\spverb|-e|,
анг. <<exclude>>, исключить). Команда ниже выполнит все тесты кроме
интеграционных:

\begin{english}
  \begin{bash}
clj -Atest -e :integration
  \end{bash}
\end{english}

Селекторы в \spverb|test-runner| могут быть только ключами. Сложные функции
вроде версии базы не поддерживаются. Test-runner ищет пространства по
регулярному выражению (ключ \spverb|-r|, namespace-regex). С~ним можно выбрать
тесты по доменной структуре, например, все дочерние модули \spverb|project.api|.

\section{Проблема окружения}

До сих пор мы писали примитивные тесты, которые проверяют числа. На практике вы
столкнетесь с проблемой: в коде преобладают не вычисления, а ввод-вывод
данных. Это обращение к сторонним ресурсам: базе данных, очереди задач, HTTP
API. Такой код трудно тестировать по двум причинам.

Во-первых, настроить локальное окружение стоит усилий, а иногда это
невозможно. У вас может не быть ключей или сертификатов для доступа. Во-вторых,
тесты плохо работают с данными, которые не контролируют. Если ответ сервиса
изменятся, получим сломанный билд.

Чтобы тест не зависел от окружения, применяют \emph{моки} и
\emph{стабы}. Разберемся, что это такое и в чем плюсы и минусы каждого.

\subsection{Моки}

Мок (анг. <<mock>>~--- ложный, фиктивный)~--- это объект, который временно
заменяет другой объект. При обращении к объекту сработает не исходный код, а
указанный. Часто моки накладывают на функции с доступом в сеть. С их помощью
проверяют, как поведет себя код в разных ситуациях: данные получены, статус 403,
соединение не удалось.

Представим мобильное приложение для отдыха. На главном экране карта, на которую
нанесены кафе, рестораны и ближайшие события: фильмы, выставки,
фестивали. Данные приходят с сервера по протоколу HTTP в JSON. Функция
\spverb|view-main-page| принимает запрос с координатами устройства, собирает
сведения о местах и событиях и отсылает результат:

\begin{english}
  \begin{clojure}
(defn view-main-page [request]
  (let [location (-> request :params (select-keys [:lat :lon]))
        sites (get-sites-by-location location)
        events (get-events-by-location location)]
    {:status 200
     :body {:sites sites :events events}}))
  \end{clojure}
\end{english}

Данные о местах и событиях приходят из сторонних сервисов, условных Яндекс.Карт
и Афиши. Функции \spverb|get-sites...| и \spverb|get-events...| общаются с ними
по HTTP. Так выглядит функция \spverb|get-sites-by-location| для поиска кафе и
ресторанов в радиусе 300 метров:

\begin{english}
  \begin{clojure}
(ns book.views
  (:require [clj-http.client :as client]))

(defn get-sites-by-location
  [{:keys [lat lon]}]
  (-> {:method :get
       :url "https://maps.yandex.ru/search/v1"
       :as :json
       :query-params {:apikey "....."
                      :lat lat :lon lon :distance 300
                      :type "cafe,restaurant"}}
      client/request
      :body))
  \end{clojure}
\end{english}

Функция \spverb|get-events-by-location| для событий аналогична первой. Разница в
адресе (не \spverb|maps.yandex.ru|, a \spverb|events.afisha.ru|) и параметрах
запроса.

Очевидно, \spverb|view-main-page| обращается в сеть дважды, что затрудняет
тестирование. Понадобится два API-ключа; если запускать тесты часто, мы
исчерпаем квоты на число запросов. Настоящие данные меняются и поэтому не
подходят для тестов: завтра откроется новое кафе или закроется старое. Решим
проблему моками.

В главе про изменяемость мы работали с \spverb|with-redefs| \page{with-redefs}.
Макрос заменяет переменную по ее пути на что-то другое. Простейший мок подменяет
функцию с обращением в сеть на функцию с известным результатом.

Напишем тест главного экрана с двумя моками. Их данные вынесем в переменные в
\spverb|let|. Функцию \spverb|view-main-page| вызываем  внутри
\spverb|with-redefs|, иначе эффект моков теряется.

\begin{english}
  \begin{clojure}
(deftest test-main-page
  (let [sites [{:name "Cafe1"} {:name "Cafe2"}]
        events [{:name "Event1"} {:name "Event2"}]]
    (with-redefs
      [book.views/get-sites-by-location (constantly sites)
       book.views/get-events-by-location (constantly events)]
      (let [request {:params {:lat 55.751244
                              :lon 37.618423}}
            result (view-main-page request)]
        (is (= (:body result)
               {:sites sites :events events}))))))
  \end{clojure}
\end{english}

Мы избежали похода в сеть: тест не зависит от ключей, прав доступа и новых
заведений. Запустите его в любое время, и результат не изменится. Теперь
рассмотрим, как улучшить тест.

\subsection{Вынести данные в файл}

Переменные \spverb|sites| и \spverb|events|, которые якобы пришли из сторонних
сервисов, крайне скудны. Сообщите им данные из реального источника, скажем,
десять ресторанов и семь мероприятий. Чтобы не засорять код огромными словарями,
поместите ответы в .json-файлы в папку ресурсов. Считайте их в переменную
модуля:

\begin{english}
  \begin{clojure}
(def data-events
  (-> "data/events.json"
      clojure.java.io/resource
      slurp
      (json/parse-string true)))
  \end{clojure}
\end{english}

На практике функция \spverb|get-sites-by-location| сложнее, чем в
примере. Заведения фильтруют согласно внутренней логике. Например, исключают
кафе с низкой оценкой и те, что сейчас закрыты. Это порождает отдельный тест:
добавьте кафе с низкой оценкой и убедитесь, что его нет в ответе.

\subsection{Мок в фикстуре}

Если тестов несколько, оборачивать каждый в \spverb|with-redefs|
утомительно. Поместим мок в фикстуру. Так мы снизим повторы в коде, и ей смогут
пользоваться другие тесты.

\begin{english}
  \begin{clojure}
(defn fix-mock-sites-events [t]
  (with-redefs [...]
    (t)))

(use-fixtures :each fix-mock-sites-events)
  \end{clojure}
\end{english}

Недостаток фикстуры в том, для всех тестов она задает позитивный сценарий (оба
вызова прошли без ошибок). Для полноты картины нужно проверить ошибки.

\subsection{Негативные сценарии}

Недоступность сервиса не должна влиять на приложение в целом. Если возникли
проблемы с кафе, пользователь увидит мероприятия и наоборот. Возможно, он даже
не заметит проблемы: если локации не пришли, приложение покажет старые данные.

Чтобы проверить главный экран на устойчивость, нужны сценарии:

\begin{itemize}

\item
  получили кафе, но не события;

\item
  наоборот: кафе в порядке, ошибка с событиями;

\item
  не работают оба сервиса.

\end{itemize}

Под общими словами <<не работают>> имеют в виду:

\begin{itemize}

\item
  проблемы доступа или квот: статусы 403 и 429;

\item
  недоступность сервиса: статусы 500 и 503;

\item
  проблемы связи: таймаут соединения или поиск хоста.

\end{itemize}

Напишем макрос \spverb|with-mock|, который принимает путь к функции, результат
мока и тело, которое он оборачивает.

\begin{english}
  \begin{clojure}
(defmacro with-mock
  [path result & body]
  `(with-redefs
     [~path (fn [& ~'_] ~result)]
     ~@body))
  \end{clojure}
\end{english}

Важно: для замены \spverb|path| мы создаем функцию с помощью \spverb|fn|, а не
\spverb|constantly|. Дело в том, что \spverb|result| может быть не только
выражением, но и произвольным кодом. \spverb|Constantly| это функция, поэтому
аргументы вычисляются до того, как она сработает. Нельзя передать в не\"{е} код
c~исключением. Строка:

\begin{english}
  \begin{clojure}
(constantly (throw (new Exception "error")))
;; Execution error at...
;; error
  \end{clojure}
\end{english}

\noindent
бросит исключение до входа в \spverb|with-redefs|. Наоборот, об\"{е}ртка в
функцию откладывает исполнение кода до вызова:

\begin{english}
  \begin{clojure}
(fn [] (throw (new Exception "error")))
;; #function[eval53371/fn--53372]
  \end{clojure}
\end{english}

С помощью \spverb|with-mock| собирают комбинации успехов и неудач. Тест ниже
проверяет сценарий, когда сервис кафе сработал без ошибок, а с событиями что-то
не так. Если запустить тест, окажется, что главный экран не ловит исключения, и
пользователь не получит ничего.

\begin{english}
  \begin{clojure}
(deftest test-sites-ok-events-err
  (with-mock book.views/get-sites-by-location [...]
    (with-mock book.views/get-events-by-location
      (throw (new java.net.UnknownHostException "DNS error"))
      (let [request {...}
            response (view-main-page request)
            {:keys [status body]} response]
        (is (= 200 status))
        (is (= {...} body))))))
  \end{clojure}
\end{english}

Доработайте \spverb|view-main-page| так, чтобы тест получил ответ со статусом
200 и полем \spverb|:sites|. Добавьте тесты с другими комбинациями: ошибка
соединения в \spverb|get-sites|, статусы 403 и 500, недоступность сразу двух
сервисов.

Подумайте, как улучшить \spverb|with-mock|. Сейчас макросы вложены друг в друга,
что дает отступы и лишний код. Сделайте так, чтобы \spverb|with-mock| был похож
на \spverb|let|~--- принимал форму связывания, где неч\"{е}тный элемент это путь
к функции, а ч\"{е}тный~--- результат. Тогда каскад макросов можно заменить на
один.

\subsection{Сбор данных}

Не заглядывая в конец главы, подумайте, как сделать мок промышленного уровня. Он
не только возвращает результат, но и:

\begin{itemize}

\item
  считает, сколько раз его вызвали;

\item
  на каждый вызов запоминает аргументы;

\item
  принимает список результатов и возвращает их поочередно, например, первый и
  второй раз словари, а на третий исключение;

\item
  предлагает удобный доступ к собранным данным.

\end{itemize}

Продвинутый мок следит за особой логикой. Например, поиск ресторанов вернет
только базовые сведения о них. Чтобы получить подробные сведения о месте, нужно
слать отдельный запрос. Чтобы трафик не рос линейно, данные извлекают только для
трех лучших ресторанов. Мок проверяет, что функцию \spverb|get-site-details|
вызвали не больше трех раз. В конце главы мы рассмотрим библиотеки для мокинга в
Clojure.

\subsection{Недостатки}

Кажется, что проблема ушла: тесты автономны и не зависят от третьих
сервисов. Однако у моков свои недостатки, которые нужно иметь в виду.

Мок нарушает принцип закрытости (черного ящика). При написании теста мы
подглядываем в код, чтобы узнать, какие функции подменить. Это нечестно: чем
меньше тест знает о коде, который тестирует, тем он надежней.

Мок исключает сетевое взаимодействие, что иногда выходит боком. Передача данных
по HTTP это сложный процесс. Подмена запроса это грубое приближение. С его
помощью нельзя предугадать возможные ошибки.

Наконец, мок повышает связанность тестов с логикой. Если переименовать функцию,
которую мокают, форма \spverb|(with-redefs...)| перестанет работать. Мелкие
правки в коде повлияют на тесты, что неудобно и неверно идеологически.

Эти проблемы решают стабы, о которых речь в следующем разделе.

\subsection{Стабы}

Стаб (анг. \emph{stub}, заглушка)~--- это сущность, которая заменяет часть
системы на время тестов. От моков она отличается тем, что тест не может ей
управлять. Если моку указывают, что и чем заменить, то стаб это черный ящик с
точки зрения теста.

Удачный пример стаба это локальный сервер для поиска ресторанов. По адресу
\spverb|/search/v1| он вернет данные, которые скопировали из настоящего
источника. Не будем мокать функцию \spverb|get-sites-by-location|. Изменим ее
так, чтобы поле запроса \spverb|:url| указывало на
\spverb|http://127.0.0.1:8808/|.  Если запустить тест, функция запросит
локальный сервер.

У подхода со стабом преимущество: он снимает зависимость теста от логики. Тест
не знает устройство кода, нужно только перенацелить запросы на другой адрес. Это
вопрос конфигурации, а логика программы не меняется.

Другое достоинство в том, что во время теста происходит \emph{настоящий} обмен
данными по сети. Да, это локальный сервер без шифрования и проверки
сертификатов. Но основная часть HTTP протекает как в боевом запуске. Приложение
выполняет разбор заголовков, чтение и парсинг JSON, проверку статуса. С моком мы
пропускаем эти шаги.

Сервера-заглушки еще называют фейками (анг. <<fake>>, подделка). С~помощью фейка
моделируют нештатное поведение сервера. Например, долгое ожидание, медленную
передачу данных или недоступность. Достичь таких эффектов моком трудно. С его
помощью можно только бросить исключение, которое ожидают от настоящего
HTTP-клиента. Тесты на исключениях далеки от реального положения дел.

Технически стаб это фикстура. До запуска теста она включает сервер, а после
выключает его. На роль сервера подходит Jetty или другой адаптер для Ring. Мы
подробно рассмотрели их в главе по веб-разработку \page{ring-jetty}.

Напишем фикстуру для сервера карт. Для пути \spverb|/search/v1| сервер вернет
данные в JSON, для остальных запросов 404:

\label{sites-handler}

\begin{english}
  \begin{clojure}
(defn sites-handler* [{:keys [uri]}]
  (case uri
    "/search/v1"
    {:status 200
     :body [{:name "Cafe1" :address "..."}
            {:name "Cafe2" :address "..."}]}
    {:status 404
     :body "page not found"}))
  \end{clojure}
\end{english}

Обернем функцию в middleware. Мы хотим, чтобы коллекция из ответа приводилась в
JSON автоматически. Кроме того, пригодятся параметры запроса:

\begin{english}
  \begin{clojure}
(def sites-handler
  (-> sites-handler*
      wrap-keyword-params
      wrap-params
      wrap-json-response))
  \end{clojure}
\end{english}

Сервер не меняет состояние от теста к тесту, потому что мы только читаем
данные. Подключим фикстуру с ключом \spverb|:once|, чтобы она сработала один
раз:

\begin{english}
  \begin{clojure}
(defn fix-fake-sites-server [t]
  (let [opt {:port 8808 :join? false}
        server (run-jetty sites-handler opt)]
    (t)
    (.stop server)))

(use-fixtures :once fix-fake-sites-server)
  \end{clojure}
\end{english}

Исправим функцию \spverb|get-sites-by-location| так, чтобы бекенд обращался к
локальному хосту. Очевидно, базовый URL должен быть в конфигурации. Вы уже
знаете, как устроена конфигурация, поэтому не будем приводить весь код. Для
краткости представим, что переменная \spverb|config| это словарь
параметров. Полный URL для карт построим так:

\begin{english}
  \begin{clojure}
{:url (str (:maps-base-url config ) "/search/v1")}
  \end{clojure}
\end{english}

Вернемся к тесту главного экрана. Нам не нужны моки, поэтому уберем макросы
\spverb|with-mock|. Прямо сейчас тест не пройдет, потому что мы решили проблему
только с поиском заведений. Для событий мы по-прежнему обращаемся к чужому
серверу. Чтобы проверить стаб, закомментируем события, а поле \spverb|:events|
сделаем \spverb|nil|. Тест пройдет без ошибок: в поле \spverb|:sites| окажутся
данные, что вернул стаб.

\begin{english}
  \begin{clojure}
(deftest test-main-page
  (let [request {:params {:lat 55.751244
                          :lon 37.618423}}
        result (view-main-page request)]
    (is (= (:body result) {:sites [...] :events nil}))))
  \end{clojure}
\end{english}

Верните поиск событий и напишите для него заглушку. Оформите ее в фикстуру и
убедитесь, что тест проходит.

Рассмотрим, что еще интересного предлагают стабы. Как и в случае с моками,
удобно, когда ответы лежат в файлах. Чтобы фейковый сервер читал файл, измените
ответ следующим образом:

\begin{english}
  \begin{clojure}
{:status 200
 :body (-> "data/events.json"
           clojure.java.io/resource
           clojure.java.io/file)}
  \end{clojure}
\end{english}

\noindent
В файл \spverb|resources/data/events.json| запишите ответ реального сервера.

Стаб всегда возвращает успешный ответ, что неудобно. Научим его негативным
сценариям. Проще всего это сделать с помощью параметров. Договоримся, что
некоторые координаты, которые посылает пользователь, особые. Для пары
\tuple{0,0} сервер вернет пустой результат; при \tuple{66,66} получим ошибку
доступа. С координатами \tuple{42,42} сервер отвечает с задержкой в 35
секунд. Перепишем \spverb|sites-handler*|:

\begin{english}
  \begin{clojure}
(defn sites-handler* [request]
  (let [{:keys [uri params]} request
        {:keys [lat lon]} params]
    (case uri
      "/search/v1"
      (case [lat lon]
        [ "0"  "0"] {:status 200 :body []}
        ["66" "66"] {:status 403 :body {:error "ACCESS_ERROR"}}
        ["42" "42"] (do (Thread/sleep (* 1000 35))
                        {:status 200 :body []})
        {:status 200
         :body [{:name "Cafe1" :address "..."}
                {:name "Cafe2" :address "..."}]})
      {:status 404 :body "page not found"})))
  \end{clojure}
\end{english}

Напишите тесты для этих координат. Особенно интересен случай с долгим ответом
для пары \tuple{42,42}. Убедимся, что если сервис карт не отвечает, мы ждем
разумное время, а не стандартные 30 секунд. Передайте в параметры
\spverb|(client/get)| поля \spverb|:socket-timeout| и
\spverb|:connection-timeout| со значением 5000, пять секунд. Этого хватит для
боевого запуска. Замерьте выполнение \spverb|view-main-page|. Добавьте проверку,
что ожидание не превышает 5 секунд с небольшой погрешностью.

Тест на экстремальный случай: что произойдет, если сервер недоступен? Чтобы
смоделировать ситуацию, у теста должен быть доступ к серверу. Поместим его в
динамическую переменную:

\begin{english}
  \begin{clojure}
(defonce ^:dynamic *server* nil)

(defn fix-fake-sites-server [t]
  (let [opt {:port 8808 :join? false}]
    (binding [*server* (run-jetty sites-handler opt)]
      (t)
      (.stop *server*))))
  \end{clojure}
\end{english}

Напишем тест, в котором временно отключим сервер. В конце его нужно включить,
чтобы не задеть остальные тесты.

\begin{english}
  \begin{clojure}
(deftest test-the-website-is-down
  (.stop *server*)
  (let [request {:params {:lat 1 :lon 2}}
        result (view-main-page request)]
    (is (= (:body result) {...})))
  (.start *server*))
  \end{clojure}
\end{english}

В обработчике запроса получим ошибку <<Connection refused>>. Доработайте код,
чтобы пользователю ушел пустой результат.

\subsection{Ресурсы и пути}

Мы упомянули, что данные для моков или стабов хранят в файлах. Так мы не
засоряем код объемными словарями. Файлы в папке \spverb|resources| называются
ресурсами приложения. У них особая роль: при сборке jar-файла ресурсы становятся
его частью. Код из jar читает ресурсы, словно это файлы рядом на диске.

Мы поместили файл \spverb|events.json| в папку \spverb|resources/data| и читали
как ресурс. Проблема в том, что файл нужен только для тестов. Если собрать
проект, \spverb|events.json| попад\"{е}т в jar-файл, что напрасно увеличит его
объем.

Чтобы отбросить лишние ресурсы при сборке, в проекте указывают
\spverb|:resource-paths|. Это вектор, в котором перечислены пути для поиска
ресурсов. По умолчанию он равен \spverb|["resources"]|. При слиянии профилей
векторы соединяются в один. Если другой профиль задал ресурсы иначе:

\begin{english}
  \begin{clojure}
:resource-paths ["resources_test"]
  \end{clojure}
\end{english}

\noindent
, то в итоге программа ищет ресурсы в папках \spverb|resources| и
\spverb|resources_test|. Для боевой сборки оставим только \spverb|resources|,
чтобы не вбирать ресурсы для тестов.

Путь \spverb|resources_test| выбран неудачно. Более удобный способ связать файлы
и окружение это \emph{env-директория}. На е\"{е} первом уровне находятся папки с
именами профилей, а в них~--- \spverb|src| и \spverb|resources| с кодом и
ресурсами, которые относятся к этому профилю.

Создайте папки и переместите ресурсы для теста:

\begin{english}
  \begin{bash}
$ mkdir -p env/test/resources/data
$ mv resources/data/events.json env/test/resources/data/
$ rm -rf resources/data
  \end{bash}
\end{english}

\noindent
Задайте профили в проекте:

\begin{english}
  \begin{clojure}
:profiles {:test {:resource-paths ["env/test/resources"]}
           :dev  {:resource-paths ["env/test/resources"]}}
  \end{clojure}
\end{english}

На первый взгляд странно, что для \spverb|:dev| и \spverb|:test| мы указали один
и тот же путь. По умолчанию REPL запускается с профилем \spverb|dev|, но не
\spverb|test|. В разработке у нас не будет доступа к ресурсам из
\spverb|env/test|. Это влечет неудобства: мы исправили код и вызвали тест из
REPL, но он не проходит из-за путей.

Если вам не нравятся повторы в конфигурации, запустите REPL с~ключом
\spverb|with-profile +test|. Плюс перед профилем означает \emph{добавить} его к
стандартному. Без плюса профиль \emph{заменит} его.

Обратите внимание: вы переместили файл между папками, но для JVM это один и тот
же ресурс \spverb|data/events.json|:

\begin{english}
  \begin{clojure}
(def data-events
  (-> "data/events.json"
      clojure.java.io/resource
      slurp
      (json/parse-string true)))
  \end{clojure}
\end{english}

Любознательным читателям предлагаем эксперимент. Скомпилируйте \spverb|uberjar|
с разными ресурсами: в первый раз как обычно, а во второй с опцией
\spverb|:resource-paths|:

\begin{english}
  \begin{clojure}
:profiles {:uberjar {:resource-paths ["env/test/resources"]}
  \end{clojure}
\end{english}

Сохраните первый файл \spverb|target/uberjar/<project>.jar| в другую папку,
чтобы второй не зат\"{е}р его. Распакуйте оба архива командой \spverb|jar|:

\begin{english}
  \begin{bash}
$ jar xf <project>.jar
  \end{bash}
\end{english}

Среди файлов из второго архива вы найдете \spverb|data/events.json|. Убедитесь,
что его нет в первом архиве. Вашу программу распакует любой желающий, поэтому в
ресурсах не должно быть приватных данных вроде паролей и ключей. Это касается и
данных для тестирования~--- включать их в jar будет ошибкой.

Пользуясь случаем, изучите другие файлы из jar, их структуру и содержимое.

\subsection{База данных}

При написании тестов вы столкнетесь с проблемой: почти каждое действие требует,
чтобы в базе лежали особые данные. Для создания пользователя нужна учетка
администратора. Чтобы добавить участника в группу, нужна эта группа. Чтобы
оформить заказ, нужны товары, история покупок для расчета скидки и многое
другое. Чем шире логика приложения, тем больше нужно данных для теста.

Если каждый каждый раз готовить данные вручную, с тестами начнутся
проблемы. Нужна единая точка входа, чтобы каждый тест имел доступ к одинаковым
данным. В идеале данные декларативны и приходят из файла. Чтобы добавить запись,
мы исправляем файл, а не код. Когда тесты закончились, содержимое базы
исчезает. Недопустимо, чтобы тест зависел от изменений, выполненных в другом
тесте.

В этом разделе мы рассмотрим, как работать с базой на время тестов. Начнем с
первого шага~--- подготовки базы. Под этим понимают создание таблиц, индексов и
других сущностей \emph{без вставки} данных.

Базу создают до запуска тестов. Способ зависит от того, как работает база:
локально или в Docker. Если локально, вызовем нескольких утилит. Ниже мы
обращаемся к \spverb|createuser| и \spverb|createdb| из поставки PostgreSQL.

\begin{english}
  \begin{bash}
$ createuser book_test -S -W
$ createdb -O book_test book_test
  \end{bash}
\end{english}

Первая команда готовит пользователя с именем \spverb|book_test|.
Ключ~\spverb|-S| означает, что это \emph{супер}пользователь,
обладатель высших привилегий. Супердоступ нужен, чтобы включать расширения:
триграммный поиск, прогрев индексов и другие. В боевом запуске приложению
выделяют минимальные права, но для тестов это неважно. Вторая команда создает
одноименную пустую базу. Ключ \spverb|-O| (owner) задает владельца
базы. Владелец имеет полный доступ ко всем ее сущностям.

Если база работает в Docker, обратитесь к главе про системы. Там мы узнали, как
задать параметры базы в .yaml-файле и переменными среды \page{docker-db}.

Пустую базу наполняют таблицами. Если приложение ведет миграции, их запускают до
тестов из командной строки. Команда \spverb|migratus| приходит из
одноим\"{е}нного\footurl{https://github.com/yogthos/migratus-lein} плагина к
Lein для управления миграциями.

\begin{english}
  \begin{clojure}
lein migratus migrate
lein test
  \end{clojure}
\end{english}

Прогона миграций можно избежать, если скопировать их в папку образа
\spverb|/docker-entrypoint-initdb.d|. Образ устроен так, что при запуске
выполнит все sql-файлы из этой папки. Убедитесь, что скопировали только
\spverb|up|-миграции, иначе их \spverb|down|-версии сведут пользу на нет.

Таблицы и индексы готовы, остались данные. Перечислим способы их вставки и
очистки во время тестов.

\subsubsection*{Ручная вставка}

В простом случае данные для тестов хранят в виде коллекции. Специальная функция
пробегает по списку словарей; каждый словарь это запись в базе. Функцию
оборачивают в фикстуру, которая запускает тест после вставки записей.

\begin{english}
  \begin{clojure}
(def db-data
  [[:users {:name "Ivan" :email "ivan@test.com"}]
   [:users {:name "Juan" :email "Juan@test.com"}]
   [:groups {:name "Dog fans" :topics 6}]
   [:groups {:name "Cat fans" :topics 7}]])
  \end{clojure}
\end{english}

Здесь и ниже мы полагаем, что переменная \spverb|*db*| означает подключение к
базе. Нам не важно, откуда пришла эта переменная. Читатель знаком с системами и
конфигурацией и вправе задать \spverb|*db*| на свое усмотрение.

\begin{english}
  \begin{clojure}
;; loader
(defn load-data []
  (doseq [[table row] db-data]
    (jdbc/insert! *db* table row)))

;; fixture
(defn fix-db-data [t]
  (load-data)
  (t))
  \end{clojure}
\end{english}

Заметим, что для каждой записи нужно знать ее таблицу. Это может быть первый
элемент пары \tuple{таблица, запись} или поле метаданных. Когда записей одного
типа много, таблица становится избыточной. Сгруппируем записи по таблицам и
вставим не по одной, а разом, что быстрее.

\begin{english}
  \begin{clojure}
(def db-data
  [[:users [{:name "Ivan" :email "ivan@test.com"}
            {:name "Juan" :email "Juan@test.com"}]]
   [:groups [{:name "Dog fans" :topics 6}
             {:name "Cat fans" :topics 7}]]])

(defn load-data []
  (doseq [[table rows] db-data]
    (jdbc/insert-multi! *db* table rows)))
  \end{clojure}
\end{english}

\subsubsection*{Вставка из CSV}

Когда ещ\"{е} больше, словари становятся неудобны. Ключи словаря занимают в
среднем половину его описания. Для тысячи записей мы столько же раз укажем
ключи. Если структура данных одинакова, их хранят в виде массива, а ключи
указывают один раз.

Именно так устроен формат CSV\footurl{https://en.wikipedia.org/wiki/Comma-separated\_values}.
Это текст, где каждая строка означает запись, а поля разделены запятой.
В первой строке обычно идет заголовок~--- имена полей с тем же разделителем.

У CSV несколько преимуществ. Записи это массивы, а не словари, поэтому имена
полей не повторяются. CSV компактней JSON и аналогов. С ним работают табличные
редакторы Excel и OpenOffice. Можно открыть CSV как электронную таблицу,
добавить или удалить столбец, пересчитать ячейки формулой и сохранить результат.

Отдельные базы читают и пишут CSV напрямую. На больших объемах вставка из CSV
работает быстрее, чем обычный \spverb|INSERT|.

Предположим, нам скинули данные о пользователях в CSV. Поместим их в ресурс
\spverb|data/users.csv|. Несколько первых строк:

\begin{english}
  \begin{text}
name,email
ivan,ivan@test.me
juan,juan@example.com
ioan,ioan@dunno.org
  \end{text}
\end{english}

Составим запрос с командой \spverb|COPY|. Это особый оператор, который работает
в PostgreSQL. Укажем таблицу, путь к CSV-файлу и его свойства. Под свойствами
понимают разделитель, формат файла и признак заголовка. Разделителем может быть
не только запятая, но и точка с запятой или табуляция. Если заголовка нет,
считается, что поля идут в том же порядке, что и в таблице.

\begin{english}
  \begin{clojure}
(jdbc/execute! *db*
 "COPY users(name, email)
  FROM '/Users/ivan/work/book/env/test/resources/data/users.csv'
  DELIMITER ',' CSV HEADER")
  \end{clojure}
\end{english}

Заметим, что запрос ожидает полный путь к файлу, что мешает командной
работе. Вряд ли ваши коллеги держат проект в папке
\spverb|/Users/ivan/work/book|. На практике полный путь к файлу вычисляют из
относительного.

Если в запросе нет ошибок, и файл на месте, и пользователя хватает прав, в
таблице \spverb|users| окажутся данные из файла. Аналогично работает сброс
таблицы в файл: оператор \spverb|FROM| заменяют на \spverb|TO|.

Выполним загрузку из CSV силами Clojure. Усложним задачу тем, что в файле очень
много записей, и потому он сжат алгоритмом GZip. Это полезная практика: все
большие файлы в репозитории должны быть сжаты. Чтобы читать такой файл на лету
(не выгружая во временную папку), понадобятся классы:

%% -------------

\begin{english}
  \begin{clojure}
(:import java.io.FileInputStream
         java.util.zip.GZIPInputStream
         org.postgresql.copy.CopyManager)
  \end{clojure}
\end{english}

Класс \spverb|CopyManager| это обертка над командой \spverb|COPY|. Чтобы создать
его экземпляр, нужно соединение с базой (именно TCP-соединение, а не словарь
спеки). Получим его функцией \spverb|jdbc/get-connection|. Метод \spverb|copyIn|
принимает SQL-запрос и поток данных CSV. Поток получим из ресурса
\spverb|data/users.csv.gz|, пропустив его через серию функций и
классов. \spverb|GZIPInputStream| оборачивает поток так, что при чтении сжатых
данных получаем исходные. В конце работы закрываем соединение с базой.

\begin{english}
  \begin{clojure}
(defn load-data-gz []
  (let [conn (jdbc/get-connection *db*)
        copy (CopyManager. conn)
        stream (-> "data/users.csv.gz"
                   clojure.java.io/resource
                   clojure.java.io/file
                   FileInputStream.
                   GZIPInputStream.)]
    (.copyIn copy "COPY users(name, email)
                   FROM STDIN (FORMAT CSV, HEADER true)"
             stream)
    (.close conn)))
  \end{clojure}
\end{english}

Нельзя загрузить один файл сразу в несколько таблиц. Соблюдайте правило один
файл~--- одна таблица. Если файлов несколько, перепишите функцию так, чтобы она
принимала путь к файлу и имя таблицы. Импорт \emph{очень больших} данных
(миллион и больше) можно вести параллельно с помощью футур и \spverb|pmap|.

\subsubsection*{Проблема ключей}

Таблицы редко существуют сами по себе. Чаще всего они связаны: пользователь
ссылается на профиль, заказ на пользователя и так далее. Для этого служат
\emph{первичный} и \emph{внешний} ключи. Первичный ключ (PK, Primary key) это
поле, которое однозначно определяет запись в таблице. Внешний ключ (FK, Foreign
key) это поле, которое ссылается на первичный ключ другой таблицы. Вместе это
называют \emph{отношением} таблиц.

Как правило, первичные ключи это целые числа с автоматической нумерацией. Чтобы
добавить запись в таблицу, не нужно подбирать новый PK. Если ключ с
автонумерацией, база ведет для него уникальный счетчик. Для очередной записи
счетчик вырастет на единицу. Имя счетчика в PostgreSQL строится по шаблону
\spverb|<table>_<pk>_seq|, например \spverb|users_id_seq|.

При вставке связанных данных их PK и FK должны быть равны. Проще всего задать их
вручную, например 3. Ниже пользователь ссылается на группу:

\begin{english}
  \begin{clojure}
(jdbc/insert! *db* :groups {:id 3 :name "Clojure users"})
(jdbc/insert! *db* :users {:group_id 3 :name "Ivan"})
  \end{clojure}
\end{english}

Проблема в том, что случайный PK может вступить в конфликт со
счетчиком. Предположим, при вставке в \spverb|groups| счетчик был равен
нулю. Поскольку мы задали \spverb|id| явно, счетчик не увеличился. Теперь нужны
еще три группы. Нам не важен их \spverb|id|, поэтому опустим его в словаре. База
назначит первичный ключ из счетчика. Две первых группы получат значения 1 и
2. Для третьей возникнет ошибка:

\begin{english}
  \begin{text}
ERROR:  duplicate key value violates unique constraint 'groups_pkey'
DETAIL:  Key (id)=(3) already exists
  \end{text}
\end{english}

Ключ 3 в конфликте с записью, которую мы добавили раньше. Избежать этого можно
двумя способами: коррекцией счетчиков и особым загрузчиком данных. Разница в
том, кто контролирует счетчики: вы или база данных.

Под ручной коррекцией имеют в виду следующее. В тестовых данных явно указывают
первичные ключи. Для удобства некоторые из них выносят в переменную. Если на
админа ссылаются несколько сущностей, напишем так:

\begin{english}
  \begin{clojure}
(def id-user-admin 1)

(def db-data
  [[:users [{:id id-user-admin :name "Ivan"}]]
   [:profiles [{:id 1 :user_id id-user-admin :avatar "..."}]
    :posts [{:id 1 :user_id id-user-admin :title "New book"}
            {:id 2 :user_id id-user-admin :title "Some post"}]]])
  \end{clojure}
\end{english}

После вставки поправим счетчики так, чтобы они перескочили значения, которые мы
использовали. Предположим, в тестовый набор входит семь пользователей, и мы
назначили им ключи от 1 до 7. Если установить счетчик в 100, он уже не вернется
к числам от 1 до 7. Новые записи в \spverb|users| получат ключи 101, 102 и так
далее.

Чтобы сбросить счетчик, выполните запрос:

\begin{english}
  \begin{clojure}
(jdbc/execute! *db* "ALTER SEQUENCE users_id_seq RESTART WITH 100")
  \end{clojure}
\end{english}

Если первичный ключ называется \spverb|id| (что рекомендуется), нужно знать
только имя таблицы. Чтобы не перечислять таблицы вручную, получим их из массива
\spverb|db-data|. Добавьте в конец функции \spverb|load-data| код:

\begin{english}
  \begin{clojure}
(let [value 100
      tables (set (map first db-data))
      query "ALTER SEQUENCE %s_id_seq RESTART WITH %s"]
  (doseq [table tables]
    (jdbc/execute! *db* (format query (name table) value))))
  \end{clojure}
\end{english}

Он подхватит все таблицы, указанные в \spverb|db-data| и выставит счетчики в
100. На больших данных из CSV эту цифру придется увеличить на порядок или
два. Если потребуется, по \spverb|id| вы отличите исходную запись от добавленной
в процессе. Например, запрос \spverb|DELETE ... WHERE id > 100| удалит записи,
которые попали в базу после <<накрутки>> счетчиков.

Теперь рассмотрим свой загрузчик. В этом случае полагаются на ключи, которые
вернула база данных. Конкретные их значения не важны. Главное, чтобы у связанных
записей первичный и внешний ключи совпадали. Для этого пойдем на хитрость.

Для начала удалим первичные ключи из тестовых данных. Ни в одном словаре нет
поля \spverb|:id| с конкретным значением. Сущностям, на которые нужно ссылаться,
присвоим псевдонимы. Проще всего это сделать через поле в метаданных. В связных
записях в поле-ссылке укажем псевдоним.

\begin{english}
  \begin{clojure}
;; user
^{:entity :users/admin} {:name "Ivan" :email "ivan@test.com"}
;; profile
{:user_id :users/admin :avatar "/images/ivan.png"}
;; posts
{:user_id :users/admin :title "New book"}
{:user_id :users/admin :title "Some post"}
  \end{clojure}
\end{english}

Загрузчик объявляет атом с пустым словарем и наполняет его в процессе. Если у
записи псевдоним, мы связываем его с ключом, который вернула база. Например, при
вставке первого словаря получили \spverb|id| 42. Атом станет \spverb|{:users/admin 42}|.

Дополнительно загрузчик ищет ссылки в словаре перед записью. Если значение
кейворд, считаем, что это внешний ключ. В профиле встречается поле
\spverb|:user_id| со значением \spverb|:users/admin|. Очевидно, нужно заменить
его на то, что лежит в атоме по этому ключу (42). Если в атоме нет ключа,
бросить исключение.

Напишите такой загрузчик самостоятельно. Важно: ответ \spverb|jdbc/insert!|
зависит от типа базы данных. Приведем примеры для двух популярных баз:
PostgreSQL и MariaDB (форк MySQL). Обратите внимание, что даже для одной записи
вы получите список.

\begin{english}
  \begin{clojure}
(jdbc/insert! *db* :users {:name "Ivan"})
({:id 42 :name "Ivan}) ;; for PostgreSQL
({:generated_key 42})  ;; for MariaDB
  \end{clojure}
\end{english}

\subsubsection*{Удаление данных}

Очистка базы столь же важна, как и наполнение. Она должна занимать минимум
усилий и не оставлять артефактов.

Наивный способ очистить базу~--- выполнить \spverb|DELETE FROM <table>| для всех
таблиц, с которыми работают тесты. Проблема \spverb|DELETE| в том, что он
учитывает зависимость ключей. Нельзя удалить пользователя, если на него
ссылается запись из другой таблицы. Придется вызвать \spverb|DELETE| в
правильном порядке: сначала для связной таблицы, затем \spverb|users|. Чем
больше у вас таблиц, тем сложнее запомнить порядок.

Читатель, знакомый с PostgreSQL, возразит: каскадное удаление возможно при
помощи \spverb|ON DELETE| в описании ключа. Но это продвинутая техника, и сейчас
мы не будем о ней говорить. Нас интересует как очистить данные независимо от
того, как заданы ключи и ссылки.

Команда \spverb|TRUNCATE| (анг. \emph{подрезать}) нужна для быстрой очистки
таблицы. В отличии от \spverb|DELETE|, она не вызывает триггеры и не сканирует
всю таблицу. \spverb|TRUNCATE| принимает несколько таблиц за раз. Ее
преимущество в каскадном режиме. Если передан флаг \spverb|CASCADE|, все
таблицы, которые входят в граф связей, тоже очищаются. Каскадная очистка пары
таблиц вызывает цепную реакцию по всей базе.

Напишем функцию очистки. Она посылает запрос, в котором через запятую указаны
таблицы из данных для тестов. Добавьте ее в фикстуру \spverb|fix-db-data| после
вызова \spverb|(t)|.

\begin{english}
  \begin{clojure}
(defn delete-data []
  (let [tables (set (map first db-data))
        tables-comma (clojure.string/join "," (map name tables))
        query (format "TRUNCATE %s CASCADE" tables-comma)]
    (jdbc/execute! *db* query)))
  \end{clojure}
\end{english}

\subsubsection*{Транзакция с откатом}

Другой способ избавиться от изменений в базе~--- обернуть действия с ней в
особую транзакцию. Она завершается оператором не \spverb|COMMIT|, а
\spverb|ROLLBACK|, что значит откатить все команды. С точки зрения базы это
выглядит так:

\begin{english}
  \begin{sql}
BEGIN;
INSERT INTO users ...
INSERT INTO profiles ...
UPDATE users SET name=...
ROLLBACK;
  \end{sql}
\end{english}

При выходе из транзакции мы не увидим эффекта \spverb|INSERT|, \spverb|UPDATE| и
других изменений.

В пакет JDBC входит функция \spverb|db-set-rollback-only!|. Она принимает
транзакционное соединение и выставляет ему флаг \spverb|rollback|. Если флаг
установлен, JDBC завершает блок откатом.

Вы уже знакомы с макросом \spverb|with-db-transaction|: внутри него действует
транзакционное соединение, которые получают из JDBC-спеки. Наш макрос
\spverb|with-db-rollback| делает то же самое, но устанавливает откат:

\begin{english}
  \begin{clojure}
(defmacro with-db-rollback
  [[t-conn & bindings] & body]
  `(jdbc/with-db-transaction [~t-conn ~@bindings]
     (jdbc/db-set-rollback-only! ~t-conn)
     ~@body))
  \end{clojure}
\end{english}

Макрос в действии:

\begin{english}
  \begin{clojure}
(with-db-rollback [tx *db*]
  (println "Inserting the data...")
  (jdbc/insert! tx :users {:name "Ivan"})
  (let [...]
    (do-something-with-db tx)))
  \end{clojure}
\end{english}

Следите за тем, чтобы все действия с базой протекали через \spverb|tx|, а не
\spverb|*db*|. Изменения в рамках обычного соединения останутся в базе. Этим
страдает загрузчик \spverb|load-data|: он ссылается на глобальную переменную
\spverb|*db*|. Если загрузчик <<упал>> на середине, половина данных останется в
базе. Чтобы загрузка была в транзакции, передайте параметр или свяжите
\spverb|*db*| формой \spverb|binding|.

Пример с параметром: \spverb|load-data| принимает \spverb|tx|, который мы
установили на вершине теста.

\begin{english}
  \begin{clojure}
(deftest test-user-with-rollback
  (with-db-rollback [tx *db*]
    (load-data tx)
    (let [user (get-user-by-name tx "Ivan")]
      (is (= "Ivan" (:name user))))))
  \end{clojure}
\end{english}

Вариант с динамической переменной. В этом случае мы считаем, что все функции
ссылаются на \spverb|*db*|. Внутри макроса она станет транзакционным соединением
с откатом.

\begin{english}
  \begin{clojure}
(deftest test-user-with-rollback
  (with-db-rollback [tx *db*]
    (binding [*db* tx]
      (load-data) ;; parameter is not needed
      (let [user (get-user-by-name "Ivan")]
        (is (= "Ivan" (:name user)))))))
  \end{clojure}
\end{english}

Выбор зависит от того, как в проекте работают с базой. Решение с откатом
подходит для Mount и похожей архитектуры, где база это глобальная
переменная. Подумайте, как написать фикстуру с макросом
\spverb|with-db-rollback|. Будет ли она работать с системой компонентов? Что
необходимо в этом случае?

\section{Тестирование веб-приложений}

До сих пор мы тестировали отдельные функции, связанные с расчетами. Эти тесты
необходимы, но недостаточны. Они защищают проект от случайных изменений, но не
обещают, что система устойчива. Поднимемся выше и рассмотрим, как тестировать
приложение целиком.

В главе про веб-разработку мы пришли к важному выводу. На каждом уровне
веб-приложение это функция одного аргумента. Обработчик запроса, маршруты,
цепочка middleware это функции, которые принимают запрос и возвращают ответ.

Чтобы написать тест, составим запрос и вызовем приложение как функцию. Проверим
статус ответа на успех (200, 201) или неудачу (404, 403). Если это ответ с JSON,
считаем тело в коллекцию и сравним с образцом.

Вспомним приложение из первой главы \page{compojure}. Отдельные страницы мы
соединили в маршруты с помощью Compojure. Получилось <<голое>> приложение. Мы
назвали его так потому, что оно многого не умеет. Например, читать параметры,
работать с JSON, сессиями и так далее. Эти возможности несут middleware, в
которые оборачивают приложение.

\begin{english}
  \begin{clojure}
(defroutes app-naked
  (GET "/"      request (page-index request))
  (GET "/hello" request (page-hello request))
  page-404)

(def app
  (-> app-naked
      wrap-session
      wrap-keyword-params
      wrap-params
      wrap-json-body
      wrap-json-response))
  \end{clojure}
\end{english}

Напишем несколько тестов для приложения. Проверим главную страницу и путь,
которого нет в маршрутах. Для экономии места сверим только статус ответа.

\begin{english}
  \begin{clojure}
(deftest test-app-index
  (let [request {:request-method :get :uri "/"}
        response (app request)
        {:keys [status body]} response]
    (is (= 200 status))))

(deftest test-app-page-not-found
  (let [request {:request-method :get :uri "/missing"}
        response (app request)
        {:keys [status body]} response]
    (is (= 404 status))))
  \end{clojure}
\end{english}

Как видно из примеров, писать тесты для веб-приложения нетрудно. Если страница
обращается в сеть, вызов <<мокают>> или запускают стаб. Работают все приемы, что
мы уже рассмотрели. Перечислим несколько новых, которые облегчат вашу работу.

\subsubsection*{Приложение целиком}

Избегайте ситуации, когда тест вызывает не приложение, а один из
обработчиков:

\noindent
\begin{tabular}{ @{}p{5.5cm} @{}p{5.2cm} }

\begin{english}
  \begin{clojure}
;; bad
(deftest test-some-view
  (let [resp (page-index {...})
        {:keys [status]} resp]
    (is (= 200 status))))
  \end{clojure}
\end{english}

&

\begin{english}
  \begin{clojure}
;; good
(deftest test-some-view
  (let [resp (page-index {...})
        {:keys [status]} resp]
    (is (= 200 status))))
  \end{clojure}
\end{english}

\end{tabular}

Даже если страница работает, нет гарантии, что запрос пройдет сквозь
middleware. В боевых проектах они несут важную логику. Это права доступа, работа
с JWT, сессии. Убрав middleware теста, вы обманываете себя. Объект \spverb|app|,
который вы тестируете, должен быть максимально <<заряжен>>, то есть близок к
настоящему веб-серверу.

\subsubsection*{Библиотека запросов}

Выше мы объявили запрос в виде словаря. Это удобно, когда нет параметров или
тела. Если странице нужны параметры, придется писать их вручную, что нечитаемо и
тяжело в поддержке.

\begin{english}
  \begin{clojure}
{:request-method :get
 :uri "/users/?page=2&order=name&name=ivan&search_type=relevance"}
  \end{clojure}
\end{english}

Чтобы избежать ошибок, подключите
Ring-mock\footurl{https://github.com/ring-clojure/ring-mock}~--- библиотеку
запросов к Ring-приложению. Она покрывает основные сценарии в тестах. Функция
\spverb|request| принимает метод и путь. Если добавить словарь параметров, то
для \spverb|GET| они станут частью адреса, а для \spverb|POST|~--- его
телом. Библиотека берет на себя URL-кодирование. Функция \spverb|json-body|
подготовит тело с потоком, в который записана коллекция.

Рассмотрим несколько примеров. \spverb|GET|-запрос по адресу \spverb|/help| :

\begin{english}
  \begin{clojure}
(mock/request :get "/help")
  \end{clojure}
\end{english}

Поиск фильмов с параметрами:

\begin{english}
  \begin{clojure}
(mock/request :get "/movies" {:search "batman" :page 1})
  \end{clojure}
\end{english}

Отправка формы, \spverb|POST /users|. Тело запроса станет классом
\spverb|ByteArrayInputStream|. Заголовок \spverb|Content-Type| равен
\spverb|application/x-www-form-urlencoded|.

\begin{english}
  \begin{clojure}
(mock/request :post "/users" {:name "Ivan" :email "test@test.com"})
  \end{clojure}
\end{english}

Случай для HTTP API. Ресурс \spverb|/users| ожидает не поля формы, а
JSON. Запрос составляют в два шага:

\begin{english}
  \begin{clojure}
(-> (mock/request :post "/users")
    (mock/json-body {:name "Ivan" :email "test@test.com"}))
  \end{clojure}
\end{english}

Эти и другие функции смотрите в документации на странице проекта.

\subsubsection*{Проверка тела}

Выше мы проверяли только статус ответа. На практике статуса недостаточно: число
200 еще не говорит, что пришло именно то, что нужно. Проверка тела зависит от
его типа. Если это текст или HTML, хватит и регулярного выражения. Например, по
фразе <<Login>> мы определим, что на этой странице пользователь не авторизован.

Интереснее вариант с JSON. Нужно восстановить коллекцию и сравнить с ее
образцом. Для простоты вызовем приложение \spverb|sites-handler|. Это заглушка,
с которой мы тестировании поиск кафе \page{sites-handler}.

\begin{english}
  \begin{clojure}
(let [request (mock/request :get "/search/v1" {:lat 11 :lon 22})
      response (sites-handler request)
      body (-> response :body (json/parse-string true))]
  (is (= {...} body)))
  \end{clojure}
\end{english}

Недостаток в том, что мы сравниваем данные как есть. В ответе могут быть поля,
которые меняются, например даты или \spverb|id|. Мы уже выяснили, что
\spverb|id| это ключ с автонумерацией. Однажды данные удалят и загрузят
повторно, и номера изменятся.

Перед сравнением лишние поля исключают с помощью \spverb|dissoc| и
\spverb|map|. Представим, что поиск кафе вернул результат:

\begin{english}
  \begin{clojure}
{:sites [{:name "Site1" :date-updated "2019-11-12" :id 42}
         {:name "Site2" :date-updated "2019-11-10" :id 99}]}
  \end{clojure}
\end{english}

Напишем функцию, которая очищает тело от лишних полей.

\begin{english}
  \begin{clojure}
(defn fix-sites [body]
  (update body :sites
          (fn [sites]
            (map (fn [site]
                   (dissoc site :id :date-updated))
                 sites))))
  \end{clojure}
\end{english}

Теперь данные сравнивают не в лоб, а с помощью функции.

\begin{english}
  \begin{clojure}
(is (= {:sites [{:name "Site1"} {:name "Site2"}]}
       (fix-sites body)))
  \end{clojure}
\end{english}

Иногда проверяют не конкретные значения, а структуру ответа. Это удобно, когда
ответ развесистый: длинные списки, объемные словари. В таком случае пользуются
спекой или JSON-схемой. Преимущества спеки в том, что она лояльна к новым
полям. Если в ответ добавили поле, спека не упадет.

\begin{english}
  \begin{clojure}
(let [;; obtain the response
      body (-> response :body (json/parse-string true))]
  (is (s/valid? :api.search/result body)))
  \end{clojure}
\end{english}

Затраты на спеку окупаются в будущем. Ей проверяют входные параметры, генерируют
данные для тестов, описывают REST API (Swagger, RAML).

\section{Тестирование систем}

Коротко о том, как пишут тесты в проектах с системами, о которых мы говорили в
отдельной главе. Напомним, система это набор компонентов со связями между
ними. Покрыть тестами каждый компонент нетрудно; проблемы возникают при их
взаимодействии. В проекте обязательно должен быть тест, где система работает как
единое целое.

На время теста кто-то должен запустить систему и остановить ее. На эту роль
подходит фикстура. Предположим, система и функции \spverb|start!| и
\spverb|stop!| находятся в модуле \spverb|system.clj|. Напишем фикстуру
\spverb|fix-system|:

\begin{english}
  \begin{clojure}
(defn fix-system [t]
  (system/start!)
  (t)
  (system/stop!))
  \end{clojure}
\end{english}

На время теста в переменной \spverb|system/system| будет рабочая система. Другие
фикстуры, например для работы с базой, могут обратиться к компонентам
напрямую. Важно, чтобы в вызове \spverb|use-fixtures| они шли в правильном
порядке (левее~--- раньше), иначе вы получите NPE и другие странности.

\begin{english}
  \begin{clojure}
(defn fix-db-data [t]
  (let [{:keys [db]} system/system]
    (prepare-test-data db)
    (t)
    (clear-test-data db)))

(use-fixtures :once fix-system fix-db-data)
  \end{clojure}
\end{english}

Фикстура \spverb|fix-db-data| нарушает правило, по которому к системе нельзя
обращаться напрямую и вынимать компоненты. Для тестов в этом плане действуют
послабления. Тесты это не промышленный код, поэтому на небольшие нарушения порой
закрывают глаза.

Фикстура \spverb|fix-system| неслучайно стоит под ключом \spverb|:once|. Запуск
и остановка системы занимают много времени. В наших интересах прогнать как можно
больше тестов, пока система работает. Если делать это поштучно, процесс
затянется надолго. При запуске теста из CIDER тоже придется ждать, пока
сработает \spverb|fix-system|.

Кажется, что пять секунд это немного. Но представьте, что работаете над задачей
и запускаете тест раз за разом~--- подобные паузы раздражают. Рассмотрим, как
избавиться от ожидания.

Потребуется два шага. Первый~--- улучшить систему, чтобы она знала о своем
состоянии: включена она сейчас или выключена. Проще всего это сделать полем с
метаданными. Вынесем имя поля в отдельную переменную. Перепишем \spverb|start!|,
чтобы в метаданных системы появился флаг со значением \spverb|true|.

\begin{english}
  \begin{clojure}
(def state-field ::started?)

(defn start! []
  (let [sys (-> system
                component/start-system
                (with-meta {state-field true}))]
    (alter-var-root #'system (constantly sys))))

(defn started? []
  (some-> system meta (get state-field)))
  \end{clojure}
\end{english}

Аналогично изменим \spverb|stop!|, чтобы флаг становился ложью. Функция
\spverb|started?| вернет флаг из метаданных системы.

Второй шаг~--- перед тем, как включить систему в фикстуре, проверяем, была ли
она уже запущена. Если нет, фикстура работает как обычно: запуск, тест,
остановка. Если система уже работает, значит, кто-то управляет ей вручную. В
этом случае фикстура только выполнит тест, что гораздо быстрее.

\begin{english}
  \begin{clojure}
(defn fixture-system [t]
  (let [started-manually? (system/started?)]
    (when-not started-manually?
      (start!))
    (t)
    (when-not started-manually?
      (stop!))))
  \end{clojure}
\end{english}

Выполните в REPL (system/start!). Теперь вызывайте тест сколько угодно раз~---
не придется ждать систему.

\section{Интеграционные тесты}

Не протяжении главы мы постепенно усложняли тесты. С каждым шагом они вс\"{е}
меньше зависят от технических деталей и делают упор на бизнес-логику. Этот
принцип называют пирамидой тестов. В ее основании лежат юнит-тесты~--- множество
отдельных проверок. Поднимаясь к вершине, мы абстрагируемся от технических
деталей. В какой-то момент тестируют не функции, а части приложения.

Каждый уровень требует специальных знаний. Читатель готов к тому, чтобы
подняться на последний этаж~--- освоить \emph{интеграционное}
тестирование. По-другому их еще называют UI- или Selenium-тестами в честь
одноименного фреймворка. В них запросы шлют не программно, а имитируют действия
человека. Для этого управляют браузером или телефоном: вводят данные в форму,
нажимают кнопку и проверяют, что появился нужный текст.

Интеграционные тесты работают медленно, потому что включают полный цикл
приложения. Это загрузка страницы, выполнение скриптов, реакция браузера. Если
возникнет ошибка, ее трудно расследовать из-за длины цепи. Представьте, что вы
нажали на кнопку, но текст не появился. Возможны десятки причин, почему этого не
произошло.

Рассмотрим, как писать UI-тесты на Clojure. С подготовкой вы уже знакомы: нужно
запустить систему и наполнить базу тестовыми данными. Тест ведет себя иначе: он
берет контроль над браузером и командует им. Например, открывает страницу
\spverb|http://127.0.0.1:8080/| и щелкает по ссылкам. В любой момент мы получим
адрес страницы, ее заголовок и HTML-код. В тест добавляют формы \spverb|(is (= ...))|,
чтобы проверить, на какой странице мы оказались или что видит пользователь.

Чтобы управлять браузером, нужен драйвер и библиотека к нему. Под драйвером
понимают утилиту командной строки. Когда драйвер запущен, он принимает запросы
по протоколу HTTP. Одновременно драйвер запускает браузер в особом режиме, и
между ними образуется связь. Драйвер это посредник между программой и
браузером. Он переводит HTTP-запросы в бинарный протокол и наоборот.

Каждый браузер работает со своим драйвером. Для Chrome он называется
\spverb|chromedriver|, для FireFox~--- \spverb|geckodriver|. Одноименные утилиты
ставятся из пакетных менеджеров \spverb|apt|, \spverb|yum| или
\spverb|brew|. Пользователи Windows скачают бинарные файлы с сайта
проектов. Драйвер к Safari называется \spverb|safaridriver|. С версии 13 он идет
в комплекте с Mac OS.

Для работы с драйвером подойдет библиотека Etaoin\footurl{https://github.com/igrishaev/etaoin}.
Добавьте ее в зависимости профиля \spverb|:dev| (только для разработки и тестов):

\begin{english}
  \begin{clojure}
:dev {:dependencies [[etaoin "0.3.6"]]}
  \end{clojure}
\end{english}

Убедитесь, что драйвер находится по одному из путей, указанных в \spverb|PATH|,
например \spverb|/usr/local/bin|. Для этого выполните \spverb|chromedriver| или
\spverb|geckodriver| в терминале. Путь до драйвера можно задать в опциях
библиотеки, но проще положить его в \spverb|PATH|.

Напишем первый тест. Представим, что локальный сервер работает на порту
8080. Тест открывает форму входа, заполняет поля и нажимает кнопку
<<Login>>. Браузер обновляет страницу, появляется приветствие. Видны элементы,
которые прежде были скрыты (ссылки <<My profile>>, <<Logout>>).

\begin{english}
  \begin{clojure}
(ns project.integration-tests
  (:require [etaoin.api :as e]))

(deftest test-ui-login-ok
  (e/with-chrome {} driver
    (e/go driver "http://127.0.0.1:8080/login")
    (e/wait-visible driver {:fn/has-text "Login"})
    (e/fill driver {:tag :input :name :email} "test@test.com")
    (e/fill driver {:tag :input :name :password} "password")
    (e/click driver {:tag :button :fn/text "Login"})
    (e/wait-visible driver {:fn/has-text "Welcome"})
    (is (e/visible? driver {:tag :a :fn/text "My profile"}))
    (is (e/visible? driver {:tag :button :fn/text "Logout"}))))
  \end{clojure}
\end{english}

Разберем отдельные выражения. Форма \spverb|with-chrome| это макрос, который
запускает Chrome на время исполнения кода. Макрос нужен, чтобы выключить драйвер
при выходе или в случае ошибки. Без него пришлось бы добавить
\spverb|try/catch|, что порождает вложенность и в целом неудобно:

\begin{english}
  \begin{clojure}
(let [driver (e/chrome)]
  (try
    (e/go driver "http://...")
    (e/click driver {:tag :button})
    (finally
      (e/quit driver))))
  \end{clojure}
\end{english}

Функция \spverb|wait-visible| ждет до тех пор, пока элемент не появится на
экране. К ней прибегают, чтобы дождаться, пока браузер нарисует верстку. Если не
отделить команды ожиданием, между ними будет разница в несколько
миллисекунд. Браузер либо не успеет выполнить первое действие, либо отбросит
второе.

Ожидание в UI-тестах это нормально. Основное время уходит на то, чтобы выполнить
действия с задержкой, как это свойственно человеку. \spverb|Wait-visible| это
лишь одна из семейства \spverb|wait|-функций. В них входит
\spverb|wait-has-text| (дождаться текст на экране), \spverb|wait-has-class|
(ждать, пока у элемента не появится класс) и другие.

Драйвер ищет элементы на странице с помощью \emph{селекторов}. Это строки с
особыми выражениями. Различают CSS\footurl{https://www.w3schools.com/cssref/css\_selectors.asp}-
и XPath\footurl{https://www.w3schools.com/xml/xpath\_syntax.asp}-селекторы. Мы не будем разбирать их
синтаксис: это долго и заслуживает отдельной главы.

Для краткости рассмотрим альтернативу: на элемент можно сослаться по
словарю. Ключи \spverb|tag| и \spverb|id| означают имя тега и
идентификатор. Другие ключи означают атрибуты тега. В примере выше селектор
\spverb|{:tag :input :name :email}| станет строкой
\spverb|.//input[@name="email"]| на языке XPath.

Ключ \spverb|index| ссылается на $i$-ый элемент, если селектор нашел
несколько. Например, \spverb|{:tag :button :index 1}| вырождается в
\spverb|.//button[1]|. Индекс считают от нуля. Ключи с пространством \spverb|fn|
особые: они означают вызов функции XPath. Ниже ключ \spverb|:fn/has-class|
вызывает \spverb|contains| на имени класса:

\begin{english}
  \begin{clojure}
{:tag :a :fn/has-class "active"}
;; .//a[contains(@class, "active")]
  \end{clojure}
\end{english}

При помощи словаря можно сослаться на любой элемент страницы. Если путь к
элементу слишком сложен, исправьте верстку: добавьте класс или идентификатор. В
особых случаях элемент ищут в рамках другого элемента, чтобы сузить выборку. Эти
и другие случаи описаны в документации к библиотеке.

Наш тест можно улучшить по следующим пунктам.

\textbf{Конфигурация.} Порт 8080 <<захардкожен>> в теле теста. Мы уже знаем, что
подобные значения приходят из конфигурации. Исправьте тест так, чтобы и сервер,
и драйвер работали с одним и тем же портом.

\textbf{Фикстура.} Вспомним, как работает \spverb|with-chrome|: он создает новый
драйвер, выполняет тело и выключает его. Это напоминает систему. Если каждый
тест обернут в \spverb|with-chrome|, мы теряем время на включение и запуск
драйвера. Сделаем так, чтобы драйвер работал на протяжении всего
прогона. Объявим динамическую переменную и фикстуру, которая связывает драйвер
на время тестов. Зарегистрируем ее с ключом \spverb|:once|.

\begin{english}
  \begin{clojure}
(defonce ^:dynamic *driver* nil)

(defn fix-chrome [t]
  (e/with-chrome {...} driver
    (binding [*driver* driver]
      (t))))

(use-fixtures :once fix-chrome)
  \end{clojure}
\end{english}

\textbf{Второй браузер.} Тест работает в Chrome, но вот приходит задача~---
убедиться, что мы поддерживаем Firefox. Тесты для Chrome нужно выполнить в
другом браузере, при этом нельзя из копировать.

Поможет мульти-фикстура, с которой мы знакомились на середине главы. Она
пробегает по списку \emph{типов} браузеров. В примере ниже это \spverb|:chrome|
и \spverb|:firefox|. Макрос \spverb|with-driver| это общий случай
\spverb|with-chrome|. Отличие в том, что \spverb|with-driver| ожидает первым
аргументом тип браузера. На каждом шаге фикстура связывает драйвер с переменной
и выполняет тест.

\begin{english}
  \begin{clojure}
(defn fix-multi-driver [t]
  (doseq [driver-type [:chrome :firefox]]
    (e/with-driver driver-type {} driver
      (binding [*driver* driver]
        (testing (format "Browser %s" (name driver-type))
          (t))))))
  \end{clojure}
\end{english}

Тесты сработают дважды обоих браузерах. Для ясности мы обернули тест сообщением
о том, в каком браузере его вызывают. Чтобы добавить новый браузер, например
Safari, добавьте его в вектор \spverb|driver-type|.

\textbf{Пакетные действия.} Еще один способ улучшить тесты~--- вынести
одинаковые действия в фикстуру или функцию. Предположим, каждый тест начинается
с авторизации и заканчивается выходом из системы. Чтобы не повторять эти
действия каждый раз, напишем фикстуру \spverb|fix-login-logout|. В отличии от
предыдущих фикстур, ей назначают ключ \spverb|:each|.

\begin{english}
  \begin{clojure}
(defn fix-login-logout [t]
  (doto *driver*
    (e/go "http://127.0.0.1:8080/login")
    (e/fill {:tag :input :name :email} "test@test.com")
    (e/click {:tag :button :fn/text "Login"}))
  (t)
  (doto *driver*
    (e/click {:tag :button :fn/text "Logout"})
    (e/wait-has-text "Login")))
  \end{clojure}
\end{english}

Другие сложные действия выносите в функции. Например, ввод даты в виджет
календаря может занять до десяти команд. Напишите условную функцию
\spverb|set-widget-date|, которая принимает драйвер, селектор поля и дату,
которую нужно ввести.

\textbf{Используйте doto.} Когда несколько форм принимают одинаковый первый
аргумент, их объединяют в макрос \spverb|doto|. Он подставит \spverb|*driver*|
на второе место в каждый список тела. С \spverb|doto| код становится немного
короче и чище.

\section{Другие решения}

Перечислим другие библиотеки, полезные для тестов. Не будем изучать их подробно:
ограничимся кратким описанием и примером. Подробности ищите в документации к
проектам.

\subsection{Продвинутые моки}

На минуту вернемся к мокам~--- подмене функции через
\spverb|with-redefs|. Макрос слишком многословен, чтобы работать с ним
напрямую. Появились библиотеки, которые описывают мокинг короче и
выразительнее. Одна из них называется Mockery\footurl{https://github.com/igrishaev/mockery}.
Библиотека предлагает макрос \spverb|with-mock|:

\begin{english}
  \begin{clojure}
(with-mock mock
  {:target :project.path/get-geo-point
   :return {:lat 14.23 :lng 52.52}}
  (get-geo-point "cafe" "200m"))
  \end{clojure}
\end{english}

Первый аргумент это символ, с которым связан будущий мок. Ключ \spverb|:target|
задает путь к объекту, который нужно заменить, а \spverb|:return|~---
результат. Выше мы <<замокали>> функцию \spverb|get-geo-point|, которая, судя по
названию, обращается к сервису карт.

Объект \spverb|mock| это атом, внутри которого словарь. Он наполняется данными
по мере того, как вызывают цель: сколько раз к ней обратились и с какими
аргументами. Ниже мы добавили проверки, что функцию вызвали один раз с
параметрами \spverb|cafe| и \spverb|200m|.

\begin{english}
  \begin{clojure}
(let [{:keys [called? call-count call-args]} @mock]
  (is called?)
  (is (= 1 call-count))
  (is (= '("cafe" "200m") call-args)))
  \end{clojure}
\end{english}

Библиотека Spy\footurl{https://github.com/alexanderjamesking/spy} работает
похожим образом. На функцию садят <<шпиона>>, который копит данные о вызове.

\begin{english}
  \begin{clojure}
(defn adder [x y] (+ x y))
(def spy-adder (spy/spy adder))

(testing "calling the function"
  (is (= 3 (spy-adder 1 2))))

(testing "calls to the spy can be accessed via spy/calls"
  (is (= [[1 2]] (spy/calls spy-adder))))
  \end{clojure}
\end{english}

\subsection{Альтернативный синтаксис}

Проект Midje\footurl{https://github.com/marick/Midje} предлагает другой способ
писать тесты. В этой библиотеке работают с \emph{фактами}. Факт это набор
проверок, сгруппированных по смыслу. Запишем факты о функции \spverb|split|:

\begin{english}
  \begin{clojure}
(facts "about split"
 (str/split "a/b/c" #"/") => ["a" "b" "c"]
 (str/split "" #"irrelvant") => [""])
  \end{clojure}
\end{english}

Стрелка между выражениями это особый оператор, который называется
\emph{extended equality}, продвинутое равенство. С ее помощью величины сравнивают
короче. Например, форма \spverb|1 => even?| сводится к \spverb|(even? 1)|.
\spverb|Midje| предлагает и другие, более сложные стрелки для коллекций и
макросов.

\subsection{Вывод XUnit}

Плагин Test2junit\footurl{https://github.com/ruedigergad/test2junit} пишет отчет
в XML-файл формата XUnit. Системы вроде CircleCI или TeamCity знают, как его
разобрать и вывести на экран. Такой отчет легче читать, чем вывод консоли:
ошибки выделены красным, стектрейсы спрятаны под выпадающие элементы.

Добавьте плагин в Lein и задайте путь, куда писать файл. После прогона тестов
проверьте, что появился XML-отчет.

\begin{english}
  \begin{clojure}
:plugins [[test2junit "1.1.2"]]
:test2junit-output-dir "target/test2junit"
  \end{clojure}
\end{english}

\subsection{Генерация данных}

Иногда для тестов нужен большой объем данных: сто, двести тысяч записей или
миллион. При этом данные должны быть разнообразны~--- не подойдет один и тот же
набор, скопированный тысячу раз. Поможет библиотека
Test.check\footurl{https://github.com/clojure/test.check}. Модуль \spverb|gen|
производит случайные данные по заданным правилам. Особенно полезна генерация
записей. В примере ниже мы получим кортеж строки, числа и булева типа. Затем
передаем его в конструктор \spverb|->User|.

\begin{english}
  \begin{clojure}
(defrecord User [user-name user-id active?])

(def user-gen
  (gen/fmap (partial apply ->User)
            (gen/tuple (gen/not-empty gen/string-alphanumeric)
                       gen/nat
                       gen/boolean)))

(last (gen/sample user-gen))
;; #user.User{:user-name "dfgJKSHF3"
;;            :user-id 5
;;            :active? false}
  \end{clojure}
\end{english}

Spec, которой мы посвятили главу, идет еще дальше. С помощью Test.check она
генерирует данные \emph{по спеке}. Появляется еще одно свойство спек: кроме
проверки, они подходят для тестовых данных. Перепишем генерацию на спеку:

\begin{english}
  \begin{clojure}
(s/def :user/id int?)
(s/def :user/name string?)
(s/def :user/active? boolean?)
(s/def ::user (s/keys :req-un [:user/id :user/name :user/active?]))

(gen/generate (s/gen ::user))
{:id 88546920, :name "Z4MO7GH80k3mRD", :active? true}
  \end{clojure}
\end{english}

Возможности \spverb|spec.gen| обширны. С ее помощью порождают связанные данные,
например, пользователей, которые ссылаются на профили и наоборот. Вместо
случайных величин можно задать варианты (список имен, фамилий, городов). Спеки
могут быть любой вложенности, что открывает поле для экспериментов.

\section{Заключение}

Тест это код, который проверяет другой код. Тест защищает проект от случайных
изменений. Без тестов появляется эффект бабочки: изменение в одном месте ломает
другое.

Различают несколько видов тестов. Чаще всего работают с юнит-тестами, которые
еще называют модульными. Они проверяют функции из одноименного пространства. С
юнит-тестами связан термин <<покрытие>>. Это доля кода, который сработал во
время тестов. Достаточным считается покрытие около 80\%. При анализе покрытия
смотрят, все ли ветки алгоритма приняли участие.

Чтобы подготовить тест к запуску, применяют фикстуры. Это функции, которые
вызывают тест до или после каких-то действий. Различают одно- и многоразовые
фикстуры, \spverb|:once| и \spverb|:each|. Фикстуры разнообразны: это может быть
запись в базу, локальный сервер, система компонентов. Иногда фикстура вызывает
тест несколько раз в цикле.

Чтобы выполнить часть тестов, применяют теги и селекторы. Тег это поле
метаданных теста. Селектор~--- функция, которая принимает метаданные и
возвращает истину или ложь. Тесты группируют в наборы, потому что не всегда
можно выполнить их все в текущем окружении.

Если код обращается в сеть, помогут моки и стабы. Мок это подмена кода в момент
исполнения. Например, функцию \spverb|get-events| заменяют на
\spverb|(constantly {...})|. Стаб это полноценный сервер на локальном
порту. Функцию \spverb|get-events| нацеливают на локальный хост с помощью
конфигурации. Сервер отдает данные, которые скопировали из настоящего
источника. Чтобы не засорять код, их хранят в файле.

Файлы в папке \spverb|resources| особые: JVM считает их \emph{ресурсами}. Проект
настраивают так, что в режиме тестов видны дополнительные ресурсы. Убедитесь,
что при сборке \spverb|uberjar| в него не попали данные для тестов.

Негативные сценарии важны в той же степени, что и позитивные. Проверяйте
неверные параметры и права доступа. Чтобы не повторяться, выносите повторы в
списки и пробегайте по ним в \spverb|doseq|. Если вы \emph{ожидали} ошибку, но
ее не было, это \emph{тоже} ошибка. Для таких ситуаций служит
макрос \spverb|(is (thrown?...))|.

Интеграционные тесты проверяют программу целиком. Они имитируют действия
человека, что занимает больше времени. Тест управляет браузером или мобильным
устройством. Типичный сценарий теста~--- нажать на элементы и проверить, что
появился или исчез другой элемент. Интеграционные тесты дороги в поддержке,
поэтому их внедряют в последнюю очередь.

Clojure предлагает библиотеки и утилиты, чтобы помочь с тестами. Из коробки
доступен Clojure.test, базовый фреймворк. Утилита Lein содержит модуль для
прогона тестов. Сторонники Deps.edn пользуются Test-runner. Другие решения
меняют синтаксис тестов, отчетность, упрощают подмену функций.
