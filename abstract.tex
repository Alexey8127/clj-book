\thispagestyle{empty}

\small

Книга рассказывает о Clojure, современном диалекте Лиспа. Это функциональный
язык с акцентом на неизменяемость и многопоточность. Clojure появилась десять
лет назад и плавно набирает популярность в России. В семи главах мы рассмотрим,
как работать с Clojure на производстве.

Эта книга не для тех, кто учит язык с нуля. Ожидается, что читатель знаком с
Clojure или другим диалектом Лиспа. Чтобы лучше усвоить материал, желательно
иметь практический опыт программирования. Для аудитории продвинутого уровня.

\normalfont

\vspace{5em}

\noindent
Верстка: Иван Гришаев\\
Сверстано в \LaTeX\\
Сборка \input{.commit}от \today
