\thispagestyle{empty}

\small
Книга рассказывает о Clojure, современном диалекте Лиспа. Это функциональный
язык с акцентом на неизменяемость данных и многопоточность. Clojure появился
десять лет назад и плавно наблюдает популярность в России. В семи главах мы
рассмотрим основные проблемы, которые ждут Clojure-программиста на
производстве. Эта книга не для тех, кто хочет выучить язык с нуля. Ожидается,
что читатель уже знаком с Clojure или другим диалектом Лиспа. Чтобы лучше
усвоить материал, желательно иметь практический опыт программирования. Для
аудитории продвинутого уровня.
