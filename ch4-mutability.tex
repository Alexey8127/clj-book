\chapter{Изменяемость}

\begin{teaser}
В классических языках данные по умолчанию меняются, а стандартная библиотека
несет ограничения: локи, атомарные действия, постоянные коллекции. Clojure
устроена наоборот: данные не меняются, а мутабельные типы задвинуты на второй
план. Это сделано намеренно, потому что неизменяемость~--- центральная идея
языка.
\end{teaser}

Руководства по Clojure учат постоянными коллекциям. Это правильный подход. Но
когда в программе появляется состояние, новички испытывают трудности. На время
этой главы мы займем другую позицию: рассмотрим, как управлять состоянием в
программах.

\section{Общие проблемы}

На Clojure трудно писать в императивном стиле. Стиль называют императивным,
когда акцент делают на изменении данных. Например, чтобы получить список
удвоенных чисел, выполняют шаги:

\begin{itemize}
\item
  создать новый пустой список;
\item
  пройти по всем элементам исходного списка;
\item
  на каждом шаге вычислить новый элемент;
\item
  добавить его к новому списку.
\end{itemize}

Базовые типы Clojure не меняются, и к ним нельзя применить алгоритм выше. Те,
кто пришел из императивных языков, поначалу не могут писать код без изменения
коллекций. Привычка менять данные настолько укрепилась в них, что
иммутабельность кажется физическим ограничением.

Это сделано нарочно. Создатель Clojure полагает, что изменяемость~--- основная
проблема в разработке ПО. Когда мы пишем код, то видим его начальное
состояние. Таким он будет только первый такт машинного времени. Затем код
инициирует классы, заполнит поля, и объекты начнут менять друг друга.

Код в редакторе это снимок системы в начальный момент. Некоторые ошибки трудно
расследовать, потому что код и состояние расходятся. Мы не знаем, в каком
состоянии была система на момент сбоя. Чтобы исправить ошибку, ее повторяют в
локальном окружении. Но как задать конкретное состояние, назначить двадцати
классам те поля, что были тогда? Это не всегда возможно.

Неизменяемые данные отсекают пласт ошибок, от которых страдают императивные
языки. Рассмотрим примеры на Python.

В модуле задан словарь. Это параметры запроса к стороннему API по
умолчанию. Функция принимает дополнительные параметры, объединяет теми по
умолчанию и передает в HTTP-клиент:

\begin{verbatim}
DEFAULT_PARAMS = {
    "allow_redirects": True,
    "timeout": 5,
    "headers": {"Content-Type": "application/json"},
    "auth": ("username", "password"),
}

def api_call(**params):
    params.update(DEFAULT_PARAMS) # !!!
    resp = requests.post("https://api.host.com", **params)
    return resp.json()
\end{verbatim}

Строка с комментарием критична, хотя на первый взгляд это незаметно. Метод
\spverb|update| словаря дополняет его другим словарем. В нашем случае изменяют
словарь \spverb|params|. Он живет внутри функции и скрыт от внешнего мира.

В коде ниже грубейшая ошибка: переменная \spverb|api_params| получает не копию
глобальных параметров, а \emph{ссылку}. Изменяя \spverb|api_params|, мы на самом
деле меняем \spverb|DEFAULT_PARAMS|. На каждый вызов словарь изменяется, что
ведет к странному поведению программы. Тот случай, когда код и состояние
<<разъехались>>.

\begin{verbatim}
def api_call(**params):
    api_params = DEFAULT_PARAMS
    api_params.update(params)
    resp = requests.post("https://api.host.com", **api_params)
    return resp.json()
\end{verbatim}

На собеседованиях задают следующий вопрос. Представьте функцию ниже. Объясните,
что в ней не так и приведите пример ошибки?

\begin{verbatim}
def foo(bar=[]):
\end{verbatim}

Ответ: параметры по умолчанию создаются однажды. В данном случае \spverb|bar|
равен пустому списку. В Python список изменяется. Если в \spverb|bar| ничего не
передали, получим ссылку на исходный список. Изменим его методом
\spverb|append|, и тогда \spverb|bar| накапливает значения:

\begin{verbatim}
def foo(bar=[]):
    bar.append(1)
    return bar
\end{verbatim}

Вызов \spverb|foo| вернет списки \spverb|[1]|, \spverb|[1, 1]|, \spverb|[1, 1, 1]|
и так далее. Обидно: в коде все в порядке, но программа работает не так.

Современные IDE исследуют код на неявные ошибки. Про \spverb|bar=[]| в сигнатуре
знают все анализаторы и линтеры. Но мы не можем целиком положиться на
утилиты. Когда данные меняются постоянно, трудно понять, где ошибка, а где
умысел.

Начинающих кложуристов выдает код:

\begin{verbatim}
(let [result (atom [])
      data [1 2 3 4 5]]
  (doseq [item data]
    (let [new-item (* 2 item)]
      (swap! result conj new-item)))
  @result)
\end{verbatim}

Программист следует привычкам из императивного прошлого. Атом-аккумулятор
лишний, достаточно \spverb|map| или \spverb|for|:

\begin{verbatim}
(map (partial * 2) [1 2 3 4 5])

(for [n [1 2 3 4 5]]
  (* n 2))
\end{verbatim}

Оба выражения короче и понятней. Не нужно создавать новую коллекцию и писать в
нее элементы. Это делают встроенные функции. Если код опирается на атом или
Java-коллекцию, скорее всего это слабое решение.

Авторы Clojure сделали все, чтобы выделить состояние на фоне общего кода. К
состоянию прибегают только в крайних случаях. Если вы написали код на атомах без
уважительной причины, в лучшем случае вам сделают замечание. В худшем~--- не
примут код.

Дальше мы поиграем в адвоката дьявола~--- изучим императивные возможности
Clojure. Рассмотрим, когда это действительно нужно и как ими пользоваться.

\subsection{В защиту состояния}

Мы говорили, что состояние несет потенциальные ошибки. Это слишком линейное
заявление. Без состояния работают только небольшие программы. Например, скрипты,
которые запускают раз в день. Писать промышленный код без состояния невозможно.

Постоянные данные избавляют нас от ошибок с перезаписью полей. Это значимый
выигрыш, но кроме данных приложение полагается на ресурсы. В их отношении
действует правило: дешевле работать с открытым ресурсом, чем постоянно открывать
и закрывать его. \emph{Состояние повышает скорость}.

Много лет назад веб-серверы работали по протоколу CGI, Common Gateway
Interface\footurl{https://en.wikipedia.org/wiki/Common\_Gateway\_Interface}. Получив
запрос, сервер запускал в отдельном процессе скрипт или бинарный файл. Входные
данные скрипт получал из переменных среды. Программа писала ответ в стандартный
поток. Сервер перехватывал поток и выводил пользователю.

Схема была простой и удобной. Приложение могло быть скриптом на Perl или
программой на C++. У сервера не было состояния. Разработчик в любой момент менял
файл на новый, и изменения вступали в силу немедленно.

За преимущества платили скоростью. Каждый запрос к серверу порождал новый
процесс. Даже если программа написана на C, запуск процесса занимает
время. Индустрия пришла к тому, что приложение должно работать постоянно, а не
по запросу.

Типичное FastCGI-приложение устроено как самостоятельный сервер. Его
производительность на два порядка выше, чем у CGI. Но появилось состояние~---
открытый порт и цикл ввода-вывода. Цикл читает запрос и делегирует отдельному
потоку. Это усложнило разработку, появились новые парадигмы и фреймворки.

Похоже устроены соединения с базой данных. Представим, что на каждый запрос мы
открываем соединение, работаем с ним и закрываем. В машинном времени открыть
соединение это долгая операция. Так появились пулы соединений.

Пул это объект, который держит несколько открытых соединений. Пул знает, какое
из них занято или свободно. Чтобы работать с базой, мы занимаем одно из
свободных соединений, передаем по нему данные и возвращаем. С точки зрения
наблюдателя пул~--- это примитивный объект, который выдает и забирает
соединения.

Внутренняя логика пула довольно сложная. Если соединений не хватает, он
увеличивает емкость, а при избытке сокращает. Для каждого соединения пул считает
время работы и сколько раз им пользовались. Он же решает, когда закрыть
соединение и заменить его новым. Пул работает в отдельном потоке, чтобы не
блокировать основную программу.

Столь сложное устройство пула компенсирует скорость доступа. Каждый запрос к
базе протекает по заранее открытому соединению, что намного быстрее, чем
открывать его каждый раз.

Сама архитектура машин поощряет изменять данные. В школе нам объясняют память
как массив ячеек. Запись в ячейку по адресу дешева. И в C++, и в Python
одинаково легко обновить элемент массива:

\begin{verbatim}
items[i] = 5;
\end{verbatim}

Постоянные структуры ложатся на модель памяти хуже. Это объясняет их сложность:
неизменяемый список уже не цепочка, а дерево узлов с указателем. Постоянные
коллекции умны, чтобы копировать данные не полностью, а частично. На больших
данных выгоднее работать с изменяемыми структурами.

Это не отменяет достоинства их постоянных аналогов. Мы не призываем всюду
внедрять состояние. Иерархия и замедление это цена, которую платят за
удобство. Инженер должен знать, на что идет, когда добавляет состояние или
избавляется от него.

\section{Атомы}

Clojure предлагает несколько способов изменять данные. Самый простой из них~---
атом. Это объект, который прячет в себе другой объект. Атом получают из
одноименной функцией с начальным значением:

\begin{verbatim}
(def store (atom 42))
\end{verbatim}

Если напечатать атом, получим следующее:

\begin{verbatim}
#<Atom@10ed2e87: 42>
\end{verbatim}

Чтобы извлечь значение, применяют оператор \spverb|@|. Запись \spverb|@store|
это укороченный вариант \spverb|(deref store)|. Функция \spverb|deref| принимает
атом и возвращает его внутренний объект. Семантически это то же самое, что
получить значение по указателю. В русской литературе операцию называют
<<разыменование>>. В разговорном языке про оператор \spverb|@| говорят
<<дереф>>, <<дерефнуть>>.

\begin{verbatim}
@store ;; 42
\end{verbatim}

В отличии от коллекций, атом меняет содержимое; при этом он остается тем самым
объектом. Это важное отличие от коллекций. Если добавить к словарю ключ, получим
новый словарь, при этом старый не изменится. Если изменить содержимое атома, это
будет все тот же атом с номером \spverb|10ed2e87|.

Наивный способ изменить атом это функция \spverb|reset!|. Она принимает атом и
другое значение. Оно может быть любого типа, в том числе \spverb|nil|,
коллекцией, исключением:

\begin{verbatim}
(reset! store nil)
(reset! store {:items [1 2 3]})
(reset! store (ex-info "error" {:id 42}))
\end{verbatim}

Если выполнить \spverb|@store| после каждого выражения, получим то же значение,
что передали в \spverb|reset!|.

\subsection{Приращение атома}

Мы назвали \spverb|reset!| наивным, потому что функция не учитывает текущее
значение атома. На практике атом изменяют, отталкиваясь от текущего состояния.
Например, если это счетчик, нам неважно, какое значение в нем сейчас. Атому
посылают команду <<прибавь к содержимому единицу>>. Если это вектор, сообщение
выглядит как <<добавь к содержимому новый элемент>>.

Значение атома уходит на второй план; нас интересует \emph{действие}. Чтобы
обновить атом с учетом текущего состояния, ему посылают функцию. Она принимает
текущее значение атома и возвращает новое, которое заменит содержимое.

Функция \spverb|swap!| принимает атом и функцию для расчета нового значения:

\begin{verbatim}
(def counter (atom 0))
(swap! counter inc) ;; 1
\end{verbatim}

Если повторять вызов \spverb|swap!|, значение \spverb|counter| каждый раз
увеличится на единицу.

\spverb|Swap!| принимает дополнительные параметры для расчета. Предположим, мы
хотим увеличить счетчик сразу на три позиции или отмотать назад. Вместо
\spverb|inc| возьмем сложение и вычитание: $+$ и $-$. Их первым аргументом
станет текущее значение атома, а второй аргумент передают в \spverb|swap!|:

\begin{verbatim}
(swap! counter + 3) ;; increase by 3
(swap! counter - 2) ;; decrease by 2
\end{verbatim}

Новое содержимое будет рассчитано по принципу:

\begin{verbatim}
(+ <current> 3)
(- <current> 2)
\end{verbatim}

\noindent
, где \spverb|<current>|~--- текущее значение атома.

Это был частный случай \spverb|swap!| с одним аргументом. В общем случае функция
принимает их неограниченное количество:

\begin{verbatim}
(swap! storage func arg2 arg3 arg4 ...)
\end{verbatim}

\noindent
Тогда новое значение вычисляется формой:

\begin{verbatim}
(func <current> arg2 arg3 arg4 ...)
\end{verbatim}

До сих пор мы хранили в атомах счетчики-числа. На практике редко считают одну
сущность. Гораздо чаще счетчики ведут в разрезе чего-то. Например, просмотры
страниц по адресам, число сообщений у пользователя и так далее.

Чтобы не создавать по атому на каждую сущность, их объединяют в
словарь. Рассмотрим подсчет системных ресурсов. Это атом, внутри которого
словарь. Ключи означают тип ресурса, значения~--- степень потребления в байтах
или процентах.

\begin{verbatim}
(def usage
  (atom {:cpu 35
         :store 63466734
         :memory 10442856}))
\end{verbatim}

Отдельная функция вычисляет потребление диска. Чтобы записать новое значение в
атом по ключу \spverb|:store|, вызовем \spverb|swap!| следующим образом:

\begin{verbatim}
(defn get-disk-usage []
  (rand-int 99999999))

(let [store (get-disk-usage)]
  (swap! usage assoc :store store))
\end{verbatim}

Эта форма перепишет значение в \spverb|:store| на новое. Возможен и другой
подход, когда мы не вычисляем все занятое место на диске, а фиксируем разницу на
каждое изменение. Например, если пользователь создал или удалил файл, мы читаем
событие и обновляем \spverb|:store| с приращением.

Пусть функция \spverb|get-file-event| каким-то образом вернет событие файловой
системы. Это словарь с ключами \spverb|:action| и \spverb|:size|. В зависимости
от \spverb|:action| мы наращиваем или уменьшаем потребление диска. Наша версия
\spverb|get-file-event| будет заглушкой, которая случайно вернет одно из двух
событий:

\begin{verbatim}
(defn get-file-event []
  (rand-nth
   [{:action :delete
     :path "/path/to/deleted/file.txt"
     :size 563467}
    {:action :create
     :path "/path/to/new/photo.jpg"
     :size 7345626}]))
\end{verbatim}

\noindent
Пересчет изменится:

\begin{verbatim}
(let [{:keys [action size]} (get-file-event)]
  (case action
    :delete
    (swap! usage update :store - size)
    :create
    (swap! usage update :store + size)))
\end{verbatim}

Так считают ресурсы в облачных платформах. Обращение к диску это дорогая
операция. Нельзя в любой момент пробежаться по дереву папок и посчитать
потребление. Иногда файлы одного пользователя лежат на разных дисках и
серверах. Поэтому ресурсы считают итеративно и раз в интервал сверяют цифры.

Усложним пример с ресурсами: теперь их считают в разрезе пользователей. Ключи
верхнего уровня означают номер пользователя, а значения~--- словари
ресурсов. Для каждого пользователя ведем список его процессов. Это множество
идентификаторов, \spverb|PID|.

\begin{verbatim}
(def usage-all
  (atom {1005 {:cpu 35
               :store 63466734
               :memory 10442856
               :pids #{6266, 5426, 6542}}}))
\end{verbatim}

Чтобы добавить процесс пользователю 1005, выполним \spverb|swap!|:

\begin{verbatim}
(swap! usage-all update-in [1005 :pids] conj 9999)
\end{verbatim}

%% ------------

Это комбо: в \spverb|swap!| передали функцию, которая принимает функцию. Опишем
по шагам что произошло:

\begin{itemize}

\item
  получим множество процессов \spverb|<pids>|: \spverb|(get-in <current> [1005 :pids])|;

\item
  добавим к нему новый процесс: \spverb|(conj <pids> 9999)|. Обозначим новое
  множество \spverb|<pids*>|;

\item
  обновим \spverb|<current>| этим множеством: \spverb|(assoc-in <current> [1005 :pids] <pids*>)|.

\end{itemize}

Чтобы удалить процесс, замените \spverb|conj| на \spverb|disj|. Это
противоположная функция: она удаляет элемент из множества.

\begin{verbatim}
(swap! usage-all update-in [1005 :pids] conj 9999)
\end{verbatim}

Очевидно, \spverb|swap!| мощнее \spverb|reset!|. Последний используют в основном
чтобы сбросить атом в исходное состояние. В других случаях важно текущее
значение атома: от него рассчитывают новое.

\subsection{Совместный доступ}

Функция, которую передают в \spverb|swap!|, должна быть без побочных эффектов. В
функциональном программировании ее бы назвали \emph{чистой функцией}. Она не
должна обращаться к базе данных, файлам, выводу на экран. Технически это
возможно, но вы столкнетесь со странным поведением. Иногда функция, переданная в
\spverb|swap!|, запускается \emph{несколько раз}. Причина в способе, которым
атом обновляет содержимое.

Предположим, ресурсы считают в нескольких потоках. Один слушает события файловой
системы, другой мониторит процессы, подсчитывает память и так далее. Возникает
проблема совместного доступа. Возможна ситуация, когда два потока обновляют одни
и те же данные. Первый поток справился быстрее и записал в атом свою версию
данных. Второй поток рассчитал другую версию. Когда он изменит атом, эффект
первого аннулируется.

Это классическая задача про терминал и семейную пару. Муж и жена вносят наличные
на общий пустой счет. Жена вносит 100 рублей, терминал прибавляет эту сумму к
нулю и записывает в базу. Муж вносит 50 рублей, терминал делает то же
самое. Итого на счете 50 рублей, а 100 пропали.

Атом не допустит такого поведения. Он запоминает свое значение в момент
вычисления нового. Назовем это значение \emph{начальным}. Перед тем как обновить
содержимое, атом проверяет, что текущее значение совпадает с начальным. Если они
не равны, атом обновили из другого потока.

В этом случае атом повторяет цикл. Текущее становится начальным, от него
вычисляют новое значение. Атом еще раз сравнивает текущее и начальное
значения. Цикл повторяется до тех пор, пока они не равны. Это значит, что за
время вычислений атом не обновили. Атом меняет текущее значение на новое, цикл
закончен.

%% ------------

Покажем сказанное на примере. Возьмем атом со словарем:

\begin{verbatim}
(def sample (atom {:number 0}))
\end{verbatim}

Понадобится функция \emph{медленного} сложения. Она принимает текущее значение,
приращение и время простоя. Для ясности добавим вывод в консоль.

\begin{verbatim}
(defn +slow
  [num delta timeout]
  (println (format "Current: %s, timeout: %s" num timeout))
  (Thread/sleep timeout)
  (+ num delta))
\end{verbatim}

Обновим атом одновременно из двух потоков. Для этого запустим две футуры. В
первой ждем две секунды, во второй пять:

\begin{verbatim}
(do (future (swap! sample update :number +slow 1 2000))
    (future (swap! sample update :number +slow 2 5000)))
\end{verbatim}

\noindent
Проверим атом:

\begin{verbatim}
@sample ;; {:number 3}
\end{verbatim}

\noindent
Это правильное значение: к нулю прибавили 1 и 2. Вывод консоли:

\begin{verbatim}
Current: 0, timeout: 2000
Current: 0, timeout: 5000
Current: 1, timeout: 5000
\end{verbatim}

Вторая функция сработала два раза, что следует алгоритму выше. Второй
\spverb|swap!| начал расчеты с начальным значением \spverb|{:number 0}|, а к их
завершению значение стало \spverb|{:number 1}|~--- его записал первый
\spverb|swap!|. Чтобы избежать ошибки, атом запустил второй \spverb|swap!| еще
раз относительно \spverb|{:number 1}|.

Когда атом меняют из нескольких потоков, перезапуск случается больше двух
раз. Это недопустимо для функций, которые меняют окружение.

\subsection{Валидаторы и вотчеры}

Поведение атомов расширяют валидаторы и вотчеры (анг. \emph{watcher},
наблюдатель). Валидаторы это функции-проверки. Они принимают новое значение
\emph{до того}, как оно записано в текущее. Если валидатор вернул ложь, вызов
\spverb|swap!| обернется ошибкой.

Функция \spverb|set-validator!| добавляет валидатор к атому. Для счетчика
предположим, что он не может быть отрицательным. Попытка понизить его в нулевом
состоянии вызовет исключение:

\begin{verbatim}
(def counter (atom 2))
(set-validator! counter (complement neg?))
(swap! counter dec) ;; repeat 3 times...
;; Execution error (IllegalStateException)
;; Invalid reference state
\end{verbatim}

Вотчеры это побочные эффекты атома. Они срабатывают \emph{после того}, как атом
перешел в новое состояние. Вотчер связан с уникальным ключом и функцией. Эта
функция принимает четыре аргумента: ключ, атом, старое и новое значения. Одному
атому можно назначить несколько вотчеров.

Разберемся, когда полезны вотчеры. Вспомним подсчет ресурсов: система получает
события и обновляет атом. Мы не можем бросить исключение, если потребление
памяти или диска превысило лимит. В этом нет смысла, потому что события
поступают из внешней системы. Исключение на нашей стороне не остановит поток
событий.

Вынесем проверку в вотчер. Если один из ресурсов превысил порог, вотчер выполнит
нужные действия. Например, уведомит пользователя письмом, что ресурс
исчерпан. Или пошлет запрос, чтобы запретить пользователю добавлять файлы.

В примере ниже, если потребление вышло за лимит, в лог пишут ошибку. Объявим
функцию вотчера:

\begin{verbatim}
(def STORE_LIMIT (* 1024 1024 1024 25)) ;; 25 Gb
(defn store-watcher
  [_key _atom _old value]
  (let [{:keys [store]} value]
    (when (> store STORE_LIMIT)
      (log/errorf "Disk usage %s has reached the limit %s"
                  store STORE_LIMIT))))
\end{verbatim}

\emph{Примечание:} из четырех параметров мы пользуемся только последним
\spverb|value|. Это новое значение атома. В боевом коде мы бы назначили первым
трем символы подчеркивания. Оно отсекает лишние переменные и потому работает
быстрее. Здесь мы оставили имена для семантики.

Назначим вотчер атому с ключом \spverb|:store|:

\begin{verbatim}
(def usage
  (atom {:cpu 35
         :store 63466734
         :memory 10442856}))

(add-watch usage :store store-watcher)
\end{verbatim}

\noindent
Если потребление диска превысит лимит, увидим запись в лог:

\begin{verbatim}
(swap! usage update :store + STORE_LIMIT)
;; Disk usage 26907012334 has reached the limit 26843545600
\end{verbatim}

Валидацию и вотчеры рассматривают как пре- и постэффекты. Разница в том, что
первые могут прервать исполнение, а вторые нет. У них разная семантика:
предварительные эффекты проверяют то, что \emph{может случиться}, а
постэффекты~--- то, что \emph{уже случилось}. Поэтому на них реагируют
по-разному.

\subsection{Другие примеры}

На атомы опираются некоторые функции Clojure, например, \spverb|memoize|. Это
декоратор, который возвращает улучшенную версию функции. Она запоминает
результат относительно аргументов и записывает во внутреннюю таблицу. Если
вызвать функцию с теми же аргументами, полуим результат из таблицы без
вычислений.

Роль таблицы играет атом. Функция-результат \spverb|memoize| замкнута на атоме,
который виден только ей. Вот как выглядит декоратор:

\begin{verbatim}
(defn memoize [f]
  (let [mem (atom {})]
    (fn [& args]
      (if-let [e (find @mem args)]
        (val e)
        (let [ret (apply f args)]
          (swap! mem assoc args ret)
          ret)))))
\end{verbatim}

Заметим, что для поиска в словаре авторы пользуются \spverb|find| вместо
\spverb|get|. Разница в том, как функции трактуют пустое значение. Если по ключу
записан \spverb|nil|, то get тоже вернет \spverb|nil|, и форма \spverb|if-let|
выполнит вторую ветку. Но \spverb|find| вернет сущность \spverb|MapEntry|,
значение из которой извлекают функцией \spverb|val|.

Убедимся, что декоратор работает на примере \spverb|+slow|, которую мы объявили
выше. Объявим аналог этой функции и замерим вызовы:

\begin{verbatim}
(def +mem (memoize +slow))

(time (+mem 1 2 2000))
"Elapsed time: 2004.699832 msecs"

(time (+mem 1 2 2000))
"Elapsed time: 0.078052 msecs"
\end{verbatim}

Первый вызов долгий, потому что срабатывает \spverb|Thread/sleep| в теле
функции. Второй вызов получает результат из атома, минуя тело.

Атомы полезны в веб-разработке. Это дешевый способ хранить состояние между
запросами. На атомах легко сделать счетчики просмотров, сессии, кэш. Например,
так выглядит счетчик просмотренных страниц. Это комбинация атома и middleware:

\begin{verbatim}
(def page-counter
  (atom {"/" 0}))

(defn wrap-page-counter
  [handler]
  (fn [request]
    (let [{:keys [request-method uri]} request]
      (when (= request-method :get)
        (swap! page-counter update uri (fnil inc 0)))
      (handler request))))
\end{verbatim}

На каждый GET-запрос мы увеличиваем счетчик для текущего адреса. Обратите
внимание на форму \spverb|fnil|, которую передаем в \spverb|update|. Она
возвращает особую версию \spverb|inc|, которая не вызовет исключение, если
первый аргумент \spverb|nil|. Такое возможно, если в словаре еще нет нужного
ключа; вместо \spverb|nil| функция получит ноль.

Функция \spverb|page-seen| вернет число просмотров по адресу страницы. Напишем
компонент подвала сайта, где мелким шрифтом указано, сколько раз смотрели
страницу. Для HTML-разметки подойдет библиотека
\spverb|hiccup|\footurl{https://github.com/weavejester/hiccup} и аналоги.

\begin{verbatim}
(defn page-seen [uri]
  (get @page-counter uri 0))

(defn component-footer [uri]
  [:div {:class "footer"}
   (let [seen (page-seen uri)]
     [:p [:small "This page has been seen " seen " times."]])])
\end{verbatim}

\subsection{Замечания к атому}

У решений на атомах есть и недостатки. Атомы не связаны с другими экземплярами
программы. Если веб-приложение состоит из нескольких нод, каждая хранит свой
счетчик. Из-за этого на каждый запрос пользователь увидит разные данные. Чтобы
избежать странного поведения, данные хранят в сетевых сервисах, например, Redis.

Атомы непостоянны: если завершить программу, они теряют состояние. Допустим
вариант, когда атом читает начальные данные и время от времени сохраняет
состояние в ресурс.

\section{Volatile}

На первый взгляд атом работает просто: с точки зрения инженера это несколько
удобных функций. Но внутри атом довольно сложен. Он регулирует параллельный
доступ, вызывает валидацию и отслеживает изменения. Иногда эти возможности
излишни, и пользуются упрощенной версией атома~--- \spverb|volatile|.

Объект \spverb|volatile| тоже хранит и изменяет значение. Одноименная функция
создает объект с состоянием. Функции \spverb|vreset!| и \spverb|vswap!|
аналогичным тем, что мы рассмотрели для атома. Префикс \spverb|v| означает, что
они работают с \spverb|volatile|.

Пример с ресурсами: вместо атома используем другой тип хранилища:

\begin{verbatim}
(def vusage (volatile! nil))
(vreset! vusage
         {:cpu 35
          :store 63466734
          :memory 10442856})
(vswap! vusage update :store + (* 1024 1024 5))
(println "Disk usage is" (get @vusage :store))
;; Disk usage is 68709614
\end{verbatim}

\spverb|Volatile| отличается от атома тем, что не контролирует запись из
нескольких потоков. Перепишем пример с футурами:

\begin{verbatim}
(def vsample (volatile! {:number 0}))
(do (future (vswap! vsample update :number +slow 1 2000))
    (future (vswap! vsample update :number +slow 2 5000)))
;; Current: 0, timeout: 2000
;; Current: 0, timeout: 5000
@vsample ;; {:number 2}
\end{verbatim}

Консоль говорит, что, во-первых, второе действие сработало один раз. Во-вторых,
результат первого действия утерян. Если для атома итог был 3, то с
\spverb|volatile| получилось 2. Операцию \spverb|+1| мы потеряли. Из этого
следует, что \spverb|volatile| не подходит для многопоточного кода. Валидаторы и
вотчеры тоже не предусмотрены. Это освобождает \spverb|volatile| от слежки за
содержим. Запись без проверок и эффектов работает быстрее.

\subsection{Применение}

У \spverb|volatile| две области применения~--- трансдьюсеры и императивный
код. Трансдьюсеры это особый способ работать с коллекцией. Они оборачивают
функции \spverb|map|, \spverb|reduce| и другие таким образом, что их комбинация
не порождает промежуточных коллекций.

Это возможно за счет внутреннего состояния. Для трансдьюсера важна скорости
записи, поэтому \spverb|volatile| подходит на роль состояния лучше, чем атом. Мы
надеемся на полноценную главу о трансдьюсерах в следующей книге.

\spverb|Volatile| полезен, когда пишут императивный код. Относитесь к этому
спокойно. Иногда бизнес-требования слишком сложны и противоречивы, и накладывать
их на функциональный стиль слишком дорого.

Например, нужно получить плоский список из дерева. Оно устроено по сложным
правилам: если в первой ветке одно значение, то рассматривать вторую, а иначе
третью. И дальше: если для первой и третьей веток выпоняется условие $x$,
положить в список произведение их значений.

Эти требования насквозь императивны, и нам выгодно описать их таким же
образом. Так мы сделаем код ближе к бизнес-логике и облегчим поддержку. Пример с
малым подмножеством такого дерева:

\begin{verbatim}
(def data
  {:items [{:result {:value 74}}
           {:result {:value 74}}]
   :records [{:usage 99 :date "2018-09-09"}
             {:usage 52 :date "2018-11-05"}]})
\end{verbatim}

Код похож на набор блоков, где каждый блок это каскад \spverb|when-let|. На
нижнем уровне мы изменяем коллекцию. Это императивный стиль, но в данном случае
он удобен. Если одно из правил потеряет актуальность, блок удаляют. Удобно,
когда над блоком пишут комментарий или ссылку на документацию:

\begin{verbatim}
(let [result (volatile! [])]

  ;; see section 5.4 from the official doc
  (when-let [a (some-> data :items first :result :value)]
    (when-let [b (some-> data :records last :usage)]
      (when (> a b)
        (vswap! result conj (* a b)))))

  ;; more expressions...
  @result)
\end{verbatim}

\section{Переходные коллекции}

С помощью атома мы создали подобие изменяемых коллекций. Познакомьтесь с новой
техникой: Clojure предлагает \emph{настоящие} изменяемые коллекции. По-другому
они называются \spverb|transient| (анг. временный, переходный).

Изменяемые коллекции получают из постоянных аналогов. Они поддерживают мало
действий, буквально добавить и удалить элемент. Стадартные \spverb|map|,
\spverb|filter| и другие функции не работают с
\spverb|transient|-коллекциями. Происходит своего рода размен: мы теряем мощь
стандартной библиотеки, но обретаем скорость и императивный подход.

Транзиентные коллекции в несколько раз быстрее постоянных. Исторически
компьютеры устроены так, что изменить ячейку памяти проще, чем выделить
новую. Неизменяемые структуры похожи на дерево узлов. Изменить дерево в целом
сложнее, чем плоскую структуру.

Транзиентная коллекция это слепок ее постоянной копии. В этом режиме коллекцию
меняют императивно. Когда алгоритм закончил работу, ее замораживают и получают
неизменяемую коллекцию.

Функция \spverb|transient| порождает переходную коллекцию из исходной. Для
работы с ней используют особые функции \spverb|conj!|, \spverb|assoc!|,
\spverb|dissoc!| и другие. Восклицательный знак на конце предупреждает об
изменении данных. Функции меняют \emph{содержимое} коллекции, а не возвращают их
новую копию, как это делают обычные \spverb|conj| и \spverb|assoc|.

\subsection{Технические детали}

Функция \spverb|persistent!| завершает работу с переходной коллекцией. Она
возвращает постоянную версию и одновременно <<запечатывает>> оригинал. После
\spverb|persistent!| коллекцию уже невожно изменить.

Рассмотрим переходный вектор. На него действуют две функции: \spverb|conj!| и
\spverb|pop!|, добавить и убрать элемент из хвоста:

\begin{verbatim}
(let [items* (transient [1 2 3])]
  (conj! items* :a)
  (conj! items* :b)
  (pop! items*)
  (persistent! items*))
;; [1 2 3 :a]
\end{verbatim}

\noindent
Вариант со словарем, \spverb|assoc!| и \spverb|dissoc!|:

\begin{verbatim}
(let [params* (transient {:a 1})]
  (assoc! params* :b 2)
  (assoc! params* :c 3)
  (dissoc! params* :b)
  (persistent! params*))
;; {:a 1, :c 3}
\end{verbatim}

\emph{Примечание:} в примерах выше имя изменяемой переменной заканчивается
звездочкой. Это не нарушает синтаксис Clojure. В отличии от других языков, в
имени переменной могут быть дефис, апостроф и другие символы. \emph{Особые}
переменные выделяют штрихом или звездочкой. Переходные коллекции встречаются
редко, поэтому их считают особыми.

После того как коллекция заморожена, изменить ее невозможно. Следующий пример
бросит исключение:

\begin{verbatim}
(let [params* (transient {:a 1})]
  (assoc! params* :b 2)
  (persistent! params*)
  (assoc! params* :c 3))
;; IllegalAccessError: Transient used after persistent! call
\end{verbatim}

Форма \spverb|(persistent! <mutable>)|, как правило, замыкает форму с изменяемой
переменной.

Переходные коллекции помогают там, где нужен императивный подход. Выше мы
работали с деревом; тогда мы использовали \spverb|volatile| для аккумуляции
данных. Перепишем код на транзиентный вектор:

\begin{verbatim}
(let [result* (transient [])
      push! (fn [item]
              (conj! result* item))]

  ;; see section 5.4 from the doc: http://...
  (when-let [a (some-> data :items first :result :value)]
    (when-let [b (some-> data :records last :usage)]
      (when (> a b)
        (push! (* a b)))))

  ;; more expressions
  (persistent! result*))
\end{verbatim}

Еще один прием: чтобы не писать каждый раз \spverb|(conj! result* item)|, мы
вводим локальную функцию \spverb|push!|. Она замкнута на \spverb|result*| и
принимает только значение. Чтобы добавить элемент, достаточно вызвать
\spverb|(push! x)|.  Это сокращает код и скрывает детали реализации.

\subsection{Итерация с изменением}

Мы упоминали, что переходные коллекции быстрее постоянных. Это заметно на долгих
итерациях через \spverb|loop/recur|. Как правило, одна из переменных это
коллекция-результат. В каждом \spverb|recur| мы передаем ее копию, дополненную
через \spverb|conj| или \spverb|assoc|.

Когда итераций много, прибегают к уловке: вместо постоянной коллекции передают
ее транзиентный вариант. Этим итерацию ускоряют в 2-4 раза. Изменения в коде
малы, нужно лишь учесть следующее:

\begin{itemize}

\item
  изменить тип результата на \spverb|(transient <coll>)|;

\item
  вместо \spverb|conj| или \spverb|assoc| вызывать их аналоги \spverb|conj!|,
  \spverb|assoc!|;

\item
  в конце вернуть постоянную коллекцию через \spverb|persistent!|.

\end{itemize}

Для дальнейших экспериментов объявим переменную \spverb|nums|, список из
миллиона чисел:

\begin{verbatim}
(def nums (range 999999))
\end{verbatim}

Наивная итерация с помощью \spverb|loop|. Результат~--- вектор чисел:

\begin{verbatim}
(loop [result []
       [n & nums] nums]
  (if n
    (recur (conj result n) nums)
    result))
\end{verbatim}

Эти же вычисления с изменяемым вектором:

\begin{verbatim}
(loop [result* (transient [])
       [n & nums] nums]
  (if n
    (recur (conj! result* n) nums)
    (persistent! result*)))
\end{verbatim}

Число строк осталось прежним, изменилось имя переменной и некоторые
функции. Важно, что правки не выходят за пределы \spverb|loop|. Это дает свободу
действий. На ранних стадиях пишут код без изменяемых коллекций. Затем цикл
улучшают так, что внутренние данные меняются. Если изменения изолированы,
оставляют новый вариант.

Макрос \spverb|time| выполняет тело и печатает затраченное время. Если обернуть
в \spverb|time| оба примера, получим результаты:

\begin{verbatim}
;; 166.688721 msecs
;;  69.415038 msecs
\end{verbatim}

Конкретные цифры зависят от оборудования, операционной системы и версии языка,
но разница в несколько раз сохраняется. Транзиентные коллекции действительно
быстрее постоянных аналогов.

Ускорение работает и для \spverb|reduce|. В других языках эта функция называется
\spverb|fold| или <<свертка>>. Центральная точка в \spverb|reduce|~---
коллекция-аккумулятор. Строго говоря, аккумулятором может быть любой тип,
например, число или строка для сложения или конкатенации. Но чаще всего это
списки и словари.

Различают два способа начать свертку. Первый~--- роль аккумулятора играет
начальный элемент коллекции. Для списка \spverb|(1, 2, 3)| и функции сложения им
станет единица. Второй~--- аккумулятор задают отдельным параметром, например,
пустым вектором, куда в будущем запишут данные.

Идея в том, чтобы передать в качестве аккумулятора изменяемую коллекцию. На
каждом шаге \spverb|reduce| изменяет ее функциями \spverb|conj!| и
аналогами. Сравним обычный \spverb|reduce|:

\begin{verbatim}
(reduce
 (fn [result n]
   (conj result n))
 []
 nums)
\end{verbatim}

\noindent
и его мутабельную версию:

\begin{verbatim}
(persistent!
 (reduce
  (fn [result* n]
    (conj! result* n))
  (transient [])
  nums))
\end{verbatim}

Второй \spverb|reduce| пришлось обернуть в \spverb|persistent!|. В случае с
\spverb|loop| мы втянули \spverb|persistent!| внутрь и тем самым изолировали
изменения. Но \spverb|reduce| не такой гибкий в этом плане. В анонимной функции
мы не знаем, достигли мы конца итерации или нет. Без \spverb|persistent!| второй
пример вернет транзиентную коллекцию, что недоспустимо.

\subsection{Семантика и ограничения}

Изменяемые данные это к продвинутая техника. Они более требовательны к инженеру,
причем не только в плане кода как мы выяснили, его легко сделать
императивным. Транзиентные коллекции меняют дизайн приложения.

Выигрыш в скорости еще не значит, что их применяют на каждом шагу. Худшее, что
может сделать разработчик~--- написать код, где функции обмениваются такими
коллекциями. Если это ваш случай, задумайтес, почему автор языка уделил так
много внимания неизменяемости. Будет ошибкой игнорировать эту идею.

Феномен, когда на раннем этапе пытаются выжать скорость, называется
преждевременной оптимизацией. Как заметил Доанльд Кнут, это корень всех зол.
Занимаясь оптимизацией, задавайте вопросы. Действительно важно ускорить этот
цикл? Поможет ли это продукту или вы действуете из любопытства?

Транзиентные коллекции должны быть изолированы в небольших функциях. Тогда
переход от постоянных типов к изменяемым не повлияет на результат. Рефакторинг
должен касаться только функции, а не ее потребителей.

Изменяемая коллекция не имеет право быть глобальной. Воздержитесь от форм
\spverb|(def users* (transient []))| и подобных. Вы придете к тому, что
\spverb|users*| станет буфером обмена между функциями. Результат такой функции
невозможно предсказать.

В отличии от атома, транзиентные типы не контроллируют обращение к ним из разных
потоков. Следите, чтобы только один поток изменял коллекцию. Не передавайте их в
футуры.

\section{Переменные и alter-var-root}

Атомы и переходные коллекции меняют объект, а не переменную. Это не всегда то,
что мы ожидаем. Например, переменная \spverb|size| это атом:

\begin{verbatim}
(def size (atom 0))
\end{verbatim}

Чтобы изменить его, вызовем \spverb|reset!| или \spverb|swap!| как в примерах
выше. Но переменная \spverb|size| всегда будет атомом. То же самое с переходными
коллекциями: к ним легко добавить и удалить элемент, но это будет \emph{та же
  самая} коллекция. Невозможно присвоить ей \spverb|nil|.

Иногда глобальную переменную нужно изменить. Например, сделать так, чтобы сперва
она была \spverb|nil|, затем словарем, затем снова \spverb|nil|. Авторы
намеренно усложнили этот сценарий. Менять переменные можно, но нежелательно с
точки зрения языка.

Clojure не поощряет стиль Python или Java, когда переменную меняют много
раз. Программист должен понимать, зачем она понадобилась и можно ли от нее
избавиться. Глобальные переменные без уважительной причины~--- признак плохого
кода.

Все же бывают случаи, когда глобальные переменные полезны. Это \emph{система} и
\emph{monkey-patch}. Для начала рассмотрим систему.

Условный проект на Clojure состоит из отдельных компонентов или доменов. Это
веб-сервер, база данных, очередь сообщений. Каждый домен помещают в свой модуль:
\spverb|http.clj|, \spverb|db.clj| и так далее.

Редко бывает так, что приложение работает с двумя базами или очередями. Поэтому
в модуле объявляют переменную, которая хранит состояние компонента. Например,
\spverb|project.http/server| или \spverb|project.db/conn|. Возникает проблема,
как задать переменную. Начинающие допускают ошибку, когда инициируют ее в лоб:

\begin{verbatim}
(def server (jetty/run-jetty app {:port 8080}))
\end{verbatim}

Определение запустит сервер при загрузке модуля. Это плохая практика, потому что
в загрузке не болжно быть побочных эффектов. С таким кодом невозможно работать
REPL: он делает то, о чем не просили.

Сервер, базы данных и другие компоненты должны включаться по требованию. Поэтому
переменная \spverb|server| вначале равна \spverb|nil|. Функция \spverb|start!|
запускает сервер и записывает его в \spverb|server|. Функция \spverb|stop!|
останавливает и меняет переменную на \spverb|nil|.

\subsection{Понятие переменной}

Чтобы переписать переменную, прибегают к \spverb|alter-var-root|. Это функция, которая
изменяет объекты, объявленные через \spverb|def| и \spverb|defn|. Ее вызов похож на \spverb|swap!|
для атома. Функция принимает объект \spverb|Var| и другую функцию, которая вычисляет
новое значение на базе текущего.

Рассмотрим, что такое Var. Это экземпляр класса \spverb|clojure.lang.Var| из библиотеки
Clojure. \spverb|Var| описывает переменную в пространстве имен. Чтобы получить объект
переменной, ее символ передают в макрос \spverb|var|, например \spverb|(var server)|. Эта
запись аналогична \spverb|#'server|, что немного короче.

Символ переменной и ее объект это разные сущности. Сам по себе символ ничего не
значит. Он равен только самому себе. Можно представить символ как слово
языка. Слово это определенная комбинация букв, и в языке не может быть двух
одинаковых слов. Но у слова может быть несколько значений в зависимости от
контекста. Объект, на который ссылается слово в данном контексте, и есть
переменная.

Символ это посредник между пространством имен и переменными. Когда мы пишем
\spverb|(def num 42)|, это не значит, что переменная \spverb|num| равна числу 42. На самом
деле мы создали объект \spverb|Var| со значением 42. Затем поместили этот объект в
текущее пространство под символом \spverb|num|.

Технически пространство имен устроено как словарь. Его ключи это символы, а
значения переменные. Формы \spverb|def| и \spverb|defn| наполняют этот словарь. Можно сказать,
что \spverb|def| задает слову смысл, и в момент компиляции Clojure понимает это слово.

Над символом работает операция вычисления. Если в сеансе REPL ввести \spverb|num|,
интерпретатор выполнит поиск с таким ключом в пространстве имен. Если ключ
найден, REPL вернет значение переменной, в нашем случае 42.

Clojure намеренно скрывает от нас стадию переменной, и это правильно. Если бы
выражение \spverb|num| вернуло переменную, в этом не было бы никакого смысла. Объект
\spverb|Var| это не цифра, а сложный объект. Значение 42 это лишь одно из множества его
полей.

Мы подробно поговорим о пространствах и переменных в другой главе. Пока что
отметим, что переменные, как правило, скрыты от пользователя. Обычно разработчик
видит либо их символ (\spverb|num|), либо значение (\spverb|42|). Функция \spverb|alter-var-root|~---
тот случай, когда переменные заявляют о себе.

\subsection{Запуск по требованию}

Вернемся к примеру с сервером. Объявим переменную, которая в будущем станет
объектом \spverb|jetty.server.Server|. По умолчанию это \spverb|nil|:

\begin{verbatim}
(def server nil)
\end{verbatim}

Функция \spverb|start!| заменяет \spverb|server| результатом анонимной функции. Эта функция
принимает текущее значение. Чтобы не запускать сервер дважды, мы проверяем, был
ли он уже запущен. Если нет, создаем новый сервер и возвращаем его объект. Если
да, просто вернем текущий.

\begin{verbatim}
(defn start!
  []
  (alter-var-root
   (var server)
   (fn [server]
     (if-not server
       (run-jetty app {:port 8080 :join? false})
       server))))
\end{verbatim}

Аналогично работает \spverb|stop!|: выключаем сервер если он уже был
запущен. Устанавливаем переменную в nil.

\begin{verbatim}
(defn stop!
  []
  (alter-var-root
   (var server)
   (fn [server]
     (when server
       (.stop server))
     nil)))
\end{verbatim}

Вызов \spverb|(start!)| в REPL запустит сервер в фоновом режиме. Браузер начнет
отвечать на запросы по адресу http://127.0.0.1:8080/ (путь зависит от маршрутов
приложения). Выражение \spverb|server| в REPL напечатает Java-объект сервера.

По такому же принципу устроена работа с базой. Чтобы не создавать подключение на
каждый запрос, применяют пулы соединений. В модуле для работы с БД объявляют
переменную \spverb|conn| со значением \spverb|nil|. Функция \spverb|start!| создает новый пул и
обновляет переменную.

После старта к базе посылают запросы. Это функции \spverb|query|, \spverb|update!|, \spverb|insert!|
и другие из пакета \spverb|clojure.java.jdbc|. Первым параметром они принимают
подключение к базе или пул:

\begin{verbatim}
(jdbc/query conn "select 1 as result")

(defn get-user-by-id
  [user-id]
  (jdbc/get-by-id conn :users user-id))
\end{verbatim}

Функция \spverb|stop!| выключает пул, закрывает соединения с базой и восстанавливает
\spverb|conn| в nil. Мы опустим код этих функций, потому что они отличаются только
именем глобальной переменной (\spverb|conn| вместо \spverb|server|) и строкой, где создается
сервер или пул.

\subsection{Коротко о системах}

Техника \spverb|alter-var-root| вводит модуль в одно из двух состояний: включен и
выключен. Это особенно удобно для разработки, когда нас интересует конкретная
подсистема. Например, для отладки базы не требуется включать веб-сервер, а
подсистема кеширования не зависит от рассылки писем.

В широком смысле описанное выше называют системой. Это набор компонентов, из
которых состоит приложение. С помощью \spverb|alter-var-root| строят системы для
*несложных* проектов. Как правило, это веб-приложения с сервером и базой
данных. По-другому их называют системами для бедных.

Преимущество таких систем в том, что они не зависят от сторонних
библиотек. Каждый модуль выставляет универсальные "ручки управления": функции
\spverb|start!| и \spverb|stop!|, которые обращаются к \spverb|alter-var-root|. Это простая и
понятная схема.

С другой стороны, "бедные" системы не знают, как один компонент зависит от
другого. Ручные зависимости становятся проблемой с ростом проекта. Поэтому
большие системы строят с помощью специальных библиотек. Некоторые из них тоже
опираются на \spverb|alter-var-root|. Мы подробно рассмотрим системы в другой главе.

\subsection{Патчинг}

Выше мы изменяли переменные в текущем пространстве. Фактически это улучшенный
вариант с атомом. Разница в том, что переменная не требует оператора \spverb|@|, когда
ссылаются на ее значение. Это делает код чище и короче.

Истинная мощь \spverb|alter-var-root| в том, что функция работает с переменными из
любого пространства. Под любым мы понимаем:

- текущий модуль;
- соседние модули проекта;
- сторонние библиотеки;
- стандартные модули Сlojure, например, \spverb|clojure.core|.

С помощью \spverb|alter-var-root| можно повлиять на любую точку всего проекта. В том
числе вторгнуться в чужие модули и даже те, что идут в поставке с Clojure!

Это мощная техника, но ей пользуются редко. Менять код в момент исполнения
считается сомнительной практикой. По-другому это называется monkey patch. Термин
означает изменение классов и функций не на уровне кода, а когда программа уже
запущена.

Если программист злоупотребляет патчингом, программа ведет себя неочевидным
образом. Коллегам будет трудно понять, почему в коде написано одно, а
выполняется другое. Это затрудняет поддержку проекта, привносит раздражение в
повседневную работу.

К патчингу прибегают, если одновременно сошлись несколько условий:

- без этого изменения мы не сможем двигаться дальше;
- проблема не в нашем коде, а в сторонней библиотеке или самой платформе;
- чтобы устранить проблему в коде, понадобиться существенное время.

Типичный случай, когда патчинг оправдан~--- ошибка в чужой библиотеке. Наверняка
многие сталкивались с тем, что автор не принимает изменения, которые вы
прислали. Причины могут быть разными. Автор считает, что проблема не в коде, а
неверных данных. Автор забросил проект или опасается, что часть прав будет
принадлежать вам,~--- словом, исправить код не удается.

Технически не сложно клонировать репозиторий и внести правки. По-другому это
называют форкнуть, сделать форк (анг. fork~--- вилка, ответвление). Но встают
организационные проблемы. Где хранить новый репозиторий? Как сделать так, чтобы
изменения в оригинальной библиотеке отражались на ее клоне? Как настроить
окружение, чтобы система сборки качала нашу версию, а не оригинал? Разрешает ли
автор использовать измененную версию его кода?

Эти проблемы сложны тем, что вовлекают других людей: админов, юристов,
руководство фирмы. Вряд ли мы найдем время, чтобы на середине проекта все
отложить и заняться ими. В данном случае monkey patch будет разумным решением.

Патчинг полезен во время разработки. Мы часто печатаем данные на экран, чтобы
исследовать их. Недостаток функции \spverb|println| в том, что ее вывод не
структурирован. Отдельные части коллекций "слипаются", и их трудно прочитать.

Пакет \spverb|clojure.pprint| (сокращение от pretty printing) решает эту
проблему. Функция \spverb|pprint| из этого модуля выводит данные с отступами и
переносами строк. Это особенно удобно для вложенных словарей.

Другое достоинство \spverb|pprint|~--- ограничение на глубину и длину коллекций. На
больших данных печать в лоб чревата задержкой. Функция \spverb|pprint| опирается на
глобальные настройки Clojure. Они определяют, сколько брать элементов из
коллекции и на какую глубину погружаться.

Не всегда удобно писать \spverb|(clojure.pprint/pprint data)| вместо `(println
data)\spverb|. Как и любой модуль, |clojure.pprint` сперва импортируют. Чтобы не
отвлекаться на импорт, сделаем так, что на время разработки вызов \spverb|println|
аналогичен \spverb|pprint|. Пропатчим функцию следующим образом:

\begin{verbatim}
(require 'clojure.pprint)

(alter-var-root
 (var println)
 (fn [_] clojure.pprint/pprint))
\end{verbatim}

Достаточно выполнить этот код один раз в любом месте проекта. Теперь вызов
\spverb|(println data)| напечатает данные так, как это делает \spverb|pprint|:

\begin{verbatim}
(println <vector-of-dicts-of-vectors...>)

;; [{:foo 42, :bar [1 2 3 4 5 {:foo 42, :bar [1 2 {#, #}]}]}
;;  {:foo 42, :bar [1 2 {:foo 42, :bar [1 2 {#, #}]}]}]
\end{verbatim}

Функция заменяет вложенные участки на символ \spverb|#|, чтобы не обрушить на вас лавину
данных. На глубину и длину печати влияют особые глобальные переменные. Позже мы
рассмотрим, как задать им другие значения.

Подмена \spverb|println| на \spverb|pprint| полезна только во время разработки. Чтобы не
повлиять на боевой режим, код с \spverb|alter-var-root| выносят в отдельный
модуль. Такие модули называют модулями среды, потому что они загружаются не
всегда, а по условию. Код с заменой печати логично поместить в файле
\spverb|env/dev/print.clj|. Проект настраивают так, что в режиме разработки он
просматривает дополнительные пути для загрузки кода.

Например, если это lein-проект, то в файле \spverb|project.clj| пишут следующее:

\begin{verbatim}
{:profiles
 :dev  {:source-paths ["env/dev"]}
 :test {:source-paths ["env/test"]}}
\end{verbatim}

Запись означает, что в режиме разработки мы получим доступ ко всем модулям из
папки \spverb|env/dev|, а во время прогона тестов~--- из \spverb|env/test|. Мы рассмотрим как
настроить проект в отдельной главе.

По аналогии с переходными коллекциями, патчинг должен быть изолирован от
остального кода. Его эффект замыкают в отдельной функции или даже модуле. Рядом
должен быть комментарий с описанием того, что произойдет при вызове и какую мы
преследуем цель.

\subsection{В боевом режиме}

[clj-book-exceptions]: /clj-book-exceptions/

Рассмотрим случай, когда \spverb|alter-var-root| полезен в промышленном запуске. В
предыдущей [главе об исключениях][clj-book-exceptions] мы отметили
проблему. Макросы \spverb|log/info|, \spverb|log/error| и другие принимают первым аргументом
исключение. При записи в файл мы не знаем, как система логирование запишет его в
текст. Длина трейса и цепочка исключений выглядят по-разному в зависимости от
бекенда (log4j, Logback и другие).

Мы написали функцию ex-print. Она принимает исключение и печатает его именно
так, как нужно нам. виде. Функция не вываливает стек-трейс на весь экран, а
обходит цепочку исключений и для каждого звена печатает класс, сообщение и
контекст:

\begin{verbatim}
(defn ex-print
  [^Throwable e]
  (let [indent "  "]
    (doseq [e (ex-chain e)]
      (println (-> e class .getCanonicalName))
      (print indent)
      (println (ex-message e))
      (when-let [data (ex-data e)]
        (print indent)
        (clojure.pprint/pprint data)))))
\end{verbatim}

Недостаток решения в том, что вместо \spverb|(log/error e)| приходится писать:

\begin{verbatim}
(log/error (with-out-str (ex-print e)))
\end{verbatim}

Это значительно длинее и вынуждает импортировать \spverb|ex-print| в каждый модуль, где
логируют исключения. Было бы удобней, если б мы по-прежнему писали `(log/error
e)\spverb|, а |ex-print` срабатывал самостоятельно за кадром. Это возможно с помощью
\spverb|alter-var-root|.

Отметим, что \spverb|log/error|, \spverb|log/info| и аналоги это не функции, а макросы. Макрос
это эфемерная сущность, на которую нельзя сослаться через var. Макрос живет
только во время компиляции программы. После компиляции на его месте остается
список функций и базовых форм. В этом и состоит трюк. Нельзя изменить макрос, но
можно подменить функции, на которые он опирается.

Модуль \spverb|clojure.tools.logging| он устроен так, что макрос \spverb|log/error| и другие
сводится к вызову функции \spverb|log/log*|. Это бутылочное горлышко, через которое
проходят все логи. Вот как выглядит ее сигнатура:

\begin{verbatim}
(defn log*
  [logger level throwable message])
\end{verbatim}

Параметр \spverb|throwable| это объект исключения или \spverb|nil|. Подменим \spverb|log*| на
анонимную функцию со следующей логикой:

- если \spverb|throwable| не nil, перевести исключение в текст;
- присоединить его к исходному сообщению через перенос строки;
- передать параметры в оригинальный \spverb|log*| с новым сообщением. При этом
  \spverb|throwable| будет nil. В нем отпала потребность, поскольку мы сделали его
  частью сообщения;
- если \spverb|throwable| был nil, вызвать \spverb|log*| не меняя входных параметров.

\begin{verbatim}
(defn install-better-logging
  []
  (alter-var-root
   (var clojure.tools.logging/log*)
   (fn [log*] ;; 5
     (fn [logger level throwable message]
       (if throwable
         (log* logger level nil
               (str message \newline
                    (with-out-str
                      (ex-print throwable))))
         (log* logger level throwable message))))))
\end{verbatim}

Хитрость кроется в строке 5. Анонимная функция, переданная в \spverb|alter-var-root|,
принимает прежнее значение переменной. Это оригинальная функция
\spverb|clojure.tools.logging/log*|, и параметр \spverb|log*| ссылается на нее. Новая функция
замкнута относительно переменной log* и может ее вызывать.

Получился своего рода декоратор. Новая версия \spverb|log*| только меняет входные
параметры и вызывает старую функцию. После вызова \spverb|(install-better-logging)|
логирование исключений изменится. Теперь достаточно написать \spverb|(log/error e)|,
чтобы ошибка была записана в файл именно так, как нужно нам.

Преимущество этого подхода в том, что поведение задано на уровне
Clojure-кода. Если потребуется улучшить логи, мы доработаем функцию \spverb|ex-print| в
любой момент. Это удобнее, чем наследовать Java-класс от условного
\spverb|com.logging.ThrowableRenderer| и переопределять его методы.

Мощь \spverb|alter-var-root| компенсируется вредом ее необдуманного применения. Функция
нужна, чтобы точечно менять переменные в особых случаях. Прибегайте к
\spverb|alter-var-root| только если альтернативный путь существенно увеличивает код и
временные затраты.

\section{Немного о set!}

Мы уже упоминали об особенности \spverb|pprint| для красивой печати. Функция
замечательна тем, что проверяет вложенность и длину коллекций. Эти ограничения
помогут избежать ситуации, когда данные заливают несколько экранов, а редактор
тормозит на подсветке синтаксиса. Проверка на длину особенно важна, потому что
некоторые коллекции не просто велики, а бесконечны.

Длину и вложенность вывода определяют глобальные переменные \spverb|*print-length*| и
\spverb|*print-level*|. По умолчанию \spverb|*print-length*| равен 100. Это довольно много,
особенно если учесть, что элементами коллекции могут быть другие
коллекции. Например, результат запроса к БД это список словарей. Печать ста
словарей это дорогая операция, поэтому логично уменьшить \spverb|*print-length*| на
старте приложения. Для этого используют форму \spverb|set!|:

\begin{verbatim}
(set! *print-length* 8)
\end{verbatim}

Теперь печать бесконечной коллекции отобразит только первые 8 элементов:

\begin{verbatim}
(println (repeat 1))
;; (1 1 1 1 1 1 1 1 ...)
\end{verbatim}

Многоточие означает, что в коллекции есть и другие элементы, но их отбросили.

Вложенность или уровень коллекции это воображаемый индекс. Когда одна коллекция
становится элементом другой, ее индекс увеличивается на единицу.

Объявим вложенную структуру:

\begin{verbatim}
(def data {:foo {:bar {:baz [42]}}})
\end{verbatim}

Так выглядит вывод при разных значениях \spverb|*print-level*|:

\begin{verbatim}
(set! *print-level* 4)
(println data)
;; {:foo {:bar #}}

(set! *print-level* 2)
(println data)
;; {:foo #}
\end{verbatim}

При нулевом уровне увидим только символ \spverb|#|.

Всего в Clojure около десяти переменных с "ушками". Две из них мы уже
рассмотрели, это \spverb|*print-length*| и \spverb|*print-level*| для настроек
печати. Перечислим несколько других:

- \spverb|*warn-on-reflection*|: если истина, компилятор покажет предупреждение в тех
  местах, где не удалось вывести Java-тип. Это продвинутая техника, и мы
  рассмотрим ее в другой главе.

- \spverb|*assert*|: если ложь, отключает \spverb|assert|-макрос. Это особая форма, которая
  проверет выражение на истинность. Если выражение ложно, форма выбрасывает
  исключение. По умолчанию assert-ы включены, ими пользуются на этапах
  разработки и тестирования. В боевом режиме их выключают, чтобы повысить
  произвоительность.

- \spverb|*in*|, \spverb|*out*|: каналы стандартного ввода и вывода. Это объекты \spverb|Reader| и
  \spverb|Writer|, откуда платформа читает и пишет данные.

Эти и другие переменные изменяют формой \spverb|set!|. Значение зависит от семантики
переменной. Рассмотрим несколько примеров.

Показывать предупреждения, если компилятор не смог вывести тип:

\begin{verbatim}
(set! *warn-on-reflection* true)
\end{verbatim}

Отключить \spverb|assert|-формы:

\begin{verbatim}
(set! *assert* false)
(assert (get {:foo 42} :bar))
;; won't throw an exception
\end{verbatim}

Направить вывод в файл: все, что напечатано функциями семейства \spverb|print|,
появится не в консоли, а в файле. Полезно, когда объем данных слишком велик. При
выводе в файл редактор не тратит ресурсы на подсветку синтаксиса.

Ниже мы создаем объект \spverb|Writer| с файлом \spverb|out.log| в качестве точки сбора
данных. Затем назначаем переменной \spverb|*out*| этот объект.

\begin{verbatim}
(require '[clojure.java.io :as io])

(def f (io/writer "out.log"))
(set! *out* f)
\end{verbatim}

Теперь \spverb|print| молча возвращает nil. Результат печати вы найдете в файле
\spverb|out.log| в той директории, откуда запустили REPL.

Выражения \spverb|(set! *<something>* <value>)| размещают в коде по такому же принципу,
что и \spverb|alter-var-root|. Если значение требуется только для определенного режима,
форму \spverb|set!| записывают в модуле среды, например \spverb|env/dev/settings.clj|.

Утилиты проектов принимают словарь глобальных переменных. При старте они
автоматически назначают переменным значения. Например, конфигурация lein
учитывает ключ \spverb|:global-vars|. Еще большей гибкости можно добиться, если
указывать переменные в профилях. Ниже мы задаем разные значения глобальным
переменным в зависимости от режима разработки (dev) или сборки (uberjar).

\begin{verbatim}
{:profiles
 :dev {:global-vars {*warn-on-reflection* true
                     *assert* true}}
 :uberjar {:global-vars {*warn-on-reflection* false
                         *assert* false}}}
\end{verbatim}

С помощью \spverb|set!| нельзя изменить пользовательские переменные, даже если они с
"ушками" и помечены как динамические. Пример ниже бросит исключение:

\begin{verbatim}
(def ^:dynaimc *data* nil)
(set! *data* {:user 1})
;; Unhandled java.lang.IllegalStateException
;; Can't change/establish root binding of: *data* with set
\end{verbatim}

Воспользуйтесь функцией \spverb|alter-var-root|, которую мы рассмотрели выше.

\section{Изменения в контексте. Binding}

Техники, которые мы рассмотрели к этому моменту, обладают одинаковым
свойством. Их эффект длится до конца работы программы. Изменения в атомах,
транзиентных коллекциях и глобальных переменных называют персистентными
(анг. persistent~--- постоянный).

Может показаться странным, но это не всегда то поведение, на которое мы
рассчитываем. Иногда требуется изменить данные временно. Например, глобальная
переменная равна X, но этот участок кода ожидает Y.

Технически возможно построить временные изменения на базе постоянных. Например,
вызывать \spverb|alter-var-root| с новым и старым значениями на границах блока кода. Но
такой подход влечет две проблемы: изоляцию изменений и их откат.

Проблема изоляции состоит в том, что, как правило, временные изменения
происходят в рамках одного потока. Это значит, что если мы временно изменили
глобальную переменную с X на Y, то в другом треде она по-прежнему будет X. Это
важно, поскольку мы физически не можем знать, какой участок кода сейчас
выполняет другой тред. Значит, мы не можем позволить себе менять глобальные
переменные для всех тредов в произвольный момент.

Проблема отката означает, что изменения в данных необходимо отменить. После
выполнения блока кода данные должны быть в точности такими же, как и до
него. Практика, когда код оборачивают в \spverb|alter-var-root| или \spverb|set!|, плоха тем,
что одна из форм, чаще заключительная, может потеряться во время
рефакторинга. Это чревато странным поведением программы и трудной отладкой.

С этого момента и до конца главы мы будем рассматривать временные изменения
данных. Clojure предлагает несколько форм, чтобы выполнить произвольный код в
контексте переменных с другими значениями. Первая из них это \spverb|binding|
(анг. "связывание").

Синтаксис \spverb|binding| аналогичен форме \spverb|let|: это форма связывания и произвольный
код. Форма связывания это вектор, где перечисляют символы переменных и их
значения. Символы должны ссылаться на уже объявленные переменные. Произвольный
код, или тело макроса, будет исполнен в рамках этих переменных с новыми
значениями. Результат \spverb|binding| это последнее выражение его тела. Изменения,
которые оказывает \spverb|binding|, протекают в рамках текущего треда и не влияют на
соседние.

\subsection{Динамические переменные}

\spverb|Binding| работает только с динамическими переменными. Компилятор считает
переменную динамической, если ей назначен тег \spverb|^:dynamic|. Это сокращенная форма
\spverb|^{:dynamic true}|:

\begin{verbatim}
(def ^:dynamic *server* nil)
;; or
(def ^{:dynamic true} *server* nil)
\end{verbatim}

Словарь с крышкой в \spverb|def|-определении называют метаданными. Это дополнительные
параметры будущей переменной. В данном случае мы сообщаем компилятору, что
переменная динамическая, то есть будет изменена в будущем.

Глобальные переменные принято выделять "ушками", то есть звездочками по краям. В
английском языке такую запись называют earmuffs syntax. Правило было принято еще
в старых Lisp-системах, и Clojure следует традиции.

Ушки и динамичность связаны между собой. Если переменная с ушками, но не
динамическая, компилятор предупредит о возможной ошибке:

\begin{verbatim}
(def *server* nil)
;; Warning: *server* not declared dynamic and thus [...]
;; Please either indicate ^:dynamic *server* or change the name.
\end{verbatim}

Сами по себе ушки не делают переменную динамической, это просто соглашение. Если
переменная не динамическая, \spverb|binding| бросит исключение:

\begin{verbatim}
(binding [*server* {:port 8080}]
  (println *server*))
;; Execution error (IllegalStateException)
;; Can't dynamically bind non-dynamic var: *server*
\end{verbatim}

\subsection{Отказ от set!}

Вспомним пример с ограничениями на длину и глубину печати. Чтобы ограничить
вывод, мы писали что-то вроде:

\begin{verbatim}
(set! *print-level* 4)
(println data)
\end{verbatim}

На самом деле это плохой пример. Он нарушает принципы изоляции и отката, которые
мы только что рассмотрели. Изменение переменной \spverb|*print-level*| не изолировано и
повлияет на всю систему глобально. Если в этот момент другой тред что-то
напечатает, мы увидим результат с максимальной вложенностью 4, что отличается от
величины, которую задали на старте приложения. Сразу после оператора `(println
data)\spverb| следует восстановить прежнее значение |*print-level*`. Однако, об этом
легко забыть.

Вот как выглядит правильный вариант для печати с особыми настройками:

\begin{verbatim}
(binding [*print-level* 8
          *print-length* 4]
  (println {:foo {:bar {:baz (repeat 1)}}}))
;; {:foo {:bar {:baz (1 1 1 1 ...)}}}
\end{verbatim}

Этот код избавлен от упомянутых недостатков. Вне формы \spverb|binding| переменные
останутся с прежними значениями, а соседние треды не заметят изменений.

Вспомним, как мы перенаправили печать данных в файл. Мы назначили переменной
*out* специальный объект FileWriter:

\begin{verbatim}
(set! *out* (io/writer "data.txt"))
\end{verbatim}

Этот метод не лишен недостатков. Во-первых, не ясно, кто и в какой момент
закроет файл. Программа может завершиться аварийно, и мы потеряем часть
данных. Должен быть какой-то фоновый обработчик, который закроет файл даже в
случае сбоя.

Во-вторых, не всегда данные пишут в один и тот же файл. Возможно, мы ожидаем
увидеть часть данных на эеране, а другую часть в файле. Как мы выяснили,
переключать \spverb|*out*| глобально это опасная практика: повышается риск
непредсказуемого вывода.

Правильный подход в том, чтобы открыть файл и временно связать с ним переменную
вывода:

\begin{verbatim}
(with-open [out (io/writer "dump.edn")]
  (binding [*out* out]
    (clojure.pprint/pprint {:test 42})))
\end{verbatim}

Объединим оба примера в функцию для сброса данных в файл. Такая функция полезна
для отладки больших данных. Она принимает путь к файлу и произвольное
значение. Внутри функция связывает стандартный вывод с временно открытым файлом
и печатает данные. Чтобы сделать вывод более детальным, назначаем переменным
печати большие значения:

\begin{verbatim}
(defn dump-data
  [path data]
  (with-open [out (io/writer path)]
    (binding [*out* out
              *print-level* 32
              *print-length* 256]
      (clojure.pprint/pprint data))))
\end{verbatim}

Вот как работает сброс данных в произвольный файл:

\begin{verbatim}
(dump-data "sample.edn" {:foo [1 2 3 {:foo [1 2 3]}]})
\end{verbatim}

и их восстановление:

\begin{verbatim}
(-> "sample.edn" slurp read-string)
;; {:foo [1 2 3 {:foo [1 2 3]}]}
\end{verbatim}

Доработайте функцию \spverb|dump-data| так, чтобы можно было передать значения
\spverb|*print-level*| и \spverb|*print-length*|. В идеале это третий необязательный параметр
opt, словарь, в котором функция ищет дополнительные настройки.

\subsection{Пример с переводом строк}

Приведем пример из реального проекта, когда \spverb|binding| чрезвычайно полезен. Речь
пойдет об интернационализации веб-приложения. Под термином понимают вывод текста
на разных языках в зависимости от настроек пользователя.

В приложениях с переводами текст хранят в виде огромных словарей. Они состоят из
двух уровней: локали и тегов. Локаль это международный код языка, например ru,
en. Локаль может состоять из доменов, разделенных подчеркиванием или дефисом,
например \spverb|en_US| или \spverb|en_GB|. В данном случае US и GB означают американский и
британский диалекты английского. В восточных языках встречается даже тройная
вложенность доменов, чтобы обозначить локальный диалект в провинции.

Под тегом понимают короткую машинную строку. Она описывает семантику фразы,
которая позже заменит ее в момент перевода. Например, по тегу \spverb|ui/add-to-cart|
становится ясно, что это надпись на кнопке "добавить в корзину".

В зависимости от фреймворка или библиотеки словари находятся в коде, в файлах
или базе данных. Но принцип интернационализации остается прежним: по локали и
тегу библиотека совершает т.н. lookup, то есть поиск перевода в
дереве. Изобразим наивную реализацию такого подхода на Clojure:

\begin{verbatim}
(def tr-map
  {:en {:ui/add-to-cart "Add to Cart"}
   :ru {:ui/add-to-cart "Добавить в корзину"}})

(defn tr
  [locale tag]
  (get-in tr-map [locale tag]
          (format "<%s%s>" locale tag)))
\end{verbatim}

Функция tr возвращает перевод то локали и тегу. Если перевод не найден,
результатом будет машинное выражение, например \spverb|<:en:ui/sign-in>|.

Недостаток функции в том, что каждый раз ей нужно передавать локаль. Это
утомительно, особенно с учетом того, что локаль вычисляется один раз на старте
запроса и не меняется в процессе. Иногда на один запрос требуется перевести
полсотни фраз. Это 50 вызовов tr, и каждый дополнительный параметр зашумляет
код.

Еще один недостаток локали в том, что место ее определения слишком далеко от
мест перевода. Между этими участками кода лежит дистанция как в физическом, так
и ментальном плане. Под физическим мы понимаем стек вызовов. Обычно мы
определяем локаль в особом middleware. Это может быть параметр адресной строки
(/?lang=ru) или поддомен (en.wikipedia.org). Но middleware считаются логикой
высокого уровня, а переводы расположены где-то внизу внутри обработчика. Даже
если сообщить запросу поле \spverb|:locale|, будет физически трудно спустить его до
уровня перевода и передать \spverb|(:locale request)| в каждый вызов \spverb|tr|.

Под ментальной дистанцией имеют в виду следующее. На уровне переводов нам не
хочется знать, откуда приходит локаль. Наоборот, эти сведения избыточны, потому
что они завязывают нас на конкретную реализацию. Это особенно очевидно, когда мы
работаем с шаблонной системой, устроенной по принципу Django.

Аналог такой системы в Clojure называется Selmer. Шаблонизатор Django
подразумевает, что у вас обычные HTML-файлы со вставками в фигурных
скобках. Выражения в скобках вычисляются в конечные значения. В шаблонной
системе выделяют фильтры. Это символьное обозначение функции, которую нужно
применить к переменной. Например, запись:


\begin{verbatim}
<p>{{ user.name|lower }}</p>
\end{verbatim}


означает, что между тегами параграфа следует разместить поле \spverb|:name| словаря
\spverb|user|. При этом привести имя к нижнему регистру. В Clojure эта запись выглядела
бы так:

\begin{verbatim}
(str/lower-case (:name user))
\end{verbatim}


Фильтром может быть любая функция, в том числе наша \spverb|tr|. Достаточно внести ее в
регистр фильтров. Нам бы хотелось, чтобы код шаблона выглядел так:


\begin{verbatim}
<div class="widget">
  <a href="/login">{{ "ui/log-in"|tr }}</a>
  <a href="/register">{{ "ui/register"|tr }}</a>
  <a href="/help">{{ "ui/help"|tr }}</a>
</div>
\end{verbatim}


Тогда фильтр "tr" должен быть функцией одной переменной. Она принимает машинную
строку и возвращает ее перевод.

Очевидно, локаль должна быть как-то предопределена. Мы должны сделать так, чтобы
в рамках конкретного запроса фильтр опирался на рассчитанную в middleware
локаль. При этом ни в коем случае не влиять на перевод в параллельных запросах.

Поможет локальное связывание через \spverb|binding|. Определим глобальную переменную
\spverb|*locale*|. В терминах Clojure такая переменная называется несвязанной, потому
что ей не сообщили значение. Можно рассматривать ее как ячейку, в которой еще
нет данных.

Изменим функцию \spverb|tr|: теперь она принимает только тег, а в качестве локали
ссылается на глобальную \spverb|*locale*|:

\begin{verbatim}
(def ^:dynamic *locale*)

(defn tr
  [tag]
  (get-in tr-map [*locale* tag]))
\end{verbatim}

Чтобы изолировать \spverb|*locale*| от внешних потребителей, предоставим специальный
макрос \spverb|with-locale|. Он выполняет блок кода в момент, когда переменная временно
связана с переданной локалью. Теперь любой перевод, вызванный внутри макроса,
сработает для этой локали:

\begin{verbatim}
(defmacro with-locale
  [locale & body]
  `(binding [*locale* ~locale]
     ~@body))

(with-locale :en
  (tr :ui/add-to-cart))
;; "Add to Cart"

(with-locale :ru
  (tr :ui/add-to-cart))
;; "Добавить в корзину"
\end{verbatim}

Напишем middleware, который определяет локаль по запросу. Для простоты решим,
что это параметр lang из адресной строки. Если не удалось найти параметр, берем
локаль по умолчанию. Весь нижележащий middleware-стек будет выполнен под
макросом \spverb|with-locale|:

\begin{verbatim}
(defn wrap-locale
  [handler]
  (fn [request]
    (let [locale (get-in request [:params "lang"] :en)]
      (with-locale locale
        (handler request)))))
\end{verbatim}



Наконец, напишем фильтр tr для шаблонной системы. Это обертка над одноименной
функцией. Внутри шаблона мы не можем указывать ключевые слова, только
строки. Это значит, вместо \spverb|{{ :ui/sign-in }}| пишут \spverb|{{ "ui/sign-in"}}|.
Фильтр \spverb|tr| переводит это строку в ключ, а затем ищет по нему
перевод. Функция \spverb|add-filter!| заносит функцию в регистр фильтров под именем
"tr".



\begin{verbatim}
(require '[selmer.filters :refer [add-filter!]])

(add-filter! :tr
 (fn [line]
   (-> line keyword tr)))
\end{verbatim}



Теперь мы не заботимся об источнике локали уровне перевода. С нашей точки зрения
ее предоставил кто-то другой, а кто и как именно в данном случае не важно. В
любой момент мы изменим код \spverb|with-locale| и \spverb|wrap-locale|, но это не отразится
на шаблонах. Запись \spverb|{{ "ui/log-in"|tr }}| останется прежней, даже если механизм
переводов изменится.



\section{Локальные переменные в контексте}

Форма \spverb|binding| связывает переменные с новыми значениями один раз. В блоке кода
невозможно задать одной из переменных новое значение. Это возможно только в
рамках вложенного \spverb|binding|, что не всегда удобно, особенно когда мы пишем
императивный код.

Макрос \spverb|with-local-vars| объявляет набор локальных переменных. Их особенность в
том, что внутри макроса им можно назначать произвольные значения. Каждая
переменная это маленький объект, для которого работают операции \spverb|get| и \spverb|set|,
то есть получить и установить значение.

Локальные переменные полезны, когда описывают запутанную бизнес-логику. Макрос
\spverb|with-local-vars| не сдвигает код вправо, как это делают \spverb|let| или
\spverb|binding|. Блок с локальными переменными выглядит линейно, его проще читать.

Форма \spverb|with-local-vars| похожа на \spverb|let|: это вектор связывания и произвольный
блок кода. Разница в том, что внутри макроса работают функции \spverb|var-get| и
\spverb|var-set|. С их помощью из переменных читают и записывают значения. Например,
если макрос задал переменную \spverb|a|, то форма \spverb|(var-set a 9)| установит ее
содержимое в 9.

Важно, что символ переменной вернет ее объект, а не значение. Убедимся в этом на
примере ниже:

\begin{verbatim}
(with-local-vars [a 0]
  a)
;; #<Var: --unnamed-->
\end{verbatim}

Выражение вернуло не ноль, а объект типа Var. Поэтому запись \spverb|(+ a 1)| приведет
к ошибке приведения типов.

Чтобы извлечь значение из переменной, ее следует разыменовать или
"дерефнуть". Для этого служит функция \spverb|var-get|; для краткой записи прибегают к
оператору \spverb|@|: \spverb|(+ @a 1)|.

\subsection{Императивный подход}

Выше мы приводили пример с обработкой дерева. Из массивной структуры данных
нужно извлечь несколько величин и вернуть их композицию: сумму, произведение или
другое выражение. В прошлый раз мы использовали атом. Теперь решим задачу на
локальных переменных.

Функция \spverb|calc-billing| рассчитывает сумму к оплате для клиента. Параметр data
это сводный отчет с метриками потребленных ресурсов. На уровне Clojure это
комбинация списков и словарей. Согласно бизнес-правилам, итоговую сумму находят
из трех составляющих. Каждую составляющую рассчитывают из данных рангом ниже и
так далее. Поскольку логика включает много условий, удобно выразить ее на
изменяемых переменных.

\begin{verbatim}
(defn calc-billing [data]
  (with-local-vars
    [a 0 b 0 c 0]

    ;; find a
    (when-let [usage (->some data :usage last)]
      (when-let [days (->some data :days first)]
        (var-set a (* usage days))))

    ;; find b
    (when-let [limits ...]
      (when-let [vms ...]
        (var-set b (* limits vms))))

    ;; find c
    ;; ...

    ;; result
    (+ (* @a @b) @c)))
\end{verbatim}

Локальные переменные не настолько продвинуты как атомы. Для переменных нет
аналога \spverb|swap!|, когда значение меняют функцией. Поэтому \spverb|with-local-vars| не
подходит для наращивания коллекций. Если user это локальный словарь, добавить к
нему новое поле будет затруднительно. Функция \spverb|var-set| может задать только
новый словарь, а комбинация \spverb|var-set| и \spverb|var-get| выглядит неуклюже:

\begin{verbatim}
(with-local-vars [user {:name "Ivan"}]
  ;; (var-set user assoc :age 33) ;; won't work
  (var-set user (assoc (var-get user) :age 33))
  @user)
\end{verbatim}

Макросом \spverb|with-local-vars| пользуются, когда сложная логика завязана на простых
типах (числах, строках). На локальных переменных удобно писать конечные автоматы
и алгоритмы с состоянием. Эта техника редко встречается в проектах на Clojure,
но в нужный момент сэкономит время и код.

\section{Глобальные изменения в контексте}

Преимущество \spverb|binding| в том, что изменения происходят только в текущем
треде. Вспомним пример с переменной \spverb|*out*|. Если беспорядочно менять ее в
процессе работы, получим непредсказуемый вывод. Говорят, что эффект binding
изолоированный или потокобезопасный, что расценивается как благо. И все же
бывают ситуации, когда изменения должны охватить систему глобально. Для этого
служит форма \spverb|with-redefs|.

Ее синтаксис похож \spverb|binding|: вектор связывания и произвольный блок кода. В
отличии от \spverb|binding|, эффект \spverb|with-redefs| распространяется на всю среду
исполнения. Это значит, изменения вступят в силу для каждого потока. Например,
веб-сервер обрабатывает сотни запросов в секунду в нескольких потоках. Если одна
из страниц выполняет часть логики в \spverb|with-redefs|, это повлияет на параллельные
запросы. Аналогично \spverb|binding| и \spverb|let|, изменения откатываются в момент выхода из
макроса.

Наивный пример ниже объясняет принципы \spverb|with-redefs|. Мы подменяем функцию
\spverb|println| на суррогат, который печатает фиксированный текст.

В теле макроса мы запускаем футуру с телом \spverb|(println 42)|. Футура (анг. future,
будущее) или фьючер это особый объект из области многопоточности. Футура
принимает блок кода и исполняет его в пуле тредов. В таком пуле каждый его тред
никогда не завершается, а только помечается как занятый или свободный. Если тред
свободен, он принимает задачу от футуры, исполняет ее и возвращает
результат. Футура это посредник между клиентом и внутренним механизмом
многопоточности.

Если коротко, тело \spverb|(println 42)| будет выполнено в другом потоке. Оператор \spverb|@|
перед футурой предписывает ждать до тех пор, пока не будет получен результат из
пула. Код ниже напечатает "fake print":

\begin{verbatim}
(with-redefs
  [println (fn [_] (print "fake print"))]
  @(future (println 42)))
;; fake print
\end{verbatim}

Заметим, что если убрать оператор \spverb|@|, футура напечатает 42. Причина в том, что
на запуск футуры, передачу задания в пул, исполнение и остальные шаги требуется
время. С машинной точки зрения это сложный цикл, каждый этап которого занимает
такты проццессора. Без оператора \spverb|@| мы только запускаем футуру и сразу выходим
из \spverb|with-redefs|. Пул тредов доберется до задания \spverb|(println 42)| уже в тот
момент, когда эффект макроса закончился.

Изменения в нескольких потоках это особая веха в разработке ПО. На эту тему
пишут книги, этому учатся годами. Мы коснемся многопоточности в будущих главах
книги, а пока что рассмотрим пример с \spverb|with-redefs| из реального проекта.

Документация \spverb|with-redefs| подчеркивает, что макрос особенно полезен для
тестирования. Это связано с тем, что иногда приложение опирается на сторонние
сервисы. Например, географичесий поиск или граф связей социальных
сетей. Некоторые сервисы отвечают долго, поэтому к ним обращаются в фоне.

При тестировании возникает проблема доступа к этим сервисам. Нельзя допустить,
чтобы на каждый прогон тестов приложение обращалось к настоящему ресурсу. Это
усложняет экосистему, влечет утечку ключей, исчерпывает квоту на доступ к
сервису.

Идея в том, чтобы с помощью \spverb|with-redefs| заменить ключевые функции, которые
обращаются в сеть. Тогда кроме нормального поведения мы сможем имитировать
ошибки. Это возможно, если в качестве замены передать функцию, которая бросает
нужное исключение.

\subsection{Приложение с координатами}

Предположим, мобильное приложение отправляет на сервер текущее положение
пользователя. Это пара чисел: долгота и широта. Позже пользователь просматривает
историю путешествий. Очевидно, в списке локаций он ожидает не машинные цифры, а
названия мест и стран с фотографией. Поэтому для каждой пары координат мы должны
найти данные об этом месте.

Технически это устроено следующим образом. Страница \spverb|POST /location| принимает
коорданаты в JSON-теле запроса. Чтобы узнать данные о месте, мы посылаем запрос
в гео-сервис Гугла. Извлекаем из ответа основные поля и записываем их в базу
вместе с координатами. Затем возвращаем ответ 200 OK. Для мобильного приложения
это знак, что новая локация записана в базу.

Начальная версия кода. В данном случае функция \spverb|geo/place-info| обращается к
серверам Гугла. Она возвращает словарь с ключами \spverb|:title|, \spverb|:country|,
\spverb|:image_url| и другими. Мы объединяем эти данные с координатами и записываем в
базу.

\begin{verbatim}
(defn location-handler
  [request]
  (let [{:keys [params]} request
        point (select-keys params [:lat :lon])
        place (geo/place-info point)]
    (db/create-location (merge {} place point))
    {:status 200 :body "OK"}))
\end{verbatim}

В код закрался неочевидный недостаток. Пока мы извлекаем данные из Гугла,
мобильное приложение ждет ответа. С машинной точки зрения это долго, ведь сеть
не гарантирует мгновенную доставку данных. Чем больше мобильных клиентов
отправляют координаты на сервер, тем больше сетевых запросов мы посылаем в
Гугл. На сервере все больше висящих сетевых соединений, число запросов в секунду
снижается. Это дегенеративное поведение системы.

Заметим, что \spverb|geo/place-info| не гарантирует, что все данные получится извлечь
за один запрос. API Гугла со временем меняются. Например, поля с фотографиями
вполне могут переехать из условного geosearch в photosearch, что порождает
второй запрос.

Быстрое решение проблемы в том, чтобы записать координаты в базу и сразу же
ответить мобильному клиенту. А сбор данных о месте вынести в футуру. Тем самым
мы сократим время ожидания клиента. Теперь мобильное приложение ждет только
запись в базу и запуск футуры, что намного меньше, чем несколько запросов в
сеть.

В новой версии функция \spverb|db/create-point| записывает коорданаты и возвращает id
новой записи. Этот id нужен, чтобы позже обновить локацию данными о месте. Поиск
данных и запись в базу протекают в футуре.

\begin{verbatim}
(defn location-handler
  [request]
  (let [{:keys [params]} request
        point (select-keys params [:lat :lon])
        row-id (db/create-point point)]
    (future
      (let [place (geo/place-info point)]
        (db/update-place row-id place)))
    {:status 200 :body "OK"}))
\end{verbatim}

Заметим, что быстрое решение не значит лучшее. В нашем случае возможны несколько
вариантов, например, с фоновым обработчиком или очередью задач. Но они дольше в
реализации, а вариант с футурой затрагивает лишь три строки. Это дешевое
временное решение, которое даст время на поиск оптимального.

\subsection{Тесты}

Напишем тест для этого обработчика. Чтобы не обращаться к реальным серверам
Гугла, временно заменим функцию \spverb|geo/place-info|. Для полноты картины проверим,
как поведет себя обработчик, если \spverb|geo/place-info| бросит сетевое исключение. В
таких случаях целевую функцию заменяют на анонимную, которая кидает нужное
исключение.

Каждый тест будет начинаться с выражения \spverb|with-redefs| для замены
\spverb|geo/place-info|. Чтобы уменьшить код, напишем макрос \spverb|with-place-info|. Он
принимает будущий результат функции и тело теста:

\begin{verbatim}
(defmacro with-place-info
  [result & body]
  `(with-redefs [geo/place-info
                 (fn [~'point] ~result)]
     ~@body))
\end{verbatim}

Вот как выглядит тест для положительного сценария. Внутри макроса
\spverb|with-place-info| вызов \spverb|geo/place-info| вернет переданный словарь. Мы вызываем
обработчик с координатами и проверяем, что получили код 200. Затем мы должны
убедиться, что футура извлекла данные и записала их базу. Добавляем небольшую
задержку и читаем из базы последнюю локацию. В ее полях должны быть данные из
тестового словаря.

\begin{verbatim}
(deftest test-place-ok
  (with-place-info
    {:title "test_title"
     :country "test_country"}

    (let [request {:params {:lat 11.111 :lon 22.222}}
          {:keys [status body]} (location-handler request)]

      (is (= 200 status))
      (is (= "OK" body))

      (Thread/sleep 100)

      (let [location (db/get-last-location)
            {:keys [title country]} location]

        (is (= title "test_title"))
        (is (= country "test_country"))))))
\end{verbatim}

Чтобы проверить, как поведет себя приложение во время ошибки, смоделируем
негативный сценарий. Пусть при обращении к \spverb|geo/place-info| будет выброшено
исключение с кодом 429. Такое возможно на практике, когда превышен лимит на
число запросов к службам Гугла. Объявим такое исключение:

\begin{verbatim}
(def ex-quota
  (ex-info "429 Quota reached"
           {:status 429
            :headers {}
            :body {:error_code :QUOTA_REACHED
                   :error_message "..."}}))
\end{verbatim}

и напишем тест:

\begin{verbatim}
(deftest test-place-quota-reached
  (with-place-info
    (throw ex-quota)

    (let [request {:params {:lat 11.111 :lon 22.222}}
          {:keys [status body]} (location-handler request)]
      ;; ...
      )))
\end{verbatim}

Текущая версия \spverb|location-handler| не перехватывает потенциальные ошибки, поэтому
тест выше пройдет неудачно. Подумайте, как улучшить обработчик страницы и тест к
нему для случая с негативным HTTP-ответом.

Возможен сценарий, когда мы получили от сервиса даже негативный ответ. Например,
из-за сбоя в сети на нашей стороне. Это значит, что вызов \spverb|geo/place-info|
бросает особое сетевое исключение. Напишем отдельный тест для недоступной сети:

\begin{verbatim}
(deftest test-place-conn-err
  (with-place-info
    (throw (new java.net.ConnectException "test_timeout"))
    ;; ...
    ))
\end{verbatim}

Прием с подменой функций и классов называется мок или мокинг (анг. mock~---
поделка). Мы подробно рассмотрим тесты в следующих главах и познакомимся с
другими техниками. Пока что заметим, что макрос with-redefs это простой способ
что-то "замокать", то есть подменить окружение на время тестов.

Макрос \spverb|with-redefs| это улучшенный вариант формы \spverb|with-redefs-fn|. Их отличия в
синтаксисе. Макрос \spverb|with-redefs| напоминает привычные \spverb|let| и \spverb|binding|. Это
вектор связывания и блок кода. Макрос \spverb|with-redefs-fn| принимает словарь и
функцию без аргументов. Ключи словаря это переменные, то есть объекты
\spverb|Var|. Функция будет вызвана в момент, когда каждое значение словаря заменит
парную ему переменную.

Пример с заменой \spverb|geo/place-info| другим способом выглядит так. Напомним, что
синтаксис \spverb|#'<something>| аналогичен \spverb|(var <something>)|:

\begin{verbatim}
(with-redefs-fn
  {#'geo/place-info (fn [point] {:title "test"})}
  (fn []
    (geo/place-info {:lat 1 :lon 2})))
\end{verbatim}

Если тело макроса это одна большая форма, например, \spverb|let|, то не обязательно
оборачивать его в \spverb|(fn [])|. Достаточно подставить спереди знак \spverb|#|, чтобы
превратить форму в анонимную функцию.

\begin{verbatim}
(with-redefs-fn
  {#'geo/place-info (fn [point] {:title "test"})}
  #(let [point {:lat 1 :lon 2}
         place (geo/place-info point)]
     ;; ... do something else
     ))
\end{verbatim}

Недостаток формы \spverb|with-redefs-fn| в том, что ее синтаксис более шумный. Легко
забыть, что тело должно быть не произвольным блоком кода, а функцией. Это делает
форму непохожей на другие макросы Clojure. \spverb|With-redefs| скрывает эти отличия, и
по возможности следует пользоваться им.

\section{Все вместе}

Clojure предлагает разнообразные приемы для изменения данных. Но в отличии от
императивных языков с их знаком равенства, в Clojure пользуются специальными
формами. Язык спроектирован так, что к изменениям прибегают выборочно. Это
считается продвинутым уровнем, к которому переходят после азов неизменяемости.

К этому моменту читатель может запутаться, в чем разница между отдельными
техниками и когда ими пользоваться. Перечислим объекты и функции, рассмотренные
выше и типовые сценарии, когда они полезны.

\emph{Атом} это объект-обертка вокруг целевого значения. Чтобы получить его
содержимое, применяют оператор \spverb|@| или форму \spverb|(deref <atom>)|. Можно сообщить
атому новое значение функцией \spverb|reset!|. Но чаще атом изменяют итеративно с
помощью функции, которая рассчитывает новое значение на базе старого.

В атомах хранят состояние отдельных частей проекта. Это счетчики, сессии,
локальный кеш для ускорения расчетов. Атомы участвуют в императивном коде как
изменяемые переменные. Иногда в атомах хранят состояние модуля, например,
текущее подключение к БД. Но в этом случае перед ним приходится ставить оператор
\spverb|@|, что не всегда удобно.

Особый объект \emph{volatile} это облегченный вариант атома. В отличии от него,
volatile не поддерживает валидаторы и вотчеры.

Переходные или \emph{транзиентные коллекции} это особая форма их постоянных
аналогов. Когда коллекция транзиентна, меняются ее внутренные элементы. На таких
коллекциях работают функции \spverb|conj!|, \spverb|assoc!| и другие с восклицательным знаком
на конце. Особая функция persistent! запечатывает изменяемую коллекцию и
возвращает ее постоянную версию.

Транзиентные коллекции полезны на больших объемах данных, поскольку быстрее
неизменяемых. Изменять коллекцию можно только из одного потока. Изменения должны
быть изолированы строго внутри функции. Другие части кода не должны знать о том,
что внутри коллекция изменяется. Обмен переходными коллекциями между функциями
считается грубой ошибкой.

Функция \emph{alter-var-root} заменяет \spverb|def|-определение на произвольное значение. С
помощью функции можно вторгнутся в чужое пространство имен и что-то исправить
уже после его загрузки. Как правило, к \spverb|alter-var-root| прибегают, чтобы
скорректировать код, которым мы не распоряжаемся. Например, улучшить
логирование, улучшить поведение чужой функции.

Форма \emph{set!} назначает новое значение глобальной переменной Clojure. Обычно
это служебные переменные с "ушками". Форма редко встречается в коде, чаще всего
значения этих переменных задают в настройках проекта.

Форма \emph{binding} выполняет произвольный код в рамках временных
изменений. Макрос связывает динамические переменные с новыми значениями на
период его работы. Чтобы переменная была динамической, в момент определения ей
сообщают флаг \spverb|^:dynamic|. Синтаксически такие переменные выделяют "ушками".

К binding прибегают, чтобы временно назначить глобальным переменным Clojure
другие значения. Например, направить стандартный вывод в файл для конктретного
участка кода. Эффект binding действует только в текущем потоке. В других случах
макрос сокращает число аргументов для функции. Например, любой вызов tr внутри
формы \spverb|(with-locale :en)| вернет перевод для английского языка.

Макрос \emph{with-local-vars} выполняет тело в контексте произвольных
переменных. Эти переменные похожи на объекты с двумя действиями: прочитать и
записать значения. Форма полезна, когда имеем дело со сложной императивной
логикой. При выходе из макроса переменные становятся недоступны.

Конструкция \emph{with-redefs} временно изменяет \spverb|def|-определения. В отличии от
\spverb|binding|, она действует глобально. Все фоновые процессы (треды, футуры,
агенты), если они работают в момент действия макроса, подхватят
изменения. Технически \spverb|with-redefs| это обертка над более низкоуровневой формой
\spverb|with-redefs-fn|. В основном \spverb|with-redefs| используют чтобы подготовить систему
к прогону тестов (мокинг).
