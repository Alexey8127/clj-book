\chapter{Что дальше}

Дорогой читатель! Если вы дошли сюда от начала книги, примите искренние
поздравления. Это был трудный путь, но с наградой в конце. Теперь вы знаете
больше, ст\'{о}ите больше и можете сделать больше.

Чтобы удержать знания, поскорей закрепите их практикой. Устройтесь в проект на
Clojure и отточите навыки в работе. Если компания не использует этот язык,
предложите руководству эксперимент. Заинтересуйте коллег, проведите мастер-класс
и~сделайте проект на Clojure. Презентуйте его руководству и опишите
преимущества: неизменяемость, скорость разработки, простота.

Русское сообщество развивается. Если возникли трудности, обращайтесь
в~Телеграм-канал \verb|clojure_ru|. На канале \verb|clojure_jobs| публикуют
резюме и~вакансии, связанные с~Clojure. Автор отвечает на письма по адресу
\verb|ivan@grishaev.me| и в~блоге \verb|grishaev.me|.

Желаю читателю успеха во всех начинаниях.

\vspace{1em}

\noindent
\textit{Иван Гришаев\\Россия, Швейцария\\2019~--- 2020}
