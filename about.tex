
\section*{Об этой книге}

У вас в руках книга о языке программирования Clojure. Это современный диалект
Лиспа на платформе JVM. От устаревших диалектов он отличается тем, что делает
ставку на функциональный подход и~неизменяемость данных. Язык устроен так, чтобы
решать сложные задачи простым способом.

Эта книга~--- не перевод, она изначально написана на русском языке. Вы не найдёте
тяжёлых предложений, в которых слышна английская речь. Вам не придётся читать
<<маркер>> вместо <<токен>> и~другую нелепицу. Термины написаны в том виде,
чтобы быть понятными программисту.

В книге нет вводной части, где написано, что скачать и установить. Также мы не
рассматриваем азы вроде чисел и строк. На тему введения в Clojure уже написаны
статьи и посты в блогах. Будет нечестно предлагать материал, где половина
повторяет сказанное ранее. Эта книга от начала и до конца~--- то, о чём ещё
никто не писал.

Другое её достоинство~--- упор на практику. Примеры кода взяты из реальных
проектов. Все техники и приёмы автор опробовал лично. В~описании проблем мы
отталкиваемся от того, что вас ждёт на~производстве. Покажем, где теория
расходится с практикой и что предпочесть в таком случае.

Коротко о том, что вас ждёт. Начнём с веб-разработки~--- вспомним протокол HTTP
и как с ним работать в Clojure. Затем рассмотрим Clojure.spec~--- библиотеку для
проверки данных. Третья глава расскажет про исключения, четвёртая~--- про
изменяемые данные. Далее переходим к конфигурации. В шестой главе знакомимся с
системами. В последней научимся писать тесты.

Даже если вы не любите Лисп и книга попала к вам случайно, не спешите её
откладывать. Clojure~--- это новые правила и другой мир, а~книга~--- шанс туда
попасть. Может быть, Clojure изменит ваше мнение о программировании. Обнаружит
вопросы там, где, казалось бы, всё решено.

Во втором издании исправлены опечатки и ошибки, в том числе смысловые. В
некоторых местах текст сокращён, в других, наоборот, стал подробней. По просьбе
читателей добавлены примеры к сложным разделам.

Желаем читателю терпения, чтобы прочесть книгу до конца.

\newpage

\section*{Благодарности}

Спасибо стартапу Flyerbee, моей первой работе на Clojure. Именно там я закрепил
скромные знания языка.

Я счастлив работать в компании Exoscale в окружении талантливых
инженеров. Многие вещи, не только технические, я узнал в~этом коллективе.

Спасибо Петру Маслову и Евгению Климову за крупные партии найденных
опечаток. Досбол Жантолин внёс важные замечания к последней главе. Молодцы все,
кто указал на ошибки в~комментариях в~блоге.

Алексей Шипилов адаптировал книгу под мобильные устройства и выполнил много
рутинных задач по вёрстке.

Вместе с Евгением Бартовым мы перевели книгу на английский язык. Во время
перевода Евгений нашёл неточности в русской версии, которые мы тоже исправили.

\ifx\PUBLISHER\RIDERO
Благодарю коллектив издательства Rider\'{o} за то, что взяли рукопись в
работу. Их усилиями вы читаете эту книгу сейчас.
\fi

\section*{Обратная связь}

Автор будет признателен за указанные опечатки и неточности. Присылайте их
по~адресу \EMAILLINK. Возможно, в промежутках между тиражами
получится обновить макет, и следующий читатель не~увидит ошибки, о~которой вы
сообщили. Автор учтёт все замечания при переводе на английский язык.
