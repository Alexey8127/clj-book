\chapter{Введение в~веб-разработку}

\begin{teaser}
В первой главе мы рассмотрим, как писать веб-приложения на Clojure. Поговорим о
том, как передают данные по протоколу HTTP. Какие абстракции строят поверх
протокола и что предлагает Clojure. Чем хорош функциональный подход и почему
разработка с~ним быстрее и~удобнее.
\end{teaser}

Каждый год компания Cognitect опрашивает\footnote{blog.cognitect.com/blog/2017/1/31/clojure-2018-results}
разработчиков на Clojure. Среди прочего в анкете вопрос о том, в какой области вы работаете? В~2010 году под
веб писали 50\% опрошенных, то есть каждый второй. К~2018 году
эта цифра выросла до 80\%. Это уже четыре человека из пяти. Похожую динамику показывают
опросы StackOverflow\footnote{https://insights.stackoverflow.com/survey/2018/}.
Согласно им, все больше разработчиков переходит в веб из смежных областей.

Если вы найдете работу на Clojure, скорее всего это будет веб-приложение. Мы
специально не говорим <<сайт>>, чтобы подчеркнуть: термин уходит в
прошлое. Сегодня веб-приложение это не только HTML с картинками. В широком плане
это сложный обмен данными по протоколу HTTP.

Протокол разработан для передачи разметки, но удивительным образом подошел для
данных. Для этого даже не пришлось менять стандарт. Причина кроется в его
изящном дизайне, простоте и гибкости. Прежде чем перейти к Clojure, освежим в
памяти устройство протокола. Из каких частей он состоит и по каким правилам с
ним работает сервер. Это важно, потому что языки и фреймворки меняются, а
протокол нет.

\section{Основы HTTP}

HTTP это протокол, который работает поверх TCP/IP. В широком смысле протоколы~---
это соглашения о том, как читать и писать данные. Обычно они записаны в официальных
документах. Для HTTP такой документ называется RFC~2616\footnote{https://tools.ietf.org/html/rfc2616}.
Разработчики браузеров сверяются с~ним, чтобы технология работала
на разных языках и платформах.

HTTP удобен тем, что это текст. Разработчику не нужно декодировать байты, чтобы
понять, что происходит. Протокол передает и бинарные данные, но главные части
все же выражены текстом. В HTTP различают запрос и ответ. Оба состоят из трех
частей: \emph{первая строка}, \emph{заголовки} и \emph{тело}.

\emph{Первая (стартовая)} строка несет самую важную информацию. Ее формат
различается для запроса и ответа. Для запроса это метод, путь и версия, для
ответа~--- статус, сообщение и версия.

\emph{Заголовки} это пары ключ-значение. В современных фреймворках их
описывают словарем. Заголовки содержат дополнительные сведения о запросе или
ответе. Например, заголовок \spverb|Content-Type| сообщает, как трактовать тело
запроса. Был ли это XML- или JSON-документ? Программа проверяет заголовок
и~читает тело должным образом.

После заголовков следует \emph{тело}. Им может быть что угодно~--- текст, данные
в виде <<поле=значение>>, JSON-документ, картинка, фильм, электронное
письмо. Стандарт допускает смешанный тип, \spverb|multipart-encoding|. Тело
такого запроса разбито на ячейки, в каждом из которых свое содержимое. Например,
текст, картинка, снова текст, двоичный файл.

Рассмотрим примеры HTTP-трафика. Именно в таком виде он передается
по сети. Это запрос к главной странице Google, поисковой терм~--- \spverb|clojure|:

\begin{verbatim}
GET /search?q=clojure HTTP/1.1
Host: google.com
Accept-Language: en-us
Accept-Encoding: gzip, deflate
User-Agent: Mozilla/4.0 (compatible; MSIE 6.0; Windows NT 5.1)
(blank line)
\end{verbatim}

\noindent
А это POST-запрос с JSON-документом в теле:

\begin{verbatim}
POST /api/users/ HTTP/1.1
Host: example.com
Content-Type: application/json
User-Agent: Mozilla/4.0 (compatible; MSIE 6.0; Windows NT 5.1)

{
  "username": "John",
  "city": "NY"
}
\end{verbatim}

\noindent
Ответ на этот запрос:

\begin{verbatim}
HTTP/1.1 200 OK
Date: Tue, 19 Mar 2019 15:57:11 GMT
Server: Nginx
Connection: close
Content-Type: application/json

{
  "code": "CREATED",
  "message": "A user has been created successfully"
}
\end{verbatim}

Видно, как изящно устроен протокол. Данные в нем расположены по убыванию важности.
Прочитав только первую строку, клиент и сервер готовы принять решение о~том,
что делать дальше.

Рассмотрим сценарий: в запросе указаны метод и путь \spverb|GET /about|, но
такой страницы не существует. Сервер проверит это заранее, например, сверив путь
с конфигурацией маршрутов. Когда маршрута нет, вернется ответ со статусом
404. Не придется читать тело запроса, что ускорит работу сервера. Получив ответ,
клиент прочитает статус 404 из первой строки, что означает ошибку. Логика
клиента может быть такова, что для негативного статуса он не читает тело.

Чтение и разбор содержимого это дорогая операция. Современные фреймворки
исключают случаи, когда чтение происходит зря. Например, по заголовку
\spverb|Content-Type| мы определяем, стоит ли читать тело. Наше приложение
работает только с JSON, поэтому для значения \spverb|text/xml| вернем
ошибку. Аналогично с заголовком \spverb|Content-Length|, где содержится длина
тела в байтах. Если значение больше заданного лимита, сервер отклонит запрос еще
до чтения.

Центральные параметры запроса это \emph{метод} и \emph{путь}. Путь указывает на
определенный ресурс на сервере. Иногда сервер трактует путь как файл
относительно заданной папки. Например, \spverb|/images/map.jpg| означает вернуть
такой файл из \spverb|/var/www/static|. Но чаще всего путь обрабатывают по
внутренним правилам. Ответом может быть не только файл, но и js-скрипт,
HTML-разметка или JSON-документ.

\emph{Метод запроса} означает действие, которые мы намерены выполнить над
ресурсом. Основные методы это \spverb|GET|, \spverb|POST|, \spverb|PUT| и
\spverb|DELETE|. Их семантика в том же порядке~--- прочитать, создать, обновить,
удалить ресурс. Так, запрос \spverb|POST /users/| означает создать пользователя,
а \spverb|GET /users/|~--- получить их список.

Главный параметр ответа это \emph{статус}~--- целое положительное число. Статусы
группируют по старшему разряду. Значения с 200 до 299 (или \spverb|2хх|) считают
положительными. Они означают, что сервер обработал запрос без ошибки.

Значения из группы \spverb|3хх| связаны с перенаправлением на другую страницу.
В заголовке \spverb|Location| приходит путь, на который нужно послать
новый запрос. Браузеры и HTTP-клиенты достаточно умны, чтобы сделать это
автоматически. Например, при запросе страницы \spverb|http://yandex.ru| вы
получите пустой документ с заголовком \spverb|Location: https://yandex.ru|.
Сервер обязывает нас перейти на безопасное соединение. Но мы даже не заметим
этого: браузер сам сменит адрес.

Статусы из группы \spverb|4хх| означают ошибку на стороне клиента. Чаще всего это 404~---
страница не найдена. На ошибочные данные сервер отвечает 400~--- \spverb|Bad request|.
Когда нет прав доступа, клиент получит код 403.

Статусы из группы \spverb|5хх| говорят об ошибке сервера или его
недоступности. Это деление на ноль, отказ базы данных, недостаток места
на диске.

Принято считать, что ответ со статусом вне диапазона \spverb|2хх| означает
ошибку. Многие HTTP-клиенты бросают исключение на ответ с негативным статусом.
Строго говоря, это верно только на высоком, абстрактном уровне. С точки
зрения протокола ответ \spverb|404 Not Found| такой же правильный, как и \spverb|200 OK|.

Когда действий с ресурсом много, используют другие, более редкие
методы. Например, \spverb|HEAD|~--- получить краткие сведения о сущности. Сервис
Amazon S3 в ответ на HEAD-запрос вернет только статус и заголовки. В них указаны
тип файла и его размер, контрольная сумма, дата последнего изменения. В данном
случае \spverb|HEAD|-запрос предпочтительней GET. Метаданные хранят в особом
хранилище отдельно от файла. Доступ к нему обычно быстрее, чем к файлу на диске.

Подход <<метод-ресурс>> со временем вырос в то, что сегодня называется
\spverb|REST|\footnote{restapitutorial.com}. Сторонники
\spverb|REST| выделяют сущности и \spverb|CRUD|-операции над ними (\textbf{C}reate,
\textbf{R}ead, \textbf{U}pdate, \textbf{D}elete). Считается хорошим подход,
когда сущность задают через путь, например \spverb|/users/1|, а операцию~---
методом. Если это создание или изменение, данные читают из тела, где обычно записан
JSON-документ. Мы не будем задерживаться на \spverb|REST|, потому что это всего лишь
свод рекомендаций, не идеальный и не единственный.

Отметим, что протокол не заставляет следовать этим правилам. Разработчик вправе
работать с HTTP так, как это удобно в данном проекте. Например, принимать только
POST-запросы с данными в теле. Или только GET с параметрами строки. Верную
стратегию определяют инструменты, бизнес и потребители сервиса.

\section{Возвращаясь к Clojure}

Современные фреймворки строят абстракции над HTTP. Разработчик не читает запрос
по байтам вручную. Эту задачу берут на себя библиотеки. Взамен разработчик
получает набор классов, чтобы с их помощью выразить логику приложения. Типичный
веб-проект на Python или Java это комбинация нескольких классов. Как правило,
это \spverb|Application|~--- главная сущность проекта. Класс \spverb|Router|
определяет, на какой обработчик переключить входящий запрос~--- \spverb|Request|.
Обработчик~--- это класс \spverb|Handler| с методами \spverb|.onGet|,
\spverb|.onPost| и тд. Ожидается, что он вернет экземпляр класса \spverb|Response|.

По такому принципу устроены промышленные веб-фреймворки: Django, Rails,
Symfony. Названия классов и их комбинация отличаются, но смысл остается
прежним. Это приложение, роутер, обработчик, запрос и ответ. Проблема в том, что
классы одного фреммворка не работаю с другим и наоборот.

Рассмотрим язык Python и фреймворки Django и Flask. Оба следуют той же
структуре. Так, запрос в Django описан в классе \spverb|django.http.HttpRequest|,
а во Flask~--- \spverb|flask.Request|. Даже беглого взгляда достаточно,
чтобы увидеть, насколько они отличаются. У классов разные методы и поля.
То, что есть в первом классе, отсутствует во втором. Использовать \spverb|flask.Request|
в проекте на Django невозможно.

Со временем проект увязает в архитектуре фреймворка. Переезд на другое решение
становится все менее возможным. Хорошие практики предлагают делить проект на
слои. Слой транспорта отвечает за ввод и вывод данных по протоколу HTTP. Слой
логики исполняет внутренний код, ничего не зная о HTTP. С таким подходом логика
не зависит от транспорта, и последний можно сменить в любой момент. Но на
практике это работает не всегда. По разным причинам слои перемешиваются. Мы
обещаем, что в следующий раз не допустим этого, и все повторяется сначала.

В Clojure другой подход.

Разработчик Джеймс Ривз (James Reeves) известен своим вкладом в экосистему
Clojure. Он разработал 60 библиотек\footnote{github.com/weavejester} для разных
задач. Нет такого проекта на Clojure, который бы не использовал его
наработки. Заслуга Джеймса в том, что он вывел стандарт веб-разработки для
Clojure на заре языка. Вместо того, чтобы писать фремворк под сиюминутные нужды,
он придумал, как сделать удобно всем.

Джеймс предложил несколько простых идей. Первая~--- если приложение принимает
запрос и возвращает ответ, то резонно \emph{выразить его функцией}. Действительно,
приложения бывают сколь угодно сложным. Они полагаются на сторонние сервисы,
машинное обучение, учитывают сотню фактов о пользователе. Но они принимают
запрос и возвращают ответ, поэтому на абстрактном уровне это \emph{функция}.

Скептики заметят, что мысль не нова. В том же Django обработчик бывает
не классом, а функцией. Разница в том, что отдельный обработчик~--- это
еще не приложение. Ему не хватает роутера, middleware и других абстракций.
Поэтому в Django или Flask выразить обработчик функцией~--- всего лишь
приятная возможность, сахар.

В Clojure приложение остается функцией на всех уровнях. Маршрутизатор~--- это
функция, которая принимает запрос, определяет обработчик и передает ему
управление. Middleware это тоже функция, которая дополняет приложение новой
логикой. Каждую тяжелую абстракцию (классы \spverb|Application|,
\spverb|Router|, \spverb|Handler|) в мире Clojure заменяют функцией. Это удобно,
потому что в отличии от классов функции стыкуются между собой.

Вторая идея Джеймса в том, чтобы зафиксировать структуру запроса и
ответа. Должны быть документы (не код, а \emph{именно документы}), где описаны
поля структур и их семантика. Это напоминает протокол HTTP. Спецификация
упрощает код и делает его переносимым. Два веб-проекта на Clojure принимают и
отдают одинаковые структуры данных. Разработчик очередного фреймворка учитывает
спецификацию. Если фреймворк следует стандарту, проще привлечь сообщество.

\subsection{Ring}

Идеи нашли применение в проекте Ring\footnote{github.com/ring-clojure/ring}.
Сегодня это стандарт веб-разработки на Clojure. Репозиторий содержит спецификацию
запроса и ответа и базовый код для работы с ними. Прилагаются основные
middleware, запуск Jetty-сервера и документация. Удивительно, как мало кода
понадобилось проекту, чтобы попасть на компьютер каждому Clojure-разработчику.

Со временем появился термин <<Ring-совместимость>>. Ему следуют все
фреймворки на Clojure. Типичное Ring-приложение запускается на многих платформах:
Jetty, Netty и других без изменений в коде.

Библиотека Ring разбита на отдельные части, чтобы можно было установить только
необходимое. Перечислим компоненты, которые понадобятся по ходу главы:


\begin{itemize}
\item
  \spverb|ring-core|~--- базовая функциональность: параметры, разбор тела, куки, сессии;

\item
  \spverb|ring-jetty-adapter|~--- запуск сервера из функции-приложения.

\end{itemize}

Первое приложение вы напишете даже без библиотеки. Вот оно:

\begin{verbatim}
(defn app [request]
  (let [{:keys [uri request-method]} request]
    {:status 200
     :headers {"Content-Type" "text/plain"}
     :body (format "You requested %s %s"
                   (name request-method) uri)}))
\end{verbatim}

Приложение извлекает путь и метод из запроса и формирует ответ. Его статус
положительный~--- 200. Мы выставили заголовок с типом документа <<простой
текст>>. Поле \spverb|:body| содержит строку, которую строим функцией \spverb|format|.

Поскольку \spverb|app| это функция, вызовем ее с различными запросами:

\begin{verbatim}
(app {:request-method :get :uri "/index.html"})
{:status 200,
 :headers {"Content-Type" "text/plain"},
 :body "You requested get /index.html"}

(app {:request-method :post :uri "/users"})
{:status 200,
 :headers {"Content-Type" "text/plain"},
 :body "You requested post /users"}
\end{verbatim}

Работает. Но пока что это структуры данных, и не ясно, что будет в
браузере. Запустим приложение в виде сервера.

Сервер это отдельная сущность. Он связывает структуры данных с сетевым
вводом-выводом. Сервер принимает приложение с параметрами и запускает в фоне
сложный процесс. Он слушает указанный порт и читает байты. Из бинарных
данных получается словарь запроса. В отдельном треде сервер вызывает проложение с
этим запросом. Результатом будет словарь ответа. Сервер переводит ответ в
байтовый поток и пишет в удаленный порт клиента. Этот цикл повторяется для
каждого запроса.

Добавим в проект зависимости:

\begin{verbatim}
[ring/ring-core "1.7.1"]
[ring/ring-jetty-adapter "1.7.1"]
\end{verbatim}

Запустим сервер:

\begin{verbatim}
(require '[ring.adapter.jetty :refer [run-jetty]])
(run-jetty app {:port 8080 :join? true})
\end{verbatim}

Происходит следующее. Мы импортировали в текущее пространство функцию
\spverb|run-jetty|. Она принимает два параметра~--- приложение и словарь
опций. Ключ \spverb|join?| определяет, будет ли заблокирован текущий тред до
конца работы сервера. Если передать \spverb|false|, сервер запустится в
фоне. Чтобы остановить его, нужно запомнить результат \spverb|run-jetty| в
переменную и вызвать у него метод \spverb|.stop|:

\begin{verbatim}
(def server
  (run-jetty app {:port 8080 :join? false}))

;; after a while
(.stop server)
\end{verbatim}

Если флаг был \spverb|true|, как в первом случае, главный поток повиснет до
конца работы сервера. Придется либо завершить программу, либо нажать
\spverb|Ctrl-C|. Пока сервер работает, откройте браузер по адресу \spverb|127.0.0.1:8080|.
Вы увидите строку из примера выше. Впишите другой путь, например \spverb|/hello| или
\spverb|/path/to/file.txt|. Ответ сервера изменится.

\section{Запросы и ответы}

Мы написали приложение, которое на все запросы печатает метод и путь. Кроме этих
полей, запрос содержит порт и адрес сервера, строку параметров, тип протокола,
заголовки и тело. Все вместе это неизменяемый словарь с ключами типа
\spverb|keyword| (кейворды или ключевые слова, в других языках~--- теги). Полное
описание запроса и ответа лежит в репозитории
на Гитхабе\footnote{github.com/ring-clojure/ring/blob/master/SPEC}.

Обратим внимание на поля \spverb|:headers| и \spverb|:body|.

Заголовки это неизменяемый словарь, но его ключи не кейворды, а строки. Такой
словарь не работает привычным с destructuring assignment. Ниже переменная \spverb|host|
окажется \spverb|nil|:

\begin{verbatim}
(defn some-handler
  [request]
  (let [{:keys [headers]} request
        {:keys [host]} headers]
    ...))
\end{verbatim}

Чтобы извлечь заголовки правильно, используйте \spverb|:strs|:

\begin{verbatim}
(defn some-handler
  [{:keys [headers]}]
  (let [{:strs [host user-agent]} headers]
    ...))
\end{verbatim}

\noindent
или обычный \spverb|get| со строкой:

\begin{verbatim}
(get headers "host") ;; "127.0.0.1"
\end{verbatim}

Имя заголовка всегда в нижнем регистре. С точки зрения HTTP варианты
\spverb|Content-Type| и \spverb|content-type| одинаковы. \spverb|Ring| приводит заголовки к
нижнему регистру, чтобы избежать недоразумений.

Значения заголовков тоже строки. Даже если стандарт задает типы некоторым
заголовкам, \spverb|Ring| не выводит их. Например,  \spverb|Content-Length|
передает длину тела в байтах. Современные фреймворки
приводят его к числу и помещают в отдельное поле. По умолчанию Ring не делает
ничего подобного, но это легко исправить.

С заголовками связана одна проблема. В Clojure ключи словаря почти всегда
кейворды. Легко забыть, что у заголовков они строки. Так появляется ошибка,
когда вместо правильного значения приходит \spverb|nil|:

\begin{verbatim}
(get headers :host) ;; nil
\end{verbatim}

Можно обработать заголовки, заменив тип ключей. Для одного случая это
нормально. Но если этим занят каждый обработчик, получается лишняя
работа. Приложение меняют так, чтобы в функцию приходили уже исправленные
заголовки. Эта техника называется Middleware, и мы рассмотрим ее ниже.

Поле запроса \spverb|:body| опционально. Согласно HTTP тела может и не быть. При
попытке считать \spverb|:body| проверяйте его на \spverb|nil|. Обратите внимание
на тип тела. Это не строка, а входящий поток~---
\spverb|java.io.InputStream|. Поток~--- это источник данных, который можно
прочесть только раз. По умолчанию \spverb|Ring| не читает поток. Делать это или нет
остается на ваше усмотрение.

Чтение и разбор тела это сложная операция. По заголовкам
определяют тип документа, его длину и читают нужное число байт. Из них
восстанавливают данные (JSON, XML). Результат каждого шага проверяют по разным
критериям. Когда из \spverb|Content-Length| получают число, перехватывают
исключение разбора строки. Результат \spverb|-42| неверный, потому что число байт не
может быть отрицательным. Технически возможно послать серверу JSON-документ, но
указать тип \spverb|text/xml|. Сервер должен быть готов к подобным сценариям.

Легче всего считать тело в строку функцией \spverb|slurp|:

\begin{verbatim}
(defn handler [request]
  (when-let [content (some-> request :body slurp)]
    (process-content content))
  {:status 200})
\end{verbatim}

Но в современном вебе все меньше работают с текстом: на его место приходят данные в виде
JSON. Позже мы рассмотрим, как подружить Ring с этим форматом.

\subsection{Структура ответа}

Ответ \spverb|Ring| устроен проще. Это словарь с тремя полями:
\spverb|:status|, \spverb|:headers| и \spverb|:body|.

\begin{itemize}

\item
  \spverb|:status|~--- целое положительное число, признак успеха или неудачи. Мы
  рассмотрели семантику статуса в начале главы.

\item
  \spverb|:headers|~--- заголовки ответа. В отличии от запроса, ключи и
  значения не обязательно строки. Вариант ниже работает:

\begin{verbatim}
{:status 302
 :headers {:content-length 0
           :location "/new/page.html"}}
\end{verbatim}

\item
  \spverb|:body|~--- тело ответа. Как и в запросе, его может не быть. Обычно тело
  это строка, но может быть и файлом, ресурсом или потоком.

\end{itemize}

\section{Маршруты}

Мы запустили приложение и проверили его в браузере. На любой запрос оно выдает
текст с небольшими отличиями. Это слабый подход: невозможно поддерживать
приложение, в котором все запросы сходятся в одну точку. На практике пишут
отдельные обработчики для каждой задачи. Затем распределяют по ним входящие
запросы согласно правилам. Это называется маршрутизация или роутинг.

В мире Clojure и Ring нет класса роутера. Это функция, которая принимает
обработчики (хендлеры) и возвращает новую функцию-приложение. Она принимает
запрос и по методу и пути подбирает нужный обработчик. Затем вызывает его с
запросом и возвращает ответ.

Рассмотрим тривиальный случай. Вообразим, что адресу \spverb|/| мы бы хотели видеть
название сайта, а по \spverb|/hello|~--- приветствие. Все другие адреса вернут
\spverb|404 Page not found|. Напишем обработчики:

\begin{verbatim}
(defn page-index [request]
  {:status 200
   :headers {:content-type "text/plain"}
   :body "Learning Web for Clojure"})

(defn page-hello [request]
  {:status 200
   :headers {:content-type "text/plain"}
   :body "Hi there and keep trying!"})

(defn page-404 [request]
  {:status 404
   :headers {:content-type "text/plain"}
   :body "No such a page."})
\end{verbatim}

Каждый обработчик можно запустить как сервер и проверить в браузере. Осталось
связать их в единое целое.

\subsection{Наивный подход}

Сделаем самое простое, что приходит в голову. Напишем обработчик, который
вручную определяет маршрут. Для этого проверим путь оператором \spverb|case|:

\begin{verbatim}
(defn app [request]
  (let [{:keys [uri]} request]
    (case uri
      "/"      (page-index request)
      "/hello" (page-hello request)
      (page-404 request))))
\end{verbatim}

Ответ такой функции зависит от поля запроса \spverb|:uri|. Запустите приложение
в браузере и проверьте разные адреса. Это наивный перебор, но он работает.

Недостатки функции очевидны. Мы не учитываем метод запроса. \spverb|GET /users| и
\spverb|POST /users| отличаются по смыслу. Мы сравниваем пути в лоб без
учета параметров. В правильном роутинге запросы \spverb|GET /users/1|
и \spverb|GET /users/99| приходят в один обработчик, но с разным параметром \spverb|id|.
В целом код зашумлен. Хотелось бы иметь маршруты в виде правил без перебора.

Эти и другие проблемы решены в библиотеках. Мы рассмотрим две
из них: \spverb|Compojure| и \spverb|Bidi|. Обе решают задачу роутинга по-своему, их
подходы ортогональны.

\subsection{Compojure}

Библиотека \spverb|Compojure|\footnote{github.com/weavejester/compojure}
предлагает макросы для описания маршрутов. Макросы похожи на таблицу правил.
Добавим зависимость в проект:

\begin{verbatim}
[compojure "1.6.1"]
\end{verbatim}

\noindent
Вот как выглядит приложение на Compojure:

\begin{verbatim}
(require '[compojure.core :refer [GET defroutes]])

(defroutes app
  (GET "/"      request (page-index request))
  (GET "/hello" request (page-hello request))
  page-404)
\end{verbatim}

\noindent

Эта запись чище и короче той, что мы написали вначале.

Разберемся, что получилось на выходе. Переменная \spverb|app|~--- функция, которая
принимает запрос. Мы объявили ее не через \spverb|def| или \spverb|defn|,
а особым макросом. Мы поговорим о макросах в отдельной главе. Пока что
скажем, что \spverb|defroutes| делает две вещи: строит функцию-роутер и связывает ее с
переменной \spverb|app|. Это уловка, чтобы писать меньше кода.

Макрос принимает правила. Правило это форма <<метод, путь, запрос,
выражение>>. Первые два правила заданы макросом \spverb|GET|. Они читаются так: если
метод равен \spverb|GET| и путь \spverb|"/"|, то для запроса \spverb|request|
вернуть \spverb|(page-index request)|.

Правило компилируется в функцию, которая принимает запрос. Функция проверяет,
что метод и путь запроса совпадают с заданными. Если да, функция вычислит
выражение и вернет его результат, в нашем случае \spverb|(page-index request)|.

Если совпадения не было, функция вернет \spverb|nil|. Это
значит, надо взять следующее правило и так далее. Макрос \spverb|defroutes|
работает по этой схеме. Он оборачивает правила в особый цикл. На каждом
шаге макрос берет очередное правило, применяет к нему запрос и оценивает
результат. Первое значение, отличное от \spverb|nil|, станет ответом к текущему запросу.

Что будет, если не подошло ни одно правило? Это вполне возможно. Если
приложение вернуло \spverb|nil|, это вызовет ошибку сервера. Чтобы избежать \spverb|nil|,
к правилам добавляют еще одно, которое сработает всегда. В нашем случае это функция \spverb|page-404|.
Ее результат не зависит от запроса. Этим мы гарантируем, что даже если запрос не подошел
первым двум правилам, последнее сработает обязательно.

Так работает роутинг на \spverb|Compojure|. Приложение разбивают на обработчики.
Их пишут в отдельных модулях, затем с помощью макросов \spverb|GET|, \spverb|POST|
и других оборачивают в правила. Правило возвращает функцию, которая
проверяет, что метод и путь подходят. Если да, в результате получим вызов
обработчика с запросом.

\subsection{Продвинутые возможности}

Выше мы обозначили проблему: правила \spverb|GET /users/1| и \spverb|GET /users/99|
это один обработчик с параметром. Его записывают так:

\begin{verbatim}
(GET "/users/:id" [id :as request] (page-user request))
\end{verbatim}

Обратите внимание на двоеточие перед \spverb|id| и квадратные скобки в середине.
Часть пути с двоеточием означает параметр. \spverb|Compojure| поместит его в словарь
\spverb|params|. Обработчик \spverb|page-user| обратится к нему следующим образом:

\begin{verbatim}
(defn page-user [request]
  (when-let [user-id (-> request :params :id)]
    (let [user (get-user-by-id user-id)
          {:keys [fname lname]} user]
      {:status 200
       :body (format "User %s is %s %s"
                     user-id fname lname)})))
\end{verbatim}

В данном случае мы решили, что функция \spverb|get-user-by-id| вернет словарь
пользователя по номеру. Из словаря извлекаем имя и фамилию, формируем
строку и возвращаем ответ.

\spverb|Compojure| решает проблему вложенных путей. Предположим, приложение показывает и
редактирует товары. По адресу \spverb|/content/order/1/view| открывается карточка
товара. Страница \spverb|/content/order/1/edit| показывает форму редактирования этого товара.
Чтобы сохранить товар, нужно отправить форму по тому же пути, но методом \spverb|POST|.

Очевидно, правила пересекаются. Чтобы избежать повторов, используем макрос \spverb|context|:

\begin{verbatim}
(context "/content/order/:id" [order-id]
  (GET  "/view" request (order-view request))
  (context "/edit" []
    (GET  "/" request (order-form request))
    (POST "/" request (order-save request))))
\end{verbatim}

Каждое правило под макросом \spverb|context| наследует параметры запроса. Это значит,
обработчики \spverb|order-view|, \spverb|order-form| и \spverb|order-save| получат
параметр \spverb|:order-id|.

До сих пор качестве ответа в правилах мы писали что-то вроде
\spverb|(some-handler request)|. Бывает, что ответ заранее известен,
поэтому нет смысла помещать его в отдельную функцию. Рассмотрим это на примере
\spverb|healthcheck|-обработчика.

Современные приложения запускают в контейнерах и облачных сервисах.  Чтобы
узнать, работает приложение или нет, специальная служба периодически опрашивает
его. Привычный способ сделать это~--- послать приложению \spverb|GET|-запрос по адресу
\spverb|/health| и проверить статус. Тело и заголовки ответа не играют роли.

Чтобы не создавать лишний обработчик \spverb|(page-health request)|, поместим ответ в
тело:

\begin{verbatim}
(ANY "/health" _ {:status 200 :body "ok"})
\end{verbatim}

Можно сделать еще проще. В \spverb|Compojure| предусмотрен случай, когда
выражение это строка. Для \spverb|Compojure| она становится телом положительного
ответа:

\begin{verbatim}
(ANY "/health" _ "ok")
\end{verbatim}

\subsection{Роутинг с Bidi}

[bidi]: https://github.com/juxt/bidi

Библиотека [Bidi][bidi] решает проблему роутинга иным способом. Compojure
предлагает макросы, чтобы описать правила и сделать по ним перебор. Bidi
опирается на данные~--- списки и словари. Сценарий роутинга в Bidi состоит из
нескольких шагов.

На первом этапе объявить особое дерево маршрутов. Это дерево~--- комбинация
векторов и словарей по определенным правилам. В листьях дерева поместить теги~---
уникальные метки для обозначения листа. Особая функция принимает это дерево и
запрос. Функция пытается понять, на какую ветвь дерева ложиться запрос. Если
таковая нашлась, результатом будет тег ветки и, возможно, параметры
пути. Например, \spverb|{:route :show-user, :route-params: {:id 1}}|.

На втором этапе написать middleware~--- промежуточный обработчик запроса. Такой
middleware принимает запрос, добавляет в него тег и передает дальше по цепочке.

На третьем этапе~--- объявить обработчик запроса. Но это будет не функция, а
мультиметод. Его функция-диспачер возвращает тег. Метод \spverb|:default| возвращает
ответ 404, \spverb|:show-user|~--- страницу пользователя, и так далее.

На первый взгляд схема кажется сложной. Но однажды настроив, ее легко
расширять. Чтобы сервер подхватил новый путь, достаточно поместить в дерево
нужную ветку и расширить мультиметод.

Перепишем на Bidi все то, что сделали на Compojure. Добавьте в проект
зависимость:

\begin{verbatim}
[bidi "2.1.5"]
\end{verbatim}

Начнем с дерева маршрутов. Вариант с page-index, page-hello и page-404 будет выглядеть так:

\begin{verbatim}
(def routes
  ["/" {""      :page-index
        "hello" :page-hello
        true    :not-found}])
\end{verbatim}

Проверим, как работает матчинг пути по этому дереву. Функция match-route
принимает маршруты и путь и возвращает словарь с тегом:

\begin{verbatim}
(require '[bidi.bidi :as bidi])

(bidi/match-route routes "/hello")
{:handler :page-hello}

(bidi/match-route routes "/test")
{:handler :not-found}
\end{verbatim}

Ответ функции следует объединить со словарем запроса. Чтобы сделать это за один
шаг, воспользуемся функцией match-route*. Это альтернативная версия match-route,
которая принимает словарь-накопитель.

\begin{verbatim}
(let [request
      {:request-method :get
       :uri "/test"}]
  (bidi/match-route* routes (:uri request) request))

{:request-method :get
 :uri "/test"
 :handler :not-found}
\end{verbatim}

Видим, что \spverb|match-route*| вернула переданный запрос, но добавила в него поле
handler. Перенесем код выше в middleware. Это функция, которая принимает
обработчик запроса и возвращает его альтернативную версию. Такой обработчик,
получив запрос, сперва добавит к нему поле handler и вызовет исходный обработчик
с новым запросом.

\begin{verbatim}
(defn wrap-handler
  [handler]
  (fn [request]
    (let [{:keys [uri]} request
          request* (bidi/match-route*
                    routes uri request)]
      (handler request*))))
\end{verbatim}

Мы еще не касались техники middleware, но вынуждены применить ее на данном
этапе.  Ниже мы рассмотрим во деталях, как устроены middleware и почему так
важны.

Проверим \spverb|wrap-handler| на скорую руку. Будем считать, что обработчик запроса
это стандартная функция identity. Она всегда возвращает переданный в нее
аргумент:

\begin{verbatim}
((wrap-handler identity)
 {:request-method :get
  :uri "/hello?foo=42"})

{:request-method :get,
 :uri "/hello?foo=42",
 :handler :page-hello}
\end{verbatim}

Конечный обработчик запроса будет мультиметодом. Его функция-диспатчер просто
\spverb|:handler|.

\begin{verbatim}
(defmulti multi-handler
  :handler)

(defmethod multi-handler :page-index
  [request]
  {:status 200
   :headers {:content-type "text/plain"}
   :body "Learning Web for Clojure"})

(defmethod multi-handler :page-hello
  [request]
  {:status 200
   :headers {:content-type "text/plain"}
   :body "Learning Web for Clojure"})

(defmethod multi-handler :not-found
  [request]
  {:status 404
   :headers {:content-type "text/plain"}
   :body "No such a page."})
\end{verbatim}

Теперь обернем \spverb|multi-handler| в middleware. Это и будет финальное приложение.

\begin{verbatim}
(def app
  (wrap-handler multi-handler))
\end{verbatim}

Запустите веб-сервер и проверьте результат в браузере.

Это был простой вариант роутинга на Bidi. Рассмотрим пример с заказами:
просмотр, редактирование и сохранение.

Новое дерево выглядит так:

\begin{verbatim}
(def routes
  ["/" {["content/order/" :id]
        {"/view" {:get  :page-view}
         "/edit" {:get  :page-form
                  :post :page-save}}}])
\end{verbatim}

В этой версии листья уже не теги, а словари. Ключ такого словаря~--- метод
HTTP-запроса, а значение~--- тег. Запрос "GET /content/order/1/edit" разрешается в
тег \spverb|:page-form|, а POST с таким же адресом~--- в \spverb|:page-save|. При прохождении
через wrap-handler запрос получит поле route-params. Для нашего случая это будет
словарь \spverb|{:id "1"}|.

Вот так бы мог выглядеть обработчик \spverb|page-edit|. Получаем словарь заказа по его
id. Если заказ найден, рисуем HTML страницу с формой редактирования. Если нет,
отдаем 404 и сообщение об ошибке.

\begin{verbatim}
(defmethod multi-handler :page-edit
  [request]
  (let [order-id (get-in request [:route-params :id])
        order (get-order-by-id order-id)]
    (if order
      {:status 200
       :headers {:content-type "text/html"}
       :body (render-order-form order)}
      {:status 404
       :headers {:content-type "text/html"}
       :body "<h1>Order not found</h1>"})))
\end{verbatim}

\subsection{Выбор между Compojure и Bidi}

Автору приходилось работать с роутингом обоих типов. По субъективным ощущениям,
с Compojure легче начать. У библиотеки достойная документация с
примерами. Compojure написал тот же разработчик, что и Ring. Проекты близки и
дополняют друг друга.

Дерево маршрутов Bidi сложно для понимания. Оно многословно и не
интуитивно. Легко допустить ошибку, перепутать вектор и словарь. С другой
стороны, логика на мультиметодах несет преимущества. Код становится линейным,
более организованным, приложение легче наращивать.

Если вы начинающий Clojure-разработчик или проект небольшой, выбирайте
Compojure. Когда проект сложный со множеством эндпоинтов, рассмотрите переезд на
Bidi.

\section{Middleware}

Выше мы упоминали про middleware и даже кинули пробный шар~--- написали
wrap-route. В этом разделе мы разберем все вопросы о middleware и лучших
практиках по работе с ними. Автор считает этому тему самой важной в главе.

В переводе с английского Middleware значит промежуточный слой, середина. В
программировании под middleware понимают код, который обрабатывает данные между
посредниками. Обработка данных это приведение типов, добавление новых полей,
проверка прав доступа.

Паттерн <<декоратор>> это частный случай middleware. Декоратор это функция А,
которая принимает функцию B и возвращает функцию C. Говорят, что A декорирует
B. Результат декорирования это C. В ходе исполнения функция C вызывает B, но с
изменениями. Например, корректирует входные или выходные данные B.

Приведем примеры простых декораторов. with-echo добавляет к функции побочные
эффекты: печатает аргументы и результат.

\begin{verbatim}
(defn with-echo
  [func]
  (fn [& args]
    (apply println "The args are" args)
    (let [result (apply func args)]
      (println "The result is" result)
      result)))
\end{verbatim}

\spverb|With-catch| оборачивает целевую функцию в форму \spverb|try/catch|. Если во время
работы выброшено исключение, результатом будет его объект.

\begin{verbatim}
(defn with-catch
  [func]
  (fn [& args]
    (try
      (apply func args)
      (catch Throwable e
        e))))
\end{verbatim}

Мы уже рассматривали структуру Ring-запроса. Возможно, читатель заметил, что в
нем нет полей, с которыми он работал в других языках. Например, классы
\spverb|django.http.HttpRequest| и \spverb|flask.Request| в Python содержат поля \spverb|.params| или
\spverb|.values|. Это словари, полученные из адресной строки или тела запроса.

Почему в стандарте Ring нет столь важных вещей? Потому что не каждое приложение
в них нуждается. Фреймворк предоставляет только базовую информацию о
запросе. Остальные данные могут быть получены из исходных.

Представим, что на каждый запрос Ring парсит строку параметров и тело. Это
удобно разработчику, но резко снижает производительность сервера. Нет гарантии,
что параметры строки пригодятся в запросе. Но сервер потратит время и память на
их обработку. Еще хуже с обработкой тела. Вспомним, что это дорогая
операция. Возможен сценарий, когда сервер прочитал огромный JSON-документ, но
внутри обработчика выяснилось, что у пользователя нет прав на запись. Эту
проверку следовало выполнить раньше!

Как мы помним, обработчик запроса в Ring это функция, которая принимает запрос и
возвращает ответ. Техника middleware как нельзя лучше подходит, чтобы добавить
промежуточную логику. Параметры запроса, сессии, куки, права доступа~--- все это
функция, которая возвращает функцию.

Вам не придется писать все middleware с нуля. Ring уже содержит основные из
них. Остается только применить их к приложению. Рассмотрим некоторые middleware
и принципы их работы.

\subsection{Параметры запроса}

Стандарт HTTP разрешает передавать данные в адресной строке. Это пары вида
\spverb|"name=John&city=NY"| после знака вопроса. Удобно, когда параметры доступны в виде
словаря. В нашем случае это была бы структура \spverb|{:name "John" :city "NY"}|.

Аналогично с параметрами из тела запроса. Их передают в теле по разным
причинам. В основном это ограничение на длину и проблемы безопасности. Длина
адресной строки ограничена 2048 байтами, в то время как на тело запроса
ограничений нет. Пароли и адреса почты небезопасно передавать в адресной строке,
потому что они остаются в логах и истории браузера.

Функция wrap-params из модуля ring.middleware.params меняет функцию-обработчик
следующим образом. Переданный в нее запрос дополняется тремя полями:

%% TODO list
- \spverb|:query-params|~--- словарь параметров адресной строки;
- \spverb|:form-params|~--- словарь данных из тела запроса;
- \spverb|:params|~--- их комбинированная версия.

Пусть \spverb|app|~--- ваше веб-приложение. Чтобы получить его обернутую версию,
достаточно вызвать wrap-params c app. Результат будет финальным приложением. На
жаргоне разработчиков это называется "врапнуть" (анг. wrap~--- обернуть).

\begin{verbatim}
(require '[ring.middleware.params
           :refer [wrap-params]])

(def final-app
  (wrap-params app))
\end{verbatim}

Чтобы не запутаться в именах, придерживайтесь правил. Не обернутое приложение
называйте \spverb|app-naked| или \spverb|app-raw| (голое, сырое), а финальное просто \spverb|app|.

Доработайте веб-приложение из примера выше так, чтобы оно учитывало параметры
строки. Например, чтобы имя того, кого приветствовать, можно было задать
параметром who: /hello?who=John.

Подсказка: добраться до параметра who можно так:

\begin{verbatim}
(defn page-hello
  [request]
  (let [who (get-in request [:params "who"])]
    ...))
\end{verbatim}

или так:

\begin{verbatim}
(defn page-hello
  [request]
  (let [who (-> request :params (get "who"))]
    ...))
\end{verbatim}

Обратите внимание, что ключи \spverb|:params| это строки. Это нормально, но Clojure
всячески поощряет нас, когда ключи словаря кейворды. Исправим это. В поставке
Ring есть особое middleware, которое приводит поле \spverb|:params| к удобному
виду. Это \spverb|wrap-keyword-params| из модуля \spverb|ring.middleware.keyword-params|:

\begin{verbatim}
(require '[ring.middleware.keyword-params
           :refer [wrap-keyword-params]])

(def app
  (wrap-keyword-params (wrap-params app-naked)))
\end{verbatim}

Мы подошли к новой проблеме: когда врапперов много, от них возникает
шум. Типичное приложение включает десять-пятнадцать middleware:

\begin{verbatim}
(def app
  (wrap-something-else
    (wrap-current-user
      (wrap-session
        (wrap-keyword-params
          (wrap-params app-naked))))))
\end{verbatim}

Это кашу невозможно поддерживать. Представьте, что требуется добавить еще один
враппер где-то в середине. Это каскадно сдвинет элементы ниже. Чтобы победить
сложность, сделаем структуру линейной. Применим стрелочный оператор:

\begin{verbatim}
(def app
  (-> app-naked
      wrap-params
      wrap-keyword-params
      wrap-session
      wrap-current-user
      wrap-something-else))
\end{verbatim}

Такая форма напоминает обычный список, поэтому ее легко поддерживать.

Запись в стрелочном виде имеет особенность. Не заглядывая в следующее
предложение, догадайтесь, в каком порядке будут выполнены middleware? Правильный
ответ: снизу вверх для запроса и сверху вниз для ответа. Это может показаться
странным, но становится очевидным при мысленном разборе.

Сперва запрос зайдет в \spverb|wrap-something-else|. Код внутри него вызовет
обработчик, который получен из \spverb|wrap-current-user|. Обработчик внутри него –
результат \spverb|wrap-session|, и так далее. Вершиной подъема станет
\spverb|app-naked|. Структура ответа начнет опускаться по стеку вниз. Сначала он
пройдет через \spverb|wrap-params| и \spverb|wrap-keyword-params|. Эти два middleware не
изменяют ответ и просто возвращают его. \spverb|Wrap-session| и \spverb|wrap-current-user|,
возможно, допишут в него новые заголовки. Последним сработает
\spverb|wrap-something-else|. Цикл запроса и ответа пройден.

Цепочку middleware следует рассматривать как восхождение в гору и спуск с
нее. Другой аналогией может быть пузырек, который всплывает и опускается (не
имеет отношения к сортировке пузырьком).

По тому же принципу устроены middleware в Django, промышленном
Python-фреймворке. Хоть в Django их роль играют не функции, а классы, их порядок
обхода такой же.

Порядок middleware порой критичен. Некоторые из них опираются на данные, которые
подготовили предыдущие middleware. Рассмотрим уже знакомые \spverb|wrap-params| и
\spverb|wrap-keyword-params|. Последний отыскивает в запросе поле params и меняет тип
ключей. Подразумевается, что params был подготовлен
\spverb|wrap-keyword-params|. Поэтому \spverb|wrap-keyword-params| ставят строго после
\spverb|wrap-params|.

Посмотрим на форму \spverb|(def app...)| выше. В нее закралась ошибка. Запрос
поднимается снизу вверх, поэтому \spverb|wrap-keyword-params| сработает раньше. Он
попытается найти поле params в запросе, но безуспешно. Следом сработает
\spverb|wrap-params|. Он заполнит это поле словарем из адресной строки. В результате
params будет словарем с ключами-строками. В следует поменять wrap-params и
\spverb|wrap-keyword-params| местами.

Неверный порядок middleware стоит часов отладки. Но есть трюк. Если два и более
middleware идут в строгой последовательности, можно "схлопнуть" их в одно
целое. Стандартная функция comp принимает произвольное число функций и
возвращает супер-функцию, которая последовательно применяет их к
аргументу. Определим умный враппер параметров:

\begin{verbatim}
(def wrap-params+
  (comp wrap-keyword-params wrap-params))
\end{verbatim}

Плюс на конце означает, что это улучшенная версия обычного wrap-params. Теперь
заменим в стеке \spverb|wrap-params| и \spverb|wrap-keyword-params| на \spverb|wrap-params+|. Цепочка
middleware станет короче, а логика параметров соберется в отдельном месте.

Перечислим другие полезные middleware. Мы не будем останавливаться на детальном
описании каждого. Это скорее индекс, к которому можно обратиться в случае
надобности.

\subsection{Cookie}

В стандарте HTTP куки~--- это маленькие кусочки информации. Между сервером и
браузером особое соглашение о том, как хранить и передавать их. Если сервер
выставил куки, браузер запоминает их для этого сайта. В следующий раз браузер
отправит куки на сервер автоматически. Так продолжается до тех пор, пока сервер
не удалит куки или истечет их срок жизни.

Простейший случай, когда нужны куки~--- определить, был ли уже пользователь на
сайте. При первом визите приложение ищет в запросе куки с именем visited. Если
значение не установлено, сервер выставляет заголовок вроде:

\begin{verbatim}
Set-Cookie: visited=true;
\end{verbatim}

В последующих запросах браузер отправит это значение на сервер
самостоятельно. Приложение проверяет: если \spverb|visited true|, значит пользователь
уже был на сайте. Такие проверки влияют на показ рекламы, всплывающие окна,
попапы с обновлениями.

Технически куки~--- это один длинный заголовок, где значения и атрибуты разделены
специальными точками с запятой. Middleware wrap-cookie значительно облегчает
работу с куки. Во время запроса заголовок преображается в словарь в поле
\spverb|:cookies|. Чтобы сообщить клиенту новые куки, добавьте поле \spverb|:cookies| в
ответ. Из такого словаря образуется заголовок \spverb|Set-Cookie|.

Напишем простую страничку, которая определяет, видим ли мы ее в первый раз.

\begin{verbatim}
(require '[ring.middleware.cookies
           :refer [wrap-cookies]])

(defn page-seen
  [request]
  (let [seen-path [:seen :value]
        {:keys [cookies]} request
        seen? (get-in cookies seen-path)
        cookies (assoc-in cookies seen-path true)]
    {:status 200
     :cookies cookies
     :body (if seen?
             "Already seen"
             "The first time you see it") }))

(defn app
  (-> page-seen
      wrap-cookies))
\end{verbatim}

Запустите приложение в браузере. После обновления страницы надпись изменится на
"Already seen". Обратите внимание, что даже после перезагрузки сервера ответ
по-прежнему будет "Already seen", потому что флаг хранится в браузере. Только
очистив куки вы снова увидите "The first time you see it". Для полноты
эксперимента откройте приватную вкладку или другой браузер.

Куки тесно связаны с безопасностью. Даже если вам понятны технические детали,
убедитесь, что куки защищены от кражи и не раскрывают секретные данные (пароли,
ключи доступа). В этом разделе мы не обсуждаем тему веб-безопасности. Она
слишком обширна для этой главы и заслуживает отдельной книги.

\subsection{Сессии}

Стандарт HTTP не предполагает связи между двумя запросами. Считается, что два
запроса с соседних компьютеров разницей в пять минут связаны так же, как с
разных континентов разницей в год. Но сразу с рождения веба разработчики
нарушили стандарт. Понадобилось хранить состояние конкретного пользователя. Даже
если компьютеры за одним столом, сервер должен различать их. Это назвали сессией
или сеансом.

\spverb|Wrap-session| это довольно сложное middleware. Оно дополняет запрос полем
\spverb|:session|, в котором словарь. Его ключи~--- поля сессии. Чтобы обновить сессию,
следует положить ее новую версию в ответ по аналогии с \spverb|:cookie|. Middleware
различает \spverb|nil| и факт отсутствия сессии в ответе. Если поле \spverb|:session| \spverb|nil|, вся
сессия удаляется. Если ключа нет, ничего не происходит.

Сессия это абстрактное понятие, поэтому различают бэкенды сессии. Это разные
способы хранить значения физически. Сессия может храниться в памяти, на диске, в
базе данных, Memcached/Redis или даже куках. При выборе бэкенда важно учитывать,
способен ли он работать на нескольких машинах одновременно. Что получится, если
каждый запрос на случайно выбранной из десяти машин?

Если сессия хранится в памяти приложения, то на каждой машине будет ее разная
версия. Это чревато странным поведением и трудной отладкой. Аналогично с файлами
— машины не делят их между собой. А вот база данных или Redis это общее
хранилище. Оно гарантирует актуальность сессии для всех клиентов.

Интересно, что сессия на базе куки тоже работает на множестве машин. На каждый
запрос браузер передает и получает полную сессию в HTTP-заголовках. В этом
случае сессия хранится на клиенте. Но если пользователь очистит куки или
запустит другой браузер, сессия будет утеряна.

В стандартной поставке Ring сессия хранится в памяти или куках. Хранилище
определяется настройками wrap-session. Ring закладывает необходимые абстракции,
чтобы хранить сессию в базе или key-value системах типа Redis.

Рассмотрим пример со счетчиком посещений. Будем считать, сколько раз
пользователь зашел на наш сайт. Для простоты храним сессию в памяти.

\begin{verbatim}
(require '[ring.middleware.session
           :refer [wrap-session]])

(defn page-counter
  [request]
  (let [{:keys [session]} request
        session (update session :counter (fnil inc 0))]
    {:status 200
     :session session
     :body (format "Seen %s time(s)" (:counter session))}))

(defn app
  (-> page-counter
      wrap-session))
\end{verbatim}

Запустите app в веб-сервере и откройте браузер. Обновляйте страницу, и счетчик в
сообщении возрастет с каждым просмотром. Ради интереса проделайте то же самое в
другом браузере. Это будет вторая сессия, которая не зависит от
первой. Поскольку данные хранится в памяти, они будут утеряны при перезагрузке
сервера.

Упражнение: в примере выше мы считаем просмотры для всего сайта. Сделайте так,
чтобы счетчик работал в рамках страниц. Например, главная страница \spverb|/|
просмотрена пять раз, а справка \spverb|/help|~--- три раза. Параметры командной строки
не влияют на подсчет.

\subsection{JSON}

Формат JSON предназначен для передачи данных. Среду прочих его достоинств~---
типы, вложеность и совместимость с JavaScript.

JSON различает базовые типы данных~--- числа, строки, логический тип. Это выгодно
отличает его от параметров адресной строки или XML, где все значения строки.

Формат предусматривает основные коллекции~--- массив и словарь~--- и их произвольную
вложенность. В разное время были попытки передать вложенные данные в адресной
строке. Общий подход был в том, чтобы ключ содержал путь внутри
структуры. Например, если в данных несколько адресов, а у каждого адреса
несколько строк (line 1, line 2, etc), то получается что-то вроде:

\begin{verbatim}
address[0].line[0].value=SomeStreet
\end{verbatim}

Требовалось писать и поддерживать код для работы с такими парами. Каждый
фреймворк нес на борту собственный модуль, чтобы упаковывать и восстанавливать
коллекции. К счастью, сегодня с этим покончено. Сегодня только старые системы
передают коллекции через командную строку.

JSON совместим с JavaScript. Если передать такой документ в функцию \spverb|eval|, она
вернет данные~--- комбинацию списков и словарей.

Все это способствовало тому, чтобы JSON стал главным способом передать данные в
интернете.

Ring предлагает набор middleware для JSON. Они вынесены в отдельный пакет для
удобства разработки. Добавим в проект зависимость:

\begin{verbatim}
[ring/ring-json "0.4.0"]
\end{verbatim}

wrap-json-response облегчает возврат JSON-данных. Это middleware проверяет поле
ответа \spverb|:body|. Если это коллекция (вектор, словарь), то middleware заменяет его
на кодированную строку и выставляет заголовок \spverb|Content-Type: application/json|.


Рассмотрим пример:

\begin{verbatim}
(require '[ring.middleware.json
           :refer [wrap-json-response]])


(defn page-data
  [request]
  {:status 200
   :body {:some {:json ["data"]}}})

(def app
  (-> page-data
      wrap-json-response))
\end{verbatim}

Более сложный пример. Если пользователь нашелся, возвращаем его модель. Если
нет, то структуру ошибки.

\begin{verbatim}
(defn page-data
  [request]
  (let [user-id (-> request :params :id)]
    (if-let [user (get-user-by-id user-id)]
      {:status 200
       :body user}
      {:status 404
       :body {:error_code "MISSING_USER"
              :error_message "No such a user"
              :error_data {:id user-id}}})))
\end{verbatim}

Для входящего JSON-документа в библиотеке два middleware. Это \spverb|wrap-json-body| и
\spverb|wrap-json-params|. На фазе запроса оба проверяют, что заголовок \spverb|Content-Type|
содержит \spverb|application/json|. Если да, они парсят тело с учетом возможных
исключений. При ошибке разбора ответ будет 400 "JSON body malformed".

Разница между \spverb|wrap-json-body| и \spverb|wrap-json-params| в том, куда они складывают
полученные данные.

\spverb|Wrap-json-body| заменяет поле \spverb|:body| запроса на полученную структуру данных. В
примере ниже обработчик \spverb|page-body| извлекает имя и город пользователя из
\spverb|:body|. Тело запроса уже не входящий поток, а структура данных, о чем заботится
\spverb|wrap-json-body|. Обратите внимание, middleware принимает опциональные
параметры. Флаг \spverb|:keywords? true| Означает, что ключи словарей должны быть
приведены к кейвордам.

\begin{verbatim}
(require '[ring.middleware.json
           :refer [wrap-json-body]])

(defn page-body
  [request]
  (let [{:keys [body]} request
        {:keys [username city]} body]
    (create-user username city)
    {:status 200
     :body {:code "CREATED"
            :message "User created"}}))

(def app
  (-> page-body
      (wrap-json-body {:keywords? true})))
\end{verbatim}

Чтобы отправить JSON-запрос к серверу, понадобится специальная программа. Это
может быть утилита cURL или графическое приложение Postman. Пример с cURL:

\begin{verbatim}
curl \
  --header "Content-Type: application/json" \
  --request POST \
  --data '{"username":"John","city":"NY"}' \
  http://localhost:8080/
\end{verbatim}

Вариант с \spverb|wrap-json-params| Отличается тем, где хранится структура данных. Это
middleware заносит данные в поле \spverb|:json-params|. В дополнение, если данные были
словарем, они вливаются в поле \spverb|:params|. Это поле, как мы помним, используются
другими врапперами, например, \spverb|wrap-params|.

Таким образом, \spverb|:params| выступает универсальным аккумулятором
параметров. Продвинутое API может быть устроено так, что клиент вправе
передавать данные удобным ему способом. Например, GET-запросом с параметрами
строки, если это данные для чтения. POST с переменными в теле, чтобы изменять
сущности. Или POST с JSON-телом, если данные с глубокой вложенностью.

Вспомним, что params это словарь с ключам-строками. По этой причине
\spverb|wrap-json-params| сохраняет строки в ключах, чтобы слияние прошло
правильно. Чтобы исправить ключи \spverb|:params| на кейворды, используйте уже знакомое
нам \spverb|wrap-keyword-params|. Оно должно быть ниже \spverb|wrap-json-params| по стеку.

Разработчики не случайно выделяют поле \spverb|:json-params|. Тело JSON-документа не
обязательно словарь, это может быть массив. Такую структуру невозможно влить в
\spverb|:params|. Документ помещают в \spverb|:json-params|, и если это словарь, объединяют с
\spverb|:params|.

Продемонстрируем сказанное на примере. Передаем данные гибридно: \spverb|username| в
теле JSON-документа и \spverb|city| в параметрах строки. Обратите внимание на стек
middleware. Сперва мы парсим параметры строки, затем тело документа. Оба словаря
накапливаются в \spverb|:params|. Затем, уже после их накопления, исправляем тип
ключей.

\begin{verbatim}
(require '[ring.middleware.json
           :refer [wrap-json-params]])

(defn page-params
  [request]
  (let [{:keys [params]} request
        {:keys [username city]} params]
    (create-user username city)
    {:status 200
     :body {:code "CREATED"
            :message "User created"}}))

(def app
  (-> page-params
      wrap-keyword-params
      wrap-json-params
      wrap-params))
\end{verbatim}

Пример обращения к серверу:

\begin{verbatim}
curl \
  --header "Content-Type: application/json" \
  --request POST \
  --data '{"username":"John"}' \
  http://localhost:8080/?city=NY
\end{verbatim}

\subsection{Собственные middleware}

До сих пор мы использовали сторонние врапперы. Это те, что идут в поставке Ring
и смежных библиотек. Но рано или поздно вам потребуются собственные. Обычно их
накапливают в модуле с именем \spverb|<projectname>.middleware|. Рассмотрим примеры из
реальных проектов.

%% TODO
\emph{wrap-headers-kw} Этот простой враппер меняет ключи заголовков со строковых
на кейворды. Полезно, когда приложение часто обращается к заголовкам.

\begin{verbatim}
(require
 '[clojure.walk :refer [keywordize-keys]])

(defn wrap-headers-kw
  [handler]
  (fn [request]
    (-> request
      (update :headers keywordize-keys)
      handler)))
\end{verbatim}



\emph{wrap-request-id} В протоколе HTTP запрос и ответ не связаны друг с
другом. Порой трудно понять, к какому запросу относится тот или иной ответ и
наоборот. Важно, чтобы система могла их сопоставить. Например, была серия
ответов с кодом 500, но какие именно запросы вызвали ошибку?

Для этого ввели заголовок \spverb|X-Request-Id|. Если клиент не передал идентификатор
запроса, мы назначаем ему случайный. Тот же идентификатор возвращаем в
ответе. Все записи в лог содержат этот идентификатор.

Обратите внимание, что мы обращаемся к заголовкам как к ключевым словам. Мы
ожидаем, что \spverb|wrap-headers-kw| был выше по стеку.

\begin{verbatim}
(import 'java.util.UUID)

(defn wrap-request-id
  [handler]
  (fn [request]
    (let [uuid (or (get-in request [:headers :x-request-id])
                   (str (UUID/randomUUID)))]
      (-> request
          (assoc-in [:headers :x-request-id] uuid)
          (assoc :request-id uuid)
          handler
          (assoc-in [:headers :x-request-id] uuid)))))
\end{verbatim}

Мы храним идентификатор не только в заголовках, но и на уровне запроса в поле
\spverb|:request-id|. Для записи в лог мы будем часто обращаться к нему. Поэтому
вынесем в отдельную переменную вместе с другими полями в начале функции:

\begin{verbatim}
(defn some-handler
  [request]
  (let [{:keys [params request-id]} request]
    (log/info "Request id: %s" request-id)))
\end{verbatim}

\emph{wrap-current-user} Этот враппер определяет текущего пользователя
системы. Стратегия в том, что в запросе содержится идентификатор пользователя. В
данном случае мы ищем его в сессии. Если идентификатор найден, читаем модель
пользователя и присоединяем к запросу. Ожидается, что функция
\spverb|get-user-by-id| знает, как извлекать данные о пользователе. Чаще всего
это запрос к базе данных.

\begin{verbatim}
(defn wrap-current-user
  [handler]
  (fn [request]
    (let [user-id (-> request :session :user-id)
          user (when user-id
                 (get-user-by-id user-id))]
      (-> request
          (assoc :user user)
          handler))))
\end{verbatim}

Условно говоря, хранить \spverb|user-id| в сессии безопасно. Сессия подписана секретным
ключом, поэтому только сервер может менять ее значения. Не допускайте, чтобы
\spverb|user-id| передавался в параметрах командной строки.

\subsection{Прерывание стека}

Выше мы рассмотрели непрерывную цепочку middleware, где каждое звено передает
управление следующему. Но логика middleware не всегда линейна. Бывает, цепочку
необходимо разорвать. Например, еще на уровне middleware мы определили, что
пользователь не имеет прав к данному ресурсу. Продолжать цепочку по обычному
пути не имеет смысла. Наоборот, мы должны как можно скорее вывалиться из стека.

Стандартные врапперы из примеров выше работают на условиях. Так,
\spverb|wrap-json-params| читает тело только в том случае, если заголовок
\spverb|Content-Type| установлен в \spverb|application/json|. Если в нем что-то другое, он
оставит поток нетронутым. При разборе JSON-документа ловится возможное
исключение. Такое возможно, если документ сформирован с ошибками или поврежден
при передачи. В таком случае \spverb|wrap-json-params| не продолжает цепочку. Он
возвращает ответ с текстом "JSON body malformed". Ни одно middleware ниже по
стеку не сработает.

Рассмотрим частный случай с проверкой доступа. Предположим, приложение доступно
только авторизованным пользователям. Мы уже определили текущего пользователя в
\spverb|wrap-current-user|. То middleware только определяет пользователя, но не
ограничивает доступ. Добавим ниже по стеку другое:

\begin{verbatim}
(defn wrap-auth-user-only
  [handler]
  (fn [request]
    (if (:user request)
      (handler request)
      {:status 403
       :headers {:content-type "text/plain"}
       :body "Please sign in to see that page."})))
\end{verbatim}

Теперь все middleware ниже \spverb|wrap-auth-user-only| не сработает если пользователь
не авторизован.

Вспомним, цепочку middleware можно представить как восхождение и спуск с
горы. Если один из элементов терпит неудачу, мы как будто срезаем
верхушку. Словно добрались до середины, столкнулись с проблемой и повернули
обратно. Общее правило: чем раньше мы обнаружим проблему, тем меньше потратим
сил. Поэтому более общие проверки мы ставим выше по стеку.

Еще один вариант middleware с развилкой это перехват ошибок. Это критически
важный обработчик. Вы не найдете его в стандартных библиотеках, потому что
логика работы с ошибками меняется от проекта к проекту. Мы просто копируем такой
middleware из прошлого проекта с небольшими изменениями.

Что случится, если при обработке запроса выброшено исключение? Не существует
четких правил на этот счет. Каждый сервер или фреймворк обрабатывает исключения.

Один сервер покажет в браузере стек-трейс. Другой сервер вернет HTML-страницу с
отладочной информацией. Разработчики третьего посчитали, что выводить стек-трейс
небезопасно. Исключение пишут в лог, а в ответе статус 500 и фраза "Internal
Server Error".

Хорошо, когда разработчик сам определяет, что делать с исключениями. Ниже
простое middleware, которое перехватывает потенциальную ошибку, пишет ее в лог и
возвращает ответ-заглушку:

\begin{verbatim}
(defn wrap-exception
  [handler]
  (fn [request]
    (try
      (handler request)
      (catch Throwable e
        (let [{:keys [uri
                      request-method]} request]
          (log/errorf e "Error, method %s, path %s"
                      request-method uri)
          {:status 500
           :headers {:content-type "text/plain"}
           :body "Sorry, please try later."})))))
\end{verbatim}

В примере выше \spverb|log/errorf| это макрос для записи ошибок. Он принимает объект
исключения, шаблон и параметры. Мы хотим знать, какие были метод и путь запроса,
поэтому записываем их тоже. Это значительно облегчит анализ логов в будущем.

Чем выше \spverb|wrap-exception| расположено в стеке, тем меньше шансов возникнуть не
пойманному исключению. В идеале оно стоит на вершине цепочки, чтобы
гарантированно ловить все исключения.

Порой даже используют стратегию двойного перехвата. Дело в том, что ошибки в
разных частях системы заслуживают разного подхода. Например, нам важно знать все
об ошибках в бизнес-логике. Если пользователь не смог создать сущность на
сервере, мы обязаны записать дату, номер пользователя, имя модуля и функции,
данные, которые вызвали ошибку. Возможно, этот случай запишут сразу в несколько
журналов. Но проблемы при чтении JSON-документа нас интересуют меньше. Это
техническая проблема, не связанная с бизнесом.

Чтобы разделять бизнес- и технические проблемы, на границах стека middleware
расставляют разные wrap-exception. Самое нижнее оборачивает непосредственно
app-naked. Оно отлавливает исключения в бизнес-логике. Такую ошибку обрабатывают
подробно, во всех деталях. На вершине стека другая, облегченная версия
wrap-exception. Его задача~--- ловить мелкие ошибки, связанные с предварительной
обработкой запроса. По большей части это для того, чтобы возвращать адекватный
ответ пользователю.

\subsection{Middleware вне стека}

Интересен сценарий, когда middleware должно оказать эффект только на запросы по
определенному пути. Вернемся к wrap-auth-user-only. В чем его недостаток? Если
включить его в стек, анонимный пользователь не увидит ни одну
страницу. Абсолютно любой запрос будет отклонен со статусом 403. Главная
страница, контактные данные, форма входа~--- все страницы недоступны. В этом нет
никакого смысла.

Очевидно, \spverb|wrap-auth-user-only| должен перекрывать только некоторое подмножество
запросов. Например, тех, что начинаются с \spverb|/account|: \spverb|/account/cart|,
\spverb|/account/orders| и т.д. Место \spverb|wrap-auth-user-only| не в общем стеке, а ниже~---
на уровне роутинга.

Дальнейшая реализация зависит от того, как мы строим маршруты. В Compojure есть
особое middleware под названием wrap-routes. Оно принимает правило и другое
middleware. Если правило накладывается на текущий запрос, то целевой обработчик
оборачивается в переданное middleware. Столь сложная логика нужна, чтобы не
вызывать middleware, пока запрос не совпадет с правилом.

Вынесем семейство маршрутов для аккаунта в отдельную ветку:

\begin{verbatim}
(defroutes account-routes
  (with-context "/account" []
    (GET "/profile" request (account-profile request))
    (GET "/orders" request (account-orders request))
    (GET "/cart" request (account-cart request))))
\end{verbatim}

Обернем аккаунты в маршрутах верхнего уровня:

\begin{verbatim}
(defroutes app
  (GET "/" request (page-index request))
  (GET "/help" request (page-help request))
  (wrap-routes account-routes wrap-auth-user-only))
\end{verbatim}

Теперь \spverb|wrap-auth-user-only| сработает только для обработчиков
\spverb|account-profile|, \spverb|account-orders| и \spverb|account-cart|.

\section{Все вместе}

Middleware, которое принимает middleware~--- довольно крутая абстракция. Если вы
действительно поняли, как это работает и почему именно так~--- примите
поздравления. Это серьезный рубеж.

Пожалуй, это все, что можно сказать о middleware. Небольшое обобщение: все, что
мы проделали выше работает на функциях. Типичное middleware это функция, которая
принимает функцию и возвращает функцию. Middleware~--- универсальный строительный
материал.

Цепочку middleware называют стеком. Во время запроса мы движемся по стеку снизу
вверх, во время ответа~--- сверху вниз. Такой обход можно сравнить с восхождением
в гору. Каждое middleware может прервать цепочку в зависимости от
обстоятельств. Легче всего выразить стек с помощью стрелочного оператора. Это
экономит скобки и делает структуру наглядней.

%% TODO summary

%% static files
