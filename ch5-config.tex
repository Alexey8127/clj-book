\chapter{Конфигурация}

\label{chapter-config}

\begin{teaser}
В этой главе мы рассмотрим, как сделать проект на Clojure удобным в
настройке. Разберем основы конфигурации: форматы файлов, переменные среды,
библиотеки, их достоинства и недостатки.
\end{teaser}

\section{Постановка проблемы}

В материалах по Clojure встречаются примеры:

\begin{english}
  \begin{clojure}
(def server
  (jetty/run-jetty app {:port 8080}))

(def db {:dbtype   "postgres"
         :dbname   "test"
         :user     "ivan"
         :password "test"})
  \end{clojure}
\end{english}

Это сервер на 8080 порту и параметры подключения к базе. Польза примеров в том,
что их можно скопировать в REPL и оценить результат: открыть страницу в браузере
или прочитать таблицу.

На практике код пишут так, чтобы в нем не было конкретных цифр и строк. С точки
зрения проекта плохо, что серверу явно задали порт. Это подойдет для
документации и примеров, но не боевого запуска.

Порт 8080 и другие комбинации нулей и восьмерок популярны у
программистов. Велики шансы, что порт занят другим сервером. Это случается,
когда запускают не отдельный сервис, а их связку на время разработки или тестов.

Код, написанный программистом, проходит несколько стадий. В~разных фирмах набор
отличается, но в целом это разработка, тестирование, предварительный и боевой
запуск.

На каждой стадии приложение запускают бок о бок с другими
проектами. Предположение, что порт 8080 свободен в любой момент это утопия. На
жаргоне разработчиков ситуацию называют <<хардкод>> (анг. <<hardcode>>) или
<<прибито гвоздями>>. Когда в коде <<прибитые>> значения, это вносит проблемы в
его цикл. Вы не сможете одновременно работать над проектом и тестировать его.

Приложение не должно знать порт сервера~--- информация об этом приходит извне. В
простом случае это файл настроек. Программа читает из него порт и запускает
сервер именно так, как это нужно на конкретной машине.

В более сложных сценариях файл составляет не человек, а специальная
программа~--- менеджер конфигураций. Менеджер хранит информацию о топологии
сети, адреса машин, параметры доступа к базам. По запросу он выдает файл для
определенной машины или сегмента сети.

Процесс, когда приложению сообщают параметры, а оно принимает их, называется
конфигурацией. Это интересный и важный процесс. Когда он устроен удачно, проект
легко проходит по всем стадиям производства.

\section{Семантика}

Цель конфигурации в том, чтобы управлять программой, не меняя ее код. К ней
приходят с ростом кодовой базы и инфраструктуры. Если у вас мелкий скрипт на
Python, нет ничего зазорного в том, чтобы открыть его в блокноте и поменять
константу. На предприятиях такие скрипты работают годами.

Чем сложнее инфраструктура фирмы, тем больше в ней ограничений. Современный
подход сводит на нет спонтанные изменения в проекте. Нельзя сделать \spverb|git push|
напрямую в мастер; запрещен \spverb|merge|, пока вашу работу не одобрят
двое коллег; приложение не попадет на сервер, пока не пройдут тесты.

Это приводит к тому, что малейшее изменение в коде займет час, чтобы попасть в
бой. Правка в конфигурации дешевле, чем выпуск новой версии продукта. Из этого
следует правило: если можно вынести что-то в конфигурацию, сделайте это сейчас.

\label{feature-flags}

В крупных фирмах практикуют то, что называют <<feature flag>>. Это логическое
поле, которое включает целый пласт логики в приложении. Например, новый
интерфейс, систему обработки заявок, улучшенный чат. Обновления тестируют внутри
фирмы, но всегда остается риск, что в бою \emph{что-то пойдет не так}. Тогда
флаг меняют на ложь и перезапускают сервер. Это не только сэкономит время, но и
сохранит репутацию фирмы.

\section{Цикл конфигурации}

При запуске приложение ищет конфигурацию. Чем лучше устроено приложение, тем
больше его частей опирается на параметры. Обработка конфигурации это не
монолитная задача, а набор шагов. Перечислим наиболее важные из них.

На первом этапе программа \textbf{читает конфигурацию}. Чаще всего это файл или
переменные среды. Данные в файлах хранят в форматах JSON, YAML и
других. Приложение несет на борту код, чтобы разобрать формат и получить
данные. Мы рассмотрим плюсы и минусы известных форматов ниже.

Переменные среды это часть операционной системы. Представьте~их как глобальный
словарь в памяти. Каждое приложение наследует его при запуске. Языки и
фреймворки предлагают функции, чтобы считать переменные в строки и словари.

Файлы и переменные среды дополняют друг друга. Например, приложение читает
данные из файла, но путь к нему ищет в переменных среды. Или в файле опущены
критические данные: пароли, API-ключи. Так поступают, чтобы их не увидели другие
программы, в том числе шпионские. Приложение читает параметры из файла, а
секретные данные из переменных.

Продвинутые конфигурации используют теги. В файле перед значением
ставят тег: \spverb|:password #env DB_PASSWORD|. Это значит, в поле
\spverb|password| не строка \spverb|DB_PASSWORD|, а значение одноименной
переменной.

Первый этап завершается тем, что мы получили данные. Неважно, был ли это файл,
переменные среды или что-то другое. Приложение переходит ко второму этапу~---
\textbf{выводу типов}.

JSON или YAML выделяют базовые типы: строки, числа, булево и~null. Легко
заметить, что среди них нет даты. С помощью дат задают промо-акции или события,
связанные с праздниками. В файлах даты указывают либо строкой в формате ISO,
либо числом секунд c 1 января 1970 года (эпоха UNIX). Специальный код пробегает
про данным и приводит даты к типу, принятому в языке.

Вывод типов применяют и для коллекций. Иногда словарей и массивов не хватает для
комфортной работы. Типы события хранят в виде множества, потому что оно отсекает
дубли и предлагает быструю проверку на вхождение. Скаляры тоже оборачивают в
классы, например, \spverb|UUID| для идентификаторов.

Переменные среды не настолько гибки, как современные форматы. Если JSON выделяет
скаляры и коллекции, то переменные несут только текст. Вывод типов для них не
просто желателен, а необходим. Нельзя передать порт в виде строки туда, где
ожидают число.

После вывода типов приступают к \textbf{валидации данных}. В главе про Spec мы
выяснили, что тип не обещает верное значение \page{type-and-pred}.
Проверка нужна, чтобы в конфигурации нельзя было указать порт 0, -1 или 80.

Из той же главы мы помним, что иногда значения верны по отдельности, но не могут
быть в паре. Пусть в конфигурации задан период акции. Это массив из двух дат,
начало и завершение. Легко перепутать даты местами, и проверка любой даты
на интервал верн\"{е}т ложь.

После валидации переходят к последней стадии. Приложение решает, где
\textbf{хранить конфигурацию}. Это может быть глобальная переменная или
компонент системы. Другие части программы читают параметры уже оттуда, а не из
файла.

\section{Ошибки конфигурации}

На каждом этапе может возникнуть ошибка: не найден файл, нарушения в синтаксисе,
неверное поле. В этом случае программа выводит сообщение и завершается.
Сообщение должно четко ответить на вопрос, что случилось. Часто программисты
держат в голове только положительный путь и забывают об ошибках. При запуске
их программ виден стек-трейс, который трудно понять.

Если ошибка случилась на этапе проверки, объясните, какое поле тому виной. В
главе про Spec мы рассмотрели, как улучшить отчет спеки \page{spec-messages}.
Это требует усилий, но окупается со временем.

В IT-индустрии одни сотрудники пишут код, а другие управляют им. Ваш
коллега-DevOps не знает, что такое Clojure и не поймет сырой
\spverb|explain|. Рано или поздно он попросит доработать сообщения
конфигурации. Сделайте это заранее из уважения к коллегам.

Если с конфигурацией что-то не так, программа не должна работать в надежде, что
вс\"{е} обойдется. Бывает, когда один из параметров задан неверно, но программа к
нему не обращается. Избегайте этого: ошибка появится в неподходящий
момент.

Когда один из шагов конфигурации не сработал, программа завершается с кодом,
отличным от нуля. Сообщение пишут в канал \spverb|stderr|, чтобы подчеркнуть
внештатную ситуацию. Продвинутые терминалы пчатают \spverb|stderr| красным
цветом.

\section{Загрузчик конфигурации}

Чтобы закрепить теорию, напишем систему конфигурации. Это отдельный модуль
примерно на сто строк. Прежде чем садиться за редактор, обдумаем основные
положения.

Будем хранить конфигурацию в JSON-файле. Считаем, что фирма недавно перешла на
Clojure, и у DevOps уже написаны скрипты на Python для управления
настройками. Формат EDN усложнит работу коллегам.

Путь к файлу задают в переменной среды \spverb|CONFIG_PATH|. От файла мы ожидаем
порт сервера, параметры базы данных и диапазон дат для промо-акции. Даты должны
стать объектами \spverb|java.util.Date|. Дата начала строго меньше конца.

Готовый словарь запишем в глобальную переменную \spverb|CONFIG|. Если на одном
из шагов случилась ошибка, покажем сообщение и завершим программу.

Начнем со вспомогательной функции \spverb|exit|. Она принимает код завершения,
текст и параметры форматирования. Если код равен нулю, пишем сообщение в
\spverb|stdout|, иначе в \spverb|stderr|.

\begin{english}
  \begin{clojure}
(defn exit
  [code template & args]
  (let [out (if (zero? code) *out* *err*)]
    (binding [*out* out]
      (println (apply format template args))))
  (System/exit code))
  \end{clojure}
\end{english}

Переходим к загрузчику. Это набор шагов, каждый из которых принимает результат
предыдущего. Логику каждого легко понять по имени. Вывод типов и валидация
совмещены в \spverb|coerce-config|, поскольку технически это вызов
\spverb|s/conform|.

\begin{english}
  \begin{clojure}
(defn load-config! []
  (-> (get-config-path)
      (read-config-file)
      (coerce-config)
      (set-config!)))
  \end{clojure}
\end{english}

Теперь опишем каждый шаг. Функция \spverb|get-config-path| читает переменную
среды и проверяет, есть ли такой файл на диске. Если вс\"{е} в~порядке, функция
верн\"{е}т путь к файлу, а иначе вызовет \spverb|exit|:

\begin{english}
  \begin{clojure}
(import 'java.io.File)

(defn get-config-path []
  (if-let [filepath (System/getenv "CONFIG_PATH")]
    (if (-> filepath (new File) .exists)
      filepath
      (exit 1 "File %s does not exist" filepath))
    (exit 1 "File path is not set")))
  \end{clojure}
\end{english}

Шаг \spverb|read-config-file| читает файл по его пути. Для разбора JSON служит
библиотека Cheshire. Функция \spverb|parse-string| верн\"{е}т данные из
строки документа.

\begin{english}
  \begin{clojure}
(require '[cheshire.core :as json])

(defn read-config-file
  [filepath]
  (try
    (-> filepath slurp (json/parse-string true))
    (catch Exception e
      (exit 1 "Malformed config file: %s" (ex-message e)))))
  \end{clojure}
\end{english}

Вывод типов и проверка это самый важный этап. Нельзя допустить, чтобы приложение
получило неверные параметры. Шаг \spverb|coerce-config| пропускает данные из
файла через \spverb|s/conform|. Вызов опасен исключением, поэтому обернем его в
\spverb|pcall| (безопасный вызов, который верн\"{е}т ошибку и результат). Если
результат \spverb|::s/invalid|, формируем отчет об ошибке и завершаем
программу. Для отчета возьмем библиотеку Expound.

\begin{english}
  \begin{clojure}
(require '[clojure.spec.alpha :as s])
(require '[expound.alpha :as expound])

(defn coerce-config [config]
  (let [[e result] (pcall s/conform ::config config)]
    (cond
      (some? e)
      (exit 1 "Wrong config values: %s" (ex-message e))

      (s/invalid? result)
      (let [report (expound/expound-str ::config config)]
        (exit 1 "Invalid config values: %s %s" \newline report))

      :else result)))
  \end{clojure}
\end{english}

Не хватает спеки. Откроем конфигурацию и изучим ее структуру:

\begin{english}
  \begin{json}
{
    "server_port": 8080,
    "db": {
        "dbtype":   "mysql",
        "dbname":   "book",
        "user":     "ivan",
        "password": "****"
    },
    "event": [
        "2019-07-05T12:00:00",
        "2019-07-12T23:59:59"
    ]
}
  \end{json}
\end{english}

\noindent
Опишем спеку сверху вниз. На верхнем уровне это словарь с ключами:

\begin{english}
  \begin{clojure}
(s/def ::config
  (s/keys :req-un [::server_port ::db ::event]))
  \end{clojure}
\end{english}

Порт сервера это комбинация двух предикатов: число и вхождение в диапазон.
Проверка на число нужна, чтобы во второй предикат не попали \spverb|nil| и
строка. Это вызовет исключение там, где его не ждали.

\begin{english}
  \begin{clojure}
(s/def ::server_port
  (s/and int? #(<= 1024 % 65534)))
  \end{clojure}
\end{english}

\noindent
Подключение к базе данных. Напомним, что \spverb|::ne-string| означает
непустую строку.

\begin{english}
  \begin{clojure}
(s/def :db/dbtype   #{"mysql"})
(s/def :db/dbname   ::ne-string)
(s/def :db/user     ::ne-string)
(s/def :db/password ::ne-string)

(s/def ::db
  (s/keys :req-un [:db/dbtype
                   :db/dbname
                   :db/user
                   :db/password]))
  \end{clojure}
\end{english}

Поле \spverb|event| сложное. На верхнем уровне это кортеж дат и проверка
диапазона:

\begin{english}
  \begin{clojure}
(s/def ::event
  (s/and (s/tuple ::->date ::->date)
         ::date-range))
  \end{clojure}
\end{english}

Спека \spverb|::->date| выводит дату из строки. Чтобы не парсить е\"{е} вручную,
возьмем функцию \spverb|read-instant-date| из пакета \spverb|clojure.instant|.
Функция лояльна к формату и читает неполные даты, например, только год.
Обернем ее в \spverb|s/conformer|. Впереди ставим \spverb|::ne-string|,
чтобы отсечь мусор.

\begin{english}
  \begin{clojure}
(require '[clojure.instant :as inst])

(s/def ::->date
  (s/and ::ne-string (s/conformer read-instant-date)))
  \end{clojure}
\end{english}

Опишем проверку диапазона. Она принимает пару объектов \spverb|java.util.Date| и
сравнивает их. Даты нельзя сравнить знаками <<больше>> или <<меньше>>. Для этого
служит функция \spverb|compare|, которая верн\"{е}т -1, 0 и 1 для случаев меньше,
равно и больше.

\begin{english}
  \begin{clojure}
(s/def ::date-range
  (fn [[date1 date2]]
    (neg? (compare date1 date2))))
  \end{clojure}
\end{english}

На этом месте можно вызвать \spverb|load-config!| и убедиться, что на выходе
словарь с правильными типами. Последний шаг \spverb|set-config!| пишет словарь в
глобальную переменную \spverb|CONFIG|. Имя в верхнем регистре, чтобы не затенить
его локальной \spverb|config|. Для подмены переменной ипользуем
\spverb|alter-var-root|.

\begin{english}
  \begin{clojure}
(def CONFIG nil)

(defn set-config!
  [config]
  (alter-var-root (var CONFIG) (constantly config)))
  \end{clojure}
\end{english}

На старте программы выполните \spverb|(load-config!)|, чтобы в переменной
появилась конфигурация. Другие модули импортируют \spverb|CONFIG| и читают
нужные им ключи. Вот как запустить сервер или выполнить запрос с учетом
конфигурации:

\begin{english}
  \begin{clojure}
(require '[project.config :refer [CONFIG]])

(jetty/run-jetty app {:port (:server_port CONFIG)
                      :join? false})

(jdbc/query (:db CONFIG) "select * from users")
  \end{clojure}
\end{english}

\subsection{Работа над ошибками}

Мы написали загрузчик конфигурации. Он удобен в поддержке: каждый шаг это
функция, которую легко доработать. Это не промышленное решение, но подойдет для
небольших проектов.

В любой момент можно считать конфигурацию заново. Это удобно при разработке:
измените файл и выполните \spverb|load-config!| в REPL. В \spverb|CONFIG|
появится новая конфигурация.

Недостаток загрузчика в том, что код привязан к функции \spverb|exit|, которая
завершает JVM. Это верный подход в боевом запуске: нельзя продолжать, если в
параметрах ошибка. Однако в разработке от завершения больше проблем: любая
ошибка убивает REPL, и нужно включать его заново.

Завершение JVM это слишком радикальное действие. Ошибка и реакция на не\"{е}
Наивный способ~--- вызвать \spverb|load-config!| в рамках подмены
\spverb|exit|. Функция ниже не завершит JVM, а только бросит исключение с
текстом, который передали в \spverb|exit|:

\begin{english}
  \begin{clojure}
(defn load-config-repl! []
  (with-redefs
    [exit (fn [_ template & args]
            (let [^String message
                  (apply format template args)]
              (throw (new Exception message))))]
    (load-config!)))
  \end{clojure}
\end{english}

Более удачное решение~--- передать в \spverb|load-config!| дополнительные
параметры. Один из них это \spverb|:die-fn|, <<функция смерти>>, которая
принимает исключение. В боевом запуске она завершает JVM, а в разработке пишет
сообщение в REPL. Доработайте загрузчик так, чтобы он поддерживал параметр
\spverb|:die-fn|. Продумайте поведение по умолчанию, если он не задан.

Для вывода типов загрузчик опирается на \spverb|s/conform|. В главе про Spec мы
рассмотрели случай, когда \spverb|s/conform| добавляет логические теги и меняет
структуру данных \page{jdbc-conform-warning}. Если заменить спеку \spverb|::db|
на \spverb|::jdbc/db-spec|, получим тот самый случай. Чтобы избежать этого, мы
задали свою версию \spverb|::db| без макросов \spverb|s/or|.

По-другому типы выводят с помощью \emph{тегов}. Эту технику мы обсудим в
следующих разделах главы.

\section{Подробнее о переменных среды}

Загрузчик читает данные из файла. Из переменных среды он берет лишь малую
часть~--- путь к файлу. Изменим загрузчик: пусть он читает данные из переменных
среды и не нуждается в файлах. Чтобы понять преимущества этого подхода,
поговорим о переменных в отрыве от языка.

Переменные среды называют ENV, <<энвы>> (анг. <<environment>>, окружающая
среда). Это фундаментальное свойство операционной системы. Представьте
переменные как глобальный словарь, который наполняется во время запуска
компьютера. В словаре основные параметры системы: локаль, список путей, где
система ищет бинарные файлы и многое другое.

Чтобы увидеть текущие переменные, запустите в терминале \spverb|env| или
\spverb|printenv|. На экране появится пары \spverb|ИМЯ=значение|. Имена
переменных в верхнем регистре, чтобы выделить на общем фоне и подчеркнуть
приоритет. Большинство систем различают регистр, поэтому \spverb|foo| и
\spverb|FOO| это разные переменные. Пробелы и дефисы недопустимы; лексемы
отделяют подчеркиванием. Фрагмент \spverb|printenv|:

\begin{english}
  \begin{bash}
LANG=en_US.UTF-8
PWD=/Users/ivan
SHELL=/bin/zsh
TERM_PROGRAM=iTerm.app
COMMAND_MODE=unix2003
  \end{bash}
\end{english}

Каждый процесс получает копию этого словаря. Процесс может добавить или удалить
переменную, но изменения видны только ему и потомкам. Потомок процесса наследует
переменные родителя.

\subsection{Локальные и глобальные переменные}

Различают переменные \emph{среды} и \emph{шелла}, они же глобальные и локальные
переменные. Их часто путают новички. Выполните в терминале команду:

\begin{english}
  \begin{bash}
> FOO=42
  \end{bash}
\end{english}

Вы задали переменную шелла (командной оболочки). Чтобы сослаться на значение по
имени (<<дерефнуть>>), поставьте перед ней доллар. Пример напечатает 42:

\begin{english}
  \begin{bash}
> echo $FOO
42
  \end{bash}
\end{english}

Если выполнить \spverb|printenv|, мы не увидим \spverb|FOO| в выводе. Команда
\spverb|FOO=42| задает переменную шелла, а не среды. Эти переменные видны только
ему и не наследуются потомками. Проверим это: из текущего шелла запустим новый и
повторим печать.

\begin{english}
  \begin{bash}
> sh
> echo $FOO
  \end{bash}
\end{english}

Получим пустую строку, потому что у потомка нет такой переменной Выполните
\spverb|exit|, чтобы вернуться в родительский шелл.

Команда \spverb|export| помещает переменную в среду. Заданная таким образом,
переменная видна \spverb|printenv| и доступна потомкам:

\begin{english}
  \begin{bash}
> export FOO=42

> printenv | grep FOO
FOO=42

> sh
> echo $FOO
42
  \end{bash}
\end{english}

Иногда нужно запустить процесс с переменной, но так, чтобы не повлиять на
текущее состояние. Тогда команду указывают после выражения
\spverb|ИМЯ=значение|:

\begin{english}
  \begin{bash}
> BAR=99 printenv | grep BAR
BAR=99
  \end{bash}
\end{english}

\spverb|Printenv| порождает новый процесс, которому доступна переменная
\spverb|BAR|. Если снова напечатать \spverb|$BAR|, получим пустую строку.

Программы читают параметры из переменных среды. Клиент к базе данных PostgreSQL
различает два десятка переменных: \spverb|PGHOST|, \spverb|PGDATABASE|,
\spverb|PGUSER| и другие. У переменных среды выше приоритет, чем у параметров
\spverb|--host|, \spverb|--user| и аналогов. Если в текущем шелле выполнить:

\begin{english}
  \begin{bash}
> export PGHOST=host.com PGDATABASE=project
  \end{bash}
\end{english}

\noindent
, то каждая утилита PostgreSQL сработает на заданном сервере и базе. Это удобно
для серии команд: не прид\"{е}тся каждый раз указывать \spverb|--host| и другие
аргументы.

Обратите внимание на префикс \spverb|PG|. Он нужен, чтобы не затереть чужую
переменную \spverb|HOST|. В среде нет пространств имен, поэтому префикс это
единственный способ отделить ваши переменные от других.

\section{Конфигурация в среде}

Переменными среды можно задать конфигурацию. Каждый язык предлагает функции,
чтобы читать переменные в строки или словарь. Разберем плюсы и минусы этого
подхода.

Поскольку окружение находится в памяти, приложение не обращается к диску при их
чтении. Дело не в том, что память быстрее диска: человек не отличит сотую долю
секунды от тысячной. Приложение, которое не зависит от файлов более автономно и
потому удобней в поддержке.

Иногда файл конфигурации оказался в другой папке, и приложение не может его
найти. Или, что хуже, запускается со старой версией файла. Это замедляет работу
и вносит путаницу.

\label{password-note}

Хранить пароли и ключи в переменных безопаснее, чем в файлах. В случае с файлами
их могут прочесть другие программы, в том числе вредоносные. По ошибке файл
может попасть в репозиторий и остаться в истории. Существуют скрипты, которые
ищут в открытых репозиториях AWS-ключи (и порой находят, к сожалению).

Даже если файлом владеет другой пользователь, он может быть доступен для чтения
остальным. Переменные среды эфемерны: они живут только памяти операционной
системы. Один пользователь не может прочесть переменные другого.

Индустрия переходит от файлов к \emph{контейнерам}. Если раньше мы копировали
файлы по FTP, то сегодня приложение запускают из образов. Это архив, в который
упакован код и его окружение. В отличии от настоящего архива, образ нельзя
изменить. Чтобы обновить файл в образе, нужно собрать его заново, что усложняет
процесс.

Наоборот, виртуализация лояльна к переменным среды. Их указывают при запуске
образа в параметрах. Один и тот же образ запускают с разными переменными; новая
сборка не требуется. Чем больше опций можно задать переменными, тем удобней
образ в работе. Ниже сервер PostgreSQL включается с готовой базой и
пользователем:

\begin{english}
  \begin{bash}
> docker run \
  -e POSTGRES_DB=book \
  -e POSTGRES_USER=ivan \
  -e POSTGRES_PASSWORD=**** \
  -d postgres
  \end{bash}
\end{english}

Принцип <<конфигурация в среде>> описан в <<The Twelve-Factor
App>>\footurl{https://12factor.net/}. Это свод правил для разработки надежных
приложений. Его третий пункт посвящен конфигурации. Автор упоминает те же плюсы
переменных, что мы рассмотрели: независимость от файлов, безопасность, поддержка
на всех платформах.

\section{Недостатки среды}

Переменные не поддерживают типы: любое значение это текст. Вывод типов остается
на ваше усмотрение. Делайте это декларативно, а не вручную. Неудачный пример на
Python:

\begin{english}
  \begin{python}
db_port = int(os.environ["DB_PORT"])
  \end{python}
\end{english}

Когда переменных больше двух, код становится уродливым. Задайте словарь, где
ключ это имя переменной, а значение~--- функция вывода. Специальный код обходит
словарь и наполняет результат. Для краткости опустим перехват ошибок:

\begin{english}
  \begin{python}
import os
env_mapping = {"DB_PORT": int}

result = {}
for (env, fn) in env_mapping.iteritems():
    result[env] = fn(os.environ[env])
  \end{python}
\end{english}

\noindent
Подход справедлив и для других языков. В Clojure для этого служит спека.

Переменные среды не работают с иерархией. Это плоский набор ключей и значений,
что не всегда ложиться на конфигурацию. Чем больше параметров, тем чаще их
группируют по смыслу. Предположим, подключение к базе задают задают десять
параметров. Чтобы не ставить перед каждым префикс, их выносят в дочерний
словарь.

\noindent
\begin{tabular}{ @{}p{5cm} @{}p{5cm} }

\begin{english}
  \begin{clojure}
;; so-so
{:db-name "book"
 :db-user "ivan"
 :db-pass "****"}
  \end{clojure}
\end{english}

&

\begin{english}
  \begin{clojure}
;; better
{:db {:name "book"
      :user "ivan"
      :pass "****"}}
  \end{clojure}
\end{english}

\end{tabular}

В разных системах вложенные переменные читают по-разному. Например, одинарное
подчеркивание разделет лексемы, но не меняет структуру. Двойное подчеркивание
означает вложенность:

\noindent
\begin{tabular}{ @{}p{5cm} @{}p{5cm} }

\begin{english}
  \begin{clojure}
"DB_NAME=book"
;; {:db-name "book"}
  \end{clojure}
\end{english}

&

\begin{english}
  \begin{clojure}
"DB__NAME=book"
;; {:db {:name "book"}}
  \end{clojure}
\end{english}

\end{tabular}

Массив задают в квадратных скобках или через запятую. При разборе массива
возникает риск ложного разбиения. Это случается, когда запятая или скобка
относится к слову, а не к синтаксису.

Форматы JSON и YAML задают четкий стандарт, как описывать коллекции. Для
переменных среды нет единого соглашения. Ситуация ухудшается, когда ожидают
параметр большой вложенности, например, список словарей. Переменные среды не
справляются с такой структурой.

В разработке встает еще один недостаток переменных~--- в некоторых системах
они доступны только для чтения. Это верно идеологически, но вынуждает заново
включать REPL на каждое изменение конфигурации. Для файла достаточно изменить
его и прочитать заново.

\subsection{Env-файлы}

Когда переменных много, вводить их вручную через \spverb|export|
утомительно. Переменные выносят в файл, который называют
\emph{env}-конфигурацией.

Технически это шелл-скрипт, но чем меньше в нем скриптовых возможностей, тем
лучше. В идеале в файле только пары \spverb|ИМЯ=значение| по одной на каждую
строку. Назовем файл просто \spverb|ENV| без расширения.

\begin{english}
  \begin{bash}
DB_NAME=book
DB_USER=ivan
DB_PASS=****
  \end{bash}
\end{english}

Чтобы считать переменные в шелл, вызывают \spverb|source <file>|. Это одна из
команд \spverb|bash|, которая выполняет скрипт в \emph{текущем} сеансе. Cкрипт
добавит переменные в шелл, и вы увидите их после \spverb|source|. Это
важное отличие от команды \spverb|bash <file>|, которая выполнит скрипт в новом
шелле.

\begin{english}
  \begin{bash}
> source ENV
> echo $DB_NAME
book
  \end{bash}
\end{english}

Если запустить из текущего шелла приложение, оно не получит переменные
файла. Вспомним, что выражение \spverb|VAR=value| задает локальную
переменную. \spverb|DB_NAME| и другие не попадут в окружение, и программа не
унаследует их. Проверим это с помощью \spverb|printenv|:

\begin{english}
  \begin{bash}
> source ENV
> printenv | grep DB
# exit 1
  \end{bash}
\end{english}

Есть два способа решить проблему. Первый~--- открыть файл и расставить перед
каждой парой выражение \spverb|export|. Тогда \spverb|source| этого файла
добавит переменные в окружение:

\begin{english}
  \begin{bash}
> cat ENV
export DB_NAME=book
export DB_USER=ivan
export DB_PASS=****

> source ENV
> printenv | grep DB
DB_NAME=book
DB_USER=ivan
DB_PASS=****
  \end{bash}
\end{english}

Недостаток в том, что в файле появилась логика скрипта. Если не поставить
\spverb|export| перед переменной, и приложение не прочитает ее.

Второй способ основан на параметре \spverb|-a| (\textbf{a}llexport) текущего
шелла. Когда он установлен, локальная переменная попадает в окружение. Перед
тем, как читать переменные из файла, флаг возводят в истину, а затем снова в
ложь.

\begin{english}
  \begin{bash}
> set -a
> source ENV
> printenv | grep DB
# prints all the vars
> set +a
  \end{bash}
\end{english}

Выражение \spverb|set| противоречиво: параметр с минусом \emph{включает}
параметр, а с плюсом~--- отключает. Это исключение, которое прид\"{е}тся запомнить.

Если считать переменную, которая уже в окружении, она заменит прежнее
значение. Так появляются файлы с переопределениями. Если нужны особые настройки
для тестов, не обязательно копировать файл. Создайте файл с полями,
которые нужно заменить и выполните после главного.

Пусть тестовые настройки отличаются именем базы. Файл \spverb|ENV| несет
основные параметры. В \spverb|ENV_TEST| поместим новое значение
\spverb|DB_NAME=test|, затем прочтем оба файла.

\begin{english}
  \begin{bash}
> set -a
> source ENV
> source ENV_TEST
> set +a

> echo $DB_NAME
test
  \end{bash}
\end{english}

Читатель заметит, что ENV-файлы противоречивы. Мы сказали, что переменные
снимают зависимость от файлов, но в итоге поместили их в файл. В ч\"{е}м смысл?

Разница между JSON- и ENV-файлами в том, кто их читает. В первом случае это
приложение, а во втором~--- операционная система. Файл находится в строго
заданной директории, а переменные среды доступны ото всюду. Мы избавим
приложение от кода, который ищет и читает файл. Облегчим работу коллегам DevOps:
они задают переменные по-разному в зависимости от инструмента (шелл, Docker,
Kubernetes).

\section{Переменные среды в Clojure}

Clojure это гостевая платформа, поэтому язык не предлагает доступ к системным
ресурсам. В главном модуле нет функции для чтения переменных среды. Получим их
из класса \spverb|java.lang.System|. Импортировать класс не нужно, он доступен в
любом пространстве имен.

Статический метод \spverb|getenv| верн\"{е}т либо одну переменную по имени, либо
весь словарь, если имя не указали.

\begin{english}
  \begin{clojure}
;; a single variable
(System/getenv "HOME")
"/Users/ivan"

;; all variables
(System/getenv)
{"JAVA_ARCH" "x86_64", "LANG" "en_US.UTF-8"} ;; truncated
  \end{clojure}
\end{english}

Во втором случае получили не Clojure-, а Java-коллекцию. Это постоянная версия
класса \spverb|Map|, поэтому переменные нельзя изменить после запуска JVM.

Чтобы облегчить работу со словарем, приведем его к типу Clojure. Заодно исправим
ключи: сейчас это строки в верхнем регистре и подчеркиваниями. В Clojure
пользуются кейвордами и записью <<kebab-case>>: нижний регистр с дефисами.

Напишем функцию для перевода ключа:

\begin{english}
  \begin{clojure}
(require '[clojure.string :as str])

(defn remap-key [^String key]
  (-> key
      str/lower-case
      (str/replace #"_" "-")
      keyword))
  \end{clojure}
\end{english}

\noindent
и убедимся в ее работе:

\begin{english}
  \begin{clojure}
(remap-key "DB_PORT")
:db-port
  \end{clojure}
\end{english}

Функция \spverb|remap-env| проходит по словарю Java и возвращает его
<<кложурную>> версию с привычными ключами:

\begin{english}
  \begin{clojure}
(defn remap-env [env]
  (reduce
   (fn [acc [k v]]
     (let [key (remap-key k)]
       (assoc acc key v)))
   {}
   env))
  \end{clojure}
\end{english}

\noindent
Приведем небольшую часть словаря:

\begin{english}
  \begin{clojure}
(remap-env (System/getenv))

{:home "/Users/ivan"
 :lang "en_US.UTF-8"
 :term "xterm-256color"
 :java-arch "x86_64"
 :term-program "iTerm.app"
 :shell "/bin/zsh"}
  \end{clojure}
\end{english}

Когда переменные это словарь, он идет по тому же конвейеру: вывод типов,
валидация спекой. Поскольку все значения строки, спеку нужно доработать, чтобы
она выводила числа из строк. Раньше в этом не было нужды, потому что числа
приходили из JSON в готовом виде. Будет удачной спека, которая учитывает и
число, и строку. Вот как выглядит умный парсер числа:

\begin{english}
  \begin{clojure}
(s/def ::->int
  (s/conformer
   (fn [value]
     (cond
       (int? value) value
       (string? value)
       (try (Integer/parseInt value)
            (catch Exception e
              ::s/invalid))
       :else ::s/invalid))))
  \end{clojure}
\end{english}

\noindent
С этой спекой можно менять источник данных без изменений в коде.

\subsection{Проблема лишних ключей}

У словаря переменных недостаток~--- много посторонних полей. Приложению не нужно
знать версию терминала или путь к Python. Эти поля вносят шум при печати и
записи в лог. Если спека не прошла проверку, мы увидим лишние данные в
\spverb|explain|.

На последнем шаге \spverb|s/conform| нужно выбрать из словаря только полезную
часть. Функция \spverb|select-keys| верн\"{е}т подмножество по списку ключей. Но где
взять ключи? Перечислять их вручную долго, и к тому же мы дублируем код. Мы уже
указали ключи в спеке \spverb|::config|, и делать это второй раз не хочется. С
помощью трюка вытащим ключи \emph{из спеки}.

Функция \spverb|s/form| принимает ключ спеки и возвращает <<замороженную>> форму
того, что передали в \spverb|s/def|. Это список, где каждый элемент примитив
(символ, кейворд) или коллекция (вектор, словарь). Для спеки \spverb|::config|
получим форму:

\begin{english}
  \begin{clojure}
(clojure.spec.alpha/keys
 :req-un [:book.config/server_port
          :book.config/db
          :book.config/event])
  \end{clojure}
\end{english}

Обратите внимание: это \emph{список}, а не код. Нужные ключи находятся в третьем
элементе после \spverb|:req-un|. Ст\'{о}ит учесть и другие типы ключей, например
\spverb|:opt-un|. Напишем универсальную функцию, которая верн\"{е}т все ключи из
\spverb|s/keys|.

Отбросим первый символ формы. Останется список, где нечетные элементы тип ключа,
а четные~--- их вектор. Перестроим список в словарь и объединим значения. Для
ключей \spverb|-un| отбросим пространство. Все вместе дает нам функцию:

\begin{english}
  \begin{clojure}
(defn spec->keys
  [spec-keys]
  (let [form (s/form spec-keys)
        params (apply hash-map (rest form))
        {:keys [req opt req-un opt-un]} params
        ->unqualify (comp keyword name)]
    (concat req
            opt
            (map ->unqualify opt-un)
            (map ->unqualify req-un))))
  \end{clojure}
\end{english}

\noindent
Проверим спеку загрузчика:

\begin{english}
  \begin{clojure}
(spec->keys ::config)
(:server_port :db :event)
  \end{clojure}
\end{english}

\noindent
Перепишем чтение переменных в словарь:

\begin{english}
  \begin{clojure}
(defn read-env-vars []
  (let [cfg-keys (spec->keys ::config)]
    (-> (System/getenv)
        (remap-env)
        (select-keys cfg-keys))))
  \end{clojure}
\end{english}

\noindent
Так мы получим только полезные ключи, то есть те, что указаны в спеке.

\subsection{Загрузчик среды}

Доработаем загрузчик, чтобы он работал с переменными среды. Замените первые два
шага на \spverb|read-env-vars|. Теперь программа не зависит от файла
конфигурации.

\begin{english}
  \begin{clojure}
(defn load-config! []
  (-> (read-env-vars)
      (coerce-config)
      (set-config!)))
  \end{clojure}
\end{english}

Сделайте так, чтобы источник данных можно было задать параметром. Например,
\spverb|:source "/path/to/config.json"| означает считать файл, а
\spverb|:source :env|~--- переменные среды.

Ещ\"{е} сложнее: как считать оба источника и объединить их? Важен ли порядок и
как его обеспечить? Как объединить словари ассиметрично, то есть когда второй
только \emph{замещает} поля первого, но не дополняет новыми?

\subsection{Вывод структуры}

Редко бывает так, что конфигурация это плоский словарь. Близкие параметры
выносят во вложенные словари, например, отдельно поля сервера и базы
данных. Сгруппированные настройки легче в поддержке.

Улучшим загрузчик: научим его читать вложенные переменные. Договоримся, что
двойное подчеркивание означает смену уровня. Поместим в файл \spverb|ENV_NEST|
следующие переменные:

\begin{english}
  \begin{bash}
DB__NAME=book
DB__USER=ivan
DB__PASS=****
HTTP__PORT=8080
HTTP__HOST=api.random.com
  \end{bash}
\end{english}

\noindent
Прочитаем его и запустим REPL с новой средой:

\begin{english}
  \begin{bash}
> set -a
> source ENV_NEST
> lein repl
  \end{bash}
\end{english}

Изменим обход окружения и разбор ключа. Функция \spverb|remap-key-nest|
принимает ключ-строку и возвращает вектор составных частей (лексем):

\begin{english}
  \begin{clojure}
(defn remap-key-nest
  [^String key]
  (-> key
      str/lower-case
      (str/replace #"_" "-")
      (str/split #"--")
      (->> (map keyword))))

(remap-key-nest "DB__PORT")
;; (:db :port)
  \end{clojure}
\end{english}

Функция обхода делает не \spverb|assoc|, а \spverb|assoc-in|, что порождает
вложенность:

\begin{english}
  \begin{clojure}
(defn remap-env-nest
  [env]
  (reduce
   (fn [acc [k v]]
     (let [key-path (remap-key-nest k)]
       (assoc-in acc key-path v)))
   {}
   env))
  \end{clojure}
\end{english}

Код ниже верн\"{е}т параметры по группам, как и задумано. Приведем их подмножество:

\begin{english}
  \begin{clojure}
(-> (System/getenv)
    (remap-env-nest)
    (select-keys [:db :http]))

{:db {:user "ivan", :pass "****", :name "book"},
 :http {:port "8080", :host "api.random.com"}}
  \end{clojure}
\end{english}

\noindent
Дальше действуем как обычно: пишем спеку, выводим типы из строк и так далее.

Подумайте, как задать в переменной массив. Чем разделить элементы? Когда
возможно ложное разбиение и как его предотвратить?

\section{Простой менеджер конфигурации}

К этому моменту читатель решит, что конфигурация в файле это плохая идея. Однако
не бросайтесь переписывать код на переменные среды. На практике работают с
\emph{гибридными} моделями, которые сочетают оба подхода. Приложение читает
основные параметры из файла, а пароли и ключи доступа из среды.

Рассмотрим, как подружить файлы и окружение. Наивное решение не потребует писать
код: оно работает на утилитах командной строки. Программа \spverb|envsubst| из
пакета <<GNU gettext>> предлагает простую шаблонную систему. Чтобы установить
\spverb|gettext|, выполните в терминале команду:

\begin{english}
  \begin{bash}
> <manager> install gettext
  \end{bash}
\end{english}

\noindent
, где \spverb|<manager>| это мастер пакетов вашей системы (\spverb|brew|,
\spverb|apt|, \spverb|yum| и другие).

Текст шаблона приходит из \spverb|stdin|, а роль контекста играют переменные
среды. Утилита заменяет выражения \spverb|$VAR_NAME| на значения одноименной
переменной. Поместим шаблон в файл \spverb|config.tpl.json|. Частичка <<tpl>>
означает <<template>>.

\begin{english}
  \begin{json}
{
    "server_port": $HTTP_PORT,
    "db": {
        "dbtype":   "mysql",
        "dbname":   "$DB_NAME",
        "user":     "$DB_USER",
        "password": "$DB_PASS"
    },
    "event": [
        "$EVENT_START",
        "$EVENT_END"
    ]
}
  \end{json}
\end{english}

\noindent
Обратите внимание, что порт сервера не окружен кавычками, потому что это
число. Поместим переменные в файл \spverb|ENV_TPL|:

\begin{english}
  \begin{bash}
> cat ENV_TPL
DB_NAME=book
DB_USER=ivan
DB_PASS='*(&fd}A53z#$!'
HTTP_PORT=8080
EVENT_START='2019-07-05T12:00:00'
EVENT_END='2019-07-12T23:59:59'
  \end{bash}
\end{english}

\noindent
Считаем переменные и <<отрендерим>> шаблон:

\begin{english}
  \begin{bash}
> source ENV_TPL
> cat config.tpl.json | envsubst
{
    "server_port": 8080,
    "db": {
        "dbtype":   "mysql",
        "dbname":   "book",
        "user":     "ivan",
        "password": "*(&fd}A53z#$!"
    },
    "event": [
        "2019-07-05T12:00:00",
        "2019-07-12T23:59:59"
    ]
}
  \end{bash}
\end{english}

\noindent
Подстановка прошла успешно. Чтобы записать результат в файл, добавьте
в конец оператор вывода:

\begin{english}
  \begin{bash}
> cat config.tpl.json | envsubst > config.ready.json
  \end{bash}
\end{english}

Способ с \spverb|envsubst| на первый взгляд примитивен, но полезен на
практике. Шаблон снимает вопрос о структуре: переменные подставляют в
нужное место, и не приходится думать о вложенности.

Иногда приложение требует несколько файлов конфигурации, в том числе для
инфраструктуры. Порт веб-сервера нужен Nginx для проксирования. В
Sendmail нужно указать тот же адрес почты, что и в приложении. Очевидно, должен
быть единый источник данных, и рендер шаблонов ложится на эту модель.

Утилита \spverb|envsubst| становится менеджером конфигураций. Чтобы
автоматизировать процесс, добавьте скрипт, который пробегает по шаблонам и
рендерит их переменными. Решение не тянет на промышленный уровень, но подойдет
для простых проектов.

\section{Чтение среды из конфигурации}

Следующие техники делают так, что приложение читает параметры
\emph{одновременно} из файла и среды. Разница в том, на каком шаге это
происходит.

Предположим, основные параметры записаны в файле, а пароль к~базе приходит из
среды. Договоримся с командой, что в поле \spverb|:password| записан не пароль,
а имя переменной, например \spverb|"DB_PASS"|. Напишем спеку, которая выводит
значение переменной по имени:

\begin{english}
  \begin{clojure}
(s/def ::->env
  (s/conformer
   (fn [varname]
     (or (System/getenv varname)
         ::s/invalid))))
  \end{clojure}
\end{english}

Если переменную не задали, вывод верн\"{е}т ошибку. Прежде чем подключиться к базе,
отбросим пустые символы и убедимся, что строка не пустая.

\begin{english}
  \begin{clojure}
(s/def ::db-password
  (s/and ::->env
         string?
         (s/conformer str/trim)
         not-empty))
  \end{clojure}
\end{english}

Быстрая проверка: запустим REPL с переменной \spverb|DB_PASS| и прочитаем ее
спекой:

\begin{english}
  \begin{bash}
DB_PASS='*(&fd}A53z#$!' lein repl

(s/conform ::db-password "DB_PASS")
"*(&fd}A53z#$!"
  \end{bash}
\end{english}

Чтобы вынести поле из файла в среду, замените значение на имя
переменной. Обновите спеку этого поля: добавьте \spverb|::->env| в начало
цепочки \spverb|s/and|.

Другой способ прочитать переменные из файла~--- расширить его \emph{тегами}. Тег
это короткое слово, которое указывает, что значение за ним читают особым
образом. Форматы YAML и EDN поддерживают теги. Библиотеки для них предлагают
несколько основных.

В EDN тег начинается со знака решетки и захватывает следующее
значение. Например, \spverb|#inst "2019-07-10"| выводит дату из строки. Тег
связан с функцией одного аргумента, которая вычисляет значение из
исходного. Чтобы задать свой тег, в функцию \spverb|clojure.edn/read-string|
передают словарь тегов. Ключи словаря это символы, значения~--- функции.

Добавим тег \spverb|#env|, который верн\"{е}т значение переменной по имени. Имя
может быть строкой или символом. Определим функцию:

\begin{english}
  \begin{clojure}
(defn tag-env
  [varname]
  (cond
    (symbol? varname)
    (System/getenv (name varname))
    (string? varname)
    (System/getenv varname)
    :else
    (throw (new Exception "Wrong variable type"))))
  \end{clojure}
\end{english}

\noindent
Прочитаем EDN-строку с новым тегом:

\begin{english}
  \begin{clojure}
(require '[clojure.edn :as edn])

(edn/read-string
 {:readers {'env tag-env}}
 "{:db-password #env DB_PASS}")

;; {:db-password "*(&fd}A53z#$!"}
  \end{clojure}
\end{english}

Чтобы не передавать теги каждый раз, объявим \spverb|partial| от
\spverb|edn/read-string|. Новая функция принимает только строку:

\begin{english}
  \begin{clojure}
(def read-config
  (partial edn/read-string
           {:readers {'env tag-env}}))
  \end{clojure}
\end{english}

Чтобы разобрать файл с тегами, считайте его в строку и передайте в
\spverb|read-config|:

\begin{english}
  \begin{clojure}
(-> "/path/to/config.edn"
    slurp
    read-config)
  \end{clojure}
\end{english}

YAML тоже предусматривает теги. Они начинаются с одного или двух восклицательных
знаков в зависимости от семантики. У стандартных тегов два знака, а у сторонних
(пользовательских) один. Так мы сразу поймем, где какой тег.

Библиотека Yummy предлагает парсер YAML, <<заряженный>> полезными тегами. Среди
прочих нас интересует \spverb|!envvar|: он верн\"{е}т значение переменной по
имени. Опишем конфигурацию в файле \spverb|config.yaml|:

\begin{english}
  \begin{yaml}
server_port: 8080
db:
  dbtype:   mysql
  dbname:   book
  user:     !envvar DB_USER
  password: !envvar DB_PASS
  \end{yaml}
\end{english}

Подключим библиотеку и прочитаем файл. На месте тегов получим значения среды:

\begin{english}
  \begin{clojure}
(require '[yummy.config :as yummy])
(yummy/load-config {:path "config.yaml"})

{:server_port 8080
 :db {:dbtype "mysql"
      :dbname "book"
      :user "ivan"
      :password "*(&fd}A53z#$!"}}
  \end{clojure}
\end{english}

\noindent
Мы подробно рассмотрим Yummy в следующем разделе главы.

Теги несут преимущества и недостатки. С одной стороны, они делают конфигурацию
плотнее: строка с тегом несет больше смысла. Запись \spverb|#env DB_PASS|
выглядит короче и приятнее глазу. Сложные теги выносят в библиотеки и подключают
в зависимости.

С другой стороны, теги привязывают конфигурацию к платформе. Если в YAML-файле
встречается тег \spverb|!envvar|, библиотека на Python не сможет его прочитать:
в ней нет такого тега (если точнее, есть, но другим именем). Технически это
можно исправить: пропускать незнакомые теги или установить заглушку. Но подход
не гарантирует одинаковый результат на разных платформах.

С тегами конфигурация обрастает побочными эффектами. В терминах функционального
программирования она теряет чистоту. Появляется соблазн вынести в тег слишком
много логики: включить дочерний файл, форматировать строки. Теги стирают грань
между чтением конфигурации и ее обработкой. Когда их слишком много, конфигурацию
трудно поддерживать.

Оба приема~--- разбор спекой и тегами~--- оппонируют друг другу. Выбирайте тот
способ, что удобен команде и процессу.

\section{Короткий обзор форматов}

Мы упомянули три формата данных: JSON, EDN и YAML. Перечислим особенности
каждого из них. Наша цель не выявить идеальный формат, а подготовить читателя к
неочевидным моментам, которые возникнут в работе с ними.

\subsection{JSON}

JSON известен даже тем, кто не работает с вебом. Это запись данных по правилам
JavaScript. Стандарт задает числа, строки, логический тип, null и две коллекции:
массив и словарь. Коллекции могут быть вложены друг в друга.

Преимущество JSON в его популярности: это стандарт обмена данными между
клиентом и сервером. По сравнению с XML его легче читать и поддерживать. JSON
поддерживают современные редакторы, языки и платформы. Это естественный способ
хранить данные в JavaScript.

В JSON не предусмотрены комментарии. На первый взгляд это мелочь, но на практике
комментарии важны. Если вы добавили новый параметр, напишите комментарий о том,
что он делает и какие значения принимает. Посмотрите конфигурации Redis,
PostgreSQL или Nginx~--- больше половины файла занимают комментарии.

Разработчики придумали уловки, чтобы обойти ограничение. Например, поставить
одноименное поле перед тем, к которому относится комментарий.

\begin{english}
  \begin{json}
{
    "server_port": "A port for the HTTP server.",
    "server_port": 8080
}
  \end{json}
\end{english}

Расчет сделан на то, что библиотека обходит поля по очереди, и второе поле
заменит первое. Стандарт JSON ничего не говорит о порядке полей; прием остается
на ваш риск. Логика библиотеки может быть иной, например, бросить исключение или
пропустить ключ, который уже обработали.

Иногда в JSON добавляют комментарии на уровне продукта. Редактор Sublime Text
хранит настройки в .json-файлах с поддержкой JavaScript-комментариев (двойная
косая черта). В общем случае у проблемы нет решения.

JSON выгодно отличается от многословного XML, которому пришел на замену. Данные
в JSON выглядят чище и удобнее, чем дерево XML-тегов. Но современные форматы
выражают данные еще чище. В YAML любую структуру можно записать без единой
скобки за счет отступов.

Синтаксис JSON <<шумит>>: он требует кавычек, двоеточий и запятых там, где
другие форматы лояльны. Запятая в конце массива или объекта считается
ошибкой. Ключи словаря не могут быть числами. Нельзя записать текст в несколько
строк.

Сравните данные в YAML и JSON. Первая запись короче и чище:

\noindent
\begin{tabular}{ @{}p{5cm} @{}p{5cm} }

\begin{english}
  \begin{yaml}
server_port: 8080
db:
  dbtype:   mysql
  dbname:   book
  user:     user
  password: '****'
event:
  - 2019-07-05T12:00:00
  - 2019-07-12T23:59:59
  \end{yaml}
\end{english}

&

\begin{english}
  \begin{json}
{
    "server_port": 8080,
    "db": {
        "dbtype":   "mysql",
        "dbname":   "book",
        "user":     "ivan",
        "password": "****"
    },
    "event": [
        "2019-07-05T12:00:00",
        "2019-07-12T23:59:59"
    ]
}
  \end{json}
\end{english}

\end{tabular}

JSON не поддерживает теги, о которых мы говорили выше. Для работы с форматом
служат библиотеки Cheshire\footurl{https://github.com/dakrone/cheshire} и
Data.json\footurl{https://github.com/clojure/data.json}.

\subsection{YAML}

Язык YAML, как и JSON, различает базовые типы: скаляры, null и коллекции. YAML
делает ставку на краткость записи: вложенность задают отступы, а не
скобки. Запятые не обязательны там, где язык выводит их логически. Массив чисел
в одну строку выглядит как в JSON:

\begin{english}
  \begin{yaml}
numbers: [1, 2, 3]
  \end{yaml}
\end{english}

\noindent
Но при записи в столбик запятые и скобки отпадают:

\begin{english}
  \begin{yaml}
numbers:
  - 1
  - 2
  - 3
  \end{yaml}
\end{english}

YAML поддерживает комментарии в стиле Python (знак решетки), за что его любят
DevOps. Программы вроде Docker-compose и Kubernetes читают настройки из
YAML-файлов.

В YAML можно записать текст в несколько строк. Его проще читать и копировать,
чем одну строку с символом переноса \spverb|\n|.

\begin{english}
  \begin{yaml}
description: |
  To solve the problem, please do the following:

  - Press Control + Alt + Delete;
  - Turn off your computer;
  - Walk for a while.

  Then try again.
  \end{yaml}
\end{english}

Язык официально поддерживает теги.

Недостатки YAML вытекают из его преимуществ. Отступы кажутся удачным решением до
тех пор, пока файл не вырастет. Глаз прыгает на большие расстояния, чтобы
сверять уровни структур. Иногда часть данных <<съезжает>> из-за лишнего
отступа. С точки зрения YAML ошибки нет, поэтому ее трудно найти.

Отказ от кавычек приводит к неверным типам или структуре. Предположим, в
поле \spverb|phrases| перечислены фразы, которые увидит пользователь:

\begin{english}
  \begin{yaml}
phrases:
  - Welcome!
  - See you soon!
  - Warning: wrong email address.
  \end{yaml}
\end{english}

Из-за двоеточия в последней строке парсер решит, что это вложенный
словарь. Получим неверную структуру:

\begin{english}
  \begin{clojure}
{:phrases
 ["Welcome!"
  "See you soon!"
  {:Warning "wrong email address."}]}
  \end{clojure}
\end{english}

Другие примеры: версия продукта \spverb|3.3| это число, но \spverb|3.3.1|~---
строка. Телефон \spverb|+79625241745| это число, потому что знак плюса считается
унарным оператором по аналогии с минусом. Лидирующие нули означают восьмеричную
запись: если не добавить кавычки к \spverb|000042|, получится \spverb|34|.

Это не значит, что YAML неудачный формат. Случаи выше описаны в стандарте и
имеют логическое объяснение. Но иногда он ведет себя не так, как вы ожидали~---
это плата за упрощенный синтаксис.

\subsection{EDN}

Формат EDN занимает особое место в обзоре. Он хранит данные Clojure и потому
играет такую же роль в языке, как JSON в JavaScript. Это родной способ связать
данные с файлом в Clojure.

Синтаксис EDN почти полностью совпадает правилам Clojure. Формат охватывает
больше типов, чем JSON и YAML. Из скалярных типов доступны символы и
кейворды (классы \spverb|Symbol| и \spverb|Keyword| из
\spverb|clojure.lang|). Кроме вектора и словаря EDN предлагает списки и
множества.

Тег начинается с символа решетки. По умолчанию стандарт предлагает два тега:
\spverb|#inst| и \spverb|#uuid|. Первый читает строку в дату, а второй в
экземпляр \spverb|java.util.UUID|. Идентификаторы используют в распределенных
системах вроде Cassandra и Kafka. Выше мы показали, как добавить свой тег: нужно
связать его с функцией одного аргумента при чтении строки.

Пример с разными типами, коллекциями и тегами:

\begin{english}
  \begin{clojure}
{:user/banned? false
 :task-state #{:pending :in-progress :done}
 :account-ids [1001 1002 1003 nil]
 :server {:host "127.0.0.1" :port 8080}
 :date-range [#inst "2019-07-01" #inst "2019-07-31"]
 :cassandra-id #uuid "26577362-902e-49e3-83fb-9106be7f60e1"}
  \end{clojure}
\end{english}

Данные в EDN не отличаются от кода. Если скопировать их в REPL или модуль,
компилятор выполнит их. И наоборот: вывод REPL можно записать в файл для
дальнейшей работы.

Функция \spverb|pr-str| переводит данные в текст. Сброс в EDN сводится к простым
шагам: <<напечатать>> данные в строку и записать ее в файл. Ниже результат
функции пишут в файл \spverb|dataset.edn|:

\begin{english}
  \begin{clojure}
(-> (get-huge-dataset)
    (pr-str)
    (->> (spit "dataset.edn")))
  \end{clojure}
\end{english}

EDN поддерживает не только обычные комментарии. Тег \spverb|#_| игнорирует
следующий за ним элемент. Им может быть что угодно, в том числе коллекция. Если
нужно <<заигнорить>> словарь, который занимает несколько строк, поставьте перед
ним \spverb|#_|, и парсер пропустит его.

Так отключают целые части конфигурации. В следующем примере мы игнорируем третий
элемент вектора. Если поставить обычный комментарий на строку (точка с запятой),
он заденет закрывающие скобки, и выражение будет неверным.

\begin{english}
  \begin{clojure}
{:users [{:id 1 :name "Ivan"}
         {:id 2 :name "Juan"}
         #_{:id 3 :name "Huan"}]}
  \end{clojure}
\end{english}

EDN это удачный выбор, когда и бекенд, и фронтенд работают на стеке Clojure и
ClojureScript.

EDN привязан к Clojure и поэтому непопулярен в других языках. Редакторы не
подсвечивают его синтаксис без плагинов. EDN доставит проблем DevOps, которые
работают только с JSON и YAML. Если конфигурацию читают скрипты на Python или
Ruby, прид\"{е}тся ставить библиотеку для работы с форматом. EDN выбирают там, где
Clojure преобладает над другими технологиями.

\section{Промышленные решения}

Понимать конфигурацию важно, но мы не ожидаем, что в каждом проекте ее пишут с
нуля. В последнем разделе мы рассмотрим, что предлагает сообщество для
конфигурации проектов. Мы остановили внимание на Cprop, Aero и Yummy. Библиотеки
отличаются в идеологии и архитектуре. Мы специально подобрали их так, чтобы
увидеть проблему с разных сторон.

\subsection{Cprop}

Библиотека Cprop\footurl{https://github.com/tolitius/cprop} работает по принципу
<<данные отовсюду>>. В отличии от нашего загрузчика Cprop понимает больше
источников. Библиотека читает не только файл и переменные среды, но и ресурсы,
property-файлы и обычные словари.

В библиотеке задан порядок обхода источников и их приоритет. Параметры одного
источника заменяют другие. Например, у переменной среды более высокий приоритет,
чем у файла. Для частных случаев в Cprop легко задать свой порядок загрузки.

Нас интересует функция \spverb|load-config|. Без параметров она запускает
стандартный загрузчик. По умолчанию Cprop ищет два источника данных: ресурс и
property-файл. Ресурс должен называться \spverb|config.edn|. Если системное
свойство \spverb|conf| не пустое, библиотека полагает, что это путь к
property-файлу и загружает его.

Свойства это переменные Java-машины, аналог среды окружения JVM. При загрузке
она получает свойства по умолчанию: тип операционной системы, разделитель строк
и другие. Дополнительные свойства задают параметром \spverb|-D| при
запуске. Пример ниже запускает jar-файл со свойством \spverb|conf|:

\begin{english}
  \begin{bash}
> java -Dconf="/path/to/config.properties" -jar project.jar
  \end{bash}
\end{english}

Файлы \spverb|.properties| это пары \spverb|поле=значение| по одной на
строку. Поля похожи на домены: это лексемы, разделенные точкой. Лексемы убывают
по старшинству:

\begin{english}
  \begin{ini}
db.type=mysql
db.host=127.0.0.1
db.pool.connections=8
  \end{ini}
\end{english}

Библиотека трактует точки как вложенные словари. Файл верн\"{е}т структуру:

\begin{english}
  \begin{clojure}
{:db {:type "mysql"
      :host "127.0.0.1"
      :pool {:connections 8}}}
  \end{clojure}
\end{english}

Получив конфигурацию, Cprop ищет переопределения в переменных среды. Для них
работают те же правила, что и в нашем загрузчике. Например, переменная
\spverb|DB__POOL__CONNECTIONS=16| заменит значение 8 в словаре выше. Cprop
игнорирует переменные, которые не входят в конфигурацию и тем самым не
загрязняет~ее.

Нестандартные пути к ресурсу и файлу задают ключами:

\begin{english}
  \begin{clojure}
(load-config
 :resource "private/config.edn"
 :file "/path/custom/config.edn")
  \end{clojure}
\end{english}

Для тонкой работы Cprop предлагает модуль \spverb|cprop.source|. Его функция
\spverb|from-env| верн\"{е}т все переменные среды, \spverb|from-props-file| загрузит
property-файл и так далее. Легко построить такую комбинацию, которая нужна
проекту.

Ключ \spverb|:merge| объединяет конфигурацию с любым источником. В н\"{е}м
выражения, которые вернут словарь. Убер-пример из документации:

\begin{english}
  \begin{clojure}
(load-config
 :resource "path/within/classpath/to.edn"
 :file "/path/to/some.edn"
 :merge [{:datomic {:url "datomic:mem://test"}}
         (from-file "/path/to/another.edn")
         (from-resource "path/within/classpath/to-another.edn")
         (from-props-file "/path/to/some.properties")
         (from-system-props)
         (from-env)])
  \end{clojure}
\end{english}

Чтобы отследить загрузку, установите переменную среды \spverb|DEBUG=y|. С ней
\spverb|cprop| выводит служебную информацию. Это список источников, порядок
загрузки, переопределение и так далее.

Cprop только читает данные из источников, но не проверяет их. В библиотеке нет
валидации спекой, как это сделано в нашем загрузчике. Шаг остается на ваше
усмотрение.

Библиотека по-своему выводит типы. Если строка состоит только из цифр, ее
приводят к числу. Значения с запятыми становятся списками. Иногда этих правил
недостаточно для полного контроля за типами. Вам по-прежнему понадобится
\spverb|spec| и \spverb|conform| для вывода типов и сообщений об ошибке.

\subsection{Aero}

Aero\footurl{https://github.com/juxt/aero} работает с файлами EDN. Библиотека
предлагает теги, с которыми формат похож на мини-язык программирования. В нем
появляются операторы ветвления, импорта, форматирования. Подход можно назвать
<<EDN на стероидах>>.

Функция \spverb|read-config| читает файл или ресурс EDN:

\begin{english}
  \begin{clojure}
(require '[aero.core :refer (read-config)])

(read-config "config.edn")
(read-config (clojure.java.io/resource "config.edn"))
  \end{clojure}
\end{english}

Теги это главный момент в Aero, поэтому разберем основные из них. Тег
\spverb|#env| заменяет имя переменной среды на значение. Мы уже писали такой
же:

\begin{english}
  \begin{clojure}
{:db {:passwod #env DB_PASS}}
  \end{clojure}
\end{english}

Тег \spverb|#envf| форматирует строку переменными среды. Предположим,
подключение к базе состоит из отдельных полей, но вы предпочитаете JDBC URI~---
длинную строку, похожую на веб-адрес. Чтобы не повторяться, адрес вычисляют из
исходных полей:

\begin{english}
  \begin{clojure}
{:db-uri #envf ["jdbc:postgresql://%s/%s?user=%s"
                DB_HOST DB_NAME DB_USER]}
  \end{clojure}
\end{english}

Тег \spverb|#or| похож на аналог Clojure и нужен для значений по
умолчанию. Пусть в файле не задан порт базы данных. Чтобы избежать ошибки,
укажем стандартный порт PostgreSQL:

\begin{english}
  \begin{clojure}
{:db {:port #or [#env DB_PORT 5432]}}
  \end{clojure}
\end{english}

Оператор \spverb|#profile| позволяет найти значение по
\emph{профилилю}. Значение за тегом должно быть словарем. Ключи словаря это
профили, а значения~--- то, что получим в результате его разрешения. Профиль
задают в параметрах \spverb|read-config|.

Пример ниже показывает, как найти имя базы по профилю. Без профиля получим
\spverb|"book"|, но для \spverb|:test| имя станет \spverb|"book_test"|:

\begin{english}
  \begin{clojure}
{:db {:name #profile {:default "book"
                      :dev     "book_dev"
                      :test    "book_test"}}}

(read-config "aero.test.edn" {:profile :test})
{:db {:name "book_test"}}
  \end{clojure}
\end{english}

Тег \spverb|#include| помещает в конфигурацию другой EDN-файл. В нем тоже могут
быть теги, и библиотека выполнит их рекурсивно. К импорту прибегают, когда
конфигурация становится большой.

\begin{english}
  \begin{clojure}
{:queue #include "message-queue.edn"}
  \end{clojure}
\end{english}

Тег \spverb|#ref| ставит ссылку на любое место конфигурации. Это вектор ключей,
который обычно передают в \spverb|get-in|. Ссылка позволит избежать повторов.
Например, компонент нуждается в пользователе, под которым мы подключаемся к базе
данных. Чтобы не копировать его, поставим ссылку:

\begin{english}
  \begin{clojure}
;; config.edn
{:db {:user #env DB_USER}
 :worker {:user #ref [:db :user]}}

;; result
{:db {:user "ivan"}, :worker {:user "ivan"}}
  \end{clojure}
\end{english}

Aero предлагает несложный язык описания конфигурации. Библиотека подкупает идеей
и красотой реализации. Но в момент, когда вам захочется переехать с негибкого
JSON на Aero, подумайте об обратной стороне медали.

Конфигурацию не случайно отделяют от кода. Если бы не потребность индустрии, мы
бы хранили параметры в исходных файлах. Но этого не делают~--- наоборот, хорошие
практики советуют \emph{отделять} параметры от кода. В том числе потому, что в
отличие от кода конфигурация \emph{декларативна}.

Негибкие JSON-файлы обладают важным свойством: они декларативны. Если вы открыли
файл или сделали \spverb|cat|, то увидите данные. Синтаксис может быть неудобен,
но данные выражаются сами в себя, и ошибки быть не может.

Наоборот, файл с обилием тегов труден в поддержке. Это не конфигурация,
а \emph{код}. Чтобы увидеть данные, файл нужно выполнить. При чтении
файла в голове запускается мини-интерпретатор, который не гарантирует
правильный результат.

Получается своего рода круг: мы вынесли параметры в конфигурацию, добавили теги
и вернулись к коду. Подход имеет право на жизнь, но к нему приходят осознанно.

\subsection{Yummy}

Библиотека Yummy\footurl{https://github.com/exoscale/yummy} замыкает обзор. От
аналогов она отличается двумя свойствами. Во-первых, работает с файлами YAML для
чтения конфигурации (отсюда и название). Во-вторых, процесс загрузки похож на
тот, что мы рассмотрели в начале главы.

Полноценный загрузчик не только читает параметры. Цикл включает проверку данных
и вывод ошибки. Сообщение внятно объясняет, в чем причина. С помощью опций можно
задать реакцию на исключение, которые возникло в работе. Yummy предлагает все из
перечисленного.

Путь к файлу либо указывают в параметрах, либо библиотека ищет его по особым
правилам. Вариант, когда путь задан явно:

\begin{english}
  \begin{clojure}
(require '[yummy.config :refer [load-config]])

(load-config {:path "config.yaml"})
  \end{clojure}
\end{english}

Во втором случае вместо пути указали имя проекта. Yummy ищет путь к файлу в
переменной среды \spverb|<project>_CONFIGURATION| или свойстве
\spverb|<project>.configuration|:

\begin{english}
  \begin{bash}
export BOOK_CONFIGURATION=config.yaml
  \end{bash}
\end{english}

\begin{english}
  \begin{clojure}
(load-config {:program-name :book})
  \end{clojure}
\end{english}

Библиотека расширяет YAML несколькими тегами. Это знакомый \spverb|!envvar|
для переменных среды:

\begin{english}
  \begin{yaml}
db:
  password: !envvar DB_PASS
  \end{yaml}
\end{english}

\noindent
Тег \spverb|!keyword| полезен, чтобы привести строку к кейворду:

\begin{english}
  \begin{yaml}
states:
  - !keyword task/pending
  - !keyword task/in-progress
  - !keyword task/done
  \end{yaml}
\end{english}

\noindent
Результат:

\begin{english}
  \begin{clojure}
{:states [:task/pending :task/in-progress :task/done]}
  \end{clojure}
\end{english}

Тег \spverb|!uuid| аналогичен \spverb|#uuid| для EDN. Он возвращает объект
\spverb|java.util.UUID| из строки:

\begin{english}
  \begin{yaml}
system-user: !uuid cb7aa305-997c-4d53-a61a-38e0d8628dbb
  \end{yaml}
\end{english}

Тег \spverb|!slurp| читает файл, что полезно для сертификатов шифрования. Их
содержимое это длинная строка, которую неудобно хранить в общей конфигурации. В
ключах \spverb|:auth|, \spverb|:cert| и \spverb|:pkey| окажется содержимое
файлов из папки \spverb|certs|.

\begin{english}
  \begin{yaml}
tls:
  auth: !slurp "certs/ca.pem"
  cert: !slurp "certs/cert.pem"
  pkey: !slurp "certs/key.pk8"
  \end{yaml}
\end{english}

Чтобы проверить данные, в параметры \spverb|load-config| передают ключ
спеки. Когда ключ указан, Yummy выполняет \spverb|s/assert| над данными из
файла. Если проверка вернула ложь, всплывает исключение. Yummy использует
Expound, чтобы улучшить текст об ошибке.

\begin{english}
  \begin{clojure}
(load-config {:program-name :book
              :spec ::config})
  \end{clojure}
\end{english}

Словарь опций принимает параметр \spverb|:die-fn|. Это функция, которая
сработает, если любая стадия завершится с ошибкой. Функция принимает исключение
и метку с именем стадии.

Если \spverb|:die-fn| не задан, Yummy вызовет обработчик по умолчанию. Он
выводит текст в \spverb|stderr| и завершает JVM с кодом 1. На этапе разработки
мы не хотим обрывать REPL из-за ошибки в конфигурации. В интерактивном сеансе
наша \spverb|die-fn| только печатает текст и ошибку:

\begin{english}
  \begin{clojure}
(load-config
 {:program-name :book
  :spec ::config
  :die-fn (fn [e msg]
            (binding [*out* *err*]
              (println msg (ex-message e))))})
  \end{clojure}
\end{english}

\noindent
В боевом режиме запишем исключение в лог и завершим программу.

\begin{english}
  \begin{clojure}
(load-config
 {:program-name :book
  :spec ::config
  :die-fn (fn [e msg]
            (log/errorf e "Config error")
            (System/exit 1))})
  \end{clojure}
\end{english}

Для валидации Yummy вызывает \spverb|s/assert|. Функция не выводит значения, как
это делает \spverb|s/conform|, а только бросает исключение. Это сделано нарочно:
типы выводят тегами, а спека только проверяет данные.

\section{Заключение}

Перечислим основные тезисы из этой главы. Конфигурация нужна, чтобы проект
прошел стадии производства: разработку, тестирование, релиз. На каждой стадии
его запускают с разными настройками. Без конфигурации это невозможно.

Загрузка конфигурации означает чтение данных, вывод типов и проверку значений. В
случае ошибки программа выводит сообщение и завершается с аварийным
кодом. Нельзя продолжать работу с неверными параметрами.

Источником конфигурации может быть файл, ресурс, переменные окружения. Бывают
гибридные схемы, когда основные данные приходят из файла, а секретные поля из
окружения.

Переменные среды живут в памяти операционной системы. Когда переменных много, их
помещают в ENV-файл. Приложение не читает его: это делает скрипт, который
управляет приложением на сервере. Приложению неизвестно, откуда пришли
переменные.

Окружение это плоский словарь. Переменные хранят только текст, в ключах нет
вложенности или пространства имен. В разных системах свои соглашения о том, как
извлечь структуру из имени переменной. Это могут быть точки, двойные
подчеркивания или что-то еще.

Форматы данных отличаются синтаксисом и типами. Общие форматы задают строки,
числа, словари и списки. Они не настолько гибки, но работают везде. Наоборот,
формат для конкретной платформы тесно связан с ней, но не популярен в других
языках.

Некоторые форматы поддерживают теги. С помощью тегов из скаляров получают более
сложные типы, например даты. Теги опасны тем, что когда их много, конфигурация
превращается в код.

Clojure предлагает несколько библиотек для конфигурации приложения. Они
различаются замыслом и архитектурой, и каждый разработчик найдет то, что ему по
душе. Нет точного ответа на вопрос, какой формат или библиотека лучше. Выбирайте
то, что предельно д\"{е}шево решит задачу.
