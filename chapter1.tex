\chapter{ЫВА ыва Ыва ыВА !!}


Это веб-сервер на 8080 порту и параметры подключения к базе.
\spverb|foo-bar-baz-aaaa-estest|, что их можно скопировать в REPL, выполнить и оценить
результат. Например, открыть браузер со страницей веб-приложения или сделать
запрос к базе данных.

На практике эти выражения дорабатывают так, чтобы в них не было конкретных цифр
и строк. С точки зрения проекта плохо то, что в выражении веб-сервера явно
записан порт. Такая запись подходит для документации и примеров, но не для
боевого проекта.

Порт 8080, ровно как и другие комбинации нулей и восьмерок, пользуется
популярностью у разработчиков. Велика вероятность, что порт будет занят другим
веб-сервером. Это возможно, когда вы запускаете не отдельный сервис, а их связку
на время разработки или прогона тестов.

Код, написанный разработчиком, обычно проходит несколько стадий. Их набор и
структура отличаются в зависимости от фирмы, но в целом это разработка,
автоматическое и ручное тестирование (юнит-тесты и QA), пре-прод и боевой режим.



\begin{sloppypar}
\begin{verbatim}
(require '[cheshire.core :as json])

(defn read-config-file
  [filepath]
  (try
    (-> filepath slurp (json/parse-string true))
    (catch Exception e
      (exit 1 "Malformed config file: %s" (ex-message e)))))
\end{verbatim}
\end{sloppypar}


На каждой стадии приложение запускают бок о бок с другими
проектами. Предположение о том, что порт 8080 свободен в любой момент на каждой
машине, звучит как утопия. На жаргоне разработчиков это называется "хардкод"
(hardcode), что соответствует идиоме "прибито гвоздями".


%% \begin{figure}[h]
%% \begin{verbatim}

%% \end{verbatim}
%% \end{figure}

%% \begin{code}


%% \end{code}

Когда приложение опирается на "прибитые" значения, это вносит проблемы в его
жизненный цикл. Например, вы не сможете одновременно разрабатывать и тестировать
приложение. Или два тестовых сервера подключаются к одной базе данных и вы не
понимаете, почему данные меняются спонтанно.

Ошибка в том, что мы неправильно распределяем ответственность. Приложение не
должно решать, на каком порту запускать веб-сервер. Информация об этом должна
прийти извне. В простейшем случае это текстовый файл. Приложение читает из него
порт и запускает сервер именно так, как это нужно в конкретной машине.

В более сложных сценариях этот файл составляется не человеком, а специальной
программой — менеджером конфигураций. Эти менеджеры хранят информацию о
топологии сети, адреса машин, типы баз данных и параметры подключения к ним. По
запросу они выдают файл с верными параметрами для определенной машины или
сегмента сети.
