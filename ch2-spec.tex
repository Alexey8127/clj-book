\chapter{Clojure.spec}

\label{chapter-spec}

\index{spec}

\begin{teaser}
В этой главе мы рассмотрим clojure.spec~--- библиотеку для проверки данных в
Clojure. Это особенная библиотека: на ней пишут валидаторы и парсеры, с её
помощью генерируют данные для тестов. Spec фундаментальна по своей природе,
поэтому уделим ей пристальное внимание.
\end{teaser}

Название spec происходит от specification (с англ.~--- <<спецификация, описание>>). Это
набор функций и макросов, чтобы схематично описать данные. Например, из каких
ключей состоит словарь и типы его значений. Запись называют спецификацией данных или
сокращённо спекой. Далее мы будем использовать короткий термин.

Специальные функции проверяют, подходят ли данные спеке. Если нет, получим
отчёт, в каком месте произошла ошибка и почему.

\index{модули!clojure.spec.alpha}

Spec входит в поставку Clojure начиная с версии 1.9. Полностью модуль называется
\texttt{clojure.\-spec.\-alpha}. Не волнуйтесь о частичке alpha на конце
имени: она осталась по историческим причинам.

\index{валидация}
\index{парсинг}

Spec стала важной вехой в развитии Clojure. Ключевая особенность Spec в том, что
она фундаментальна. Валидация данных~--- лишь малая часть её возможностей. Spec
не только проверяет данные, но и преобразует их. На Spec легко писать парсеры.

Формально Spec~--- обычная библиотека. Но её абстракции настолько мощны, что
Clojure переиспользует их. С версии 1.10 компилятор Clojure анализирует основные
макросы с помощью Spec. Так проекты дополняют друг друга.

Прежде чем браться за техническую часть, разберёмся с теорией. Вспомним, как
связаны между собой классы, типы и валидация.

\section{Типы и классы}

\label{type-and-pred}

\index{типизация!статическая}
\index{типизация!динамическая}

Считается, что код на языке со статической типизацией безопаснее, чем с
динамической. Компилятор не позволит сложить число и~строку ещё до того, как
мы запустим программу. Однако тип переменной~--- это лишь одно из многих
ограничений. Редко случается так, что тип задаёт все допустимые
значения. Чаще всего вместе с типом учитывают границы, длину, попадание в
интервалы и перечисления. Иногда значения верны по отдельности, но не могут
стоять в паре друг с другом.

\index{порт}

Рассмотрим, как выразить в коде сетевой порт. В операционной системе это число
от 0 до $2^{16}-1$. Целые типы обычно описаны степенями двойки, поэтому найдётся
условный \verb|unsigned int|, который охватит именно этот диапазон. У нулевого
порта особая семантика, и в~прикладных программах его не используют.
Вероятность, что в~языке предусмотрен тип от~1 до~$2^{16}-1$, крайне мала.

Ещё легче увидеть проблему на диапазоне дат. Единичная дата может быть сколь
угодно разумной, но диапазон накладывает ограничение: начало строго меньше
конца. Бизнес дополняет: разница не больше недели, обе даты в рамках текущего
месяца.

\index{валидация}

В ООП знают об этой проблеме и решают её классами \verb|UnixPort|
и~\verb|DateRange|. Условный \verb|UnixPort|~--- это класс с конструктором. Он
принимает целое число и выполняет проверку на диапазон. Если число выходит за
рамки 1~\dots~$2^{16}-1$, конструктор бросит исключение. Программист уверен, что
создал новый тип. Это неверно~--- классы и типы не тождественны.

Конструктор представляет собой обычный валидатор. Он неявно сработает в
выражении \spverb|new UnixPort(8080)|. Из-за неявности возникает иллюзия, что мы
создали тип. На самом деле это валидация и синтаксический сахар.

\index{синтаксический сахар}

В промышленных языках нельзя описать класс так, чтобы выражение
\verb|new UnixPort(-42)| привело к ошибке компиляции. Найти её могут
только сторонние утилиты и плагины для IDE.

Конструктор нельзя использовать повторно. Представим классы \code{UnixPort} и
\code{NetPort} из разных библиотек. Первый класс проверяет порт на~диапазон
и~бросает исключение. Выгодно пользоваться этим классом, поскольку он совмещён
с~валидацией. Однако сторонняя библиотека принимает \verb|NetPort|. Возникает
проблема конвертации: нужно извлечь <<сырой>> порт из \verb|UnixPort| и~передать
в~\verb|NetPort|. Это лишний код и путаница с~классами.

\index{классы}

Признаки удобной валидации~--- это независимость и компоновка. Независимость
означает, что данные не привязаны к валидации. Нет ничего зазорного в том, что
порт~--- целое число. Пусть библиотека принимает \verb|integer|, а разработчик
сам решит, как его проверить. Появится выбор, насколько строгой должна быть
проверка.

\index{функции!композиция}

Компоновка означает, что полезно иметь несколько простых проверок, чтобы
составить из них сложные. Пусть заданы проверки <<это>> и~<<то>> и теперь нужны
комбинации <<это \emph{и}~то>>, <<это \emph{или}~то>>. В идеале компоновка
занимает пару строк и считается тривиальной задачей.

\index{функции}

Оба тезиса ложатся на функцию. На неё действует одна операция~--- вызов, что
упрощает схему. Функция принимает значение и возвращает истину или ложь. Это
ответ на вопрос, было ли значение правильным или нет. Функция~--- объект высшего
порядка, поэтому другие функции строят из них комбинации.

\section{Основы spec}

С багажом рассуждений мы подходим к Spec. Подключим модуль в~текущее
пространство:

\begin{english}
  \begin{clojure}
(require '[clojure.spec.alpha :as s])
  \end{clojure}
\end{english}

Синоним \verb|s| нужен, чтобы избежать конфликтов имён с модулем
\code{clo\-ju\-re.\-co\-re}. Spec несёт макросы \verb|s/and|, \verb|s/or| и другие, у
которых ничего общего с обычными \verb|and| и \verb|or|. Считается дурным тоном,
когда имена одного модуля затеняют другие, поэтому обращаемся к Spec через
синоним.

\index{spec!def}

Главная операция в Spec~--- создать новую \emph{спеку}:

\begin{english}
  \begin{clojure}
(s/def ::string string?)
  \end{clojure}
\end{english}

\index{clojure.core!string?}

Макрос \verb|s/def| принимает ключ и предикат. Он строит объект спеки из
функции \verb|string?| и помещает его в глобальный реестр с~ключом
\verb|::string|.

Важно понимать, что \verb|::string|~--- не спека, а псевдоним. Макросы Spec
работают не с объектами спеки, а с ключам. На нижнем уровне они сами найдут
спеку в реестре. Это удобно, потому что ключи глобальны: в любом месте можно
написать \verb|:my.project/string| без лишних импортов.

\index{spec!get-spec}

Вторым аргументом идёт предикат \verb|string?|. Предикатом называют функцию,
которая возвращает истину или ложь. Функция~--- это не спека, а строительный
материал для неё. Спека оборачивает функцию в объект, который реализует
внутренний протокол спеки. Технически на объект можно сослаться: функция
\verb|s/get-spec| вернёт его по~ключу спеки. На практике объект спеки не нужен,
потому что везде указывают ключи.

\ifnarrow

\begin{english}
  \begin{clojure}
(s/get-spec ::string)
;;#object[clojure.spec.alpha$reify 0x...]
  \end{clojure}
\end{english}

\else

\begin{english}
  \begin{clojure}
(s/get-spec ::string)
;; #object[clojure.spec.alpha$reify 0x3e9dde1d]
  \end{clojure}
\end{english}

\fi

\index{spec!регистр}

Спеки хранятся в глобальном реестре под своими ключами. Макрос \verb|s/def| не
проверяет, была ли уже такая спека, перед тем как поместить её в
реестр. Если была, мы потеряем её старую версию.

\index{конфликты ключей}

Spec не работает с ключами без пространства, например \verb|:name| или
\verb|:email|. С ними легко случайно затереть одну спеку другой. Чтобы назначить
ключу текущее пространство, поставьте два двоеточия: \verb|::name|,
\verb|::email|. При запуске такая запись станет \verb|:my.module/name| и
\code{:my.module/\-email}.

\index{пространства имён!текущее}

Самое простое, что можно сделать со спекой,~--- проверить, подходит ли ей
значение. Функция \verb|s/valid?| принимает ключ спеки, значение и~возвращает
\verb|true| или \verb|false|.

\begin{english}
  \begin{clojure}
(s/valid? ::string 1)      ;; false
(s/valid? ::string "test") ;; true
  \end{clojure}
\end{english}

Пустая строка пройдёт валидацию, но чаще всего в этом нет смысла. Пустые имя или
заголовок означают ошибку. Объявим спеку, которая дополнительно проверит, что
строка не пустая. Наивный способ это сделать~--- усложнить предикат:

\index{spec!::ne-string}

\begin{english}
  \begin{clojure}
(s/def ::ne-string
  (fn [val]
    (and (string? val)
         (not (empty? val)))))
  \end{clojure}
\end{english}

\noindent
Быстрая проверка:

\begin{english}
  \begin{clojure}
(s/valid? ::ne-string "test") ;; true
(s/valid? ::ne-string "")     ;; false
  \end{clojure}
\end{english}

\label{ne-string}

Ключ \verb|::ne-string| происходит от <<\textbf{n}on-\textbf{e}mpty
string>>. Спека встречается часто, поэтому логично сэкономить на её имени.

Более изящный способ задать эту же спеку~--- объединить предикаты через
\verb|every-pred|. Функция принимает предикаты и возвращает супер-предикат. Он
вернёт истину только если истинны все предикаты.

\begin{english}
  \begin{clojure}
(s/def ::ne-string
  (every-pred string? not-empty))
  \end{clojure}
\end{english}

\index{spec!and}
\index{clojure.core!every-pred}
\index{предикаты}

Мы собрали новую сущность из базовых, что короче и следует функциональному
стилю. Но ещё лучше комбинировать не предикаты, а спеки. Макрос \verb|s/and|
объединяет несколько предикатов и спек в новую спеку:

\begin{english}
  \begin{clojure}
(s/def ::ne-string
  (s/and string? not-empty))
  \end{clojure}
\end{english}

Так в Clojure строят сложные спеки: объявляют примитивы и~наращивают их
комбинации.

\section{Исключения}

\index{исключения}

Во время проверки Spec не перехватывает исключения~--- о них заботится
программист. Рассмотрим спеку для проверки URL. Проще всего это сделать
регулярным выражением:

\index{clojure.core!re-matches}
\index{регулярные выражения}

\ifnarrow

\begin{english}
  \begin{clojure}
(s/def ::url
  (partial
    re-matches #"(?i)^http(s?)://.*"))

(s/valid? ::url "test") ;;false
(s/valid? ::url "http://test.com") ;;true
  \end{clojure}
\end{english}

\else

\begin{english}
  \begin{clojure}
(s/def ::url
  (partial re-matches #"(?i)^http(s?)://.*"))

(s/valid? ::url "test")            ;; false
(s/valid? ::url "http://test.com") ;; true
  \end{clojure}
\end{english}

\fi

Что-то отличное от строки вызовет ошибку:

\index{исключения}
\index{классы!NullPointerException}

\begin{english}
  \begin{clojure}
(s/valid? ::url nil)
;; Execution error (NullPointerException)
;; at java.util.regex.Matcher...
  \end{clojure}
\end{english}

\index{NPE}

Примечание: класс \verb|NullPointerException|~--- частый гость в мире Java. Для
краткости его называют \verb|NPE|.

Причина в том, что \verb|nil| попал в функцию \verb|re-matches|. Функция
трактует аргумент как строку, что приводит к \verb|NPE|. Перепишите спеку так,
чтобы она не бросала исключения. В примере с \verb|::url| сначала убедимся,
что это строка, и только потом проверим регулярным выражением.

\ifnarrow

\begin{english}
  \begin{clojure}
(s/def ::url
  (s/and ::ne-string
         (partial re-matches
           #"(?i)^http(s?)://.*")))

(s/valid? ::url nil) ;; false
  \end{clojure}
\end{english}

\else

\begin{english}
  \begin{clojure}
(s/def ::url
  (s/and ::ne-string
         (partial re-matches #"(?i)^http(s?)://.*")))

(s/valid? ::url nil) ;; false
  \end{clojure}
\end{english}

\fi

\index{spec!::ne-string}

Макрос \verb|s/and| устроен так, что на первой неудаче цепь оборвётся. Всё, что
после звена \verb|::ne-string|, не сработает, и исключения не будет.

По аналогии проверим возраст пользователя. Это предикаты на~число и~диапазон.

\begin{english}
  \begin{clojure}
(s/def ::age
  (s/and int? #(<= 0 % 150)))

(s/valid? ::age nil) ;; false
(s/valid? ::age -1)  ;; false
(s/valid? ::age 42)  ;; true
  \end{clojure}
\end{english}

\section{Спеки-коллекции}

\index{коллекции}
\index{скаляры}

Выше мы проверяли примитивные типы, или \emph{скаляры}. Это удобно для примеров,
но редко встречается на практике. В~основном проверяют не скаляры,
а~коллекции. Spec предлагает макросы, чтобы задать спеки-коллекции из
примитивов.

\index{spec!coll-of}

Макрос \verb|s/coll-of| принимает предикат или ключ и возвращает
спеку-коллекцию. Она проверяет, что каждый элемент проходит валидацию. Вот так
мы определим список URL:

\begin{english}
  \begin{clojure}
(s/def ::url-list (s/coll-of ::url))
  \end{clojure}
\end{english}

\noindent
Быстрая проверка:

\ifnarrow

\begin{english}
  \begin{clojure}
(s/valid? ::url-list
  ["http://test.com" "http://ya.ru"])
;; true

(s/valid? ::url-list
  ["http://test.com" "dunno.com"])
;; false
  \end{clojure}
\end{english}

\else

\begin{english}
  \begin{clojure}
(s/valid? ::url-list ["http://test.com" "http://ya.ru"])
;; true

(s/valid? ::url-list ["http://test.com" "dunno.com"])
;; false
  \end{clojure}
\end{english}

\fi

Ключи \verb|:count|, \verb|:min-count| и \verb|:max-count| задают ограничения на
число элементов в списке: точное, минимальное и максимальное
количество. Наиболее частый сценарий~--- убедиться, что список не пустой, потому
что в этом случае нет смысла его обрабатывать. Если задать \verb|:min-count|
единицу, мы отсечём пустоту:

\begin{english}
  \begin{clojure}
(s/valid? ::url-list [])
;; false

(s/valid? ::url-list ["http://test.com"])
;; true
  \end{clojure}
\end{english}

\index{spec!map-of}
\index{HTTP!параметры}

Макрос \verb|s/map-of| описывает словарь. Вспомним поле \verb|:params| из
главы про веб-разработку \page{ring-params}. Его ключи~--- кейворды, а значения~---
строки. На языке спеки параметры выглядят так:

\ifnarrow

\begin{english}
  \begin{clojure}
(s/def ::params
  (s/map-of keyword? string?))

(s/valid? ::params
  {:foo "test"})  ;; true
(s/valid? ::params
  {"foo" "test"}) ;; false
  \end{clojure}
\end{english}

\else

\begin{english}
  \begin{clojure}
(s/def ::params
  (s/map-of keyword? string?))

(s/valid? ::params {:foo "test"})  ;; true
(s/valid? ::params {"foo" "test"}) ;; false
  \end{clojure}
\end{english}

\fi

В целом проверка \verb|s/map-of| довольно слабая, чтобы покрыть все
варианты. Факт того, что значения~--- строки не даёт полезной информации. Важнее
убедиться, что в словаре именно те ключи, что мы ожидаем. К тому же редко бывает
так, что тип значений одинаковый. Наоборот, словарь несёт разные сведения о
сущности: имя, возраст, дату.

\index{spec!keys}

В таких случаях используют макрос \verb|s/keys|, в котором перечислены
спеки. Имена спек совпадают с ключами словаря. Значения ключей проверяются
одноимёнными спеками.

Представим веб-страницу с адресом и описанием. Объявим спеки этих полей по
отдельности:

\begin{english}
  \begin{clojure}
(s/def :page/address ::url)
(s/def :page/description ::ne-string)
  \end{clojure}
\end{english}

\index{пространства имён}

Обратите внимание на пространство ключей. Адрес и описание относятся к странице,
поэтому им задают своё пространство. У статьи или книги тоже могут быть адрес и
описание. Пространство обещает, что спеки \verb|:page/address| и
\verb|:book/address| не заменят друг друга.

Составим спеку страницы:

\begin{english}
  \begin{clojure}
(s/def ::page
  (s/keys :req-un [:page/address
                   :page/description]))
  \end{clojure}
\end{english}

\index{spec!:req-un}

В параметре \verb|:req-un| указан вектор спек. Для каждой из них спека ищет в
словаре ключ с таким же именем и проверяет значение. Но сначала рассмотрим, что
означает \verb|:req-un| и какие ещё параметры принимает \verb|s/keys|.

Имя \verb|:req-un| состоит из частей req и un. Это признаки наличия
ключа и его типа. Req (англ.~required) означает, что ключи обязательно должны
быть в словаре. Если хотя бы одного ключа нет, получим ошибку. Противоположный
по смыслу параметр называется opt (англ.~optional). В нём указаны ключи,
которых может не быть. Их валидация происходит, только если они были в словаре.

\index{spec!:un}

Частичка un означает unqualified, неполный ключ. При проверке un-ключей
спека отбрасывает их пространство. Например, если указать \verb|:page/address|
в списке \verb|:req-un|, то в словаре ищется ключ \verb|:address|, а не
\verb|:page/address|.

Неполные ключи встречаются часто. Данные приходят из чужих API и баз данных,
которые не знают о пространствах имён. Исключения возможны, когда весь стек
фирмы построен на Clojure. В этом случае клиент и сервер обмениваются данными с
полными ключами.

Различают следующие комбинации \verb|req|, \verb|opt| и \verb|un|:

\begin{itemize}

\item
  \verb|:req|~--- необходимые полные ключи,

\item
  \verb|:req-un|~--- необходимые краткие ключи,

\index{spec!:opt}

\item
  \verb|:opt|~--- опциональные полные ключи,

\index{spec!:opt-un}

\item
  \verb|:opt-un|~--- опциональные краткие ключи.

\end{itemize}

У спеки \verb|::page| ключи обязательны и не учитывают пространство. Ниже
примеры данных \emph{с ошибками}. Это может быть неправильный адрес, пустое
описание, пропавший ключ. Если каждый из словарей подставить в выражение
\spverb|(s/valid? ::page <data>)|, результат будет ложью.

\ifnarrow

\begin{english}
  \begin{clojure}
{:address "clojure.org" ;; not a URL
 :description "Clojure Language"}

;; an empty string
{:address "https://clojure.org/"
 :description ""}

;; no description
{:address "https://clojure.org/"}

 ;; full keys
{:page/address "https://clojure.org/"
 :page/description "Clojure Language"}
  \end{clojure}
\end{english}

\else

\begin{english}
  \begin{clojure}
{:address "clojure.org" ;; not a URL
 :description "Clojure Language"}

{:address "https://clojure.org/"
 :description ""} ;; an empty string

{:address "https://clojure.org/"} ;; no description

{:page/address "https://clojure.org/" ;; full keys
 :page/description "Clojure Language"}
  \end{clojure}
\end{english}

\fi

Обратите внимание на последний случай. Значения верны, но у ключей пространство
\verb|:page|. Валидация не сработает, потому что спека ищет \verb|:address|,
а не \verb|:page/address|. Чтобы исправить последний пример, замените тип
ключей \verb|:req-un| на \verb|:req| (необходимые полные).

\index{spec!valid?}

\ifnarrow

\begin{english}
  \begin{clojure}
(s/def ::page-fq
  (s/keys :req [:page/address
                :page/description]))

(s/valid? ::page-fq
  {:page/address "https://clojure.org/"
   :page/description "Clojure Language"})
;; true
  \end{clojure}
\end{english}

\else

\begin{english}
  \begin{clojure}
(s/def ::page-fq
  (s/keys :req [:page/address
                :page/description]))

(s/valid? ::page-fq
          {:page/address "https://clojure.org/"
           :page/description "Clojure Language"})
;; true
  \end{clojure}
\end{english}

\fi

Усложним пример: добавим странице статус, который мы получили при
последнем обращении к ней. Поле опционально, потому что если к странице ещё не
обращались, в него нечего записать. Новая спека:

\begin{english}
  \begin{clojure}
(s/def :page/status int?)

(s/def ::page-status
  (s/keys :req-un [:page/address
                   :page/description]
          :opt-un [:page/status]))
  \end{clojure}
\end{english}

Словари с правильным статусом и без него пройдут валидацию:

\ifnarrow

\begin{english}
  \begin{clojure}
(s/valid? ::page-status
  {:address "https://clojure.org/"
   :description "Clojure Language"})

(s/valid? ::page-status
  {:address "https://clojure.org/"
   :description "Clojure Language"
   :status 200})
  \end{clojure}
\end{english}

\else

\begin{english}
  \begin{clojure}
(s/valid? ::page-status
          {:address "https://clojure.org/"
           :description "Clojure Language"})

(s/valid? ::page-status
          {:address "https://clojure.org/"
           :description "Clojure Language"
           :status 200})
  \end{clojure}
\end{english}

\fi

\index{nil}

Заметим, что \verb|s/keys| различает \verb|nil| и наличие ключа. Если статус
\verb|nil|, он \emph{состоит} в словаре. Сработает проверка \verb|nil| на
\verb|int?|, что приведёт к негативному результату. Это тот случай, когда пустое
значение не равно его отсутствию.

\ifnarrow

\begin{english}
  \begin{clojure}
(s/valid? ::page-status
  {:address "https://clojure.org/"
   :description "Clojure Language"
   :status nil})
;; false
  \end{clojure}
\end{english}

\else

\begin{english}
  \begin{clojure}
(s/valid? ::page-status
          {:address "https://clojure.org/"
           :description "Clojure Language"
           :status nil})
;; false
  \end{clojure}
\end{english}

\fi

\section{Вывод значений}

\label{spec-conform}

\index{вывод!значений}

До сих пор мы проверяли данные с помощью \verb|s/valid?|. Функция вернёт истину
или ложь, что значит~--- данные верны или нет. Иногда одной проверки
недостаточно: значения корректны, но требуется привести их к~нужному типу.

Предположим, на вход поступило число в виде строки. Мы убедились, что строка
состоит из цифр и не превышает допустимой длины. Однако после валидации значение
по-прежнему строка, и нужно парсить его вручную. Хотелось бы, чтобы типы вывел
за нас какой-то механизм.

\index{spec!conform}

\mnoindent
Spec предлагает такие возможности. Это функции \texttt{s/con\-for\-mer}
и~\code{s/\-con\-form} (англ.~conform~--- <<подчиняться>>).

\label{spec-invalid}

\index{spec!conformer}

\index{spec!::invalid}

Сначала пишут функцию вывода. Она принимает исходное значение и возвращает либо
новое, либо ключ \verb|::s/invalid|, что означает ошибку. Затем функцию
оборачивают в \verb|s/conformer|, чтобы получить спеку. Далее вызывают
\verb|(s/conform)| с ключом спеки и данными, которые нужно преобразовать. Если
вывод прошёл без ошибок, получим новое значение, а иначе~--- ключ
\verb|::s/invalid|.

Рассмотрим вывод числа из строки. Чтобы отличить вывод данных от валидации, к
имени добавляют стрелку, что означает приведение типа.

\begin{english}
  \begin{clojure}
(s/def ::->int
  (s/conformer
   (fn [value]
     (try
       (Integer/parseInt value)
       (catch Exception e
         ::s/invalid)))))
  \end{clojure}
\end{english}

\noindent
Эту спеку передают в \verb|s/conform| с данными:

\begin{english}
  \begin{clojure}
(s/conform ::->int "42") ;; 42

(s/conform ::->int "dunno")
:clojure.spec.alpha/invalid
  \end{clojure}
\end{english}

\index{исключения}

\verb|S/conform| не ловит исключения при работе, а вывод типов богат на
них. Будет правильно перехватить исключение и вернуть \verb|::s/invalid|, как
в примере выше.

Обе спеки~--- валидатор и конформер~--- можно объединить через \verb|s/and|,
чтобы проверить тип перед выводом. В нашем случае убедимся, что значение~---
строка. Так мы не допустим, чтобы в \verb|parseInt| попал \verb|nil| или
что-то другое:

\begin{english}
  \begin{clojure}
(s/def ::->int
  (s/and ::ne-string ::->int))

(s/conform ::->int nil)
:clojure.spec.alpha/invalid
  \end{clojure}
\end{english}

\index{веб-разработка!даты}

Рассмотрим, как восстановить из строки более сложный тип~--- дату. Это старая
проблема веб-разработки: JSON не поддерживает даты, поэтому их передают строкой
ISO или числом секунд. Возникает вопрос, как привести их к~объекту на сервере.

\index{модули!clojure.instant}
\index{функции!read-instant-date}

\mnoindent Понадобятся парсер строки и небольшая обвязка, чтобы подружить его со
спекой. Функция \verb|read-instant-date| из модуля \verb|clojure.instant| читает
дату из строки. Она лояльна к формату и учитывает неполную запись. Например,
датой может быть только год.

\index{даты}

\ifnarrow

\begin{english}
  \begin{clojure}
(require '[clojure.instant
           :refer [read-instant-date]])

(read-instant-date "2019")
#inst "2019-01-01T00:00:00.000-00:00"
  \end{clojure}
\end{english}

\else

\begin{english}
  \begin{clojure}
(require '[clojure.instant :refer [read-instant-date]])

(read-instant-date "2019")
#inst "2019-01-01T00:00:00.000-00:00"
  \end{clojure}
\end{english}

\fi

Обернём функцию в спеку:

\begin{english}
  \begin{clojure}
(s/def ::->date
  (s/and
   ::ne-string
   (s/conformer
    (fn [value]
      (try
        (read-instant-date value)
        (catch Exception e
          ::s/invalid))))))
  \end{clojure}
\end{english}

Перед разбором мы делаем минимальные проверки. Убеждаемся, что это не пустая
строка, чтобы отсечь \verb|nil| и прочий мусор. Вывод даты:

\begin{english}
  \begin{clojure}
(s/conform ::->date "2019-12-31")
#inst "2019-12-31T00:00:00.000-00:00"
  \end{clojure}
\end{english}

\noindent
Дата и время:

\ifnarrow

\begin{english}
  \begin{clojure}
(s/conform ::->date
           "2019-12-31T23:59:59")
#inst "2019-12-31T23:59:59.000-00:00"
  \end{clojure}
\end{english}

\else

\begin{english}
  \begin{clojure}
(s/conform ::->date "2019-12-31T23:59:59")
#inst "2019-12-31T23:59:59.000-00:00"
  \end{clojure}
\end{english}

\fi

По аналогии напишите спеку \verb|::ts->date|, которая выводит дату из эпохи
Unix. Добейтесь следующего результата:

\begin{english}
  \begin{clojure}
(s/conform ::ts->date 1609230152)
;; #inst "2020-12-29T08:22:32.000-00:00"
  \end{clojure}
\end{english}

\section{Спеки-перечисления}

\index{spec!перечисления}

Иногда известно заранее, какие значения принимает поле. Представим, что при
вызове API клиент передаёт архитектуру системы~--- 32 или 64 бита. Ради двух
значений нет смысла парсить число: подойдёт \verb|case| или словарь.

Ниже вариант с макросом \verb|case|. Если не нашли допустимых значений, сигналим
об ошибке ключом \verb|::s/invalid|. Заметим, что \verb|case| не означает
линейный перебор. Он строит индексы веток и сразу переходит на нужную.

\index{clojure.core!case}

\ifnarrow

\begin{english}
  \begin{clojure}
(s/def ::->bits
  (s/conformer
   (fn [value]
     (case value
       "32" 32 "64" 64
       ::s/invalid))))

(s/conform ::->bits "32") ;; 32

(s/conform ::->bits "42")
:clojure.spec.alpha/invalid
  \end{clojure}
\end{english}

\else

\begin{english}
  \begin{clojure}
(s/def ::->bits
  (s/conformer
   (fn [value]
     (case value
       "32" 32 "64" 64
       ::s/invalid))))

(s/conform ::->bits "32") ;; 32
(s/conform ::->bits "42") :clojure.spec.alpha/invalid
  \end{clojure}
\end{english}

\fi

\index{конфигурация}

Второй вариант основан на словаре для перевода значения. Удобно, когда словарь
живёт в отдельной переменной. Его легко дополнить или вынести в~конфигурацию,
при этом логика валидации не изменится.

\begin{english}
  \begin{clojure}
(def bits-map {"32" 32 "64" 64})

(s/def ::->bits
  (s/conformer
   #(get bits-map % ::s/invalid)))
  \end{clojure}
\end{english}

Похожим образом читают логические значения из строк. Нет единого соглашения о
том, как передать истину и ложь в тексте. Это могут быть \verb|True|,
\verb|TRUE|, \verb|1|, \verb|yes| для истины и противоположности \verb|FALSE|,
\verb|no| и~другие. При разборе значений их приводят к одному
регистру. В~Clojure \verb|FALSE| и \verb|false|~--- разные строки, даже если
отправитель имел в виду одно и то же. Сценарий выглядит так:

\begin{itemize}

\item
  убедиться, что значение~--- строка;

\item
  привести её к нижнему регистру;

\item
  найти значение по словарю или \verb|case|.

\end{itemize}

\noindent
Код вывода:

\ifnarrow

\begin{english}
  \begin{clojure}
(s/def ::->bool
  (s/and
   ::ne-string
   (s/conformer
     clojure.string/lower-case)
   (s/conformer
    (fn [value]
      (case value
        ("true" "1" "on" "yes")
        true

        ("false" "0" "off" "no")
        false

        ::s/invalid)))))

  \end{clojure}
\end{english}

\else

\begin{english}
  \begin{clojure}
(s/def ::->bool
  (s/and
   ::ne-string
   (s/conformer clojure.string/lower-case)
   (s/conformer
    (fn [value]
      (case value
        ("true" "1" "on" "yes") true
        ("false" "0" "off" "no") false
        ::s/invalid)))))

  \end{clojure}
\end{english}

\fi

\noindent
Вывод в действии:

\begin{english}
  \begin{clojure}
(s/conform ::->bool "True")  ;; true
(s/conform ::->bool "yes")   ;; true
(s/conform ::->bool "0")     ;; false
(s/conform ::->bool "FALSE") ;; false
  \end{clojure}
\end{english}

\section{Продвинутые техники}

\index{паттерны}

Мы написали достаточно кода, чтобы увидеть одинаковые участки~--- паттерны. В
этом разделе мы вынесем их в функции и~макросы, которые ускорят вашу работу.

\subsection{Множества}

\index{множества}

Когда значения известны, на роль спеки подходит множество. Оно ведёт себя как
функция: если аргумент найден во множестве, получим его же. Если нет, результат
будет \verb|nil|. Представим, что статус задачи может быть строкой \verb|todo|,
\verb|in_progress| и \verb|done|. Опишем спеку множеством этих значений:

\ifnarrow

\begin{english}
  \begin{clojure}
(s/def ::status
  #{"todo" "in_progres" "done"})
(s/valid? ::status "todo") ;; true
  \end{clojure}
\end{english}

\else

\begin{english}
  \begin{clojure}
(s/def ::status #{"todo" "in_progres" "done"})
(s/valid? ::status "todo") ;; true
  \end{clojure}
\end{english}

\fi

Множества подходят и для точного равенства. Если спека принимает строго одно
значение, её по-прежнему можно задать множеством одного
элемента. Предположим, параметр \verb|notify| определяет, как часто уведомлять
пользователя. Пока что мы поддерживаем только тип \verb|daily|, поэтому
запишем спеку так:

\begin{english}
  \begin{clojure}
(s/def ::notification #{"daily"})
  \end{clojure}
\end{english}

\noindent
Это же выражение можно переписать на \verb|partial| от равенства:

\ifnarrow

\begin{english}
  \begin{clojure}
(s/def ::notification
  (partial = "daily"))
  \end{clojure}
\end{english}

\else

\begin{english}
  \begin{clojure}
(s/def ::notification (partial = "daily"))
  \end{clojure}
\end{english}

\fi

Плюс множества в том, что его легче расширить, вынести в~файл или
конфигурацию. Если появятся пользователи, которых нужно уведомлять только раз в
неделю или месяц, мы просто добавим элементы \verb|:weekly| и \verb|:monthly|. В
случае с равенством придётся менять код.

\subsection{Перечисления}

\index{clojure.core!contains?}

Множество не подходит в случаях, когда \verb|false| и \verb|nil| считают верными
значениями. \verb|S/valid?| трактует их как неудачу. Если \verb|nil| или
\verb|false| допустимы, такие случаи проверяют функцией \verb|contains?|:

\begin{english}
  \begin{clojure}
(contains? #{1 :a nil} nil) ;; true
  \end{clojure}
\end{english}

Чтобы не повторяться, напишем функцию \verb|enum|. Она принимает значения и
возвращает предикат. В свою очередь, предикат принимает аргумент и проверяет,
есть ли такой среди исходных значений.

\begin{english}
  \begin{clojure}
(defn enum [& args]
  (let [arg-set (set args)]
    (fn [value]
      (contains? arg-set value))))
  \end{clojure}
\end{english}

Функция внутри замкнута на переменной \verb|arg-set|. Это множество, которое
получили из списка аргументов. Мы создали его один раз, чтобы не делать это
постоянно при вызове предиката. Новое перечисление выглядит коротко и ясно:

\begin{english}
  \begin{clojure}
(s/def ::status
  (enum "todo"
        "in_progres"
        "done"))
  \end{clojure}
\end{english}

\subsection{With-conformer}

\index{макросы!with-conformer}
\index{исключения!try}
\index{исключения!catch}

Конформеры требуют особого внимания. В них легко допустить ошибку: не
перехватить исключение или забыть о том, что функцию нужно обернуть в
\verb|s/conformer|. Чтобы снизить риск ошибки, вынесем рутину в~макрос
\verb|with-conformer|.

Макрос принимает символ переменной и произвольное тело. Он порождает функцию
одного аргумента, которая выполнит тело в блоке \verb|try/catch|. Функция
обёрнута в \verb|s/conformer|. Если исключения не было, получим последнее
выражение тела. В~противном случае вернётся \verb|::s/invalid|. Мы специально
поместили символ переменой в квадратные скобки, чтобы синтаксис напоминал
функцию (\verb|defn|).

\begin{english}
  \begin{clojure}
(defmacro with-conformer
  [[bind] & body]
  `(s/conformer
    (fn [~bind]
      (try
        ~@body
        (catch Exception e#
          ::s/invalid)))))
  \end{clojure}
\end{english}

\noindent

Вывод числа:

\begin{english}
  \begin{clojure}
(s/def ::->int
  (s/and
    ::ne-string
    (with-conformer [val]
      (Integer/parseInt val))))

(s/conform ::->int "123")
;; 123
  \end{clojure}
\end{english}

\noindent
и булева типа:

\ifnarrow

\begin{english}
  \begin{clojure}
(s/def ::->bool
  (s/and
   ->lower
   (with-conformer [val]
     (case val
       ("true"  "1" "on"  "yes")
       true

       ("false" "0" "off" "no" )
       false))))

(s/conform ::->bool "no")
;; false
  \end{clojure}
\end{english}

\else

\begin{english}
  \begin{clojure}
(s/def ::->bool
  (s/and
   ->lower
   (with-conformer [val]
     (case val
       ("true"  "1" "on"  "yes") true
       ("false" "0" "off" "no" ) false))))

(s/conform ::->bool "no")
;; false
  \end{clojure}
\end{english}

\fi

\noindent
В последнем примере переменная \verb|->lower|~--- обёртка для приведения к
нижнему регистру:

\ifnarrow

\begin{english}
  \begin{clojure}
(def ->lower
  (s/and
    string?
    (s/conformer
      clojure.string/lower-case)))
  \end{clojure}
\end{english}
\else

\begin{english}
  \begin{clojure}
(def ->lower
  (s/and
    string?
    (s/conformer clojure.string/lower-case)))
  \end{clojure}
\end{english}

\fi

\mnoindent
В \verb|case| необязательно указывать \verb|::s/invalid| на конце
макроса. Если \verb|case| не нашёл ветку и не задан вариант по умолчанию,
он~бросит исключение. \verb|With-conformer| перехватит его и вернёт
\verb|::s/invalid|.

\section{Логические пути}

\index{spec!логические пути}

Функция \verb|s/conform| не всегда возвращает то, что мы ожидаем. Некоторые
спеки оборачивают результат в вектор, где первый элемент~--- логический путь. Он
появляется там, где проверка ветвится из-за условий.

Знакомый нам \verb|s/and| обходит дочерние спеки по порядку и проверяет
данные. Это линейный подход, и порой его недостаточно: проверка
ветвится. Например, если значение~--- число, то оставить его как есть, а~если
строка, то привести к~числу. Спеки с развилками называют условными
\eng{conditional}.

Макрос \verb|s/or| принимает теги и дочерние спеки. Он применяет их к значению
до первого совпадения. Результат будет парой, где первый элемент~--- тег, а
второй~--- результат \verb|s/conform| для спеки.

Тег становится частью пути, по которому шла проверка. Логический путь помогает
расследовать, в каком месте произошла ошибка. В простых спеках ошибка видна
сразу, но на практике условная спека вложена в другую условную, та тоже и так
далее. Найти причину без логического пути будет трудно.

\index{spec!explain}

Если валидация не прошла, логический путь получают из отладочной информации. Её
возвращают функции семейства \verb|s/explain*|, которые мы рассмотрим ниже.

Напишем спеку сетевого порта, которая принимает число или строку. Во втором
случае спека выводит число. Это полезно, если значение приходит из переменной
среды или INI-файла.

\begin{english}
  \begin{clojure}
(s/def ::smart-port
  (s/or :string ::->int :num int?))
  \end{clojure}
\end{english}

\noindent
Теперь \verb|s/conform| вернёт не просто значение, а пару с тегом:

\begin{english}
  \begin{clojure}
(s/conform ::smart-port 8080)
[:num 8080]

(s/conform ::smart-port "8080")
[:string 8080]
  \end{clojure}
\end{english}

\index{скаляры}
\index{spec!развилки}

Если в спеке была развилка (\verb|s/or|, \verb|s/alt|), то структура
\verb|s/conform| отличается от входных данных: на месте скаляра появится
вектор. Покажем это на вложенных данных. Пусть порт~--- одно из полей
подключения к базе:

\begin{english}
  \begin{clojure}
(s/def :conn/port ::smart-port)
(s/def ::conn
  (s/keys :req-un [:conn/port]))

(s/conform ::conn {:port "9090"})
{:port [:string 9090]}
  \end{clojure}
\end{english}

Топология результата изменилась, и это нужно учесть. Если передать поле
\verb|:port| в функцию подключения, получим ошибку типов.

\section{Обратное действие}

\index{spec!unform}

У функции \verb|s/unform| противоположный смысл: по спеке и результату она
вернёт исходное значение. Выше мы получили словарь, где поле \verb|:port|~---
вектор. Чтобы вернуться к исходным данным, выполните:

\begin{english}
  \begin{clojure}
(s/unform ::conn {:port [:num 9090]})
{:port 9090}
  \end{clojure}
\end{english}

\noindent
Но если передать пару со строкой, получим ошибку:

\ifnarrow

\begin{english}
  \begin{clojure}
(s/unform ::conn {:port [:string 9090]})
;;Execution error (IllegalStateException)
;;no unform fn for conformer
  \end{clojure}
\end{english}

\else

\begin{english}
  \begin{clojure}
(s/unform ::conn {:port [:string 9090]})
;; Execution error (IllegalStateException)
;; no unform fn for conformer
  \end{clojure}
\end{english}

\fi

Дело в том, что тегу \verb|:num| мы задали предикат \verb|int?|. Спека,
полученная из предиката, не меняет данные, а только проверяет их. Поскольку для
\verb|:int| не было преобразований, функция \verb|s/unform| просто отбросит тег
и вернёт значение. С тегом \verb|:string| сложнее: он указывает на спеку типа
\verb|conformer|, которая \emph{меняет} значение. Чтобы выполнить \verb|unform|,
нужно сказать ей, как это делается.

\index{spec!conformer}

Напомним, первый аргумент \verb|s/conformer|~--- функция вывода типа. К ней
обращается \texttt{s/con\-form}, когда его вызывают. Второй необязательный
аргумент, который мы до сих пор игнорировали,~--- функция обратного действия: по
результату \verb|s/conform| она вернёт прежнее значение. Функция сработает в
момент вызова \verb|s/unform|.

Изменим \verb|::->int| так, чтобы он поддерживал \verb|unform|. Вторая функция
переведёт число обратно в строку:

\begin{english}
  \begin{clojure}
(s/def ::->int
  (s/and
   ::ne-string
   (s/conformer
    (fn [string] ;; conform func
      (Integer/parseInt string))
    (fn [integer] ;; unform func
      (str integer)))))
  \end{clojure}
\end{english}

\noindent
Теперь порт со строкой не вызовет ошибки:

\begin{english}
  \begin{clojure}
(s/unform ::conn {:port [:string 9090]})
{:port "9090"}
  \end{clojure}
\end{english}

\verb|Unform| полезен, если вы утратили исходные данные, но сохранили их
\verb|conform|-вариант.

Читатель заметит, что нашей целью было вывести порт из строки, однако из-за
условной спеки \verb|s/or| всё пошло не так. \verb|Conform| вывел тип, но
добавил в результат теги, из-за чего изменилась структура. \verb|Unform| убрал
теги, но вернул строку на место числа. В чём смысл?

Если коротко, \verb|conform| и теги незаменимы для парсинга данных и их
обхода. Далее мы покажем, как с помощью \verb|conform| получить синтаксическое
дерево, без которого не обходится ни один компилятор. Что касается
\verb|unform|, его поведение можно исправить хитростью.

В спеке \verb|::->int| в её \verb|unform|-части мы переводим число в обратно
строку. Если вернуть число без изменений, оно встанет на место \verb|:port| в
итоговом словаре:

\begin{english}
  \begin{clojure}
(s/def ::->int
  (s/and
   ::ne-string
   (s/conformer
    (fn [string]
      (Integer/parseInt string))
    (fn [integer]
      integer))))

(s/unform ::conn {:port [:string 9090]})
{:port 9090}
  \end{clojure}
\end{english}

\index{clojure.core!identity}

Для краткости вторым аргументом указывают \verb|identity|~--- функцию, которая
возвращает переданный аргумент. С таким подходом \verb|unform| всегда вернёт то,
что получили на этапе \verb|conform|. Полный цикл этих функций:

\begin{english}
  \begin{clojure}
(->> {:port "9090"}
     (s/conform ::conn)
     (s/unform ::conn))
{:port 9090}
  \end{clojure}
\end{english}

Чтобы не забыть \verb|unform| в спеке, сделайте его частью макроса
\verb|with-conformer| (последняя строка):

\begin{english}
  \begin{clojure}
(defmacro with-conformer
  [[bind] & body]
  `(s/conformer
    (fn [~bind]
      (try
        ~@body
        (catch Exception e#
          ::s/invalid)))
    identity))
  \end{clojure}
\end{english}

\section{Анализ ошибок}

\index{ошибки!spec}

\label{spec-explain}

Когда данные неверны, \verb|s/valid?| и \verb|s/conform| возвращают \verb|false|
и~\code{::s/\-in\-va\-lid}. Этого недостаточно, чтобы понять причину
ошибки. Представьте, что у вас спека пользователя. В ней несколько адресов,
в~каждом адресе несколько строк, и проверка вернула \verb|false|. Ручной поиск
ошибки займёт час.

Функции семейства \verb|s/explain| принимают спеку и данные. Если проверка не
удалась, получим более подробную информацию~--- отчёт об ошибке. Это словарь,
где указаны проблемные значения, спеки, пути к ним и другие данные. Разница
между функциями в том, как они поступают с отчётом:

\begin{itemize}

\index{spec!explain}
\index{spec!explain-str}
\index{spec!explain-data}
\index{spec!отчёт}
\index{отчёт!spec}

\item
  \verb|s/explain| печатает его в стандартный поток (на экран);

\item
  \verb|s/explain-str| возвращает отчёт в виде строки;

\item
  \verb|s/explain-data| возвращает словарь. Это самый полный отчёт об ошибке.

\end{itemize}

Попробуем функции в действии. Подготовим для них простую спеку:

\begin{english}
  \begin{clojure}
(s/def :sample/username ::ne-string)

(s/def ::sample
  (s/keys :req-un [:sample/username]))
  \end{clojure}
\end{english}

\noindent
На корректных данных функции не проявляют себя:

\ifnarrow

\begin{english}
  \begin{clojure}
(s/explain ::sample
  {:username "some user"})
Success!
nil

(s/explain-data ::sample
  {:username "some user"})
nil
  \end{clojure}
\end{english}

\else

\begin{english}
  \begin{clojure}
(s/explain ::sample {:username "some user"})
Success!
nil

(s/explain-data ::sample {:username "some user"})
nil
  \end{clojure}
\end{english}

\fi

\noindent
Попробуем число вместо имени:

\begin{english}
  \begin{clojure}
(s/explain ::sample {:username 42})
;; 42 - failed: string? in: [:username]
;; at: [:username] spec: ::string
  \end{clojure}
\end{english}

Вывод читается так: значение \verb|42| не прошло проверку предикатом
\verb|string?|. Путь к значению внутри словаря~--- \verb|[:username]|. Ключ
спеки, где случилась ошибка,~--- \verb|::string|.

Отчёт показывает наиболее вложенные спеки и предикаты. Вспомним, что
\verb|::ne-string|~--- это комбинация \verb|::string| и функции
\verb|not-empty|. Ошибка случилась на этапе \verb|::string|, о чём и было
сказано.

Для пустой строки вывод будет другим, потому что на этот раз причиной станет
\verb|not-empty|. Проверим это:

\begin{english}
  \begin{clojure}
(s/explain ::sample {:username ""})
;; "" - failed: not-empty in: [:username]
;; at: [:username] spec: ::ne-string
  \end{clojure}
\end{english}

\verb|Explain|~--- это быстрый способ сообщить о проблеме в конфигурации или
JSON-файле. Со временем вы научитесь читать его. Но чем сложнее данные, тем
меньше понятен \verb|explain|. Когда в коллекции больше трёх уровней,
отчёт заливает экран. Трудно даже разбить его на части, не говоря уж о
понимании. Чтобы подружиться с \verb|explain|, нужно промежуточное звено, о
котором речь в следующем разделе.

\section{Понятные ошибки}

\index{сообщения}
\index{spec!ошибки}

\label{spec-messages}

Когда проверяют данные, важен не только факт ошибки. Ещё важнее объяснить
клиенту, где именно он ошибся. Под клиентом не обязательно имеют в виду
человека. Даже если это другая программа, в~ответ добавляют понятный текст. Он
попадёт в журнал, который читают сотрудники.

Часто мы видим сообщения вроде <<Ошибка: DATAERROR>> без каких-либо деталей. Или
красную надпись <<проверьте данные>> над формой в два экрана. Этих глупостей
можно было бы избежать, умей программисты переводить язык машины на человеческий.

\index{интерфейс}
\index{пользователь}

Фраза \texttt{``'' - failed: not-empty in: [:user\-name]} не только ничего не скажет
пользователю, но и отпугнёт его машинной природой. Кажется, что в интерфейсе
возникла брешь и пользователь видит то, что не должен. Это резко снижает
доверие к системе.

Чтобы составить сообщение, вернёмся к функции \texttt{s/explain\--data}. Она
возвращает словарь с полной информацией о неудаче. Пример такого отчёта:

\begin{english}
  \begin{clojure}
(s/explain-data ::sample {:username ""})

#:clojure.spec.alpha
{:problems
 ({:path [:username]
   :pred clojure.core/not-empty
   :val ""
   :via [::sample ::ne-string]
   :in [:username]})
 :spec ::sample
 :value {:username ""}}
  \end{clojure}
\end{english}

\index{языки!Python}
\index{языки!JavaScript}

На первый взгляд непонятно, что с ним делать. Некоторые программисты сдаются и
говорят, что Spec не подходит для показа ошибок. Это не так: в отчёте все
необходимые данные, нужно только правильно их обработать.

Новички спрашивают: почему бы не сделать понятные сообщения на уровне
библиотеки? Например, назначить спеке поле с текстом <<введите правильный
адрес>>? Почему не взять пример с библиотек для Python или JavaScript?

\index{фундаментальность}

Ответ на этот вопрос не устраивает новичков. Вспомним тезис из начала главы:
Spec~--- это \emph{фундаментальная библиотека}. То, что мы проверяем ей поля
формы,~--- всего лишь частный случай. У~спеки разные области применения, поэтому
структура ошибки тоже фундаментальна.

Трудно создать систему ошибок, которая устроит всех. В каждом проекте свои
правила о том, как показывать ошибки. Иногда это фиксированное сообщение, а в
других случаях~--- шаблон. Где-то учитывают язык пользователя. Всё вместе это
даёт сложные сценарии.

Если бы разработчики Spec занялись ошибками, их фокус был бы смещён с главной
цели. Вместо Spec мы получили бы валидаторы по типу тех, что пишут десятками для
Python и JavaScript. Они скучны, не гибки и без концепции.

Словарь \verb|explain-data| содержит ключи \verb|:spec|, \verb|:value| и
\texttt{:prob\-lems} с пространством \texttt{clojure.\-spec.\-alpha}. Первые два
означают спеку и значение, которые приняли участие в проверке. Нас интересует
поле \verb|:problems|. Это список словарей, где каждый описывает ошибку
валидации. Перечислим их поля и семантику:

\index{spec!развилки}

\begin{itemize}

\item
  \verb|:path|~--- логический путь валидации. Вектор ключей, где спеки
  чередуются с тегами-развилками. Условные спеки вроде \verb|s/or| и
  \verb|s/alt| пишут сюда свои метки;

\item
  \verb|:pred|~--- символ предиката с пространством, например
  \texttt{clo\-ju\-re.\-core/\-string?};

\item
  \verb|:val|~--- значение, для которого предикат вернул ложь (число,
  \verb|nil|, коллекция);

\item
  \verb|:via|~--- вектор спек, по которым прошло значение от верхнего уровня к
  нижнему;

\item
  \verb|:in|~--- физический путь к значению. Вектор ключей и индексов, который
  передают в функцию \verb|get-in|. Если выполнить \spverb|(get-in <данные> <:in>)|,
  получим значение, которое вызвало ошибку.

\end{itemize}

В отчёте всё, что нам нужно для сообщения. Построим его: из \verb|:val| возьмём
проблемное значение. Спека, на которой остановилась валидация~--- последний
элемент вектора \verb|:via|.

Составим словарь, где ключ~--- спека, а значение~--- понятный текст или
шаблон. Зная спеку, которая вызвала ошибку, получим из словаря текст. В нашем
случае последний элемент \verb|:via|~--- это \verb|::ne-string|. Назначим ей
сообщение <<Строка не должна быть пустой>> или что-то похожее.

\label{spec-errors}

\ifnarrow

%% \begin{english}
  \begin{clojure}
(def spec-errors
  {::ne-string
   "Строка не должна быть пустой"})
  \end{clojure}
%% \end{english}

\else

%% \begin{english}
  \begin{clojure}
(def spec-errors
  {::ne-string "Строка не должна быть пустой"})
  \end{clojure}
%% \end{english}

\fi

\mnoindent
Напишем функцию, которая принимает словарь ошибки (один из элементов
\verb|::s/problems|) и возвращает сообщение:

\index{функции!get-message}

\ifnarrow

%% \begin{english}
  \begin{clojure}
(defn get-message [problem]
  (let [{:keys [via]} problem
        spec (peek via)]
    (get spec-errors spec)))

(get-message {:via [::sample
                    ::ne-string]})
"Строка не должна быть пустой"
  \end{clojure}
%% \end{english}

\else

%% \begin{english}
  \begin{clojure}
(defn get-message [problem]
  (let [{:keys [via]} problem
        spec (peek via)]
    (get spec-errors spec)))

(get-message {:via [::sample ::ne-string]})
"Строка не должна быть пустой"
  \end{clojure}
%% \end{english}

\fi

Функция \verb|peek| похожа на \verb|last| (получить последний
элемент). Разница в том, что \verb|last| бежит от начала коллекции, что
неэффективно. \verb|Peek|~--- это версия \verb|last| для вектора, которая сразу
прыгает в конец.

Проверим способ на других полях. Добавим в спеку \verb|::sample| адрес почты:

\index{регулярные выражения}
\index{clojure.core!re-matches}

\ifnarrow

\begin{english}
  \begin{clojure}
(s/def ::email
  (s/and
    ::ne-string
    (partial re-matches
      #"(.+?)@(.+?)\.(.+?)")))

(s/def :sample/email ::email)

(s/def ::sample
  (s/keys :req-un [:sample/username
                   :sample/email]))
  \end{clojure}
\end{english}

\else

\begin{english}
  \begin{clojure}
(s/def ::email
  (s/and
   ::ne-string
   (partial re-matches #"(.+?)@(.+?)\.(.+?)")))

(s/def :sample/email ::email)

(s/def ::sample
  (s/keys :req-un [:sample/username
                   :sample/email]))
  \end{clojure}
\end{english}

\fi

Спека \verb|::email| проверяет, что строка не пустая и совпадает с~шаблоном
адреса. Шаблон требует, чтобы в строке были символ~\verb|@| и~точка, а~между
ними произвольные символы.

\mnoindent
Если передать в \verb|email| пустую строку, последним элементом \verb|via|
будет \verb|::ne-string|. Для экономии места сократим вывод
\verb|explain-data|:

\ifnarrow

\begin{english}
  \begin{clojure}
(s/explain-data ::sample
  {:username "test" :email ""})

{:path [:email]
 :pred clojure.core/not-empty
 :val ""
 :via [::sample ::email ::ne-string]
 :in [:email]}
  \end{clojure}
\end{english}

\else

\begin{english}
  \begin{clojure}
(s/explain-data ::sample {:username "test" :email ""})

{:path [:email]
 :pred clojure.core/not-empty
 :val ""
 :via [::sample ::email ::ne-string]
 :in [:email]}
  \end{clojure}
\end{english}

\fi

Вызов \verb|get-message| с этой ошибкой вернёт сообщение о пустой
строке. Попробуем почту, которая не совпала с шаблоном. Последним элементом
\verb|:via| станет \verb|:sample/email|. Словарь ошибки выглядит так:

\ifnarrow

\begin{english}
  \begin{clojure}
(s/explain-data ::sample
  {:username "test" :email "test"})

{:path [:email]
 :pred
 (clojure.core/partial
  clojure.core/re-matches
  #"(.+?)@(.+?)\.(.+?)")
 :val "test"
 :via [::sample ::email]
 :in [:email]}
  \end{clojure}
\end{english}

\else

\begin{english}
  \begin{clojure}
(s/explain-data ::sample {:username "test" :email "test"})

{:path [:email]
 :pred
 (clojure.core/partial
  clojure.core/re-matches
  #"(.+?)@(.+?)\.(.+?)")
 :val "test"
 :via [::sample ::email]
 :in [:email]}
  \end{clojure}
\end{english}

\fi

Чтобы \verb|get-message| вернул новое сообщение для почты, добавим в словарь
ключ \verb|::email|. В будущем наполним его другими спеками и сообщениями, пока
не покроем все варианты.

\ifnarrow

%% \begin{english}
  \begin{clojure}
(def spec-errors
  {::ne-string
   "Строка не должна быть пустой"
   ::email
   "Введите правильный почтовый адрес"})
  \end{clojure}
%% \end{english}

\else

%% \begin{english}
  \begin{clojure}
(def spec-errors
  {::ne-string "Строка не должна быть пустой"
   ::email "Введите правильный почтовый адрес"})
  \end{clojure}
%% \end{english}

\fi

\index{перевод}
\index{языки!ClojureScript}

Система переводов, которую мы построили, довольно проста. Её легко тестировать и
менять под нужды конкретного проекта. Доработанные версии этой системы работают
в бою. В одной из них формы проверяют на клиенте до отправки на
сервер. Сообщение для виджетов получают тем же способом. Это возможно, поскольку
мощь Spec в полной мере доступна в ClojureScript.

Далее рассмотрим, как улучшить систему переводов.

\subsubsection*{Сообщение по умолчанию}

Что случится, если перевода нет в словаре? В этом случае вернём нейтральное
<<исправьте ошибку в данных>>. Заодно запишем в лог событие с именем спеки. Лог
устроен так, что сообщения из модуля переводов оседают в отдельном файле. Позже
локализаторы прочтут его и~добавят перевод.

%% \begin{english}
  \begin{clojure}
(def default-message
  "Исправьте ошибку в данных")
  \end{clojure}
%% \end{english}

Теперь мы не просто берём значение по словарю. Поиск текста перешёл в функцию
\texttt{get-\-better-\-message}, через которую проходят все случаи перевода.

\ifnarrow

\begin{english}
  \begin{clojure}
(defn get-better-message [problem]
  (let [{:keys [via]} problem
        spec (peek via)]
    (or (get spec-errors spec)
        (do
          (log/warnf
            "Missing message for spec %s"
            spec)
          default-message))))
  \end{clojure}
\end{english}

\else

\begin{english}
  \begin{clojure}
(defn get-better-message [problem]
  (let [{:keys [via]} problem
        spec (peek via)]
    (or (get spec-errors spec)
        (do (log/warnf "Missing message for spec %s" spec)
            default-message))))
  \end{clojure}
\end{english}

\fi

\noindent
Для любой незнакомой спеки получим нейтральное сообщение. В~комментарии показан
лог, который возникнет в консоли при вызове:

\ifnarrow

%% \begin{english}
  \begin{clojure}
(get-better-message
    {:via [::unknown :unknown/field]})
;; WARN book.spec - Missing message for
;; spec :unknown/field
"Исправьте ошибку в данных"
  \end{clojure}
%% \end{english}

\else

%% \begin{english}
  \begin{clojure}
(get-better-message {:via [::unknown :unknown/field]})
;; WARN book.spec - Missing message for spec :unknown/field
"Исправьте ошибку в данных"
  \end{clojure}
%% \end{english}

\fi

\subsubsection*{Восходящий поиск}

До сих пор мы работали с последним элементом вектора \verb|:via|. Давайте
рассмотрим вектор в целом. Для этого спровоцируем \verb|s/explain-data| с~пустой
почтой. В поле \verb|:via| окажется цепочка:

\ifnarrow

\begin{english}
  \begin{clojure}
(s/explain-data ::sample
  {:username "test" :email ""})

;; :via field
[::sample ::email ::ne-string]
  \end{clojure}
\end{english}

\else

\begin{english}
  \begin{clojure}
(s/explain-data ::sample {:username "test" :email ""})
[::sample ::email ::ne-string] ;; :via field
  \end{clojure}
\end{english}

\fi

Её читают так: ошибка в образце из-за почты \arr{} ошибка в почте из-за
пустой строки. Очевидно, спеки убывают по старшинству: на первом месте
глобальный объект, на последнем~--- локальная проблема. Из убывания можно
извлечь пользу: искать перевод не строго последней спеки, а по нарастанию
важности. Сперва проверить \verb|::ne-string|, затем \verb|::email| и, если
ничего не нашли, \verb|::sample|. Назовём метод восходящим поиском.

\index{clojure.core!keep}

В отличие от прошлых решений, мы ищем перевод для каждого элемента в обратном
порядке. Функции \verb|reverse| и \verb|keep| порождают ленивые коллекции,
поэтому вычислений не происходит, пока мы не обратимся к ним. В качестве
результата вернём первый не пустой элемент.

\begin{english}
  \begin{clojure/lines}
(defn get-message
  [problem]
  (let [{:keys [via]} problem]
    (->> via
         reverse
         (keep spec-errors)
         first)))
  \end{clojure/lines}
\end{english}

Переменная \verb|spec-errors| в четвёртой строке указывает на словарь сообщений,
который определили выше \page{spec-errors}. Словарь ведёт себя как функция,
поэтому передаём его в \verb|keep|.

Преимущество метода в том, что он позволяет настроить нужную детализацию
сообщений. Например, \verb|::email| может состоять из нескольких мелких спек,
однако из-за их низкого уровня сообщение будет непонятным. Для более ясного
перевода удалим из-словаря \verb|::ne-string| и оставим \verb|::email|.

\ifnarrow

%% \begin{english}
  \begin{clojure}
(def spec-errors
  {::email
   "Введите правильный почтовый адрес"})

(get-message
  {:via [::sample ::email ::ne-string]})
"Введите правильный почтовый адрес"
  \end{clojure}
%% \end{english}

\else

%% \begin{english}
  \begin{clojure}
(def spec-errors
  {::email "Введите правильный почтовый адрес"})

(get-message {:via [::sample ::email ::ne-string]})
"Введите правильный почтовый адрес"
  \end{clojure}
%% \end{english}

\fi

Если в спеке \verb|::sample| появится поле, для которого забыли перевод, поиск
разрешится в ключ \verb|::sample|. Назначим ему общий перевод вроде <<образец
заполнен неверно>>, чтобы локализовать проблему. Это полезно, если
\verb|::sample| станет элементом другой спеки, та~--- ещё одной и так далее.

\subsubsection*{Сходимость}

Чтобы объяснить, что такое сходимость и как ей управлять, проведём
эксперимент. Объявим спеки книги и фильма с одним полем~--- описанием:

\ifnarrow

\begin{english}
  \begin{clojure}
(s/def :book/description ::ne-string)
(s/def ::book
  (s/keys :req-un [:book/description]))

(s/def :movie/description ::ne-string)
(s/def ::movie
  (s/keys :req-un [:movie/description]))
  \end{clojure}
\end{english}

\else

\begin{english}
  \begin{clojure}
(s/def :book/description ::ne-string)
(s/def ::book (s/keys :req-un [:book/description]))

(s/def :movie/description ::ne-string)
(s/def ::movie (s/keys :req-un [:movie/description]))
  \end{clojure}
\end{english}

\fi

Если проверить книгу с пустым описанием, в векторе \verb|:via| будет следующее:

\ifnarrow

\begin{english}
  \begin{clojure}
(s/explain-data ::book
  {:description ""})
[::book ::ne-string]
  \end{clojure}
\end{english}

\else

\begin{english}
  \begin{clojure}
(s/explain-data ::book {:description ""})
[::book ::ne-string]
  \end{clojure}
\end{english}

\fi

Заметим, что в векторе нет спеки \texttt{:book/de\-scription}, хотя мы её
ожидали. Это значит, не получится составить сообщение специально для этого поля,
например <<неверное описание книги>>. Аналогично ведёт себя \verb|::movie|: если
передать фильм с пустой строкой, на конце \verb|:via| будет спека
\verb|::ne-string|. Почему так происходит?

Дело в том, что выражение \texttt{(s/def :book/de\-scrip\-tion ::ne\--string)}
работает как ссылка. Когда одна спека ссылается на другую, Spec укорачивает
цепь, чтобы сразу достичь \verb|::ne-string|. Говорят, что спеки
\verb|:book/description| и \texttt{:movie/\-de\-scrip\-tion} \emph{сходятся} в
\verb|::ne-string|. У сходимости свои преимущества: можно задать перевод для
ключа \verb|::ne-string| и ссылаться на неё во всех полях с текстом. В случаях,
когда важен путь \verb|:via|, спеку реорганизуют.

\mnoindent
Один из способов это сделать~--- заменить ссылку на предикат. Перепишите
\verb|:book/description|, как в примере ниже:

\ifnarrow

\begin{english}
  \begin{clojure}
(def ne-string
  (every-pred string? not-empty))
(s/def :book/description ne-string)
  \end{clojure}
\end{english}

\else

\begin{english}
  \begin{clojure}
(def ne-string (every-pred string? not-empty))
(s/def :book/description ne-string)
  \end{clojure}
\end{english}

\fi

Теперь описание ссылается не на \verb|::ne-string|, а на предикат с тем же
именем. Если \verb|ne-string| вернёт ложь, \verb|:book/description| станет
последним элементом \verb|:via|:

\ifnarrow

\begin{english}
  \begin{clojure}
(s/explain-data ::book
  {:description ""})
{:via [::book :book/description]}
  \end{clojure}
\end{english}

\else

\begin{english}
  \begin{clojure}
(s/explain-data ::book {:description ""})
{:via [::book :book/description]}
  \end{clojure}
\end{english}

\fi

То же самое с фильмом: укажите в спеке \verb|:movie/description| предикат
\verb|ne-string|, и нужный ключ появится в \verb|:via| при ошибке. Добавьте
спекам отдельные переводы и выполните поиск, как мы делали выше.

\ifnarrow

%% \begin{english}
  \begin{clojure}
(def spec-errors
  {:book/description
   "Введите описание книги"
   :movie/description
   "Введите описание фильма"})

(get-message
  {:via [::book :book/description]})
"Введите описание книги"
  \end{clojure}
%% \end{english}

\else

%% \begin{english}
  \begin{clojure}
(def spec-errors
  {:book/description "Введите описание книги"
   :movie/description "Введите описание фильма"})

(get-message {:via [::book :book/description]})
"Введите описание книги"
  \end{clojure}
%% \end{english}

\fi

Другой способ повлиять на сходимость~--- обернуть ключ спеки в~функцию
\verb|s/spec|:

\ifnarrow

\begin{english}
  \begin{clojure}
(s/def :book/description
  (s/spec ::ne-string))
  \end{clojure}
\end{english}

\else

\begin{english}
  \begin{clojure}
(s/def :book/description (s/spec ::ne-string))
  \end{clojure}
\end{english}

\fi

При такой записи спека \verb|:book/description| заменит ключ \verb|::ne-string|
в~векторе \verb|:via|:

\ifnarrow

\begin{english}
  \begin{clojure}
(s/explain-data ::book
  {:description ""})
[:book.spec/book :book/description]
  \end{clojure}
\end{english}

\else

\begin{english}
  \begin{clojure}
(s/explain-data ::book {:description ""})
[:book.spec/book :book/description]
  \end{clojure}
\end{english}

\fi

Преимущество \verb|s/spec| в том, что нужно только обернуть ссылки, а~предикаты
и~остальной код останутся без изменений.

\subsubsection*{Гибкий поиск}

Решим обратную задачу. Предположим, в проекте много сущностей с~описанием:
книги, фильмы, игры и так далее. Хотелось бы, чтобы поля
\verb|:movie/description|, \texttt{:book/descrip\-tion} и другие сходились к
переводу <<неверное описание>>. Как это сделать, не добавляя каждую спеку в
перевод?

Чтобы не засорять словарь, пойдём на хитрость. Пусть функция ищет перевод по
полному ключу, а если его нет, то по имени. Тогда хватит ключа
\verb|:description|, чтобы все спеки сошлись в этот перевод. Если для
конкретного \verb|:book/description| нужна особая фраза, добавим полную
версию:

\ifnarrow

%% \begin{english}
  \begin{clojure}
(def spec-errors
  {:description
   "Неверное описание"
   :book/description
   "Введите описание книги"})
  \end{clojure}
%% \end{english}

\else

%% \begin{english}
  \begin{clojure}
(def spec-errors
  {:description "Неверное описание"
   :book/description "Введите описание книги"})
  \end{clojure}
%% \end{english}

\fi

Перепишем поиск с учётом неполного ключа и фразы по умолчанию:

\ifnarrow

\begin{english}
  \begin{clojure}
(defn get-message
  [problem]
  (let [{:keys [via]} problem
        spec (peek via)]
    (or (get spec-errors spec)
        (get spec-errors
          (-> spec name keyword))
        default-message)))
  \end{clojure}
\end{english}

\else

\begin{english}
  \begin{clojure}
(defn get-message
  [problem]
  (let [{:keys [via]} problem
        spec (peek via)]
    (or (get spec-errors spec)
        (get spec-errors (-> spec name keyword))
        default-message)))
  \end{clojure}
\end{english}

\fi

\noindent
В действии:

\ifnarrow

%% \begin{english}
  \begin{clojure}
(get-message
  {:via [::movie :movie/description]})
"Неверное описание"

(get-message
  {:via [::movie :book/description]})
"Введите описание книги"
  \end{clojure}
%% \end{english}

\else

%% \begin{english}
  \begin{clojure}
(get-message {:via [::movie :movie/description]})
"Неверное описание"

(get-message {:via [::movie :book/description]})
"Введите описание книги"
  \end{clojure}
%% \end{english}

\fi

\subsubsection*{Мультиметод}

Возможно, система переводов понравится вашим коллегам, и они захотят её
использовать. Разумно вынести код в библиотеку и подключить в зависимости. Тогда
проекты будут одинаковы в плане сообщений.

Минус словаря в том, что его трудно расширить со стороны. Чтобы добавить
перевод, нужно выпустить новую версию библиотеки и обновить зависимости, что в
целом долго. Предоставим клиентам только механизм перевода, а содержимое они
наполнят сами.

Для этого заменим словарь на мультиметод. Его диспатчер принимает словарь ошибки
и находит виновную спеку. Далее расширим мультиметод спеками. С таким подходом
каждый потребитель добавит свои переводы или заменит чужие, если они не подошли.

Объявим мультиметод и реализацию для \texttt{::ne\--string}:

\index{мультиметоды!problem\arr{}text}

%% \begin{english}
  \begin{clojure}
(defmulti problem->text
  (fn [{:keys [via]}]
    (peek via)))

(defmethod problem->text ::ne-string [_]
  "Строка не должна быть пустой")
  \end{clojure}
%% \end{english}

\noindent
Пример его работы:

\ifnarrow

%% \begin{english}
  \begin{clojure}
(problem->text
  {:val "" :via [::email ::ne-string]})
"Строка не должна быть пустой"
  \end{clojure}
%% \end{english}

\else

%% \begin{english}
  \begin{clojure}
(problem->text {:val "" :via [::email ::ne-string]})
"Строка не должна быть пустой"
  \end{clojure}
%% \end{english}

\fi

Ключ \verb|:default| отвечает за действие по умолчанию. Если перевод не
найден, вернём стандартную фразу:

\begin{english}
  \begin{clojure}
(defmethod problem->text :default [_]
  default-message)
  \end{clojure}
\end{english}

\index{наследование}

Вспомним проблему поля description для многих сущностей. Хотелось бы, чтобы
спеки \verb|:book/description| и \texttt{:movie/\-descrip\-tion} сходились в общий
перевод \verb|::description|, но так, чтобы в любой момент можно было задать им
частный перевод. Это возможно с помощью иерархии ключей. Если в мультиметоде нет
ключа, но ключ наследует родителя, мультиметод выполнит поиск для родителя.

\mnoindent
Для этого унаследуем ключ \verb|:book/description| от
\texttt{::descri\-ption}. Если ошибка случится в спеке \verb|:book/description|,
получим общий перевод для любого описания.

\index{clojure.core!derive}

\ifnarrow

%% \begin{english}
  \begin{clojure}
(defmethod problem->text
  ::description [_] "Введите описание")

(derive :book/description ::description)

(problem->text
  {:val "" :via [:book/description]})
"Введите описание"
  \end{clojure}
%% \end{english}

\else

%% \begin{english}
  \begin{clojure}
(defmethod problem->text ::description [_]
  "Введите описание")

(derive :book/description ::description)

(problem->text {:val "" :via [:book/description]})
"Введите описание"
  \end{clojure}
%% \end{english}

\fi

\noindent
Если нужен особый перевод, расширим мультиметод:

\ifnarrow

%% \begin{english}
  \begin{clojure}
(defmethod problem->text
  :book/description [_]
  "Введите описание книги")

(problem->text
  {:val "" :via [:book/description]})
"Введите описание книги"
  \end{clojure}
%% \end{english}

\else

%% \begin{english}
  \begin{clojure}
(defmethod problem->text :book/description [_]
  "Введите описание книги")

(problem->text {:val "" :via [:book/description]})
"Введите описание книги"
  \end{clojure}
%% \end{english}

\fi

\index{синтаксис!\textbf{\_} (затенение)}

Заметим, что каждый метод принимает словарь ошибки. Мы затенили его символом
\verb|_|, потому что не обращаемся к нему и не хотим порождать лишнюю
переменную. В особых случаях можно построить фразу в зависимости от полей
ошибки. Например, добавить текущее значение:

\ifnarrow

%% \begin{english}
  \begin{clojure}
(defmethod problem->text
  :book/description
  [{:keys [val]}]
  (format "Ошибка в описании книги: %s"
    val))

(problem->text
  {:val "abc" :via [:book/description]})
"Ошибка в описании книги: abc"
  \end{clojure}
%% \end{english}

\else

%% \begin{english}
  \begin{clojure}
(defmethod problem->text :book/description
  [{:keys [val]}]
  (format "Ошибка в описании книги: %s" val))

(problem->text {:val "abc" :via [:book/description]})
"Ошибка в описании книги: abc"
  \end{clojure}
%% \end{english}

\fi

\emph{Замечание:} функция \verb|derive| с двумя аргументами порождает глобальное
наследование ключей. Это может повлиять на другие части программы, где логика
зависит от наследования. Будет хорошим тоном использовать локальную иерархию
ключей. Изучите, что представляет собой третий аргумент \verb|derive| и как он
работает в паре с мультиметодом.

\subsubsection*{Шаблон}

\index{шаблоны}

Если варианты выше показались вам сложными, попробуйте сообщение по~шаблону. Оно
складывается из имени поля и значения, например: <<в~поле email неверное
значение test>>. Сообщение легко получить функцией \verb|format|. В нём
слышится машинная природа, зато способ быстрый и дешёвый.

Имя поля получим как последний кейворд поля \verb|:in|. Почему бы не взять
просто последний элемент? Вспомним, что \verb|:in|~--- это физический путь к данным,
который передают в функции \verb|get-in|, \verb|assoc-in| и~другие. Его
элементами могут быть не только кейворды для вложенных словарей, но и индексы
для векторов.

Приведём пример, когда \verb|:in| заканчивается на число. Пусть задана
спека \verb|::post| для публикации в блоге. Кроме заголовка у записи могут
быть теги, набор не пустых строк:

\ifnarrow

\begin{english}
  \begin{clojure}
(s/def :post/title ::ne-string)
(s/def :post/tags
  (s/coll-of ::ne-string))

(s/def ::post
  (s/keys :req-un [:post/title]
          :opt-un [:post/tags]))
  \end{clojure}
\end{english}

\else

\begin{english}
  \begin{clojure}
(s/def :post/title ::ne-string)
(s/def :post/tags (s/coll-of ::ne-string))

(s/def ::post (s/keys :req-un [:post/title]
                      :opt-un [:post/tags]))
  \end{clojure}
\end{english}

\fi

Предположим, один из тегов оказался пустой строкой, что не имеет
смысла. Привёдем сокращённый отчёт \verb|s/explain-data|:

\ifnarrow

\begin{english}
  \begin{clojure}
(s/explain-data ::post
  {:title "On Clojure"
   :tags ["clojure" ""]})

{:path [:tags]
 :pred clojure.core/not-empty
 :val ""
 :via [:book.spec/post
       :post/tags
       :book.spec/ne-string
       :book.spec/not-empty]
 :in [:tags 1]}
  \end{clojure}
\end{english}

\else

\begin{english}
  \begin{clojure}
(s/explain-data ::post {:title "On Clojure"
                        :tags ["clojure" ""]})

{:path [:tags]
 :pred clojure.core/not-empty
 :val ""
 :via [:book.spec/post
       :post/tags
       :book.spec/ne-string
       :book.spec/not-empty]
 :in [:tags 1]}
  \end{clojure}
\end{english}

\fi

Последний элемент 1 указывает на второй проблемный тег, что верно (нумерация от
нуля). Чтобы узнать имя поля, которое привело к ошибке, поднимемся на уровень
выше и получим \verb|:tags|. Значение, которое не прошло валидацию, находится в
поле \verb|:val|. Этого достаточно, чтобы построить фразу:

\ifnarrow

\begin{english}
  \begin{clojure}
(defn last-kw [coll]
  (->> coll
       reverse
       (filter keyword?)
       first))

(defn get-common-message
  [problem]
  (let [{:keys [in val]} problem
        field (last-kw in)]
    (format
      "The field '%s' has
                an incorrect value '%s'."
      (name field) val)))
  \end{clojure}
\end{english}

\else

\begin{english}
  \begin{clojure}
(defn last-kw [coll]
  (->> coll
       reverse
       (filter keyword?)
       first))

(defn get-common-message
  [problem]
  (let [{:keys [in val]} problem
        field (last-kw in)]
    (format "The field '%s' has an incorrect value '%s'."
            (name field) val)))
  \end{clojure}
\end{english}

\fi

Поле и значение заключим в кавычки, чтобы выделить из общего текста. Проверим,
что вернёт функция:

\ifnarrow

\begin{english}
  \begin{clojure}
(get-common-message
  {:val "" :in [:tags 1]})
"The field 'tags' has
          an incorrect value ''."
  \end{clojure}
\end{english}

\else

\begin{english}
  \begin{clojure}
(get-common-message {:val "" :in [:tags 1]})
"The field 'tags' has an incorrect value ''."
  \end{clojure}
\end{english}

\fi

Если поле множественное, одного имени недостаточно. Сообщение станет лучше, если
добавить индекс значения с~ошибкой, например \verb|'tags[1]'|. Другой подход~---
вписать номер поля естественным образом, например <<the second tags
field...>>. Подумайте, как ещё улучшить текст и доработайте алгоритм.

\subsection{Открытые вопросы}

За рамками остались несколько вопросов. Они слишком общие, чтобы претендовать на
конкретное решение. В этом разделе мы не будем писать код, а только
порассуждаем.

Что делать, если требуется локализация, то есть текст на русском или английском
в зависимости от настроек? Для этого переделаем словарь ошибок. В нём станет два
уровня: на первом~--- код локали (\verb|ru|, \verb|en|), а на втором~---
переводы спек.

\ifnarrow

%% \begin{english}
  \begin{clojure}
(def spec-errors
  {:ru {::ne-string "Строка не должна быть пустой"
        ::email "Введите правильный почтовый адрес"}
   :en {::ne-string "The field must not be blank"
        ::email "The email address is invalid"}})
  \end{clojure}
%% \end{english}

\else

%% \begin{english}
  \begin{clojure}
(def spec-errors
  {:ru {::ne-string "Строка не должна быть пустой"
        ::email "Введите правильный почтовый адрес"}
   :en {::ne-string "The field must not be blank"
        ::email "The email address is invalid"}})
  \end{clojure}
%% \end{english}

\fi

\index{локаль}
\index{перевод}

Для локали мы получим словарь переводов, затем переводим сообщение, как делали
выше. С кодом локали можно схитрить, чтобы облегчить поиск. Для отдельных фраз
выделяют более точные локали, например американский и британский диалекты
с~кодами \verb|en_US| и~\verb|en_GB|. Изменим поиск так, что сперва он ищет по
младшей локали (\verb|en_US|), а затем по старшей (\verb|en|). Если британского
текста не оказалось, получим общий английский. Так работает перевод сообщений
в~широком смысле, не только ошибок.

Вопрос, откуда читать локаль, остаётся на ваше усмотрение. Можно хранить её в
сессии, параметрах запроса, базе данных, словом~--- как это удобно в проекте. Мы
ещё вернёмся к теме перевода строк в главе про изменяемость \page{translate}.

Второй вопрос~--- как связать ошибки с интерфейсом. В классическом подходе
\footurl{Mo\-del\--View\--Cont\-roller}{https://en.wikipedia.org/wiki/Model-view-controller}[MVC]
модель отделяют от представления, и формы следуют этому правилу. Удобно, когда
форма выражена структурой данных. Операции над ней~--- чистые функции,
которые легко поддерживать. Каждая функция возвращает форму в новом состоянии.

Представим форму в виде дерева. Ключи~--- это поля, а значения~---
виджеты. Виджет содержит тип поля, текущее значение и текст ошибки. На каждый
виджет подписан React-компонент. При изменении виджета он рисует поле ввода с
текущим значением. Если ошибка не \verb|nil|, как во втором поле, над ним
появится красное сообщение.

\begin{english}
  \begin{clojure}
(def form
  {:name {:title "Your name"
          :widget :textfield
          :value "Ivan"
          :error nil}
   :email {:title "Email address"
           :widget :textfield
           :value "test@"
           :error "Email is incorrect"}})
  \end{clojure}
\end{english}

Валидация принимает форму и строит дерево значений. У него такая же топология, но
на месте виджетов значения полей ввода:

\begin{english}
  \begin{clojure}
(def values
  {:name "Ivan"
   :email "test@"})
  \end{clojure}
\end{english}

С~помощью спеки мы проверяем значения и выводим типы из строк. В~случае ошибки
получим отчёт \verb|explain|. Для каждого элемента из поля \verb|problems|
находим путь, спеку и сообщение об ошибке. Это сообщение добавляем виджету в
поле \verb|:error|. Компонент, который подписан на виджет, заново отрисует его с
ошибкой над полем.

\index{интерфейс}
\index{фреймворки!React}

Мы упомянули формы, React и проблемы интерфейса. Всё вместе это называется
\emph{фронтенд}. Мы не будем на нём останавливаться, потому что фронтенд~---
сложная тема, достойная отдельной книги. Отметим только, что если код на Clojure
не содержит побочных эффектов и не обращается к Java-библиотекам, его легко
запустить на клиенте с помощью ClojureScript. Таким образом, все подходы и
техники, описанные выше, справедливы и для фронтенда.

\section{Парсинг}

Мы научились проверять данные и выводить типы. Перейдём к более сложной
операции~--- парсингу. Под термином понимают разбор данных на части, поиск
структуры там, где прежде её не было.

Возможно, вам приходилось писать регулярные выражения. Это шаблоны на особом
языке, которые описывают структуру текста. Специальные функции принимают строку
и регулярное выражение. Они возвращают фрагменты текста, которые совпали с
шаблоном.

Пример регулярного выражения~--- IP-адрес. Он состоит из четырёх групп с
точками. Каждая группа~--- число от 0 до 255.

\index{парсинг}
\index{регулярные выражения}

\begin{english}
  \begin{text}
\d{1,3}\.\d{1,3}\.\d{1,3}\.\d{1,3}
  \end{text}
\end{english}

В шаблоне мы ставим косую черту перед точкой, потому что без слеша точка
означает любой символ. Чтобы сослаться именно на символ точки, её экранируют.

В регулярных выражениях применяют операторы \verb|+|, \verb|?|, \verb|*| и
другие. Они указывают, сколько раз встречается шаблон перед ними: один и~более
раз, ни одного или один, произвольное число. В зависимости от оператора шаблон
захватывает разные части текста.

Представьте, что регулярные выражения откусывают строку по частям. Та часть, что
легла на шаблон, уходит в результат. Остаток переходит к следующему шаблону, и
так далее.

Регулярные выражения подводят нас к \verb|regex|-спекам. Это особые спеки для
разбора данных по шаблону. Разница в том, что входные данные~--- это коллекции, а не
текст.

\index{spec!regex}

\subsection{Простой разбор}

Предположим, нужно разобрать массив пользователей. Каждый из них~--- это кортеж
вида \tuple{номер, почта, статус}. Все значения~--- строки:

\begin{english}
  \begin{text}
1,test@test.com,active
2,ivan@test.com,pending
3,user@test.com,active
  \end{text}
\end{english}

Для каждого пользователя требуется:

\begin{itemize}

\item
  убедиться, что в кортеже именно три элемента;

\item
  привести номер к числу;

\item
  проверить почту на минимальные критерии;

\item
  привести статус к перечислению (одному из кейвордов);

\item
  получить словарь с верными значениями.

\end{itemize}

Мы уже знакомы с \verb|s/conformer| и выводом типов. Можно написать функцию,
которая примет кортеж целиком и выполнит действия выше. Это несложно, но функция
будет монолитом со слишком большим \emph{скоупом}. Рассмотрим другой способ.

\index{spec!cat}

Спека \verb|s/cat| служит для разбора коллекций. Она принимает набор тегов и
других спек. На вход подают коллекцию, и \verb|s/cat| накладывает её элементы на
спеки. Если они совпали, получим словарь. Ключи словаря~--- теги, а значения~---
вызов дочерней спеки с элементом.

Составим спеку для разбора кортежа. Начнём со статуса: спека ниже приводит
строки \verb|active| и \verb|pending| к особым кейвордам. Другие значения
спровоцируют ошибку.

\begin{english}
  \begin{clojure}
(s/def :user/status
  (s/and ->lower
         (with-conformer [val]
           (case val
             "active"  :USER_ACTIVE
             "pending" :USER_PENDING))))
  \end{clojure}
\end{english}

Спеки вывода числа \verb|::->int| и проверки почты \verb|::email| мы описали
ранее, поэтому используем их повторно. Готовая спека пользователя:

\begin{english}
  \begin{clojure}
(s/def ::user
  (s/cat :id ::->int
         :email ::email
         :status :user/status))
  \end{clojure}
\end{english}

\noindent
Положительный случай даёт на выходе словарь:

\ifnarrow

\begin{english}
  \begin{clojure}
(s/conform ::user
  ["1" "test@test.com" "active"])
{:id 1
 :email "test@test.com"
 :status :USER_ACTIVE}
  \end{clojure}
\end{english}

\else

\begin{english}
  \begin{clojure}
(s/conform ::user ["1" "test@test.com" "active"])
{:id 1
 :email "test@test.com"
 :status :USER_ACTIVE}
  \end{clojure}
\end{english}

\fi

Варианты с плохим номером, почтой или статусом не пройдут разбор. Примеры ниже
вернут \verb|::s/invalid|:

\ifnarrow

\begin{english}
  \begin{clojure}
(s/conform ::user
  ["" "test@test.com" "active"])
(s/conform ::user
  ["1" "@test.com" "active"])
(s/conform ::user
  ["1" "test@test.com" "unknown"])
  \end{clojure}
\end{english}

\else

\begin{english}
  \begin{clojure}
(s/conform ::user ["" "test@test.com" "active"])
(s/conform ::user ["1" "@test.com" "active"])
(s/conform ::user ["1" "test@test.com" "unknown"])
  \end{clojure}
\end{english}

\fi

Скорее всего, для неудачных записей вы должны вывести сообщение об
ошибке. Проверьте отчёт \verb|explain| для случаев выше и сформируйте текст.

\subsection{Условный разбор}

Представим, что работаем с устаревшим форматом данных. В нём условие: если
перед номером стоит метка blocked, пользователь заблокирован. Например:

\begin{english}
  \begin{text}
blocked;1;test@test.com;active
  \end{text}
\end{english}

Это усложняет задачу, ведь теперь кортеж состоит из трёх \emph{или} четырёх
элементов. Сдвигается семантика полей: первый элемент уже не только номер, но и
флаг блокировки. В старых данных, особенно финансовых, бывают и более странные
условия.

В императивных языках они порождают каскад \verb|if/else|. В Clojure проблему
решают декларативно. Объявим спеку блокировки, которая выводит флаг из строки:

\begin{english}
  \begin{clojure}
(s/def ::blocked
  (s/and
   ->lower
   (s/conformer (partial = "blocked"))))
  \end{clojure}
\end{english}

\index{spec!?}

Добавим её в итоговую \verb|s/cat|, но укажем, что она встречается ни разу или
только один раз. Для этого \verb|::blocked| оборачивают в спеку \verb|s/?|.  В
регулярных выражениях знак вопроса делает то же самое для шаблона. Дополним
спеку новым полем:

\begin{english}
  \begin{clojure}
(s/def ::user
  (s/cat :blocked (s/? ::blocked)
         :id ::->int
         :email ::email
         :status :user/status))
  \end{clojure}
\end{english}

Теперь оба кортежа совпадают со спекой \verb|::user|. Если метки нет, получим
словарь, как в начале. Если пользователь заблокирован, в~словаре будет поле
\verb|:blocked|:

\ifnarrow

\begin{english}
  \begin{clojure}
(s/conform ::user
  ["1" "test@test.com" "active"])
{:id 1
 :email "test@test.com"
 :status :USER_ACTIVE}

(s/conform ::user
  ["BLOCKED" "1"
   "test@test.com" "active"])
{:blocked true
 :id 1
 :email "test@test.com"
 :status ...}
  \end{clojure}
\end{english}

\else

\begin{english}
  \begin{clojure}
(s/conform ::user ["1" "test@test.com" "active"])
{:id 1 :email "test@test.com" :status :USER_ACTIVE}

(s/conform ::user ["BLOCKED" "1" "test@test.com" "active"])
{:blocked true :id 1 :email "test@test.com" :status ...}
  \end{clojure}
\end{english}

\fi

Представим, что на входе коллекция кортежей. Чтобы не утруждать себя итерацией,
объявим спеку-коллекцию:

\ifnarrow

\begin{english}
  \begin{clojure}
(s/def ::users (s/coll-of ::user))

(def user-data
  [["1" "test@test.com" "active"]
   ["Blocked" "2" "joe@doe.com"
    "pending"]])

(s/conform ::users user-data)
[{:id 1
  :email "test@test.com"
  :status :USER_ACTIVE}
 {:blocked true
  :id 2
  :email "joe@doe.com"
  :status ...}]
  \end{clojure}
\end{english}

\else

\begin{english}
  \begin{clojure}
(s/def ::users (s/coll-of ::user))

(def user-data
  [["1" "test@test.com" "active"]
   ["Blocked" "2" "joe@doe.com" "pending"]])

(s/conform ::users user-data)
[{:id 1 :email "test@test.com" :status :USER_ACTIVE}
 {:blocked true :id 2 :email "joe@doe.com" :status ...}]
  \end{clojure}
\end{english}

\fi

\index{clojure.core!filter}
\index{clojure.core!complement}

Отсеем заблокированных пользователей с помощью \verb|filter| и предиката
\texttt{(complement :blo\-cked)}.

\subsection{Практика}

\index{библиотеки!Etaoin}
\index{Selenium}

\label{selenium-ide}

Рассмотрим задачу, которую автор решал в работе над библиотекой Etaoin. Один из
её модулей выполняет команды \footurl{Selenium IDE}{https://www.selenium.dev/selenium-ide}[Selenium\\*IDE][-5mm].
Это расширение для браузера, которое записывает действия пользователя в файл
JSON. Получается сценарий, который можно запустить вручную или с помощью
утилиты.

Как правило, сценарий проверяют конкретную часть приложения: авторизацию, сброс
пароля, заказ товара. Скрипт, записанный локально, запускают на тестовых
машинах. Представим, что наша задача~--- написать интерпретатор этих сценариев
на Clojure.

Внутри JSON-файла нас ожидает поле \texttt{com\-mands} с массивом команд. Каждая
команда~--- словарь с ключами \verb|id|, \verb|command|, \verb|value| и
другими. Наиболее важно для нас поле \verb|command|, в котором находится имя
команды: open, click, assert и другие. Приведём отрывок из файла:

\begin{english}
  \begin{json}
"commands": [{
  "id": "af046c62-...-53a16251705f",
  "command": "open",
  "target": "/"
}, {
  "id": "ecc21655-...-91da3f4abba7",
  "command": "setWindowSize",
  "target": "1680x960"
}]
  \end{json}
\end{english}

\index{мультиметоды}

На первый взгляд всё просто: обойдём команды в цикле и вызовем для каждой
мультиметод \verb|do-cmd|.

\begin{english}
  \begin{clojure}
(doseq [command commands]
  (do-cmd command))
  \end{clojure}
\end{english}

Мультиметод находит реализацию по полю \verb|:command|. Останется наполнить его
командами, и задача готова. Ниже~--- реализация команды \verb|open|, которая
открывает веб-страницу.

\begin{english}
  \begin{clojure}
(defmulti do-cmd
  (fn [command]
    (-> command :command keyword)))

(defmethod do-cmd :open
  [{:keys [target]}]
  (webdriber/open-url *driver* target))
  \end{clojure}
\end{english}

Ситуация меняется из-за так называемых
\footurl{flow-команд}{https://www.selenium.dev/selenium-ide/docs/en/introduction/control-flow}[Control\\*flow],
к которым относятся \verb|if|, \verb|for|, \verb|repeat| и аналоги. В отличии от
обычных действий, они включают себя дочерние команды. Их нельзя выполнить в лоб:
при обработке \verb|if| мы сперва проверяем условие, а потом переходим на нужную
ветку. В случае с \verb|for| или \verb|repeat| внутренний блок нужно выполнить
определённое число раз.

\index{control flow}

Ниже показан минимальный пример с командой \texttt{if}. С точки зрения Clojure
эти команды~--- плоский вектор словарей, но мы добавили отступы для
читаемости. Чтобы код не вышел громоздким, оставим в каждом словаре минимальный
набор полей.

\begin{english}
  \begin{clojure}
(def data-if
  [{:command "if" :condition ...}
     {:command "print" :text "True"}
   {:command "else"}
     {:command "print" :text "False"}
   {:command "end"}])
  \end{clojure}
\end{english}

Как и в настоящем языке программирования, команды Selenium IDE могут быть
неограниченно вложены друг в друга. Например, если верно условие, открыть цикл:

\begin{english}
  \begin{clojure}
[{:command "if" :condition ...}
   {:command "for"}
     ...
   {:command "end"}
 {:command "end"}]
  \end{clojure}
\end{english}

\noindent
или условие в цикле внутри цикла:

\begin{english}
  \begin{clojure}
[{:command "for"}
   {:command "for"}
     {:command "if" :condition ...}
        ...
     {:command "else"}
        ...
     {:command "end"}
   {:command "end"}
 {:command "end"}]
  \end{clojure}
\end{english}

Ожидается, что наш код принимает JSON-файл, некоторые параметры и выполняет
команды с учетом условий и повторов. На страницах книги мы не будем решать
задачу досконально, но разберём её основную часть~--- группировку команд с
учётом flow-операторов. В этом нам поможет спека, а именно её
regex-функциональность.

\index{парсинг}

Очевидно, условные операторы наделяют код структурой, и мы должны отразить это в
программе. Для этого приведем плоский список к более сложному виду. Это
будет список пар, где первый элемент~--- тег, а второй~--- команда или её
сгруппированный вариант. Например, для обычного \texttt{print} получим пару:

\begin{english}
  \begin{clojure}
[:cmd {:command "print" :text "hello"}]
  \end{clojure}
\end{english}

\noindent
, а для \texttt{if}~--- нечто следующее (элемент \verb|{...}| означает словарь
команды):

\begin{english}
  \begin{clojure}
[:if {:tag {...}
      :condition {...}
      :branch-true [{...} {...}]
      :branch-else [{...} {...}]
      :end {...}}]
  \end{clojure}
\end{english}

В первом случае мы просто выполним \verb|print|. У варианта с \verb|if| второй
элемент структурирован. Теперь это не плоский список, а словарь с входным тегом,
положительной и негативной ветками. В новом виде обработать его гораздо
легче. Сначала проверим условие из ключа \verb|:condition|. Если оно истинно,
выполним \verb|:branch-true|, а иначе \verb|:branch-else|.

Начнём код с простой функции \texttt{parse-flow}, которая принимает вектор
команд и возвращает вектор пар:

\begin{english}
  \begin{clojure}
(defn parse-flow [commands]
  (s/conform ::flow commands))
  \end{clojure}
\end{english}

Чтобы перейти к новой структуре, определим спеку \texttt{::flow}. Это один и
более элементов, каждый из которых либо \texttt{:if}, либо обычная команда
\texttt{:cmd} (для краткости пока что опустим \texttt{for}). Спека не допустит
пустого списка команд, потому что обычно в этом нет смысла.

\begin{english}
  \begin{clojure}
(s/def ::flow
  (s/+ (s/alt :if ::flow-if
              :cmd ::command)))
  \end{clojure}
\end{english}

Подумаем, какой смысл мы вкладываем в термин <<обычная команда>> с ключом
\verb|:cmd|. Если коротко, это действие, которое не зависит от других команд, то
если не \texttt{if}, \texttt{else}, \texttt{for} или \texttt{end}. В
классических языках мы бы сказали, что это не служебное слово. Чтобы отличить
обычные команды от служебных, заведем множество flow-тегов:

\label{flow-tags}

\begin{english}
  \begin{clojure}
(def flow-tags #{"if" "else" "end" "for"})
  \end{clojure}
\end{english}

Объявим спеку \texttt{::command}. Она проверяет, что поле \texttt{:command}
словаря заполнено, но не входит в особые теги:

\begin{english}
  \begin{clojure}
(s/def ::command
  (fn [obj]
    (when-let [command (get obj :command)]
      (not (contains? flow-tags command)))))
  \end{clojure}
\end{english}

Введём вспомогательную функцию \texttt{tag=}. Она принимает тег и возвращает
функцию, которая принимает команду и проверяет, что у неё такой же тег.

\begin{english}
  \begin{clojure}
(defn tag= [tag]
  (fn [obj]
    (= (get obj :command) tag)))
  \end{clojure}
\end{english}

\noindent
Функция пригодится, чтобы быстро задать предикат:

\begin{english}
  \begin{clojure}
(def tag-print? (tag= "print"))

(tag-print? {:command "print"})
;; true
  \end{clojure}
\end{english}

Самое интересное кроется в спеке \texttt{::flow-if}. Приведём её описание:

\ifnarrow

\begin{english}
  \begin{clojure/lines}
(s/def ::flow-if
  (s/cat
    :this (tag= "if")
    :flow ::flow
    :else (s/? (s/cat
                 :this (tag= "else")
                 :flow ::flow))
    :end (tag= "end")))
  \end{clojure/lines}
\end{english}

\else

\begin{english}
  \begin{clojure/lines}
(s/def ::flow-if
  (s/cat :this (tag= "if")
         :flow ::flow
         :else (s/? (s/cat :this (tag= "else")
                           :flow ::flow))
         :end (tag= "end")))
  \end{clojure/lines}
\end{english}

\fi

Спека читается так. Сперва идёт открывающий элемент с тегом \texttt{if}
\coderef{\ifnarrow{3}\else{2}\fi}. Поместим его в словарь с ключом
\texttt{:this}. Затем~--- положительная ветка, одна и более других команд,
включая вложенные \texttt{if}.  Здесь мы просто ссылаемся на спеку
\texttt{:flow}, которую объявили выше. В поле \texttt{:else}
\coderef{\ifnarrow{5}\else{4}\fi} попадёт негативная ветка, однако лишь в том
случае, если для неё было совпадение (обёртка \verb|s/?| означает ни разу или
один раз). Ветка обнаружится, если встретится тег \verb|else| с последующими
командами. Спека завершается элементом \verb|end|.

Уже на этом этапе можно проверить парсинг. Подберём минимальный пример с печатью
и условием:

\ifnarrow

\begin{english}
  \begin{clojure}
[{:command "if" :condition ...}
   {:command "print" :text "True"}
 {:command "else"}
   {:command "print" :text "False"}
 {:command "end"}]
  \end{clojure}
\end{english}

\else

\begin{english}
  \begin{clojure}
[{:command "if" :condition ...}
   {:command "print" :text "True"}
 {:command "else"}
   {:command "print" :text "False"}
 {:command "end"}]
  \end{clojure}
\end{english}

\fi

Если передать эти данные в \texttt{parse-flow}, получим результат:


\begin{english}
  \begin{clojure}
[[:if
  {:this {:command "if" :condition "..."}
   :flow [[:cmd {:command "print"
                 :text "True"}]]
   :else
   {:this {:command "else"}
    :flow [[:cmd {:command "print"
                  :text "False"}]]}
   :end {:command "end"}}]]
  \end{clojure}
\end{english}

Более сложный вариант со вложенным \texttt{if}:

\begin{english}
  \begin{clojure}
[{:command "if" :condition "..."}
   {:command "if" :condition "..."}
     {:command "print" :text "hello"}
   {:command "end"}
 {:command "end"}]
  \end{clojure}
\end{english}

Видим, что структура кода сохраняется:

\ifnarrow

\begin{english}
  \begin{clojure}
[[:if
  {:this {:command "if"
          :condition "..."}
   :flow
   [[:if
     {:this {:command "if" :condition "..."}
      :flow [[:cmd {:command "print"
                    :text "hello"}]]
      :end {:command "end"}}]]
   :end {:command "end"}}]]
  \end{clojure}
\end{english}

\else

\begin{english}
  \begin{clojure}
[[:if
  {:this {:command "if" :condition "..."}
   :flow
   [[:if
     {:this {:command "if" :condition "..."}
      :flow [[:cmd {:command "print" :text "hello"}]]
      :end {:command "end"}}]]
   :end {:command "end"}}]]
  \end{clojure}
\end{english}

\fi

Важно, что ошибки в структуре не пройдут валидацию спекой. Например, незакрытый
тег \texttt{end} породит сбой в \texttt{::flow-if}, потому что он обязателен на
конце. Двойной \texttt{end} или \texttt{else} тоже нарушат спеку, ровно как и
пустая ветка без команд. Все варианты ниже вернут \texttt{::s/invalid}.

\begin{english}
  \begin{clojure}
(def data-broken1 ;; missing end
  [{:command "if" :condition "..."}
   {:command "print" :text "hello"}])

(def data-broken2 ;; double end
  [{:command "if" :condition "..."}
   {:command "print" :text "hello"}
   {:command "end"}
   {:command "end"}])

(def data-broken3 ;; empty block
  [{:command "if" :condition "..."}
   {:command "end"}])
  \end{clojure}
\end{english}

Исследуйте отчёт \verb|s/explain| для этих команд и подумайте, как составить
текст ошибки.

\subsubsection*{Другие операторы}

Парсинг легко дополнить новым оператором ветвления. До сих пор мы игнорировали
тег \verb|for| для цикла, и теперь добавим его в три шага.

Во-первых, убедимся, что он входит в множество \texttt{flow-tags}, чтобы
\verb|for| не считался обычной командой \page{flow-tags}.

Во-вторых, расширим главную спеку \texttt{::flow}. Теперь её элемент либо
\texttt{if}, либо \texttt{for}, либо обычная команда:

\begin{english}
  \begin{clojure}
(s/def ::flow
  (s/+
   (s/alt :if ::flow-if
          :for ::flow-for
          :cmd ::command)))
  \end{clojure}
\end{english}

В-третьих, напишем спеку \texttt{::flow-for} для разбора одноимённой
команды. Она проще \texttt{if}, потому что включает в себя только один блок
команд.

\begin{english}
  \begin{clojure}
(s/def ::flow-for
  (s/cat :this (tag= "for")
         :flow ::flow
         :end (tag= "end")))
  \end{clojure}
\end{english}

Данные с циклом, в теле которого условие:

\begin{english}
  \begin{clojure}
[{:command "for"}
   {:command "if" :condition "..."}
     {:command "print" :text "hello"}
   {:command "end"}
 {:command "end"}]
  \end{clojure}
\end{english}

Результат разбора:

\begin{english}
  \begin{clojure}
[[:for
  {:this {:command "for"}
   :flow
   [[:if
     {:this {:command "if" :condition "..."}
      :flow [[:cmd {:command "print" :text "hello"}]]
      :end {:command "end"}}]]
   :end {:command "end"}}]]
  \end{clojure}
\end{english}

По аналогии легко добавить операторы \texttt{repeat} или \verb|when|, которые
выполняют блок кода несколько раз или по условию.

Подумайте, как составить спеку для оператора \texttt{switch}. Он принимает один
и более блоков \verb|case|; внутри каждого блока~--- произвольные команды,
которые выполняются, если управление переходит на эту ветку.

\begin{english}
  \begin{clojure}
[{:command "switch" :value "..."}
   {:command "case" :value 1}
     {:command "print" :text "one"}
   {:command "end"}
     {:command "case" :value 2}
   {:command "print" :text "two"}
     {:command "end"}
 {:command "end"}]
  \end{clojure}
\end{english}

Пока что не отвлекайтесь на значения полей \verb|:value|. Сначала добейтесь,
чтобы список выше обрёл форму, удобную для будущей обработки.

Мы проделали большую работу~--- из плоского списка команд получили структуру с
учётом вложенности. Интересно, для этого понадобился только один модуль из
поставки Clojure и менее ста строк кода. Система проста и может быть легко
расширена новыми тегами.

Интерпретация кода~--- крайне интересная тема. Она открывает неожиданные
решения, например языки сценариев, описание бизнес процессов и многое
другое. Сейчас мы не будем рассматривать, как выполнить полученную
структуру. Это сложный вопрос, который уведёт нас от текущего
обсуждения. Заинтересованному читателю советуем исследовать код модуля
\verb|flow.clj| из библиотеки Etaoin.

\section{Разбор Clojure-кода (теория)}

\index{парсинг}
\index{макросы}

В завершение темы поговорим о том, как парсить код. Мы уже видели, что Spec
подходит для разбора коллекций. Код на Clojure состоит из списков. Это приводит
к необычному подходу: код можно проверить спекой и вернуть ошибку до того, как
он запущен.

Проверять код удобно в макросах. Это особые функции, которые работают на этапе
компиляции. Макрос принимает код в виде списка символов. В коде могут быть
ошибки, но макрос об этом ничего не знает: для него это просто символы.

Задача макроса в том, чтобы перестроить список в другой, понятный
Clojure. Компилятор заменит вызов макроса на то, что он вернул, и запустит
код. Макросы~--- это отдельная веха в изучении Clojure, и сейчас мы не будем
разбирать их с азов. Пока что рассмотрим, как проверить макрос спекой.

Каждый макрос~--- это мини-язык с соглашением о том, что подавать на вход. В
простых случаях его тело парсят функциями \verb|first|, \verb|rest| и
условиями. Сложные макросы разбирают грамматиками, как мы делали это с командами
Selenium IDE. Если код нарушает правила, мы должны объяснить, в чём ошибка.

Иногда один и тот же макрос допускает разные записи. Хорошим примером служит
\verb|defn|~--- определение функции. Кроме обязательных параметров он принимает
дополнительные: строку документации, пре- и постпроверки. У функции может быть
несколько тел:

\ifnarrow

\begin{english}
  \begin{clojure}
(defn my-inc
  [x]
  (+ x 1))
  \end{clojure}

\splitter

  \begin{clojure}
(defn my-inc
  "Increase a number."
  [x]
  {:pre [(int? x)]
   :post [(int? %)]}
  (+ x 1))
  \end{clojure}

\splitter

  \begin{clojure}
(defn my-inc
  ([x]
   (my-inc x 1))
  ([x delta]
   (+ x delta)))
  \end{clojure}
\end{english}

\else

\begin{english}

\noindent
\begin{tabular}{ @{}p{2.5cm} @{}p{4.5cm} @{}p{3cm} }

  \begin{clojure}
(defn my-inc
  [x]
  (+ x 1))
  \end{clojure}

&

  \begin{clojure}
(defn my-inc
  "Increase a number."
  [x]
  {:pre [(int? x)]
   :post [(int? %)]}
  (+ x 1))
  \end{clojure}

&

  \begin{clojure}
(defn my-inc
  ([x]
   (my-inc x 1))
  ([x delta]
   (+ x delta)))
  \end{clojure}

\end{tabular}

\end{english}

\fi

Это одна и та же функция, записанная по-разному. Очевидно, разобрать все
варианты вручную тяжело. До версии Clojure 1.10 каждый макрос парсил код на свой
манер. Это было ситуативно и неорганизованно. С выходом Spec основные макросы
перешли на спеку. Появился общий подход, которым легко управлять.

\index{spec!cat}

Разберём устно, как бы мы построили спеку для разбора \verb|defn|. Это список,
поэтому на верхнем уровне поместим \verb|s/cat|. Первый его элемент~--- символ
\verb|defn|, второй~--- символ с именем функции. После имени следует строка
документации (её может и не быть). Далее~--- тело или список тел. Набросаем
черновик:

\begin{english}
  \begin{clojure}
(s/def ::defn
  (s/cat :tag #{'defn}
         :name symbol?
         :doc (s/? string?)
         :body ...))
  \end{clojure}
\end{english}

Что скрывается за ключом \verb|:body|, пока неизвестно. Проблема в том, что
\verb|defn| принимает либо одно тело, либо их список. Сравните первый и третий
столбики в примере с \verb|my-inc| выше.

Для начала напишем спеку обычного тела. Считаем, что тело начинается с вектора
параметров. После него идёт опциональный словарь пре- и постпроверок. Затем
произвольные формы, из которых состоит тело функции.

\begin{english}
  \begin{clojure}
(s/def :defn/body
  (s/cat :args vector?
         :prepost (s/? map?)
         :code (s/* any?)))
  \end{clojure}
\end{english}

На базе \verb|:defn/body| построим такую спеку, которая учитывает оба случая:
одно тело и несколько. Обозначим её звёздочкой: \verb|:defn/body*|. Вот её
реализация:

\ifnarrow

\begin{english}
  \begin{clojure}
(s/def :defn/body*
  (s/alt :single :defn/body
         :multi (s/+
                  (s/spec :defn/body))))
  \end{clojure}
\end{english}

\else

\begin{english}
  \begin{clojure}
(s/def :defn/body*
  (s/alt :single :defn/body
         :multi (s/+ (s/spec :defn/body))))
  \end{clojure}
\end{english}

\fi

\index{синтаксис!\textquotesingle{ }(заморозка)}

Теперь подготовим данные для спеки. Это список, который повторяет выражение
\verb|defn|. Он состоит из примитивов~--- символов, чисел, строк~--- и вложенных
коллекций. Список можно составить комбинацией \verb|list| и~экранированных
символов. Выражение \verb|'something| не вычисляет символ за апострофом,
а~только возвращает его. Эту запись называют квотированием \eng{quote} или
заморозкой. На жаргоне программистов говорят <<заквотить символ>>.

\begin{english}
  \begin{clojure}
(def defn-expr
  (list 'defn 'my-inc
        "Increase a number"
        ['x]
        {:pre [(list 'int? 'x)]} ...))
  \end{clojure}
\end{english}

Если выполнить \verb|defn-expr| в REPL, увидим код в том виде, в каком он
выглядел бы в редакторе. Однако из-за частых \verb|list| и апострофов код
получился шумный, и в нём легко допустить ошибку. Для краткости заморозим не
отдельные символы, а выражение \verb|defn| целиком. Так мы сохраним оригинальный
синтаксис:

\begin{english}
  \begin{clojure/lines}
(def defn-expr
  '(defn my-inc
     "Increase a number"
     [x]
     {:pre [(int? x)]
      :post [(int? %)]}
     (+ x 1)))
  \end{clojure/lines}
\end{english}

Обратите внимание на апостроф во второй строке: именно он удерживает \verb|defn|
от выполнения. В результате мы не объявим функцию, а только получим её код.

Заметим, что во внутреннем \verb|defn| могут быть какие угодно ошибки, но мы
о~них не узнаем. С точки зрения Clojure это список символов без всякой
семантики. Для этого мы и пишем спеку~--- убедиться, что у~списка правильная
структура. Распарсим его:

\begin{english}
  \begin{clojure}
(s/conform ::defn defn-expr)
  \end{clojure}
\end{english}

\noindent
Результат:

\ifnarrow

\begin{english}
  \begin{clojure}
{:tag defn
 :name my-inc
 :doc "Increase a number"
 :body
 [:single
  {:args [x]
   :prepost {:pre [(int? x)]
             :post [(int? %)]}
   :code [(+ x 1)]}]}
  \end{clojure}
\end{english}

\else

\begin{english}
  \begin{clojure}
{:tag defn
 :name my-inc
 :doc "Increase a number"
 :body
 [:single
  {:args [x]
   :prepost {:pre [(int? x)] :post [(int? %)]}
   :code [(+ x 1)]}]}
  \end{clojure}
\end{english}

\fi

Обратите внимание на поле \verb|:body|. Это вектор из метки и результата. Для
одного тела получим метку \verb|:single| и словарь. Для нескольких тел метка
будет \verb|:multi|, а значение~--- вектор словарей:

\ifnarrow

\begin{english}
  \begin{clojure}
[:multi [{:args [x]
          :code [(println 1)]}
         {:args [x y]
          :code [(println 2)]}]]
  \end{clojure}
\end{english}

\else

\begin{english}
  \begin{clojure}
[:multi [{:args [x] :code [(println 1)]}
         {:args [x y] :code [(println 2)]}]]
  \end{clojure}
\end{english}

\fi

Чтобы проверить метку (одно тело или несколько), пригодится оператор
\verb|case|. Ниже в переменой \verb|result| записан результат парсинга. Функция
\verb|process-body| обрабатывает словарь тела в зависимости от единичного и
множественного случаев:

\begin{english}
  \begin{clojure}
(let [{:keys [body]} result
      [tag body] body]
  (case tag
    :single
    (process-body body)
    :multi
    (doseq [body body]
      (process-body body))))
  \end{clojure}
\end{english}

\subsection{Самостоятельная работа}

\index{сообщения}

Каждый уровень спеки расширяется вглубь. Доработаем аргументы функции: разделим
их на обязательные и остаточные. Например, чтобы параметры \verb|[x y & other]|
предстали в виде словаря:

\begin{english}
  \begin{clojure}
{:req [x y] :opt other}
  \end{clojure}
\end{english}

\noindent
По аналогии разберите словари пре- и постпроверок.

Передайте в спеку данные с ошибками. Что делать в таком случае? Как составить
\emph{понятное} сообщение о том, где именно ошибка и чего вы ожидали? Подойдёт ли
словарь переводов? Получится ли у вас сделать ошибки лучше, чем в промышленных
языках?

Данные, которые вернула спека-парсер, называются \emph{абстрактным синтаксическим деревом}
(англ.~abstract syntax tree, \footurl{AST}{https://en.wikipedia.org/wiki/Abstract\_syntax\_tree}[AST][-3mm]).
Это вложенная структура, которую получают из текста. AST~--- важный этап в~исполнении
программы. Только построив дерево, можно выполнить логику, которая за ним стоит.

\index{интерпретатор}

Spec работает в том числе как парсер грамматик: можно разобрать данные и
построить дерево. Вы в шаге от того, чтобы написать простой интерпретатор
Clojure~--- программу, которая читает список и выполняет его.

За подробностями обратитесь к прошлому разделу, где мы рассматривали
интерпретатор Selenium IDE \page{selenium-ide}. Реализация языка на самом языке
станет хорошей темой для курсовой работы или даже диплома.

\section{Спецификация функций}

\index{spec!функции}

Проблема валидации касается и функций. Они получают данные от посредников, и
всегда есть шанс, что в них ошибка. Вспомним функцию с диапазоном дат. Если
поменять их местами (начало больше конца), проверка на интервал вернёт
ложь. Здесь и пригодится спека.

Опишем параметры функции. Это \verb|s/cat|, которая принимает список аргументов
и возвращает словарь. Назначим аргументам функции метки \verb|:start| и
\verb|:end|:

\ifnarrow

\begin{english}
  \begin{clojure}
(s/def ::date-range-args
  (s/and
    ;; parse args
    (s/cat :start inst? :end inst?)
    ;; compare args
    (fn [{:keys [start end]}]
      (<= (compare start end) 0))))
  \end{clojure}
\end{english}

\else

\begin{english}
  \begin{clojure}
(s/def ::date-range-args
  (s/and
   (s/cat :start inst? :end inst?) ;; parse args
   (fn [{:keys [start end]}]       ;; compare args
     (<= (compare start end) 0))))
  \end{clojure}
\end{english}

\fi

\index{clojure.core!compare}

Вторая функция в \verb|s/and| принимает словарь и сравнивает даты. Для сравнения
дат используют функцию \verb|compare|, которая вернёт \mbox{-1}, 0 и 1 для
случаев <<меньше>>, <<равно>> и <<больше>>. Быстрая проверка:

\ifnarrow

\begin{english}
  \begin{clojure}
(s/valid? ::date-range-args
          [#inst "2019" #inst "2020"])
;; true

(s/valid? ::date-range-args
          [#inst "2020" #inst "2019"])
;; false
  \end{clojure}
\end{english}

\else

\begin{english}
  \begin{clojure}
(s/valid? ::date-range-args
          [#inst "2019" #inst "2020"]) ;; true

(s/valid? ::date-range-args
          [#inst "2020" #inst "2019"]) ;; false
  \end{clojure}
\end{english}

\fi

\index{декораторы}

Напрашивается декоратор, который принимает функцию и спеку. Перед тем как
запустить функцию, он проверяет аргументы и в случае ошибки бросит исключение.
Если задана спека результата, проверим его тоже перед тем как вернуть из
декоратора.

\index{модули!clojure.spec.test.alpha}
\index{функции!instrument}

Писать декоратор не нужно, потому что его включили в Spec. Это функция
\verb|instrument| из модуля \verb|clojure.spec.test.alpha|
(англ.~instru\-ment~--- <<оснастить, оборудовать>>). Обратите внимание на
пространство: в него закралась частичка test. Оснащение функций вынесли в
отдельный модуль.

\index{monkey patch}

\verb|Instrument| принимает символ функции, которую <<заряжают>>. С тем же
символом определяют спеку этой функции. \verb|Instrument| подменяет функцию на
такую же, но с проверками. Происходит своего рода \footurl{monkey
  patch}{https://en.wikipedia.org/wiki/Monkey\_patch}[Monkey patch][-7mm], когда
один модуль изменяет другой.

Чтобы задать спеку функции, используют макрос \verb|s/fdef|. Ему передают символ
функции и спеки входящих параметров, результата и их композиции.

\index{даты}
\index{классы!Date}

Напишем функцию и спеку к ней. Пусть функция считает разницу между датами в
секундах. Если первая дата больше второй, результат отрицательный.

\ifnarrow

\begin{english}
  \begin{clojure}
(import 'java.util.Date)

(defn date-range-sec
  "Return the difference
           between two dates in seconds."
  [^Date date1 ^Date date2]
  (quot (- (.getTime date2)
           (.getTime date1))
        1000))
  \end{clojure}
\end{english}

\else

\begin{english}
  \begin{clojure}
(import 'java.util.Date)

(defn date-range-sec
  "Return the difference between two dates in seconds."
  [^Date date1 ^Date date2]
  (quot (- (.getTime date2)
           (.getTime date1))
        1000))
  \end{clojure}
\end{english}

\fi

Теги \verb|^Date| нужны, чтобы компилятор знал тип объектов \verb|date1| и
\verb|date2|. Если тега нет, Clojure ищет тип во время исполнения с помощью
рефлексии, что медленней. Посчитаем разницу в секундах между двумя сутками:

\ifnarrow

\begin{english}
  \begin{clojure}
(date-range-sec #inst "2019-01-01"
                #inst "2019-01-02")
86400
  \end{clojure}
\end{english}

\else

\begin{english}
  \begin{clojure}
(date-range-sec #inst "2019-01-01" #inst "2019-01-02")
86400
  \end{clojure}
\end{english}

\fi

\noindent
Если поменять даты местами, получим то же число с минусом.

Опишем функциональную спеку с символом \verb|date-range-sec|. В~ключе
\verb|:args| указывают спеку входящих параметров. Это список, поэтому пригодится
\verb|s/cat|. Он разбивает список на словарь, чтобы спеки ниже работали с
ключами.

Ключ \verb|:ret| означает спеку результата. Чаще всего это проверка на число или
строку: \verb|int?|, \verb|string?| или их \verb|nilable|-версии, которые
допускают \verb|nil|: \verb|(s/nilable int?)| и так далее.

\index{spec!fdef}

\begin{english}
  \begin{clojure}
(s/fdef date-range-sec
  :args (s/cat :start inst? :end inst?)
  :ret int?)
  \end{clojure}
\end{english}

В ключи \verb|:args| и \verb|:ret| можно передать готовые спеки, что полезно
для повторного использования. У вас может быть несколько спек для диапазона дат.

При создании функциональная спека не меняет функцию. Она только объявляет
проверки, но не запускает их. Чтобы подменить функцию на её оснащенную версию,
вызывают \verb|instrument|:

\ifnarrow

\begin{english}
  \begin{clojure}
(require '[clojure.spec.test.alpha
           :refer [instrument]])
(instrument `date-range-sec)
  \end{clojure}
\end{english}

\else

\begin{english}
  \begin{clojure}
(require '[clojure.spec.test.alpha :refer [instrument]])
(instrument `date-range-sec)
  \end{clojure}
\end{english}

\fi

\index{синтаксис!\textasciigrave{ }(заморозка)}

Символ функции должен быть полным (с пространством). Чтобы подставить в символ
текущее пространство, перед ним ставят обратную кавычку \verb|`| по аналогии с
\verb|::| для кейворда.

Теперь \verb|date-range-sec| проверит аргументы и результат. Что случится,
если передать в <<заряженную>> функцию не тот аргумент? Получим исключение
\verb|ExceptionInfo|:

\ifnarrow

\begin{english}
  \begin{clojure}
(date-range-sec nil #inst "2019")
;; Execution error - invalid arguments
;; to date-range-sec
;; nil - failed: inst? at: [:start]
  \end{clojure}
\end{english}

\else

\begin{english}
  \begin{clojure}
(date-range-sec nil #inst "2019")
;; Execution error - invalid arguments to date-range-sec
;; nil - failed: inst? at: [:start]
  \end{clojure}
\end{english}

\fi

\index{spec!explain}
\index{отчёт!explain}

Сообщение и тело нам знакомы. В поле \verb|message| находится текст из функции
\texttt{s/ex\-plain-\-str}, а в \verb|data|~--- отчёт из \texttt{s/ex\-plain-\-da\-ta}.
Чтобы добраться до него, передайте исключение в функцию \verb|(ex-data e)|. Всё,
что мы обсудили в разделе сообщений об ошибках, справедливо и в этом случае.

\subsection{Документация}

\index{документация}
\index{модули!clojure.repl}

Функциональная спека улучшает документацию. Функция \verb|doc| из модуля
\verb|clojure.repl| выводит справку о запрошенной функции. С появлением Spec её
поведение изменилось: кроме документации она печатает одноимённую спеку, если её
задали. Вот как выглядит справка для \verb|date-range-sec| после того, как
объявили спеку:

\ifnarrow

\begin{english}
  \begin{clojure}
(clojure.repl/doc date-range-sec)
-------------------------
([date1 date2])
  Return the difference
            between two dates in seconds.
Spec
  args: (cat :start inst? :end inst?)
  ret: int?
  \end{clojure}
\end{english}

\else

\begin{english}
  \begin{clojure}
(clojure.repl/doc date-range-sec)
-------------------------
([date1 date2])
  Return the difference between two dates in seconds.
Spec
  args: (cat :start inst? :end inst?)
  ret: int?
  \end{clojure}
\end{english}

\fi

На функцию \verb|doc| полагаются IDE и редакторы, чтобы подсказывать аргументы в
коде. Утилиты для сборки документации добавляют спеки в HTML-файлы.

\subsection{Производительность}

\index{производительность}

\verb|Instrument| помогает в тестировании. На время тестов функции <<заряжают>>,
чтобы выявить ошибки. Это делают с помощью модуля, который выполняет серию
\verb|(instrument ...)| при загрузке. Проект устроен так, что модуль загружается
только на время тестов. Если функция получила не те аргументы, что мы ожидали,
это станет заметно.

\index{бенчмарк}

\verb|Instrument| не подходит для боевого режима, потому что замедляет
код. Напишем простой замер, который вызывает функцию много раз:

\index{clojure.core!time}

\ifnarrow

\begin{english}
  \begin{clojure}
(time
 (dotimes [n 10000]
   (date-range-sec #inst "2019"
                   #inst "2020")))
;; Elapsed time: 1.783962 msecs
  \end{clojure}
\end{english}

\else

\begin{english}
  \begin{clojure}
(time
 (dotimes [n 10000]
   (date-range-sec #inst "2019" #inst "2020")))
;; Elapsed time: 1.783962 msecs
  \end{clojure}
\end{english}

\fi

\noindent
То же самое после оснащения:

\ifnarrow

\begin{english}
  \begin{clojure}
(time
 (dotimes [n 10000]
   (date-range-sec #inst "2019"
                   #inst "2020")))
;; Elapsed time: 116.984496 msecs
  \end{clojure}
\end{english}

\else

\begin{english}
  \begin{clojure}
(time
 (dotimes [n 10000]
   (date-range-sec #inst "2019" #inst "2020")))
;; Elapsed time: 116.984496 msecs
  \end{clojure}
\end{english}

\fi

\index{тесты}

Разница во времени в сто раз! Проверка \emph{существенно} замедляет приложение. Вот почему
\verb|instrument| не претендует на запуск в бою: это слишком дорогая
цена. Однако в тестах нас не волнует скорость. На время прогона код покрывают
как можно б\'{о}льшим числом проверок, чтобы поймать все ошибки.

\section{Повторное использование спек}

\index{повторное использование}
\index{библиотеки!JDBC}

В Clojure принято снабжать библиотеки спеками, чтобы помочь другим
разработчикам. Если библиотека активно работает с какой-то структурой данных,
опишите её спекой. Хорошим примером служит
\footurl{JDBC}{https://github.com/clojure/java.jdbc}[JDBC][-7mm], обёртка для реляционных
баз данных.

Подключение к базе задают словарём с ключами \verb|:host|, \verb|:port|,
\verb|:user| и другими. Обычно словарь читают из конфигурации. Перед
подключением его нужно проверить, чтобы не получить \verb|NPE| и другие
странности.

\index{NPE}

Предположим, конфигурация описана в ресурсе EDN. Поле \verb|:db| задаёт
подключение к базе.

\index{форматы!EDN}

\begin{english}
  \begin{clojure}
{:db {:dbtype "mysql"
      :host "127.0.0.1"
      :port 3306
      :dbname "project"
      :user "user"}}
  \end{clojure}
\end{english}

Ключи \verb|:dbtype|, \verb|:host| и другие уже описаны в библиотеке. Один из её
модулей несёт спеки подключения и основных функций. Используем их повторно:

\ifnarrow

\begin{english}
  \begin{clojure}
(require
  '[clojure.java.jdbc.spec :as jdbc])

(s/def ::db ::jdbc/db-spec)
(s/def ::config (s/keys :req-un [::db]))
  \end{clojure}
\end{english}

\else

\begin{english}
  \begin{clojure}
(require '[clojure.java.jdbc.spec :as jdbc])

(s/def ::db ::jdbc/db-spec)
(s/def ::config (s/keys :req-un [::db]))
  \end{clojure}
\end{english}

\fi

\noindent
Прочитаем конфигурацию из файла и проверим спекой:

\ifnarrow

\begin{english}
  \begin{clojure}
(def config
  (read-string (slurp "config.edn")))
(s/valid? ::config config)
  \end{clojure}
\end{english}

\else

\begin{english}
  \begin{clojure}
(def config (read-string (slurp "config.edn")))
(s/valid? ::config config)
  \end{clojure}
\end{english}

\fi

\label{jdbc-conform-warning}

\index{spec!or}

Будьте внимательны с \verb|::jdbc/db-spec|: это сложная спека с макросами
\verb|s/or|, которые порождают развилки. Если передать её в
\verb|s/conform|, получим поля с тегами (строки 3, 4, 6):

\begin{english}
  \begin{clojure/lines}
(s/conform ::config config)
{:db
 [:friendly
  {:dbtype   [:name "mysql"]
   :host     "127.0.0.1"
   :port     [:port 3306]
   :dbname   "project"
   :user     "user"
   :password "********"}]}
  \end{clojure/lines}
\end{english}

Такой результат не работает с \verb|query| и другими функциями JDBC, которые
ожидают первым параметром словарь подключения. Мы ограничились \verb|s/valid?|,
чтобы оставить структуру прежней.

Иногда спеки выносят в отдельную библиотеку. Так поступили разработчики
\footurl{Alia}{https://github.com/mpenet/alia}[Alia][-2mm]~--- клиента для БД
Cassandra. Проект \verb|qbits.alia| содержит основные функции для работы с
базой. Спеки идут в пакете \verb|cc.qbits/alia-spec|, который подключают
отдельно.

\section{Дополнения}

Spec входит в поставку Clojure и поэтому меняется не так часто, как хотелось бы
разработчикам. Дополнения к Spec выпускают в виде библиотек. Среди прочих
заслуживают внимания Expound, Spec-tools и Spec-dict. Коротко опишем возможности каждой.

\subsection{Expound}

\label{expound}
\index{библиотеки!Expound}

Библиотека \footurl{Expound}{https://github.com/bhb/expound}[Expound] улучшает сообщения об
ошибках. Сигнатура функции \verb|expound| аналогична \verb|s/explain|: она
тоже принимает спеку и~данные. Новое сообщение об ошибке выглядит так:

\index{spec!ошибки}
\index{сообщения}

\begin{english}
  \begin{clojure}
(expound/expound string? 1)
-- Spec failed --------------------
  1
should satisfy
  string?
-------------------------
Detected 1 error
  \end{clojure}
\end{english}

\index{spec!explain}

Это всё ещё машинный текст, который нельзя показывать пользователю. Всё же он
лучше, чем сырой \verb|s/explain|: его прочтут инженеры, которые не знают
Clojure. Expound подойдёт для проверки конфигурации на старте
приложения. Конфигурацию обновляют часто, поэтому внятный отчёт об ошибке
полезен.

\index{организации!Metosin}
\index{библиотеки!Spec.tools}
\index{веб-разработка!Swagger}
\index{веб-разработка!REST}

\subsection{Spec-tools}

Разработчики Metosin собрали улучшения для Spec в проекте
\footurl{Spec-tools}{https://github.com/metosin/spec-tools}[Spec-tools]. Его
основная задача~--- обеспечить интеграцию спеки с REST API, а именно
предоставить клиентам JSON-схемы и описание API по стандарту Swagger. По
умолчанию Clojure ничего не знает о JSON-схемах, а современный веб~--- о
спеке. Если вы проверяете параметры API спекой, то чтобы показать её клиенту в
виде схемы, понадобится конвертер.

Spec-tools закрывает эту проблему. Добавьте библиотеку в проект:

\begin{english}
  \begin{clojure}
[metosin/spec-tools "0.10.4"]
  \end{clojure}
\end{english}

Объявим простую спеку пользователя:

\begin{english}
  \begin{clojure}
(s/def :user/name ::ne-string)
(s/def :user/age pos-int?)

(s/def :user/user
  (s/keys :req-un [:user/name :user/age]))
  \end{clojure}
\end{english}

Чтобы получить из спеки JSON-схему, передайте её в функцию \code{transform}:

\begin{english}
  \begin{clojure}
(require '[spec-tools.json-schema :as json-schema])

(json-schema/transform :user/user)
  \end{clojure}
\end{english}

В результате получим словарь:

\begin{english}
  \begin{clojure/lines}
{:type "object"
 :properties
 {"name" {:type "string"}
  "age" {:type "integer"
         :format "int64"
         :minimum 1}}
 :required ["name" "age"]
 :title "user/user"}
  \end{clojure/lines}
\end{english}

Обратите внимание, что предикату \verb|pos-int?| соответствует минимальное
значение~--- единица \coderef{6}. Spec-tools учитывает все встроенные предикаты
(\verb|string?|, \verb|int?|, \verb|boolean?|) и коллекции (\verb|s/coll-of|,
\verb|s/keys|). Функция \verb|transform| принимает ряд опций, чтобы повлиять на
генерацию схемы. Для особых случаев расширьте протокол \verb|Transformer|,
который отвечает за отображение спеки в форматы JSON и EDN.

Будет полезно изучить и другие возможности Spec-tools. Ищите их описание в
разделе Wiki на странице проекта.

\subsection{Spec-dict}

Наверняка читатель заметил, что описание словарей на языке спеки слегка
избыточно. Например, чтобы задать пользователя с именем и возрастом, нужно
сперва определить имя и возраст (спеки \verb|:user/name| и \verb|:user/age|), а
затем сослаться на них в \verb|s/keys|. Причина кроется в том, как устроен
макрос \verb|s/keys|~--- он принимает только ключи спек, но не предикаты. Это
значит, не получится описать спеку как в примере ниже:

\begin{english}
  \begin{clojure}
(s/def ::user
  {:name string?
   :age pos-int?})
  \end{clojure}
\end{english}

Библиотека \footurl{Spec-dict}{https://github.com/igrishaev/spec-dict}[Spec-dict]
разрешает такую запись. Функция \verb|dict| принимает словарь, где значения~--- предикаты
или другие спеки. Её результат~--- спека, которая проверяет ключи и значения
по заданным предикатам. Спеки-словари могут быть вложенными. Вот как задать
спеку записи в блог, где автор~--- вложенное поле:

\begin{english}
  \begin{clojure}
[spec-dict "0.2.1"] ;; project.clj

(require '[spec-dict :refer [dict dict*]])

(s/def ::post
  (dict {:title string?
         :author (dict {:name string?
                        :email string?})}))
  \end{clojure}
\end{english}

Новый синтаксис значительно сокращает описание словарей. Пеерчислим другие
особенности библиотеки.

\begin{itemize}

\item
  Ключи словаря могут быть любого типа, например строкой, числом или датой:

\begin{english}
  \begin{clojure}
(s/def ::data (dict {"name" string?}))

(s/valid? ::data {"name" "Test"})
;; true
  \end{clojure}
\end{english}

\item
  В \verb|dict| можно передать несколько словарей, и они сольются в один. Если
  предварить словарь синтаксисом \verb|^:opt|, его ключи станут необязательными:

\begin{english}
  \begin{clojure}
(dict {:name string?} ^:opt {:age int?})
  \end{clojure}
\end{english}

\item
  Спека, полученная из функции \verb|dict|, поддерживает \verb|conform| и \verb|unform|,
  генерацию данных, отчёт \verb|explain|.

\item
  Функция \verb|dict*| со звёздочкой дополнительно проверяет, что в словаре нет
  лишних ключей. По умолчанию \verb|s/keys| не предлагает такой проверки, хотя
  порой это критически важно. В примере ниже данные не пройдут валидацию из-за
  ключа \verb|:extra|, который не предусмотрен спекой:

\begin{english}
  \begin{clojure}
(s/def ::user-strict
  (dict* {:name string? :age pos-int?}))

(s/valid? ::user-strict
          {:name "test" :age 30 :extra 1})
;; false
  \end{clojure}
\end{english}

\end{itemize}

\subsection{Spec~2}

На сегодняшний день пакет Spec всё ещё не избавился от частички alpha в
названии. Трудно сказать, когда спеку официально признают стабильной, и
произойдёт ли это вообще. В любом случае, статус альфы не должен останавливать
вас от использования спеки. Полагаем, в этой главе мы сказали достаточно, чтобы
убедить вас в её преимуществах.

\index{люди!Рич Хикки}
\index{видео!Maybe Not}

В докладе \footurl{Maybe Not}{https://youtube.com/watch?v=YR5WdGrpoug}[Maybe\\*Not]
от 2018 года Рич Хикки анонсировал вторую версию Spec. В ней упростят работу со
словарями (спека \verb|select|) и множественными типами (когда значение~---
строка или число). Обсуждение второй спеки выходит за рамки главы. Предлагаем
читателю исследовать Spec~2 самостоятельно.

\section{Заключение}

Spec~--- это библиотека Clojure, которая идёт в поставке с языком. Spec
предлагает набор функций и макросов. Ими описывают правила, которым подчиняются
данные. Правила~--- это функции, которые возвращают истину или ложь. По-другому
такие функции называют предикатами.

Предикаты гибче и мощнее типов. Если известно, что значение верного типа, это не
гарантирует корректности. Классы вроде \verb|UnixPort| или \verb|DateRange| на
самом деле не типы, а валидация в рантайме. Она привязана к вызову класса с
помощью синтаксиса: оператор \verb|new| неявно вызывает конструктор. Внутри
конструктора параметры проходят проверку, и если что-то не так, получим
исключение.

В отличие от классов, предикаты компонуются друг с другом. Легко составить
предикат с логикой <<каждый из>>, <<любой из>> и другие комбинации.

Библиотека служит в том числе для преобразования данных. Функция
\verb|s/conform| оборачивает другую функцию, которая принимает текущее значение
и возвращает новое. Этим пользуются для вывода чисел и дат из строк.

Spec предлагает regex-спеки, похожие на регулярные выражения. От обычных спек
они отличаются тем, что захватывают часть коллекции. На regex-спеках пишут
парсеры данных, в том числе для самой Clojure. Основные макросы модуля
\verb|clojure.core| проверяют своё тело с помощью Spec.

Библиотека полезна в тестах. Функция \texttt{in\-stru\-ment} изменяет другую
функцию так, что её аргументы проверяются спекой. Это замедляет вызов, но делает
тесты надёжней. Функциональная спека становится частью документации.

Spec не предлагает решения для сообщений об ошибках. Как их~строить и показывать
клиенту, зависит от проекта. Один из подходов основан на словаре, где ключи~---
спеки, а значения~--- текст ошибки или тег перевода. Если заменить словарь на
мультиметод, потребителям будет проще расширить его.

Мы рассмотрели основные возможности библиотеки. В обсуждение не попали
генераторы, мультиспеки и другие интересные вещи. Всё это читатель найдёт в
\footurl{официальной документации}{https://clojure.org/about/spec}[About Spec][-4.5mm].
Мы не прощаемся со спекой~--- нам ещё доведётся поработать с ней в будущих главах.
