\chapter{Clojure.spec}

\begin{teaser}
В этой главе мы рассмотрим clojure.spec~--- библиотеку для проверки данных в
Clojure. Это особенная библиотека: на ней пишут валидаторы и парсеры, с ее
помощью генерируют данные для тестов. Spec фундаментальна по своей природе,
поэтому уделим ей пристальное внимание.
\end{teaser}

Название <<spec>> происходит от specification (анг. спецификация, описание). Это
набор функций и макросов, чтобы схематично описать данные. Например, из каких
ключей состоит словарь и каких типов его значения. Такую запись называют
спецификацией данных или сокращенно спекой. Далее по тексту мы будем
использовать короткий термин.

Специальные функции проверяют, подходят ли данные к спеке. Если нет, мы получим
отчет в каком месте произошла ошибка и почему.

Spec входит в поставку Clojure начиная с версии 1.9. Полностью модуль называется
\spverb|clojure.spec.alpha|. Не волнуйтесь о частичке <<alpha>> на конце
имени. Она осталась по историческим причинам.

Spec стала важной вехой в развитии Clojure. Ключевое свойство spec в том, что
она фундаментальна. Валидация данных это малая часть ее возможностей. Spec не
только проверяет данные, но и преобразует их. Например, на spec легко написать
процессор данных или парсер.

Формально Spec это обычная библиотека. Но ее абстракции оказались настолько
мощны, что Clojure переиспользует их. С версии 1.10 компилятор Clojure
анализирует код с помощью Spec. Так проекты дополняют друг друга.

Начнем описание Spec с валидации данных. Но прежде чем браться за техническую
часть, разберемся с теорией. Вспомним, как связаны между собой классы, типы и
валидация.

\section{Типы и классы}

\label{type-and-pred}

Принято считать, что код на языке со статической типизацией безопаснее, чем с
динамической. Компилятор не позволит сложить число и строку еще до того, как мы
запустим программу. Однако тип переменной~--- это лишь одно из многих
ограничений. Редко случается так, что тип задает все допустимые значения. Чаще
всего вместе с типом учитывают границы, длину, попадание в интервалы и
перечисления. Иногда значения верны по отдельности, но не могут стоять в паре
друг с другом.

Рассмотрим, как выразить в коде сетевой порт. Для операционной системы это число
от 0 до $2^{16}-1$. Целые типы обычно описаны степенями двойки, поэтому найдется
условный \spverb|unsigned int|, который охватит именно этот диапазон. У нулевого
порта особая семантика, и в прикладных программах им не пользуются. Поэтому
правильно считать порт с единицы. Вероятность, что в языке предусмотрен тип от 1
до $2^{16}-1$ крайне мала.

Лучше всего увидеть проблему на диапазоне дат. Единичная дата может быть сколь
угодно разумной, но диапазон накладывает ограничение: дата начала строго меньше
даты конца. Бизнес дополняет: разница в датах не больше недели, обе даты в
рамках текущего месяца.

В ООП знают об этой проблеме и решают ее классами. Программисты пишут классы
\spverb|UnixPort| и \spverb|DateRange|. Условный \spverb|UnixPort| это класс с
одним конструктором. Он принимает целое число и выполняет проверки на
диапазон. Если число выходит за рамки 1\dots$2^{16}-1$, конструктор бросит
исключение. Программист уверен, что создал новый тип. Это неверно~--- классы и
типы не тождественны.

Конструктор это обычный валидатор. Он неявно сработает, когда мы напишем
\spverb|new UnixPort(8080)|. Из-за неявности возникнет иллюзия, что мы создали
тип. На деле это всего лишь класс-обертка, валидация и синтаксический сахар.

В промышленных языках нельзя описать класс так, чтобы выражение \spverb|new UnixPort(-42)|
привело к ошибке компиляции. Найти такую ошибку могут только сторонние утилиты
и плагины IDE.

Код конструктора трудно использовать повторно. Два разработчика написали классы
\spverb|UnixPort| и \spverb|MyPort|. Первый класс проверяет порт на диапазон и
бросает исключение. Выгодно пользоваться этим классом, поскольку он совмещен с
валидацией. Однако сторонняя библиотека принимает \spverb|MyPort|. Возникает
проблема конвертации: нужно извлечь <<сырой>> порт из \spverb|UnixPort| и
передать его в \spverb|MyPort|. Это лишний код и путаница с классами.

Хорошие признаки валидации это независимость и компоновка. \emph{Независимость}
означает, что данные не привязаны к валидации. Нет ничего плохого в том, что
порт это целое число. Пусть библиотека принимает \spverb|integer|, а разработчик
сам решит, как его проверить. У него появится выбор, насколько строгой должна
быть проверка.

\emph{Компоновка} означает, что полезно иметь несколько простых проверок, чтобы
составить из них сложные. Пусть заданы проверки <<это>> и <<то>>, и теперь нужны
их комбинации: <<это \emph{и} то>>, <<это \emph{или} то>>. В идеале компоновка
занимает пару строк и считается тривиальной задачей.

Оба тезиса ложатся на функцию. Это объект, на который действует одна
операция~--- вызов. Функция-валидатор принимает значение и возвращает истину или
ложь. Это ответ на вопрос, было ли значение правильными или нет. Функция это
объект высшего порядка, поэтому другие функции принимают их и порождают
комбинации.

\section{Основы spec}

С багажом рассуждений мы подходим к тому, как работает Spec. Подключим модуль в
текущее пространство:

\begin{english}
  \begin{clojure}
(require '[clojure.spec.alpha :as s])
  \end{clojure}
\end{english}

Синоним \spverb|s| нужен, чтобы избежать конфликтов имен с
\spverb|clojure.core|. Модуль Spec несет макросы \spverb|s/and|, \spverb|s/or| и
другие, у которых ничего общего с обычными \spverb|and| и \spverb|or|. Считается
дурным тоном, если имена одного модуля затеняют другие. Поэтому обращаемся к
Spec через синоним.

Главная операция в Spec~--- задать новую \emph{спеку}:

\begin{english}
  \begin{clojure}
(s/def ::string string?)
  \end{clojure}
\end{english}

Макрос \spverb|s/def| принимает ключ и предикат. Он создал объект спеки из
функции \spverb|string?|. Затем поместил спеку в глобальный реестр с ключом
\spverb|::string|.

Важно понимать, что \spverb|::string|~--- это не спека, а ее псевдоним. Макросы
Spec устроены так, что принимают не объект спеки, а ее ключ. Далее они сами
найдут спеку в реестре. Это удобно, потому что ключи глобальны. В любом месте
можно сослаться на \spverb|::string| без лишних импортов.

Вторым аргументом идет предикат \spverb|string?|. Предикат это функция, которая
возвращает истину или ложь. Функция это не спека, а строительный материал для
нее. Спека оборачивает функцию в особый объект. Технически на него можно
сослаться: функция \spverb|s/get-spec| по ключу спеки возвращает ее
Java-объект. На практике он не пригодится, потому что везде указывают ключи.

\begin{english}
  \begin{clojure}
(s/get-spec ::string)
;; #object[clojure.spec.alpha$reify 0x3e9dde1d]
  \end{clojure}
\end{english}

Спеки хранятся в глобальном реестре под своими ключами. Макрос \spverb|s/def| не
проверяет, была ли уже такая спека. Если была, мы потеряем ее старую версию.

Spec не работает с ключами без пространства, например \spverb|:error| или
\spverb|:message|. Это повышает риск конфликта ключей. Чтобы назначить ключу
текущее пространство, поставьте два двоеточия: \spverb|::error|,
\spverb|::message|.

Самое простое, что можно сделать со спекой~--- проверить, подходит ли ей
значение. Функция \spverb|s/valid?| принимает ключ спеки, значение и возвращает
\spverb|true| или \spverb|false|.

\begin{english}
  \begin{clojure}
(s/valid? ::string 1)      ;; false
(s/valid? ::string "test") ;; true
  \end{clojure}
\end{english}

Пустая строка пройдет валидацию, но чаще всего это не имеет смысла. Пустая
строка в поле <<имя>> или <<заголовок>> означает ошибку. Объявим спеку, которая
дополнительно проверяет, что строка не пустая. Наивный способ это сделать~---
усложнить предикат:

\begin{english}
  \begin{clojure}
(s/def ::ne-string
  (fn [val]
    (and (string? val)
         (not (empty? val)))))
  \end{clojure}
\end{english}

\noindent
Быстрая проверка:

\begin{english}
  \begin{clojure}
(s/valid? ::ne-string "test") ;; true
(s/valid? ::ne-string "")     ;; false
  \end{clojure}
\end{english}

Ключ \spverb|::ne-string| это сокращение от <<\textbf{n}on-\textbf{e}mpty
string>>. Спека встречается в коде часто, поэтому логично сэкономить на ее
имени.

Более изящный способ задать эту спеку~--- объединить предикаты через
\spverb|every-pred|. Это функция, которая принимает предикаты и возвращает
супер-предикат. Он вернет истину только если истинны все предикаты.

\begin{english}
  \begin{clojure}
(s/def ::ne-string
  (every-pred string? not-empty))
  \end{clojure}
\end{english}

Мы собираем новую сущность из базовых. Это удачный способ: он короче и следует
функциональному стилю. Но еще лучше комбинировать не предикаты, а спеки. Макрос
\spverb|s/and| объединяет несколько предикатов и спек в новую спеку:

\begin{english}
  \begin{clojure}
(s/def ::ne-string
  (s/and ::string not-empty))
  \end{clojure}
\end{english}

Так в Clojure строят сложные спеки: объявляют примитивы и наращивают их
комбинации.

\section{Исключения}

Во время проверки Spec не перехватывает исключения; о них заботится
программист. Рассмотрим спеку для проверки URL. Проще всего это сделать
регулярным выражением:

\begin{english}
  \begin{clojure}
(s/def ::url
  (partial re-matches #"(?i)^http(s?)://.*"))

(s/valid? ::url "test")            ;; false
(s/valid? ::url "http://test.com") ;; true
  \end{clojure}
\end{english}

\noindent
Что-то отличное от строки вызовет ошибку:

\begin{english}
  \begin{clojure}
(s/valid? ::url nil)
;; Execution error (NullPointerException)
;; at java.util.regex.Matcher...
  \end{clojure}
\end{english}

Причина в том, что \spverb|nil| попал в функцию \spverb|re-matches|. Функция
трактует аргумент как строку, что приводит к \spverb|NPE|. Пишите спеки так,
чтобы они не бросали исключения. В примере с \spverb|::url| сначала убедимся,
что это строка, и только потом проверим на регулярное выражение.

\begin{english}
  \begin{clojure}
(s/def ::url
  (s/and ::ne-string
         (partial re-matches #"(?i)^http(s?)://.*")))

(s/valid? ::url nil) ;; false
  \end{clojure}
\end{english}

Макрос \spverb|s/and| устроен так, что на первой неудаче цепь обрывается. Все,
что следует после \spverb|::ne-string| не сработает. Теперь \spverb|nil| не
спровоцирует исключение.

По аналогии проверим возраст пользователя. Это два предиката на число и
диапазон.

\begin{english}
  \begin{clojure}
(s/def ::age
  (s/and int? #(<= 0 % 150)))

(s/valid? ::age nil) ;; false
(s/valid? ::age -1)  ;; false
(s/valid? ::age 42)  ;; true
  \end{clojure}
\end{english}

\section{Спеки-коллекции}

Выше мы проверяли примитивные типы или \emph{скаляры}. Это удобно для примеров,
но редко встречается на практике. Чаще проверяют не скаляры, а коллекции. Spec
предлагает макросы, чтобы задать спеки-коллекции из примитивов.

Макрос \spverb|s/coll-of| принимает предикат или ключ и возвращает
спеку-коллекцию. Она проверяет, что каждый элемент проходит валидацию. Вот так
мы определим список URL:

\begin{english}
  \begin{clojure}
(s/def ::url-list (s/coll-of ::url))
  \end{clojure}
\end{english}

\noindent
Быстрая проверка:

\begin{english}
  \begin{clojure}
(s/valid? ::url-list ["http://test.com" "http://ya.ru"])
;; true

(s/valid? ::url-list ["http://test.com" "dunno.com"])
;; false
  \end{clojure}
\end{english}

Макрос \spverb|s/map-of| проверяет ключи и значения словаря. Вспомним поле
запроса \spverb|:params| \page{ring-params} из главы про
веб-разработку \page{ring-params}. Его ключи кейворды, а значения строки. На
языке спеки это выглядит так:

\begin{english}
  \begin{clojure}
(s/def ::params
  (s/map-of keyword? string?))

(s/valid? ::params {:foo "test"})  ;; true
(s/valid? ::params {"foo" "test"}) ;; false
  \end{clojure}
\end{english}

Отдельно поговорим о словарях. Проверка \spverb|s/map-of| довольно слабая, чтобы
покрыть все варианты. Факт того, что значения строки не несет полезной
информации. Важнее убедиться, что в словаре именно те ключи, что мы ожидаем. К
тому же редко бывает так, что тип значений одинаковый. Наоборот, чаще всего
словарь несет разные сведения о сущности: имя, возраст, дату.

В таких случаях используют макрос \spverb|s/keys|. Он выглядит как список
спек. Ключи спек совпадают с ключами словаря. Значения ключей проверяются
одноименными спеками.

Опишем веб-страницу двумя полями: \spverb|address|, URL-строка и
\spverb|description|, текстовое описание. Объявим примитивы:

\begin{english}
  \begin{clojure}
(s/def :page/address ::url)
(s/def :page/description ::ne-string)
  \end{clojure}
\end{english}

Обратите внимание на пространство ключей. Адрес и описание относятся к странице,
поэтому логично задать им свое пространство. У статьи или книги тоже могут быть
адрес и описание. Пространство обещает, что спеки \spverb|:page/address| и
\spverb|:book/address| не заменят друг друга.

Составим спеку страницы:

\begin{english}
  \begin{clojure}
(s/def ::page
  (s/keys :req-un [:page/address
                   :page/description]))
  \end{clojure}
\end{english}

В параметре \spverb|:req-un| указан вектор спек. Для каждой из них спека ищет
ключ с таким же именем в словаре и проверяет значение. Рассмотрим, что именно
означает \spverb|:req-un| и какие еще параметры принимает \spverb|s/keys|.

Имя \spverb|:req-un| состоит из двух частей: \emph{req} и \emph{un}. Это
признаки наличия ключа и его типа. \emph{Req} (анг. required) означает, что эти
ключи должны быть в словаре. Если хотя бы одного ключа нет, получим
ошибку. Противоположный по смыслу параметр называется \emph{opt}
(анг. optional). В нем указаны ключи, которых может не быть. Валидация таких
ключей происходит, только если они были в словаре.

Частичка \emph{un} означает \emph{unqualified}, неполный ключ. При проверке
\emph{un}-ключей спека не учитывает их пространство. Например, если указать
\spverb|:page/address| в списке \spverb|:req-un|, то в словаре ищется ключ
\spverb|:address|, а не \spverb|:page/address|.

Неполные ключи встречаются в коде часто. Мы получаем данные из чужих API и баз
данных. Эти системы не знают о пространствах имен. Для Clojure пространства
нужны, чтобы разделить одноименные поля у разных сущностей. Например,
\spverb|:user/name| и \spverb|:project/name|. Исключения бывают, когда весь стек
фирмы построен на Clojure. В этом случае клиент и сервер шлют данные с полными
ключами.

Различают следующие комбинации \spverb|req|, \spverb|opt| и \spverb|un|:

\begin{itemize}

\item
  \spverb|:req|~--- необходимые полные ключи,

\item
  \spverb|:req-un|~--- необходимые краткие ключи,

\item
  \spverb|:opt|~--- опциональные полные ключи,

\item
  \spverb|:opt-un|~--- опциональные краткие ключи.

\end{itemize}

У спеки \spverb|::page| для страницы ключи обязательны и не учитывают
пространство. Составим для нее данные \emph{с ошибками}. Это может быть
неправильный адрес, пустое описание, пропавший ключ. Если каждый из словарей
ниже подставить в выражение \spverb|(s/valid? ::params <data>)|, результат будет
ложью.

\begin{english}
  \begin{clojure}
{:address "clojure.org" ;; not a URL
 :description "Clojure Language"}

{:address "https://clojure.org/"
 :description ""} ;; empty string

{:address "https://clojure.org/"} ;; missing key

{:page/address "https://clojure.org/" ;; full keys
 :page/description "Clojure Language"}
  \end{clojure}
\end{english}

Обратите внимание на последний случай. Значения верны, но у ключей пространство
\spverb|:page|. Валидация не сработает, потому что спека ищет \spverb|:address|,
а не \spverb|:page/address|. Чтобы исправить последний пример, замените тип
ключей \spverb|:req-un| на \spverb|:req| (необходимые полные).

\begin{english}
  \begin{clojure}
(s/def ::page-fq
  (s/keys :req [:page/address
                :page/description]))

(s/valid? ::page-fq
          {:page/address "https://clojure.org/"
           :page/description "Clojure Language"})
  \end{clojure}
\end{english}

Комбинированный пример. Добавим странице HTTP-статус, который мы получили при
последнем обращении к ней. Поле опционально, потому что если к странице еще не
обращались, в него нечего записать. Вот как выглядит новая спека:

\begin{english}
  \begin{clojure}
(s/def :page/status int?)

(s/def ::page-status
  (s/keys :req-un [:page/address
                   :page/description]
          :opt-un [:page/status]))
  \end{clojure}
\end{english}

Словари без статуса и с правильным статусом проходят валидацию:

\begin{english}
  \begin{clojure}
(s/valid? ::page-status
          {:address "https://clojure.org/"
           :description "Clojure Language"})

(s/valid? ::page-status
          {:address "https://clojure.org/"
           :description "Clojure Language"
           :status 200})
  \end{clojure}
\end{english}

Заметим, что \spverb|s/keys| различает \spverb|nil| и наличие ключа. Если статус
\spverb|nil|, то он есть в словаре. Сработает проверка \spverb|nil| на
\spverb|int?|, что приведет к ошибке. Тот случай, когда пустое значение не равно
его отсутствию.

\begin{english}
  \begin{clojure}
(s/valid? ::page-status
          {:address "https://clojure.org/"
           :description "Clojure Language"
           :status nil})
;; false
  \end{clojure}
\end{english}

\section{Вывод значений}

\label{spec-conform}

До сих пор мы проверяли данные с помощью \spverb|s/valid?|. Функция вернет
истину или ложь, что значит данные верны или нет. Одной проверки
недостаточно. Бывает так, что данные корректны, но требуется привести их к
нужному типу.

На вход поступило число в виде строки. Мы убедились, что строка состоит из цифр
и не превышает допустимой длины. После валидации значение по-прежнему строка, и
придется парсить его вручную. Код растет и начинает <<шуметь>>. Хотелось бы,
чтобы типы выводил за нас какой-то механизм.

Spec предлагает такие возможности. Это функции \spverb|s/conformer| и
\spverb|s/conform| (анг. conform~--- подчиняться).

\spverb|S/conformer| принимает функцию вывода и оборачивает ее в
спеку-конформер. Функция вывода принимает исходное значение и возвращает либо
новое, либо ключ \spverb|:clojure.spec.alpha/invalid|, что означает ошибку.

Функция \spverb|S/conform| принимает ключ спеки-конформера и данные. Если вывод
прошел без ошибок, результатом будет новое значение. Если с ошибками, вернется
тот же ключ \spverb|::s/invalid|.

Рассмотрим вывод числа из строки. Чтобы различать спеку-конформер от валидатора,
к ее имени добавляют стрелку, что означает приведение типа.

\begin{english}
  \begin{clojure}
(s/def ::->int
  (s/conformer
   (fn [value]
     (try
       (Integer/parseInt value)
       (catch Exception e
         ::s/invalid)))))
  \end{clojure}
\end{english}

Эту спеку передают в \spverb|s/conform| с данными:

\begin{english}
  \begin{clojure}
(s/conform ::->int "42") ;; 42

(s/conform ::->int "dunno")
:clojure.spec.alpha/invalid
  \end{clojure}
\end{english}

Как и \spverb|s/valid?|, \spverb|s/conform| не ловит исключения при
работе. Вместе с тем вывод типов богат на исключения. Будет правильно
перехватывать их и возвращать \spverb|::s/invalid|, как в примере выше.

Обе спеки~--- валидатор и конформер~--- можно объединить через \spverb|s/and|,
чтобы проверить тип перед выводом. В нашем случае убедимся, что значение
строка. Так мы не допустим, чтобы в \spverb|parseInt| попал \spverb|nil| или
что-то другое:

\begin{english}
  \begin{clojure}
(s/def ::->int
  (s/and ::ne-string ::->int))

(s/conform ::->int nil)
:clojure.spec.alpha/invalid
  \end{clojure}
\end{english}

Рассмотрим, как восстановить из строки дату. С этой проблемой знакомы
веб-разработчики. Формат JSON не поддерживает даты, поэтому их передают строкой
или числом секунд. Возникает вопрос, как привести их к объекту на сервере.

Понадобится функция разбора строки и небольшая обвязка, чтобы подружить ее со
спекой. Функция \spverb|read-instant-date| из модуля \spverb|clojure.instant|
читает дату из строки. Она лояльна к формату и учитывает разные
комбинации. Например, датой может быть только год.

\begin{english}
  \begin{clojure}
(require '[clojure.instant :refer [read-instant-date]])
(read-instant-date "2019")
#inst "2019-01-01T00:00:00.000-00:00"
  \end{clojure}
\end{english}

Обернем функцию в спеку:

\begin{english}
  \begin{clojure}
(s/def ::->date
  (s/and
   ::ne-string
   (s/conformer
    (fn [value]
      (try
        (read-instant-date value)
        (catch Exception e
          ::s/invalid))))))
  \end{clojure}
\end{english}

Как и с числом, перед разбором мы делаем минимальные проверки. Убеждаемся, что
это непустая строка, чтобы отсечь \spverb|nil| и прочий мусор. Разбор даты:

\begin{english}
  \begin{clojure}
(s/conform ::->date "2019-12-31")
#inst "2019-12-31T00:00:00.000-00:00"
  \end{clojure}
\end{english}

\noindent
Дата и время:

\begin{english}
  \begin{clojure}
(s/conform ::->date "2019-12-31T23:59:59")
#inst "2019-12-31T23:59:59.000-00:00"
  \end{clojure}
\end{english}

\section{Спеки-перечисления}

Иногда известно заранее, какие значения принимает поле. Например, при вызове API
клиент передает архитектуру системы~--- 32 или 64 бита. Ради двух значений нет
смысла парсить число: подойдет перебор или словарь.

Перебор работает через макрос \spverb|case|. Он пробегает по веткам-строкам и
возвращает аналогичные числа. Если ничего не найдено, сигналим об ошибке ключом
\spverb|::s/invalid|.

\begin{english}
  \begin{clojure}
(s/def ::->bits
  (s/conformer
   (fn [value]
     (case value
       "32" 32
       "64" 64
       ::s/invalid))))

(s/conform ::->bits "32") ;; 32
(s/conform ::->bits "42") :clojure.spec.alpha/invalid
  \end{clojure}
\end{english}

Вариант со словарем. По заданному словарю ищем результат с помощью
\spverb|get|. Если не нашли значение, вернем тег \spverb|invalid|.

\begin{english}
  \begin{clojure}
(def bits-map {"32" 32 "64" 64})

(s/def ::->bits
  (s/conformer
   (fn [value]
     (get bits-map value ::s/invalid))))
  \end{clojure}
\end{english}

Подход хорош тем, что опорная точка~--- словарь соответствий~--- живет в
отдельной переменной. Легко дополнить его новыми значениями или вынести в
конфигурацию. При этом логика проверки не изменится.

Похожим образом читают логические значения из строк. Нет единого соглашения о
том, как передать истину и ложь в тексте. Это может быть \spverb|True|,
\spverb|TRUE|, \spverb|1|, \spverb|on|, \spverb|yes| для истины и их
противоположности: \spverb|FALSE|, \spverb|no|, \spverb|off|... При разборе
значений их приводят к одному регистру. В Clojure \spverb|FALSE| и
\spverb|false| это разные строки, даже если отправитель имел в виду одно и то
же. Сценарий выглядит так:

\begin{itemize}

\item
  убедиться, что значение это строка;

\item
  привести ее к нижнему регистру;

\item
  определить значение по словарю или перебором.

\end{itemize}

\noindent
Пример из реального проекта:

\begin{english}
  \begin{clojure}
(s/def ::->bool
  (s/and
   ::ne-string
   (s/conformer clojure.string/lower-case)
   (s/conformer
    (fn [value]
      (case value
        ("true" "1" "on" "yes") true
        ("false" "0" "off" "no") false
        ::s/invalid)))))

  \end{clojure}
\end{english}

\noindent
В действии:

\begin{english}
  \begin{clojure}
(s/conform ::->bool "True") ;; true
(s/conform ::->bool "yes")  ;; true
(s/conform ::->bool "0")    ;; false
  \end{clojure}
\end{english}

\section{Продвинутые техники}

Мы написали достаточно кода, чтобы увидеть одинаковые участки~--- паттерны. В
этом разделе мы вынесем их в служебные функции и макросы. Заодно рассмотрим
приемы, которые ускорят вашу работу.

\subsection{Множества}

Когда значения известны, на роль спеки подходит множество. Оно ведет себя как
функция. Если аргумент найден в множестве, получим его же. Если нет, результат
будет \spverb|nil|. Представим, что статус задачи может быть \spverb|todo|,
\spverb|in_progress| и \spverb|done|. Опишем спеку множеством этих значений:

\begin{english}
  \begin{clojure}
(s/def ::status #{"todo" "in_progres" "done"})
(s/valid? ::status "todo") ;; true
  \end{clojure}
\end{english}

\subsection{Перечисления}

Множество не подходит в случаях, когда \spverb|false| и \spverb|nil| считают
верными значениями. \spverb|S/valid?| трактует их как неудачу. Если \spverb|nil|
или \spverb|false| входят в множество значений, их проверяют функцией
\spverb|contains?|:

\begin{english}
  \begin{clojure}
(contains? #{1 :a nil} nil) ;; true
  \end{clojure}
\end{english}

Чтобы не повторяться, напишем функцию \spverb|enum|. Она принимает значения и
возвращает функцию-предикат. Этот предикат принимает один аргумент и проверяет,
есть ли такой среди исходных значений.

\begin{english}
  \begin{clojure}
(defn enum [& args]
  (let [arg-set (set args)]
    (fn [value]
      (contains? arg-set value))))
  \end{clojure}
\end{english}

Функция внутри замкнута на переменной \spverb|arg-set|. Это множество, которое
получили из списка аргументов. Мы создали его один раз, чтобы не делать это
постоянно при вызове внутренней функции. Теперь перечисления выглядят коротко и
ясно:

\begin{english}
  \begin{clojure}
(s/def ::status
  (enum "todo" "in_progres" "done"))
  \end{clojure}
\end{english}

\subsection{With-conformer}

Спеки-конформеры требуют особого внимания. В них легко допустить ошибку: не
перехватить исключение или не обернуть функцию в \spverb|s/conformer|. Вынесем
рутину в макрос \spverb|with-conformer|. Он принимает символ переменной и
произвольное тело. Внутри макрос порождает функцию одного аргумента. Она
исполняет тело в блоке \spverb|try/catch|. Если исключения не было, результатом
будет последнее выражение тела. В противном случае вернется тег
\spverb|invalid|.

\begin{english}
  \begin{clojure}
(defmacro with-conformer
  [bind & body]
  `(s/conformer
    (fn [~bind]
      (try
        ~@body
        (catch Exception e#
          ::s/invalid)))))
  \end{clojure}
\end{english}

\noindent
Примеры из реального проекта. Вывод числа:

\begin{english}
  \begin{clojure}
(s/def ::->int
  (s/and
   ::ne-string
   (with-conformer val
     (Integer/parseInt val))))
  \end{clojure}
\end{english}

\noindent
Вывод логического типа:

\begin{english}
  \begin{clojure}
(s/def ::->bool
  (s/and
   ->lower
   (with-conformer val
     (case val
       ("true"  "1" "on"  "yes") true
       ("false" "0" "off" "no" ) false))))
  \end{clojure}
\end{english}

\noindent
Переменная \spverb|->lower| это тоже обертка для приведения регистра:

\begin{english}
  \begin{clojure}
(def ->lower
  (s/and
    string?
    (s/conformer clojure.string/lower-case)))
  \end{clojure}
\end{english}

В примере с \spverb|case| необязательно указывать \spverb|invalid| на конце
макроса. Если \spverb|case| не нашел ветку и не задан вариант по умолчанию, он
бросит исключение. \spverb|With-conformer| перехватит его и вернет
\spverb|invalid|.

\section{Логические пути}

Функция \spverb|s/conform| не всегда возвращает то, что мы ожидаем. Некоторые
спеки оборачивают результат в вектор, где первый элемент~--- логический путь. Он
появляется в случаях, где проверка ветвится.

Выше мы объединили спеки через \spverb|s/and|. Такая супер-спека обходит
дочерние по порядку и проверяет данные. Иногда линейного обхода недостаточно:
нужна развилка. Например, если значение число, то оставить его как есть, а если
строка, то привести к числу. Такие спеки называют условными (анг. conditional).

Макрос \spverb|s/or| задает условную спеку. Он принимает дочерние спеки и
теги. Макрос применяет значение к спекам до первого удачного случая. В
результате получим пару из значения и тега спеки, которая подошла.

Тег становится частью логического пути, по которому шла проверка. Логический
путь помогает расследовать, в каком месте произошла ошибка. Для простых спек
обычно это не проблема. В боевых проектах условная спека вложена в другую
условную, та тоже и так далее. Найти ошибку без логического пути будет трудно.

Если валидация не прошла, логический путь получают из отладочной информации. Ее
возвращают функции семейства \spverb|s/explain*|, которые мы рассмотрим ниже.

Напишем спеку сетевого порта, которая принимает число или строку. Во втором
случае спека парсит ее в число. Это полезно, если значение приходит из
переменной среды или ini-файла.

\begin{english}
  \begin{clojure}
(s/def ::smart-port
  (s/or :string ::->int :num int?))
  \end{clojure}
\end{english}

Теперь \spverb|s/conform| вернет не просто значение, а пару с тегом:

\begin{english}
  \begin{clojure}
(s/conform ::smart-port 8080)
[:num 8080]

(s/conform ::smart-port "8080")
[:string 8080]
  \end{clojure}
\end{english}

Если в спеке была развилка (\spverb|s/or|, \spverb|s/alt|), то структура
\spverb|s/conform| отличается от входных данных: на месте скаляра появится
вектор. Покажем это на вложенных данных. Пусть порт~--- одно из полей
подключения к базе:

\begin{english}
  \begin{clojure}
(s/def :conn/port ::smart-port)
(s/def ::conn
  (s/keys :req-un [:conn/port]))
  \end{clojure}
\end{english}

Топология результата будет другой, и это нужно учесть:

\begin{english}
  \begin{clojure}
(s/conform ::conn {:port "9090"})
{:port [:string 9090]}
  \end{clojure}
\end{english}

\section{Анализ ошибок}

Когда данные неверны, \spverb|s/valid?| и \spverb|s/conform| возвращают
\spverb|false| и \spverb|::s/invalid|. Этого недостаточно, чтобы понять причину
ошибки. Представьте, что у вас спека пользователя. В ней несколько адресов, в
каждом адресе несколько строк,... и проверка вернула \spverb|false|. Ручной
поиск ошибки в таком дереве займет час.

Функции семейства \spverb|s/explain| принимают спеку и данные. Если ошибок не
было, результат будет пустым. Если были, получим отчет о проверке. Это словарь,
где указаны проблемные значения, спеки, пути к ним и другие данные. Разница
между функциями в том, как они поступают с отчетом.

\begin{itemize}

\item
  \spverb|s/explain| печатает его в стандартный поток (на экран);

\item
  \spverb|s/explain-str| возвращает отчет в виде строки;

\item
  \spverb|s/explain-data| возвращает словарь с данными. Это самый полный отчет
  об ошибке.

\end{itemize}

Попробуем функции в действии. Их результат одинаковый, разница в том, куда он
приходит~--- в консоль или переменную. Подготовим простую спеку:

\begin{english}
  \begin{clojure}
(s/def :sample/username ::ne-string)

(s/def ::sample
  (s/keys :req-un [:sample/username]))
  \end{clojure}
\end{english}

На корректных данных функции не проявляют себя. \spverb|S/explain| печатает
\spverb|Success!|, \spverb|s/explain-data| вернет \spverb|nil|.

\begin{english}
  \begin{clojure}
(s/explain ::sample {:username "some user"})
Success!
nil

(s/explain-data ::sample {:username "some user"})
nil
  \end{clojure}
\end{english}

Попробуем число вместо имени:

\begin{english}
  \begin{clojure}
(s/explain ::sample {:username 42})
;; 42 - failed: string? in: [:username]
;; at: [:username] spec: ::string
  \end{clojure}
\end{english}

Вывод читается так: значение \spverb|42| не прошло проверку предикатом
\spverb|string?|. Путь к значению внутри словаря \spverb|[:username]|. Ключ
спеки, где случилась ошибка -- \spverb|::string|.

Отчет показывает наиболее вложенные спеки и предикаты. Вспомним, что
\spverb|::ne-string| это комбинация \spverb|::string| и
\spverb|not-empty|. Ошибка случилась на этапе \spverb|::string|, о чем и было
сказано.

Для пустой строки вывод будет другим. На этот раз причиной станет
\spverb|not-empty|. Проверим это:

\begin{english}
  \begin{clojure}
(s/explain ::sample {:username ""})
;; "" - failed: not-empty in: [:username]
;; at: [:username] spec: ::ne-string
  \end{clojure}
\end{english}

Со временем вы научитесь читать \spverb|explain|. Это быстрый способ сообщить о
проблеме в конфигурации или JSON-файле. Но чем сложнее данные, тем меньше
понятен \spverb|explain|. Когда в коллекции больше трех уровней, отчет заливает
экран выхлопом. Трудно даже разбить его на части, не говоря уж о понимании.

Чтобы подружить \spverb|explain| с человеком, нужно промежуточное звено. Речь о
нем пойдет в следующем разделе.

\section{Понятные ошибки}

Когда проверяют данные, важен не только факт ошибки. Еще важнее объяснить
клиенту, что не так с его данными. Под клиентом не обязательно имеют в виду
человека. Даже если это другая программа, в ответ добавляют понятный
текст. Скорее всего, программа ведет журнал, который читают сотрудники.

Часто мы видим сообщения вроде <<Ошибка: DATAERROR>> без каких-либо деталей. Или
красную надписью <<проверьте данные>> над формой в два экрана. Этих глупостей
можно было избежать, умей программисты переводить язык машины в человеческий.

Фраза \spverb|"" - failed: not-empty in: [:username]| не только ничего не скажет
пользователю, но и отпугнет его машинной природой. Возникнет ощущение, что в
интерфейсе возникла брешь, и пользователь видит то, что не должен. Это резко
снижает доверие к системе.

Чтобы составить сообщение об ошибке, вернемся к функции \spverb|s/explain-data|.
Она возвращает словарь с нужной информацией. Пример такого отчета:

\begin{english}
  \begin{clojure}
(s/explain-data ::sample {:username ""})

#:clojure.spec.alpha
{:problems
 ({:path [:username]
   :pred clojure.core/not-empty
   :val ""
   :via [::sample ::ne-string]
   :in [:username]})
 :spec ::sample
 :value {:username ""}}
  \end{clojure}
\end{english}

На первый взгляд непонятно, что с ним делать. Некоторые инженеры пасуют перед
проблемой и говорят, что Spec не подходит для ошибок. Это не так. В отчете все
необходимые данные, нужно только правильно их обработать.

Новички задают вопрос~--- почему бы не сделать понятные сообщения на уровне
библиотеки? Например, назначить спеке дополнительное поле с текстом <<введите
правильный адрес>>? Почему не взять пример с библиотек для Python или
JavaScript?

Ответ на этот вопрос не устраивает новичков. Вспомним тезис из начала
главы. Spec это \emph{фундаментальная библиотека}, набор абстракций и
примитивов. То, что мы проверяем спекой HTML-форму~--- всего лишь частный
случай. У спеки самые разные области применения, поэтому структура ошибки тоже
фундаментальна.

Трудно создать систему ошибок, которая устроит всех. В каждом проекте свои
правила о том, как показывать ошибки. Иногда это фиксированное сообщение, а в
других случаях шаблон. Где-то учитывают язык пользователя. Все вместе это
сложные сценарии.

Если бы разработчики Spec занялись ошибками, их фокус был бы смещен с главной
цели. Вместо Spec мы бы получили валидаторы по типу тех, что пишут десятками для
Python и JavaScript. Они скучны, не гибки и без концепции.

Словарь \spverb|explain-data| содержит ключи \spverb|:spec|, \spverb|:value| и
\spverb|:problems| с префиксом \spverb|clojure.spec.alpha|. Первые два это спека
и значение, которые принимали участие в проверке. Нас интересует поле
\spverb|problems|. Это список словарей; каждый словарь описывает ошибку
валидации. Перечислим его поля и семантику.

\begin{itemize}

\item
  \spverb|:path|~--- логический путь валидации. Вектор ключей, где спеки
  чередуются с тегами-развилками. Условные спеки вроде \spverb|s/or| пишут сюда
  свои метки.

\item
  \spverb|:pred|~--- символ предиката, например \spverb|clojure.core/string?|.

\item
  \spverb|:val|~--- конкретное значение, которое не прошло проверку на
  предикат. Например, 42, \spverb|nil|, элемент словаря.

\item
  \spverb|:via|~--- цепочка спек, по которым прошло значение от верхнего уровня
  к нижнему.

\item
  \spverb|:in|~--- физический путь к значению. Вектор ключей и индексов, который
  передают в функцию \spverb|get-in|. Если выполнить \spverb|(get-in <данные> <путь>)|,
  получим значение, которое вызвало ошибку.

\end{itemize}

Итак, в отчете все необходимое, чтобы собрать сообщение. Из \spverb|:val|
возьмем проблемное значение. Спека, на которой прервалась валидация это
последний элемент вектора \spverb|:via|.

Составим словарь сообщений, где ключ~--- спека, а значение~--- понятный текст
или шаблон. Зная спеку, которая вызвала ошибку, получим из словаря текст. В
нашем случае последний элемент \spverb|:via| это \spverb|::ne-string|. Назначим
ей сообщение <<Строка не должна быть пустой>> или что-то похожее.

%% \begin{english}
  \begin{clojure}
(def spec-errors
  {::ne-string "Строка не должна быть пустой"})
  \end{clojure}
%% \end{english}

Напишем функцию, которая принимает словарь ошибки (один из элементов
\spverb|::s/problems|) и возвращает понятное сообщение:

%% \begin{english}
  \begin{clojure}
(defn get-message [problem]
  (let [{:keys [via]} problem
        spec (last via)]
    (get spec-errors spec)))

(get-message {:via [::sample ::ne-string]})
"Строка не должна быть пустой"
  \end{clojure}
%% \end{english}

Проверим способ на других полях. В спеку \spverb|::sample| добавим поле
электронной почты:

\begin{english}
  \begin{clojure}
(s/def ::email
  (s/and
   ::ne-string
   (partial re-matches #"(.+?)@(.+?)\.(.+?)")))

(s/def :sample/email ::email)

(s/def ::sample
  (s/keys :req-un [:sample/username
                   :sample/email]))
  \end{clojure}
\end{english}

Спека \spverb|::email| убеждается, что строка не пустая и совпадает с шаблоном
адреса. Шаблон требует, чтобы в адресе был символ \spverb|@| и точка.

Если передать в \spverb|email| пустую строку, последним элементом \spverb|via|
будет \spverb|::ne-string|. Для экономии места сократим вывод
\spverb|explain-data|:

\begin{english}
  \begin{clojure}
(s/explain-data ::sample {:username "test" :email ""})

{:path [:email]
 :pred clojure.core/not-empty
 :val ""
 :via [::sample ::email ::ne-string]
 :in [:email]}
  \end{clojure}
\end{english}

Вызовем \spverb|get-message| с этой ошибкой. Получим сообщение о пустой
строке. Предположим теперь, почта была строкой, которая не совпала с
шаблоном. Тогда последним элементом \spverb|:via| будет
\spverb|:sample/email|. Полный словарь ошибки выглядит так:

\begin{english}
  \begin{clojure}
{:path [:email]
 :pred
 (clojure.core/partial
  clojure.core/re-matches
  #"(.+?)@(.+?)\.(.+?)")
 :val "test"
 :via [::sample ::email]
 :in [:email]}
  \end{clojure}
\end{english}

Чтобы \spverb|get-message| вернул другое сообщение, добавим в словарь ошибок
ключ \spverb|::email|:

%% \begin{english}
  \begin{clojure}
(def spec-errors
  {::ne-string "Строка не должна быть пустой"
   ::email "Введите правильный почтовый адрес"})
  \end{clojure}
%% \end{english}

Осталось наполнить словарь другими спеками и сообщениями, пока не покроем все
варианты. Ниже мы рассмотрим, как улучшить этот подход.

\subsubsection{Сообщение по умолчанию}

Что случится, если перевода нет в словаре? В этом случае вернем что-то
нейтральное, например, <<исправьте ошибку в данных>>. Заодно запишем в лог
событие с именем спеки. Лог настроен так, что сообщения из этого модуля идут в
отдельный файл. Позже локализаторы прочтут его и добавят перевод.

%% \begin{english}
  \begin{clojure}
(def default-message
  "Исправьте ошибку в данных")

(defn get-better-message [problem]
  (let [{:keys [via]} problem
        spec (last via)]
    (or (get spec-errors spec)
        (do (log/debugf "missing message for spec %s" spec)
            default-message))))
  \end{clojure}
%% \end{english}

\subsubsection{Гибкий поиск}

Упростим поиск ключа в словаре. Поле \spverb|email| встречается в разных спеках:
\spverb|:account/email|, \spverb|:patient/email| и других. Согласно методу выше,
каждый из ключей должен быть в словаре. Это склоняет нас к повторам в коде.

Чтобы не засорять словарь, пойдем на хитрость. Пусть функция ищет перевод по
полному ключу, а если его нет, то по имени. Хватит ключа \spverb|:email|, чтобы
все емейлы сошлись в этот перевод. Если для конкретного \spverb|:account/email|
нужна особая фраза, добавим его полную версию в словарь:

%% \begin{english}
  \begin{clojure}
(def spec-errors
  {::ne-string "Строка не должна быть пустой"
   :email "Введите правильный почтовый адрес"
   :account/email "Почтовый адрес сотрудника содержит ошибки"})
  \end{clojure}
%% \end{english}

\noindent
Обновим поиск с учетом неполного ключа и фразы по умолчанию:

\begin{english}
  \begin{clojure}
(defn get-better-message
  [problem]
  (let [{:keys [via]} problem
        spec (last via)]
    (or (get spec-errors spec)
        (get spec-errors (-> spec name keyword))
        default-message)))
  \end{clojure}
\end{english}

Система, которую мы построили, довольно проста. Ее легко тестировать и менять
под нужды конкретного проекта. Доработанные версии этой системы работают в
бою. В одном из них формы проверяют на клиенте до отправки на сервер. Это
возможно, поскольку мощь Spec в полной мере доступна в ClojureScript.

\subsubsection{Мультиметод}

Возможно, система переводов понравится коллегам, и они захотят использовать ваши
наработки. Разумно вынести код в библиотеку и подключить в зависимости. Тогда
все проекты фирмы будут одинаковы в плане сообщений об ошибках.

Недостаток словаря в том, что его трудно расширить со стороны. Чтобы добавить
перевод, придется выпустить новую версию библиотеки. Предоставим клиентам только
механизм перевода, а содержимое они наполнят сами.

Для этого заменим словарь на мультиметод. Его диспатчер принимает словарь ошибки
и находит виновную спеку. Далее расширяем мультиметод спеками. С таким подходом
каждый добавит свои переводы или заменит чужие, если они не подошли.

Объявим мультиметод с одним переводом:

%% \begin{english}
  \begin{clojure}
(defmulti problem->text
  (fn [{:keys [via]}]
    (last via)))

(defmethod problem->text ::ne-string [_]
  "Строка не должна быть пустой")
  \end{clojure}
%% \end{english}

Пример его работы:

%% \begin{english}
  \begin{clojure}
(problem->text {:val "" :via [::ne-string]})
"Строка не должна быть пустой"
  \end{clojure}
%% \end{english}

Ключ \spverb|:default| отвечает за действие по умолчанию. Если перевод не
найден, вернем стандартную фразу:

\begin{english}
  \begin{clojure}
(defmethod problem->text :default [_]
  default-message)
  \end{clojure}
\end{english}

Вспомним проблему с адресом почты. Хотелось бы, чтобы \spverb|:account/email| и
\spverb|:client/email| сходились в общий \spverb|::email|, но так, чтобы можно
было задать им частный перевод. Это возможно с помощью иерархии ключей. Если у
мультиметода нет действия для ключа, но ключ наследует родителя, мультиметод
выполнит поиск для родителя.

Добавим перевод почты и унаследуем ключ \spverb|:account/email| от
\spverb|::email|. Если ошибка случится в спеке \spverb|:account/email|, получим
общий перевод для любого типа почты.

%% \begin{english}
  \begin{clojure}
(defmethod problem->text ::email [_]
  "Введите правильный почтовый адрес")

(derive :account/email ::email)

(problem->text {:val "" :via [:account/email]})
"Введите правильный почтовый адрес"
  \end{clojure}
%% \end{english}

Если нужен особый перевод, расширим мультиметод:

%% \begin{english}
  \begin{clojure}
(defmethod problem->text :account/email [_]
  "Введите почту сотрудника")
  \end{clojure}
%% \end{english}

Заметим, что каждый метод принимает словарь ошибки. Мы затенили его символом
\spverb|_|, потому что перевод зависит только от ключа. В особых случаях можно
построить фразу в зависимости от полей словаря, например добавить текущее
значение:

%% \begin{english}
  \begin{clojure}
(defmethod problem->text :account/email
  [{:keys [val]}]
  (format "Ошибка в адресе почты: %s" val))

(problem->text {:val "test" :via [:account/email]})
"Ошибка в адресе почты: test"
  \end{clojure}
%% \end{english}

\subsubsection{Шаблон}

Если варианты выше показались сложными, попробуйте сообщение по шаблону. Оно
складывается из имени поля и значения, например: <<в поле email неверное
значение test>>. Сообщение легко получить функцией \spverb|format|. В нем
слышится машинное происхождение, но это лучше, чем ничего.

Имя поля получим как последний отличный от цифры элемент \spverb|:in|. Цифры
означают индексы вектора, поэтому их отбрасывают. Значение, которое привело к
ошибке, получим из поля \spverb|:val|.

\begin{english}
  \begin{clojure}
(defn get-common-message [problem]
  (let [{:keys [in val]} problem
        field (last (remove int? in))]
    (format "The field `%s` has got an incorrect value `%s`."
            (name field) val)))
  \end{clojure}
\end{english}

Поле и значение мы обернули в кавычки, чтобы выделить из общего
текста. Проверим, что вернет функция:

\begin{english}
  \begin{clojure}
(get-common-message {:in [::user :user/email] :val "test"})
The field `email` has got an incorrect value `test`.
  \end{clojure}
\end{english}

\subsection{Открытые вопросы}

За рамками остались несколько вопросов. Они слишком общие, чтобы претендовать на
одно решение. В этом разделе мы не будем писать код, а только порассуждаем.

Что делать, если требуется локализация, то есть текст на русском или английском
в зависимости от состояния? Переделаем словарь ошибок. Он станет вложенным
словарем, где на первом уровне код локали (\spverb|ru|, \spverb|en|), а на
втором~--- переводы спек.

На первом шаге мы получаем по локали словарь переводов. Затем переводим
сообщение как делали это выше. С кодом локали можно схитрить, чтобы облегчить
поиск. Для отдельных фраз выделяют более точные локали, например, американский и
британский английский с кодами \spverb|en_US| и \spverb|en_GB|. Напишем поиск
так, что сперва он ищет по младшей локали (\spverb|en_US|), а затем по старшей
(\spverb|en|). Если американского текста не оказалось, получим нейтральный
английский. Так устроен перевод текста в широком смысле, не только ошибок.

Вопрос откуда читать локаль остается на ваше усмотрение. Можно хранить ее в
сессии, параметрах запроса, базе данных, словом~--- как это удобно в проекте.

Второй вопрос~--- как связать ошибки с интерфейсом. Принято отделять модель от
ее представления; это относится и к формам. Удобно, когда форма это структура
данных, вложенный словарь. Операции над ней это чистые функции, которые легко
поддерживать.

Представим форму в виде дерева. Ключи это поля, а значения виджеты. Виджет
содержит тип поля, текущее значение и ошибку. На каждый лист подписан
React-компонент. При изменении виджета он рисует HTML-элемент, например поле
ввода с текущим текстом. Если ошибка не \spverb|nil|, над полем ввода появится
красный текст.

Валидаця принимает форму и строит дерево значений. У него такая же топология, но
на месте виджетов значения полей ввода. С помощью спеки мы проверяем значения и
выводим типы из строк. В случае ошибки получим отчет \spverb|explain|. Для
каждого элемента из поле \spverb|problems| находим путь, спеку и сообщение об
ошибке. Это сообщение добавляем виджету в поле \spverb|:error|. Компонент,
который подписан на виджет, заново отрисует его с ошибкой над полем.

Мы упомянули формы, React и проблемы интерфейса. Все вместе это называется
\emph{фронтенд}. Мы не будем на нем останавливаться, потому что фронтенд~---
большая и сложная тема, достойная отдельной книги. Автор надеется, что в будущем
эта книга появится.

\section{Парсинг}

Мы научились проверять данные и выводить типы. Перейдем к более сложной
операции~--- парсингу. Под термином понимают разбор данных на части, поиск
структуры там, где прежде ее не было.

Возможно, вам приходилось писать регулярные выражения. Это шаблоны, которые
описывают структуру текста. Специальные функции принимают текст и регулярное
выражение. Они возвращают фрагменты текста, которые совпали с шаблоном.

Пример регулярного выражения это IP-адрес. Он состоит из четырех групп с
точками. Каждая группа это число от 0 до 255.

\begin{english}
  \begin{text}
\d{1,3}\.\d{1,3}\.\d{1,3}\.\d{1,3}
  \end{text}
\end{english}

В шаблоне мы ставим косую черту перед точкой. Это служебный символ, поэтому его
экранируют.

В регулярных выражениях применяют операторы \spverb|+|, \spverb|?|, \spverb|*| и
другие. Они указывают, сколько раз встречается шаблон перед ними. Например, один
и более раз, ни одного или один, произвольное число. В зависимости от оператора
шаблон захватывает разные части текста.

Представьте, что регулярные выражения откусывают текст частями. Та часть, что
легла на шаблон, уходит в результат. Остаток переходит к следующему шаблону, и
так далее.

Регулярные выражения подводят нас к \spverb|regex|-спекам. Это особые спеки для
разбора данных по шаблону. Разница в том, что входные данные это коллекции, а не
текст.

\subsection{Простой разбор}

Предположим, нужно разобрать массив пользователей. Каждый из них это кортеж
$\langle$номер, емейл, статус$\rangle$. Все значения строки. Для каждого
пользователя требуется:

\begin{itemize}

\item
  убедиться, что в кортеже именно три элемента;

\item
  привести номер к числу;

\item
  проверить емейл на минимальные критерии;

\item
  привести статус к перечислению (константе).

\item
  получить словарь с верными значениями.

\end{itemize}

Мы уже знакомы с \spverb|s/conformer|. Можно написать функцию, которая принимает
кортеж и выполняет действия выше. Это несложно, но такая функция будет монолитом
со слишком большим \emph{скоупом}. Рассмотрим способ лучше.

Спека \spverb|s/cat| служит для разбора коллекций. Она принимает набор тегов и
других спек. На вход подают список или вектор. \spverb|S/cat| накладывает его
элементы на спеки. Если они совпали, результатом будет словарь. Ключи этого
словаря~--- теги, значения~--- вызов дочерней спеки с элементом.

Составим спеку для разбора кортежа. Пригодятся вывод чисел и проверка почты,
которые мы написали. Опишем статус и соберем композицию спек:

\begin{english}
  \begin{clojure}
(s/def :user/status
  (s/and ->lower
         (with-conformer val
           (case val
             "active"  :USER_ACTIVE
             "pending" :USER_PENDING))))

(s/def ::user
  (s/cat :id ::->int
         :email ::email
         :status :user/status))
  \end{clojure}
\end{english}

\noindent
Положительный случай:

\begin{english}
  \begin{clojure}
(s/conform ::user ["1" "test@test.com" "active"])
{:id 1
 :email "test@test.com"
 :status :USER_ACTIVE}
  \end{clojure}
\end{english}

Варианты с плохим номером, почтой или не тем статусом не пройдут разбор. Примеры
ниже вернут \spverb|::s/invalid|:

\begin{english}
  \begin{clojure}
(s/conform ::user ["" "test@test.com" "active"])
(s/conform ::user ["1" "@test.com" "active"])
(s/conform ::user ["1" "test@test.com" "unknown"])
  \end{clojure}
\end{english}

\subsection{Условный разбор}

В примере выше поля в кортеже не меняются. На практике так бывает не
всегда. Иногда мы работаем с устаревшими форматами данных. В них бывают условия
вроде <<если перед номером стоит метка \spverb|BLOCKED|, пользователь
заблокирован>>. Например:

\begin{english}
  \begin{clojure}
blocked;1;test@test.com;active
  \end{clojure}
\end{english}

Это усложняет задачу, ведь теперь кортеж состоит из трех \emph{или} четырех
элементов. Сдвигается семантика полей: первый элемент не только номер, но и флаг
блокировки. Бывают и более сложные условия, особенно в старых данных.

В императивных языках такие условия порождают каскад \spverb|if/else|. В Clojure
проблему решают декларативно. Объявим спеку блокировки:

\begin{english}
  \begin{clojure}
(s/def ::blocked
  (s/and
   ->lower
   (s/conformer (partial = "blocked"))))
  \end{clojure}
\end{english}

Добавим ее в итоговую \spverb|s/cat|, но укажем, что она встречается ни разу или
только один раз. Для этого \spverb|::blocked| оборачивают спеку в \spverb|s/?|.
В регулярных выражениях знак вопроса делает то же самое.

\begin{english}
  \begin{clojure}
(s/def ::user
  (s/cat :blocked (s/? ::blocked)
         :id ::->int
         :email ::email
         :status :user/status))
  \end{clojure}
\end{english}

Теперь оба кортежа совпадают со спекой \spverb|::user|. Если пользователь
заблокирован, в словаре будет поле \spverb|:blocked|:

\begin{english}
  \begin{clojure}
(s/conform ::user ["1" "test@test.com" "active"])
{:id 1 :email "test@test.com" :status :USER_ACTIVE}

(s/conform ::user ["BLOCKED" "1" "test@test.com" "active"])
{:blocked true :id 1 :email "test@test.com" :status :USER_ACTIVE}
  \end{clojure}
\end{english}

Представим, что на входе коллекция кортежей. Чтобы не утруждать себя итерацией,
объявим спеку-коллекцию:

\begin{english}
  \begin{clojure}
(s/def ::users (s/coll-of ::user))

(def user-data
  [["1" "test@test.com" "active"]
   ["Blocked" "2" "joe@doe.com" "pending"]])

(s/conform ::users user-data)
[{:id 1 :email "test@test.com" :status :USER_ACTIVE}
 {:blocked true :id 2 :email "joe@doe.com" :status :USER_PENDING}]
  \end{clojure}
\end{english}

Отсеять заблокированных пользователей можно функцией \spverb|filter| с
предикатом \spverb|(complement :blocked)|.

\subsection{Практика}

С помощью Spec парсят не только данные, но и текст. Рассмотрим, как прочитать
INI-файл в словарь данных. INI\footurl{https://en.wikipedia.org/wiki/INI\_file}
это старый формат для конфигурации. Он состоит из заголовков в квадратных
скобках и пар $\langle$поле = значение$\rangle$. Пример условного
\spverb|config.ini|:

\begin{english}
  \begin{ini}
[database]
host=localhost
port=5432
user=test

[server]
host=127.0.0.1
port=8080
  \end{ini}
\end{english}

Наша цель получить словарь, где на первом уровне заголовки, а под ними~--- пары
полей и значений.

\begin{english}
  \begin{clojure}
{:database {:host "localhost"
            :port 5432}
 :server {:host "127.0.0.1"}}
  \end{clojure}
\end{english}

Если отбросить пустые строки и комментарии, формат сводится к грамматике
\spverb|([title], (key=value)*)*|, где звездочка означает сколько угодно раз, в
том числе ничего.

Для начала прочитаем строки из файла. Спеки не должны иметь побочных эффектов,
поэтому чтение файла выносят в отдельную функцию.

\begin{english}
  \begin{clojure}
(require '[clojure.java.io :as io])

(defn get-ini-lines [path]
  (with-open [src (io/reader path)]
    (doall (line-seq src))))
  \end{clojure}
\end{english}

Форма \spverb|doall| нужна, чтобы прочитать строки до выхода из \spverb|with-open|.

Переходим к парсеру. Это спека, которая принимает строки из ini-файла. Алгоритм
спеки следующий:

\begin{itemize}

\item
  удалить пустые строки и комментарии;

\item
  сгруппировать оставшиеся строки по заголовкам;

\item
  разбить пары $\langle$поле = значение$\rangle$;

\item
  построить вложенный словарь;

\item
  вывести типы и проверить значения.

\end{itemize}

Будем писать по принципу <<сверху вниз>>, словно все компоненты уже
готовы. Ниже~--- спека, которая решает задачу:

\begin{english}
  \begin{clojure}
(s/def ::->ini-config
  (s/and
   (s/conformer clear-ini-lines)
   (s/* :ini/section)
   (s/conformer remap-ini-data)
   ::ini-config))
  \end{clojure}
\end{english}

Напишем недостающие элементы. Функция \spverb|clear-ini-lines| убирает пустые
строки и комментарии. В формате INI они начинаются с символа решетки.

\begin{english}
  \begin{clojure}
(require '[clojure.string :as str])

(defn comment? [line]
  (str/starts-with? line "#"))

(defn clear-ini-lines [lines]
  (remove (some-fn comment? str/blank?) lines))
  \end{clojure}
\end{english}

Спека \spverb|(s/* :ini/section)| читается как <<ноль и более секций>>. Под
секцией понимают заголовок и произвольное число пар $\langle$поле =
значение$\rangle$. Запишем ее в виде \spverb|s/cat|:

\begin{english}
  \begin{clojure}
(s/def :ini/section
  (s/cat :title :ini/title :fields (s/* :ini/field)))
  \end{clojure}
\end{english}

Объявим \spverb|:ini/title|. Она проверяет, что строка это заголовок. Согласно
формату, заголовок пишут в квадратных скобках. Если первый и последний символ
строки это скобки, отбросим их:

\begin{english}
  \begin{clojure}
(s/def :ini/title
  (s/and
   #(str/starts-with? % "[")
   #(str/ends-with? % "]")
   (with-conformer line
     (subs line 1 (dec (count line))))))
  \end{clojure}
\end{english}

Вариант той же спеки с регулярным выражением:

\begin{english}
  \begin{clojure}
(s/def :ini/title
  (with-conformer line
    (or (second (re-matches #"^\[(.+)\]$" line))
        ::s/invalid)))
  \end{clojure}
\end{english}

Спека \spverb|:ini/field| парсит поле и значение. Строку разбивают по знаку
равенства. Цифра 2 означает, что в конечном списке должно быть не более двух
элементов: ключ и значение. Это важно, потому что в значении может быть знак
равенства, например в base64-строке.

\begin{english}
  \begin{clojure}
(s/def :ini/field
  (with-conformer line
    (let [pair (str/split line #"=" 2)]
      (if (= (count pair) 2)
        pair
        ::s/invalid))))
  \end{clojure}
\end{english}

Проверяем, что действительно получили пару, иначе сигналим об ошибке. Запустим
урезанную версию спеки. Обернем ее в функцию \spverb|parse-ini|, которая читает
файл.

\begin{english}
  \begin{clojure}
(s/def ::->ini-config
  (s/and
   (s/conformer clear-ini-lines)
   (s/* :ini/section)))

(defn parse-ini [path]
  (let [lines (get-ini-lines path)]
    (s/conform ::->ini-config lines)))
  \end{clojure}
\end{english}

\noindent
Черновой результат:

\begin{english}
  \begin{clojure}
(parse-ini "config.ini")

[{:title "database"
  :fields [["host" "localhost"]
           ["port" "5432"]
           ["user" "test"]]}
 {:title "server"
  :fields [["host" "127.0.0.1"]
           ["port" "8080"]]}]
  \end{clojure}
\end{english}

Разбор прошел удачно. Читатель заметит, что структура отличается от той, что мы
планировали. Это неважно: главное, мы вывели данные из текста. Привести словарь
к нужному виду легко. Напишем функцию \spverb|remap-ini-data|:

\begin{english}
  \begin{clojure}
(defn remap-ini-data [data-old]
  (reduce
   (fn [data-new entry]
     (let [{:keys [title fields]} entry]
       (assoc data-new title (into {} fields))))
   {}
   data-old))
  \end{clojure}
\end{english}

\noindent
Если передать в нее вектор из последнего шага, получим то, что ожидали:

\begin{english}
  \begin{clojure}
{"database" {"host" "localhost" "port" "5432" "user" "test"}
 "server" {"host" "127.0.0.1" "port" "8080"}}
  \end{clojure}
\end{english}

Напишем спеку, чтобы вывести типы и проверить значения. Например, чтобы номера
портов стали числами, а строки не могли быть пустыми:

\begin{english}
  \begin{clojure}
(s/def :db/host ::ne-string)
(s/def :db/port ::->int)
(s/def :db/user ::ne-string)

(s/def ::database
  (s/keys :req-un [:db/host :db/port :db/user]))

(s/def :server/host ::ne-string)
(s/def :server/port ::->int)

(s/def ::server
  (s/keys :req-un [:server/host :server/port]))

(s/def ::ini-config
  (s/keys :req-un [::database ::server]))
  \end{clojure}
\end{english}

Последний штрих~--- исправить тип ключей в словаре. Сейчас это строки, но спека
\spverb|::ini-config| ожидает ключи. Модуль \spverb|walk| предлагает функцию
\spverb|keywordize-keys| на этот случай. Она обходит словарь любой вложенности и
меняет ключи. Итоговая спека:

\begin{english}
  \begin{clojure}
(require '[clojure.walk :as walk])

(s/def ::->ini-config
  (s/and
   (s/conformer clear-ini-lines)
   (s/* (s/cat :title :ini/title :fields (s/* :ini/field)))
   (s/conformer remap-ini-data)
   (s/conformer walk/keywordize-keys)
   ::ini-config))
  \end{clojure}
\end{english}

\noindent
Результат:

\begin{english}
  \begin{clojure}
(parse-ini "config.ini")

{:database {:host "localhost"
            :port 5432
            :user "test"}
 :server {:host "127.0.0.1"
          :port 8080}}
  \end{clojure}
\end{english}

Получились аккуратные данные из текста. Обратите внимание, что в коде нет
состояния, и в целом он выглядит как цепочка шагов. Чтобы улучить разбор,
исправим один из шагов или добавим новый.

\emph{Упражнение:} устраните мелкие недостатки в коде. Пусть пара
\spverb|"foo="| становится \spverb|{:foo nil}|, а не \spverb|{:foo ""}|.
Удалите пустые символы из имен полей и значений. Опробуйте код на больших
ini-файлах.

\section{Разбор кода (теория)}

В завершение темы поговорим о том, как парсить код. Мы уже видели, что Spec
подходит для разбора коллекций. Код на Clojure состоит из списков. Это приводит
к неожиданному решению: код можно проверить спекой и вернуть ошибку до того, как
он запущен.

В начале главы мы упоминали, что Clojure и Spec неразрывно связаны. Объясним
связь на макросах. Это особые функции, которые работают на этапе
компиляции. Макрос принимает код в виде списка символов. В коде могут быть
ошибки, но макрос об этом ничего не знает: для него это просто символы.

Задача макроса в том, чтобы перестроить список в другой, понятный
Clojure. Компилятор заменяет вызов макроса на то, что он вернул, и запускает
код. Макросы это отдельная веха в изучении Clojure. Мы поговорим о них отдельно;
пока что рассмотрим, как проверить макрос спекой.

Каждый макрос это мини-язык с соглашением о том, что подавать на вход. В простых
случаях код парсят функциями \spverb|first|, \spverb|rest| и условиями. Сложные
макросы разбирают грамматиками, как мы делали это с ini-файлом. Если код
нарушает правила, мы должны объяснить программисту, в чем ошибка.

Иногда один и тот же макрос допускает разную запись. Хорошим примером служит
\spverb|defn|, макрос определения функции. Кроме обязательных параметров он
принимает опциональные: строку документации, пре- и пост-проверки. У функции
может быть несколько тел:

\noindent
\begin{tabular}{ @{}p{2cm} @{}p{4cm} @{}p{3cm} }

\begin{english}
  \begin{clojure}
(defn my-inc
  [x]
  (+ x 1))
  \end{clojure}
\end{english}

&

\begin{english}
  \begin{clojure}
(defn my-inc
  "Increase the number."
  [x]
  {:pre [(int? x)]
   :post [(int? %)]}
  (+ x 1))
  \end{clojure}
\end{english}

&

\begin{english}
  \begin{clojure}
(defn my-inc
  ([x]
   (my-inc x 1))
  ([x delta]
   (+ x delta)))
  \end{clojure}
\end{english}

\end{tabular}

Это одна и та же функция, записанная по-разному. Очевидно, разобрать все
варианты вручную тяжело. До версии Clojure 1.10 каждый макрос парсил код как
придется. Это было неорганизованно. С выходом Spec основные макросы перешли на
спеку. Появился общий подход, которым легко управлять.

Разберем устно, как бы мы построили спеку для разбора \spverb|defn|. Это список,
поэтому на верхнем уровне поместим \spverb|s/cat|. Первый его элемент~--- символ
\spverb|defn|. Второй~--- символ с именем функции. После имени следует строка
документации (ее может и не быть). Далее~--- тело или список тел. Пока мы не
ушли удалеко, набросаем черновик:

\begin{english}
  \begin{clojure}
(s/def ::defn
  (s/cat :tag (partial = 'defn)
         :name symbol?
         :doc (s/? string?)
         :body :defn/body*))
  \end{clojure}
\end{english}

Что скрыто за спекой \spverb|:defn/body*| пока неизвестно. Считаем, что тело
начинается с вектора параметров. После него идет опциональный словарь пре- и
пост- проверок. Затем произвольные формы, из которых состоит тело функции.

\begin{english}
  \begin{clojure}
(s/def :defn/body
  (s/cat :args vector?
         :prepost (s/? map?)
         :code (s/* any?)))
  \end{clojure}
\end{english}

Проблема в том, что \spverb|defn| принимает либо одно тело, либо несколько
обернутых в списки. Сравните первый и третий столбики в примере с
\spverb|my-inc|. Обернем спеку \spverb|:defn/body| так, чтобы она учитывала оба
случая. Обозначим эту спеку звездочкой:

\begin{english}
  \begin{clojure}
(s/def :defn/body*
  (s/alt :single :defn/body
         :multi (s/+ (s/spec :defn/body))))
  \end{clojure}
\end{english}

Теперь <<заморозим>> форму \spverb|defn| с помощью апострофа; получится список
символов. Отправим его в \spverb|s/conform|:

\begin{english}
  \begin{clojure}
(s/conform
 ::defn
 '(defn my-inc
    "Increase a number"
    [x]
    {:pre [(int? x)]
     :post [(int? %)]}
    (+ x 1)))
  \end{clojure}
\end{english}

\noindent
Результат:

\begin{english}
  \begin{clojure}
{:tag defn
 :name my-inc
 :doc "Increase a number"
 :body
 [:single
  {:args [x]
   :prepost {:pre [(int? x)] :post [(int? %)]}
   :code [(+ x 1)]}]}
  \end{clojure}
\end{english}

Заметим, как ведет себя поле \spverb|:body|. Это вектор из метки и
результата. Для одного тела получим метку \spverb|:single| и словарь. Для
нескольких тел метка будет \spverb|:multi|, а значение~--- вектор словарей:

\begin{english}
  \begin{clojure}
[:multi [{:args [x] :code [(println 1)]}
         {:args [x y] :code [(println 2)]}]]
  \end{clojure}
\end{english}

Чтобы проверить метку (одно тело или несколько), пригодится оператор
\spverb|case|. Ниже в переменой \spverb|result| записан результат
парсинга. Функция \spverb|process-body| обрабатывает словарь тела:

\begin{english}
  \begin{clojure}
(let [{:keys [body]} result
      [tag body] body]
  (case tag
    :single
    (process-body body)
    :multi
    (doseq [body body]
      (process-body body))))
  \end{clojure}
\end{english}

\subsection{Самостоятельная работа}

Каждый уровень спеки расширяется вглубь. Доработаем аргументы функции: разделим
их на обязательные и необязательные. Например, чтобы параметры \spverb|[x y & other]|
предстали в виде словаря:

\begin{english}
  \begin{clojure}
{:req [x y] :opt other}
  \end{clojure}
\end{english}

\noindent
По аналогии разберите словари пре- и пост- проверок.

Передайте в спеку данные с ошибками. Что делать в таком случае? Как составить
\emph{понятное} сообщение о том, где именно ошибка и что вы ожидали? Подойдет ли
словарь переводов? Получится ли у вас сделать ошибки лучше, чем в промышленных
языках?

Данные, которые вернула спека-парсер, называются
\emph{абстрактным синтаксическим деревом}\footurl{https://en.wikipedia.org/wiki/Abstract\_syntax\_tree}
(анг. abstract syntax tree, AST). Это вложенная структура, которую получают из
текста. AST~--- важный этап компиляции программы. Только построив дерево, можно
выполнить логику, которая стоит за ним.

Spec работает в том числе как парсер грамматик. Можно разобрать любые данные и
построить дерево. Вы в шаге от того, чтобы написать простой интерпретатор~---
программу, которая читает код и выполняет его. Узлы дерева это функции, а
потомки~--- их аргументы. Напишите функцию, которая обходит дерево и вычисляет
его в верном порядке. Даже если это курсовая работа, свой язык или компилятор
улучшит ваши навыки.

%% ----------------

\section{Спецификация функций}

Поговорим о том, как clojure.spec связана с функциями. Мы уже упоминали проблему
с входными данными. Даже если параметры верного типа, это ничего не говорит об
их семантике. Вспомним функцию, которая принимает диапазон дат. Случай, когда ее
вызвали с параметрами \spverb|start=2019.01.01| и \spverb|end=2009.01.01|, не
имеет смысла.

Опишем параметры функции спекой. Это \spverb|s/cat|, которая <<откусывает>> от
аргументов две даты и помещает в словарь с ключами \spverb|:start| и
\spverb|:end|.

\begin{english}
  \begin{clojure}
(s/def ::date-range-args
  (s/and
   (s/cat :start inst? :end inst?)
   (fn [{:keys [start end]}]
     (<= (compare start end) 0))))
  \end{clojure}
\end{english}

Вторая функция в \spverb|s/and| принимает этот словарь и сравнивает даты. Для
сравнения дат используют специальную функцию \spverb|compare|, которая возвращает
-1, 0 и 1 для случаев меньше, равно и больше. Быстрая проверка:

\begin{english}
  \begin{clojure}
(s/valid? ::date-range-args [#inst "2019" #inst "2020"]) ;; true
(s/valid? ::date-range-args [#inst "2020" #inst "2019"]) ;; false
  \end{clojure}
\end{english}

Витает мысль написать декоратор, который принимает функцию и спеку. Перед тем,
как запустить функцию, он проверит аргументы и в случае ошибки бросит
исключение. То же самое можно проделать для результата функции.

Писать декоратор не придется, потому что его включили в clojure.spec. Речь о
функции \spverb|clojure.spec.test.alpha/instrument| (анг. instrument~---
оснастить, оборудовать). Обратите внимание на пространство: в него закралась
частичка <<test>>. Оснащение функций вынесли в отдельный модуль.

\spverb|Instrument| принимает символ функции, которую <<заряжают>>. Одновременно
идет поиск особой спеки, объявленной с таким же символом. Когда функция и спека
найдены, \spverb|instrument| подменяет функцию на такую же, но с проверками. Это своего
рода <<monkey patch>>, когда один модуль изменяет другой.

Функциональную спеку объявляют макросом \spverb|s/fdef|. Ему передают символ
функции, которую хотят оснастить. Затем отдельные спеки для проверки входящих
параметров, результата и их композиции.

Напишем функцию и спеку к ней. Пусть функция считает разницу между двумя датами
в секундах. В отличие от примера выше, мы допускаем случай, когда первая дата
больше второй. В этом случае разница отрицательная.

\begin{english}
  \begin{clojure}
(import 'java.util.Date)

(defn date-range-sec
  "Return the difference between two dates in seconds."
  [^Date date1 ^Date date2]
  (quot (- (.getTime date2)
           (.getTime date1))
        1000))
  \end{clojure}
\end{english}

Теги \spverb|^Date| нужны, чтобы компилятор знал тип объектов \spverb|date1| и
\spverb|date2|. В противном случае тип приходится вычислять во время исполнения,
что съедает ресурсы. Мы поговорим о типах в отдельной главе.

Посчитаем разницу в сутках:

\begin{english}
  \begin{clojure}
(date-range-sec #inst "2019-01-01" #inst "2019-01-02")
86400
  \end{clojure}
\end{english}

\noindent
Если поменять даты местами, результат будет отрицательным.

Опишем функциональную спеку. Ее символ будет \spverb|date-range-sec|. Под ключом
\spverb|:args| указывают спеку входящих параметров. Это список, поэтому пригодится
\spverb|s/cat|. Он разбивает список на словарь, чтобы спеки ниже работали
с отдельными ключами.

В \spverb|:ret| указана спека выходного значения. Чаще всего это проверка на число
или строку. Например, \spverb|int?|, \spverb|string?| или их \emph{nilable}-версии:
\spverb|(s/nilable int)| и так далее.

Вот как выглядит спека \spverb|date-range-sec|. Она проверяет входные параметры
и результат:

\begin{english}
  \begin{clojure}
(s/fdef date-range-sec
  :args (s/cat :start inst? :end inst?)
  :ret int?)
  \end{clojure}
\end{english}

В ключи \spverb|:args| и \spverb|:ret| можно передать уже объявленные спеки. Это
полезно для повторного использования кода. У вас может быть несколько функций,
которые принимают диапазон дат. Задайте спеку параметров и ссылайтесь на нее в
каждой из \spverb|s/fdef|.

Функциональная спека сама по себе не меняет функцию. Она только объявляет проверки,
но не запускает их. Чтобы подменить функцию на ее оснащенную версию, вызывают
\spverb|instrument|:

\begin{english}
  \begin{clojure}
(require '[clojure.spec.test.alpha :refer [instrument]])
(instrument `date-range-sec)
  \end{clojure}
\end{english}

Символ функции должен быть полным, то есть с пространством. Чтобы подставить в
символ текущее пространство, перед ним ставят обратную кавычку \spverb|`|.

Теперь \spverb|date-range-sec| проверяет аргументы и результат. Что
случится, если передать в <<заряженную>> функцию не тот аргумент, например,
\spverb|nil|? Получим исключение \spverb|ExceptionInfo|:

\begin{english}
  \begin{clojure}
(date-range-sec nil #inst "2019")
  \end{clojure}
\end{english}

Его сообщение и тело нам знакомы. В поле \spverb|message| текст из функции
\spverb|s/explain-str|:

\begin{english}
  \begin{clojure}
Execution error - invalid arguments to date-range-sec
nil - failed: inst? at: [:start]
  \end{clojure}
\end{english}

В поле \spverb|data| структура, аналогичная \spverb|s/explain-data|. Чтобы
получить данные из исключения, его ловят и передают функции \spverb|(ex-data e)|.

\subsection{Производительность}

\spverb|Instrument| помогает в тестировании. Для важных функций пишут спеки в
отдельном модуле. Проект устроен так, что во время тестов загружаются
дополнительные модули, в том числе тот, который оснащает функции спеками. Если в
тестах функция получила не те аргументы, это станет заметно.

\spverb|Instrument| не подходит для боевого режима, потому что замедляет код
проверками. Напишем бенчмарк, который вызывает оснащенную функцию десять тысяч
раз:

\begin{english}
  \begin{clojure}
(time
 (dotimes [n 10000]
   (date-range-sec #inst "2019" #inst "2020")))
"Elapsed time: 116.984496 msecs"
  \end{clojure}
\end{english}

Получили десятую долю секунды. Посчитаем время исходной функции. Поскольку
\spverb|date-range-sec| уже оснащена, объявим функцию с таким же телом, но
другим именем, например \spverb|date-range-sec-orig|:

\begin{english}
  \begin{clojure}
(time
 (dotimes [n 10000]
   (date-range-sec-orig #inst "2019" #inst "2020")))
"Elapsed time: 1.783962 msecs"
  \end{clojure}
\end{english}

Разница в сто раз! Проверка в рантайме \emph{существенно} замедляет приложение,
поэтому \spverb|instrument| не претендует на промышленный запуск. Замедление
кода в сто раз~--- слишком дорогая цена. Наоборот, во время тестов нас не
волнует скорость. Код покрывают как можно б\'{о}льшим числом проверок, чтобы
поймать все ошибки.

\subsection{Документация}

Функциональная спека улучшает документацию. Функция \spverb|doc| из модуля
\spverb|clojure.repl| выводит на экран справку о запрошенной функции. С
появлением clojure.spec ее поведение изменилось. Кроме документации она выводит
спеку функции, если она задана. Вот как выглядит справка \spverb|date-range-sec|
после того, как мы объявили спеку:

\begin{english}
  \begin{clojure}
(clojure.repl/doc date-range-sec)
-------------------------
([date1 date2])
  Return the difference between two dates in seconds.
Spec
  args: (cat :start inst? :end inst?)
  ret: int?
  \end{clojure}
\end{english}

На функцию \spverb|doc| полагаются IDE и редакторы, чтобы подсказывать аргументы
в коде. Утилиты для документации, например \spverb|autodoc| или \spverb|codox|,
тоже учитывают спеки функций.

\section{Переиспользование спек}

В Clojure принято снабжать библиотеки спеками, чтобы помочь другим. Если
библиотека активно работает какой-то структурой данных, полезно описать ее
спекой. Хорошим примером служит \spverb|clojure.jdbc|\footurl{https://github.com/clojure/java.jdbc}.
Это библиотека для реляционных баз данных. JDBC-подключение задают словарем с ключами
\spverb|:host|, \spverb|:port|, \spverb|:user| и так далее.

Конфигурацию базы проверяют перед тем, как подключиться к ней. Иначе вы рискуете
получить \spverb|NullPointerException| и другие странности. Модуль
\spverb|clojure.java.jdbc.spec| несет на борту спеки подключения и основных
функций. Импортируйте модуль, чтобы его спеки попали в глобальный реестр.

Предположим, конфигурация лежит в edn-файле. Ключ \spverb|:db| задает
подключение к базе.

\begin{english}
  \begin{clojure}
{:db {:dbtype "mysql"
      :host "127.0.0.1"
      :port 3306
      :dbname "project"
      :user "user"
      :password "********"
      :useSSL true}}
  \end{clojure}
\end{english}

Ключи \spverb|:dbtype|, \spverb|:host| и другие уже описаны в библиотеке.
Используем их повторно. Спека файла выглядит так:

\begin{english}
  \begin{clojure}
(require '[clojure.java.jdbc.spec :as jdbc])
(s/def ::db ::jdbc/db-spec)
(s/def ::config (s/keys :req-un [::db]))
  \end{clojure}
\end{english}

Прочитаем конфигурацию с помощью \spverb|read-string| и \spverb|slurp| и
проверим спекой:

\begin{english}
  \begin{clojure}
(def config (read-string (slurp "config.edn")))
(s/valid? ::config config)
  \end{clojure}
\end{english}

Иногда спеку выносят в отдельную библиотеку как дополнение к основной. Так
поступили разработчики \spverb|alia|~--- клиента для БД Cassandra. Библиотека
\spverb|qbits.alia| несет основные функции для работы с базой. Спеки для нее
идут в отдельном проекте \spverb|cc.qbits/alia-spec|.

\section{Дополнения}

Spec входит в поставку Clojure и меняется не так часто, как хотелось бы
разработчикам. Дополнения к spec выпускают в виде отдельных библиотек. Среди
прочих заслуживают внимания два проекта: \spverb|expound| и
\spverb|spec.tools|. В этом разделе мы опишем возможности каждого.

Библиотека \spverb|expound|\footurl{https://github.com/bhb/expound} улучшает сообщения
об ошибках, делает их понятней для человека. Сигнатура функции \spverb|expound|
аналогична \spverb|s/explain|. Она тоже принимает спеку и данные. Сообщение об
ошибке выглядит так:

\begin{english}
  \begin{clojure}
(expound/expound string? 1)
-- Spec failed --------------------
  1
should satisfy
  string?
-------------------------
Detected 1 error
  \end{clojure}
\end{english}

Отчет все еще выглядит машинным, и мы не можем показать его пользователю. Все же
он лучше, чем сырой \spverb|s/explain|. Например, его могут прочитать коллеги из
команды Ops, которые не знают Clojure. \spverb|Expound| подходит для проверки
конфигурации на старте приложения. Иногда код не меняется месяцами, но
конфигурацию обновляют часто, поэтому внятный отчет о проблемах важен.

Разработчики из Metosin собрали улучшения к clojure.spec в проекте
\spverb|spec.tools|\footurl{https://github.com/metosin/spec-tools}. В сердце этой
библиотеки особый объект \spverb|Spec|. Он оборачивает стандартную спеку и дополняет ее
различными методами. С помощью \spverb|spec.tools| формируют JSON-схему или
описывают REST-проект по стандарту Swagger. Библиотека полезна как промежуточный
слой между REST-фреймворком и спекой.

Обе библиотеки просты в техническом плане. Читателю не составит труда
разобраться с ними, когда на возникнет потребность.

\section{Будущее спеки}

На сегодняшний день пакет clojure.spec все еще не избавился от частички
<<alpha>> в названии. Авторы все еще экспериментируют со спекой, ищут лучшие
способы валидации. Это смущает некоторых разработчиков. Опасаясь, что по
окончании эксперимента от spec избавятся, они берут альтернативы:
\spverb|schema|\footurl{https://github.com/plumatic/schema},
\spverb|bouncer|\footurl{https://github.com/leonardoborges/bouncer} и другие.

Отдельные группы пишут обертки над спекой, чтобы расширить ее
возможности. Например, подружить ее с JSON-схемами и популярными инструментами
вроде Swagger. Это путь \spverb|spec.tools|, который мы только что рассмотрели.

В докладе <<Maybe Not>>\footurl{https://youtube.com/watch?v=YR5WdGrpoug} Рич Хикки
анонсировал вторую версию спеки. В ней упростят работу со словарями (эту
возможность назвали \emph{select}) и множественными типами (когда значением
может быть и строка, и число). Разработка идет в открытом режиме, но еще рано
говорить о результатах. Обсуждение второй спеки выходит за рамки главы.

\section{Итог}

Spec это библиотека Clojure. С версии 1.9 она идет в поставке с языком. Spec
предлагает набор функций и макросов. Ими описывают правила, которым
удовлетворяют данные. Правила это предикаты, т.е. функции, которые возвращают
истину или ложь.

Предикаты гибче и мощнее типов. Если о значении известно, что оно верного типа,
это еще не гарантирует его корректность. Значение \spverb|-1| не может быть
Unix-портом. Классы-обертки вроде \spverb|UnixPort| это не типы, а валидация в
рантайме. Она привязана к вызову класса синтаксическим сахаром.

В отличие от классов, предикаты компонуются друг с другом. Легко написать
супер-предикат с логикой <<каждый из>>, <<любой из>> и так далее.

Spec выводит новые данные из старых. Функция \spverb|S/conform| оборачивает
другую функцию, которая преобразует значение. В основном этим пользуются для
разбора текста, например, при чтении файла или полей HTTP-запроса.

Библиотека несет на борту regex-спеки, похожие на регулярные выражения. От
обычных спек они отличаются тем, что захватывают часть коллекции. На
regex-спеках пишут парсеры данных, в том числе для нужд самой Clojure. Основные
макросы проверяют свое тело с помощью spec.

Spec пригодится в тестах. Функция \spverb|instrument| изменяет другую функцию
так, что ее аргументы проверяются спекой. Это замедляет обращение к ней, но
делает тесты надежней. Функциональная спека становится частью документации.

Spec не предлагает решения для понятных сообщений об ошибках. Как их строить и
показывать клиенту, зависит от проекта. Один из подходов основан на словаре, где
ключи~--- спеки, а значения~--- текст ошибки или тег перевода.

Мы рассмотрели основные возможности библиотеки. В обсуждение не попали
спеки-комбинаторы вроде \spverb|s/alt|, генераторы, мульти-спеки и другие
интересные вещи. Все это читатель найдет в официальной документации. Мы не
прощаемся с спекой: в следующих главах мы вернемся к ней.
